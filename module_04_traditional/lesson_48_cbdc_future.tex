\documentclass[8pt,aspectratio=169]{beamer}
\usetheme{Madrid}
\usepackage{graphicx,booktabs,adjustbox,multicol,amsmath,amssymb}
\definecolor{mlblue}{RGB}{0,102,204}
\definecolor{mlpurple}{RGB}{51,51,178}
\definecolor{mllavender}{RGB}{173,173,224}
\definecolor{mllavender2}{RGB}{193,193,232}
\definecolor{mllavender3}{RGB}{204,204,235}
\definecolor{mllavender4}{RGB}{214,214,239}
\definecolor{mlorange}{RGB}{255,127,14}
\definecolor{mlgreen}{RGB}{44,160,44}
\definecolor{mlred}{RGB}{214,39,40}
\setbeamercolor{palette primary}{bg=mllavender3,fg=mlpurple}
\setbeamercolor{palette secondary}{bg=mllavender2,fg=mlpurple}
\setbeamercolor{palette tertiary}{bg=mllavender,fg=white}
\setbeamercolor{structure}{fg=mlpurple}
\setbeamercolor{frametitle}{fg=mlpurple,bg=mllavender3}
\setbeamertemplate{navigation symbols}{}
\setbeamertemplate{itemize items}[circle]
\setbeamercolor{palette primary}{bg=mllavender3,fg=mlpurple}
\setbeamercolor{palette secondary}{bg=mllavender2,fg=mlpurple}
\setbeamercolor{palette tertiary}{bg=mllavender,fg=white}
\setbeamercolor{structure}{fg=mlpurple}
\setbeamercolor{frametitle}{fg=mlpurple,bg=mllavender3}
\setbeamertemplate{navigation symbols}{}
\setbeamertemplate{itemize items}[circle]
\setbeamersize{text margin left=5mm,text margin right=5mm}

% Bottom note command for key takeaways
\newcommand{\bottomnote}[1]{%
\vfill
\vspace{-2mm}
\textcolor{mllavender2}{\rule{\textwidth}{0.4pt}}
\vspace{1mm}
\footnotesize
\textbf{#1}
}
\title{Lesson 48: CBDCs and Future of Digital Finance}
\subtitle{Module 4: Traditional Digital Finance}
\author{Digital Finance Course}
\date{2025}

\begin{document}

\begin{frame}
\titlepage
\end{frame}

\begin{frame}[t]{Learning Objectives}
\begin{itemize}
\item Understand Central Bank Digital Currency (CBDC) design principles and architectures
\item Analyze Digital Euro project and global CBDC landscape
\item Examine retail vs wholesale CBDC models
\item Evaluate programmable money and smart contract integration
\item Assess future trends in traditional digital finance
\end{itemize}
\end{frame}

\section{CBDC Fundamentals}

\begin{frame}[t]{Global CBDC Adoption Status}
\begin{center}
\includegraphics[width=0.60\textwidth]{figures/cbdc_adoption/cbdc_adoption.pdf}
\end{center}
\bottomnote{Central bank digital currency projects span research, pilots, and live deployments.}
\end{frame}

\begin{frame}[t]{What is a CBDC?}
\begin{columns}[T]
\scriptsize
\column{0.48\textwidth}
\textbf{Definition:}

\textit{Central Bank Digital Currency (CBDC) is a digital form of central bank money, distinct from balances in traditional reserve or settlement accounts.}

\vspace{2mm}
\textbf{Key Characteristics:}
\begin{itemize}
\item \textbf{Central Bank Liability:} Direct claim on central bank (like cash)
\item \textbf{Digital:} Electronic, not physical currency
\item \textbf{Legal Tender:} Government-backed, accepted for payments
\item \textbf{Programmable:} Potential for conditional payments
\item \textbf{Account-Based or Token-Based:} Identity vs bearer instrument
\end{itemize}

\vspace{2mm}
\textbf{CBDC vs Other Digital Money:}
\begin{itemize}
\item \textbf{Commercial Bank Deposits:} Bank liability, deposit insurance
\item \textbf{Cryptocurrencies:} Decentralized, volatile, no legal tender status
\item \textbf{Stablecoins:} Private issuers (Tether, USDC), reserves backing
\end{itemize}

\column{0.48\textwidth}
\textbf{Motivation for CBDCs:}

\textbf{Policy Goals:}
\begin{itemize}
\item \textbf{Payment Efficiency:} Faster, cheaper cross-border payments
\item \textbf{Financial Inclusion:} Access for unbanked populations
\item \textbf{Monetary Sovereignty:} Counter private stablecoins (Libra/Diem threat)
\item \textbf{Cash Decline:} Digital alternative as physical cash usage drops
\item \textbf{Innovation Platform:} Programmable money, smart contracts
\end{itemize}

\vspace{2mm}
\textbf{Risks and Concerns:}
\begin{itemize}
\item \textbf{Bank Disintermediation:} Flight to CBDC during crises
\item \textbf{Privacy:} Central bank surveillance potential
\item \textbf{Cybersecurity:} Attractive target for attacks
\item \textbf{Cross-Border Capital Flows:} Bypass capital controls
\item \textbf{Operational Complexity:} 24/7 availability, scalability
\end{itemize}
\end{columns}
\bottomnote{Clear definitions are essential for understanding complex technical concepts.}
\end{frame}

\begin{frame}[t]{Retail vs Wholesale CBDCs}
\begin{columns}[T]
\scriptsize
\column{0.48\textwidth}
\textbf{Retail CBDC (General Purpose):}

\textbf{Users:}
\begin{itemize}
\item Households and businesses
\item Direct access to central bank money
\item Digital cash alternative
\end{itemize}

\textbf{Use Cases:}
\begin{itemize}
\item Everyday payments (groceries, bills)
\item P2P transfers
\item E-commerce
\item Government disbursements (stimulus, benefits)
\end{itemize}

\textbf{Design Considerations:}
\begin{itemize}
\item Distribution model (direct vs two-tier)
\item Anonymity vs AML compliance
\item Interest-bearing vs non-interest
\item Holding limits (caps to prevent bank runs)
\item Offline capability (resilience)
\end{itemize}

\column{0.48\textwidth}
\textbf{Wholesale CBDC (Limited Access):}

\textbf{Users:}
\begin{itemize}
\item Banks and financial institutions
\item Authorized payment service providers
\item No public access
\end{itemize}

\textbf{Use Cases:}
\begin{itemize}
\item Interbank settlements (RTGS enhancement)
\item Securities settlement (DVP)
\item Cross-border payments (FX vs payment)
\item Central bank operations (repo, monetary policy)
\end{itemize}

\textbf{Advantages:}
\begin{itemize}
\item Less disruptive to banking system
\item Lower technology scaling requirements
\item Easier privacy/AML balance
\item Building block for retail later
\end{itemize}

\vspace{2mm}
\textit{Most advanced projects: Wholesale (e.g., Project Jura, Aber) vs retail still exploratory (Digital Euro, e-CNY pilot)}
\end{columns}
\bottomnote{Comparative analysis helps identify the right tool for specific requirements.}
\end{frame}

\section{Digital Euro Project}

\begin{frame}[t]{Digital Euro: Design and Timeline}
\begin{columns}[T]
\scriptsize
\column{0.48\textwidth}
\textbf{ECB Digital Euro Timeline:}

\textbf{2020-2021: Investigation Phase}
\begin{itemize}
\item Report on digital euro (October 2020)
\item Public consultation (8000+ responses)
\item Decision to launch investigation (July 2021)
\end{itemize}

\textbf{2021-2023: Investigation}
\begin{itemize}
\item 24-month investigation phase
\item Design choices and technology exploration
\item Prototypes and proof-of-concepts
\item Rulebook drafting
\end{itemize}

\textbf{2023-2025: Preparation Phase (Current)}
\begin{itemize}
\item Scheme development (rules, standards)
\item Technology provider selection
\item Legislative framework (EU Digital Euro Regulation)
\end{itemize}

\textbf{2025-2027: Implementation (Planned)}
\begin{itemize}
\item Platform development and testing
\item Pilot programs
\item Potential launch decision
\end{itemize}

\column{0.48\textwidth}
\textbf{Key Design Choices (ECB Announcements):}

\textbf{1. Two-Tier Distribution:}
\begin{itemize}
\item ECB issues, supervised intermediaries distribute
\item Banks and PSPs provide wallets and services
\item No direct ECB customer relationship
\end{itemize}

\textbf{2. Privacy-Preserving:}
\begin{itemize}
\item Offline payments: Full anonymity (like cash)
\item Online payments: Privacy from merchants, AML compliance
\item ECB does not track individual transactions
\end{itemize}

\textbf{3. Holding Limits:}
\begin{itemize}
\item Caps on digital euro holdings (proposed EUR 3,000-4,000)
\item Prevent bank disintermediation
\item Higher amounts: tiered remuneration (negative interest)
\end{itemize}

\textbf{4. Offline Capability:}
\begin{itemize}
\item Works without internet (NFC, Bluetooth)
\item Resilience during outages
\item Privacy benefit (no central server involved)
\end{itemize}
\end{columns}
\bottomnote{Key concepts from this slide inform practical applications in finance.}
\end{frame}

\begin{frame}[t]{CBDC Design Choices}
\begin{center}
\includegraphics[width=0.60\textwidth]{figures/cbdc_design/cbdc_design.pdf}
\end{center}
\bottomnote{CBDC design involves fundamental choices about architecture and access models.}
\end{frame}

\begin{frame}[t]{Digital Euro Technology Architecture}
\begin{columns}[T]
\scriptsize
\column{0.48\textwidth}
\textbf{Technical Approach:}

\textbf{Centralized Ledger vs DLT:}
\begin{itemize}
\item ECB leaning toward centralized database
\item DLT explored but performance concerns
\item Hybrid possible: Central ledger + DLT for settlement
\end{itemize}

\vspace{2mm}
\textbf{Core Components:}
\begin{enumerate}
\item \textbf{Settlement Layer (ECB):}
\begin{itemize}
\item Central ledger of all digital euro balances
\item Final settlement authority
\item Reconciliation and oversight
\end{itemize}

\item \textbf{Distribution Layer (Intermediaries):}
\begin{itemize}
\item Customer onboarding (KYC/AML)
\item Wallet provision (mobile, hardware)
\item Payment initiation and routing
\item Customer service and dispute resolution
\end{itemize}

\item \textbf{Interface Layer:}
\begin{itemize}
\item APIs for merchants and developers
\item POS integration, e-commerce plugins
\item Peer-to-peer transfer protocols
\end{itemize}
\end{enumerate}

\column{0.48\textwidth}
\textbf{Offline Payment Technology:}

\textbf{Hardware Wallets:}
\begin{itemize}
\item Secure element (e.g., SIM card, smart card)
\item Store digital euro locally (encrypted)
\item NFC for proximity payments (contactless)
\item Bluetooth for device-to-device transfers
\end{itemize}

\textbf{Offline Protocol:}
\begin{itemize}
\item Cryptographic signatures verify authenticity
\item Double-spending prevention (local nonce tracking)
\item Periodic online synchronization
\item Limits on offline value (EUR 100-500)
\end{itemize}

\vspace{2mm}
\textbf{Privacy Architecture:}
\begin{itemize}
\item Blind signatures or zero-knowledge proofs
\item Pseudonymous identifiers (rotate frequently)
\item ECB sees aggregates, not individual transactions
\item Intermediaries handle AML checks
\end{itemize}

\vspace{2mm}
\textit{Key challenge: Balance privacy (citizen demand) with AML compliance (regulatory requirement)}
\end{columns}
\bottomnote{Key concepts from this slide inform practical applications in finance.}
\end{frame}

\begin{frame}[t]{Digital Euro Stakeholder Perspectives}
\begin{columns}[T]
\scriptsize
\column{0.48\textwidth}
\textbf{Central Bank (ECB) Goals:}
\begin{itemize}
\item Monetary sovereignty (counter BigTech stablecoins)
\item Ensure access to central bank money (cash declining)
\item Support European payment autonomy (reduce Visa/Mastercard dependence)
\item Platform for innovation (programmable euro)
\end{itemize}

\vspace{2mm}
\textbf{Commercial Banks Concerns:}
\begin{itemize}
\item \textbf{Disintermediation:} Customers move deposits to CBDC
\item \textbf{Funding Costs:} Higher deposit rates to compete
\item \textbf{Profitability:} Lower payment fee revenue
\item \textbf{Mitigation:} Two-tier model, holding limits, no interest on CBDC
\end{itemize}

\vspace{2mm}
\textbf{Banks' Requested Safeguards:}
\begin{itemize}
\item Low holding caps (EUR 3,000 or less)
\item No interest or negative rates (unattractive for savings)
\item Compensation for distribution services
\item Gradual rollout with monitoring
\end{itemize}

\column{0.48\textwidth}
\textbf{Citizens and Businesses:}

\textbf{Potential Benefits:}
\begin{itemize}
\item Free or low-cost payments (no card fees)
\item Instant settlement (vs T+1 bank transfers)
\item Privacy for small transactions (cash-like)
\item Pan-European acceptance (vs fragmented schemes)
\end{itemize}

\textbf{Concerns:}
\begin{itemize}
\item Privacy vs surveillance fears
\item Complexity (another payment method)
\item Holding limits (restrictive for some)
\item Transition costs for merchants (POS upgrades)
\end{itemize}

\vspace{2mm}
\textbf{Payment Providers (Visa, Mastercard):}
\begin{itemize}
\item Competitive threat (direct ECB payment rail)
\item Opportunity: Provide infrastructure services
\item Push for open standards and interoperability
\end{itemize}

\vspace{2mm}
\textit{Political dimension: 2024 EU elections, Digital Euro Regulation debate (data protection, privacy, holding limits)}
\end{columns}
\bottomnote{Key concepts from this slide inform practical applications in finance.}
\end{frame}

\section{Global CBDC Landscape}

\begin{frame}[t]{CBDC Projects Worldwide}
\begin{columns}[T]
\scriptsize
\column{0.48\textwidth}
\textbf{Live CBDCs (As of 2024):}

\textbf{Bahamas - Sand Dollar (2020):}
\begin{itemize}
\item First retail CBDC globally
\item Blockchain-based (private, permissioned)
\item Focus: Financial inclusion (islands)
\item Limited adoption (competition with USD)
\end{itemize}

\textbf{Eastern Caribbean - DCash (2021):}
\begin{itemize}
\item 7 island nations, ECCB-issued
\item Blockchain (Hyperledger Fabric)
\item Challenges: Merchant acceptance, tech issues
\end{itemize}

\textbf{Nigeria - eNaira (2021):}
\begin{itemize}
\item Africa's first CBDC
\item Over 1M wallets (low active usage)
\item Push for financial inclusion
\end{itemize}

\textbf{Jamaica - JAM-DEX (2022):}
\begin{itemize}
\item Retail CBDC for unbanked
\item Gradual adoption, government incentives
\end{itemize}

\column{0.48\textwidth}
\textbf{Advanced Pilots:}

\textbf{China - e-CNY (Digital Yuan):}
\begin{itemize}
\item Largest retail CBDC pilot (2020+)
\item 260M+ wallets, \$250B+ transactions (cumulative 2024)
\item Two-tier: PBOC issues, banks distribute
\item Controlled rollout (Beijing, Shenzhen, 26+ cities)
\item Goals: Domestic efficiency, internationalizing RMB
\item Concerns: Surveillance, government control
\end{itemize}

\textbf{Sweden - e-Krona:}
\begin{itemize}
\item Investigation since 2017 (cash usage < 10\%)
\item Pilot ended 2022, no launch decision yet
\item R3 Corda DLT platform tested
\end{itemize}

\textbf{India - Digital Rupee:}
\begin{itemize}
\item Retail pilot (2022): 5M users (2024)
\item Wholesale pilot (2022): Interbank settlements
\item Hyperledger Fabric blockchain
\end{itemize}
\end{columns}
\bottomnote{CBDCs represent the digitization of central bank money.}
\end{frame}

\begin{frame}[t]{Cross-Border CBDC Projects}
\begin{columns}[T]
\scriptsize
\column{0.48\textwidth}
\textbf{Multi-CBDC Platforms (mCBDC):}

\textbf{Project mBridge (BIS Innovation Hub):}
\begin{itemize}
\item Participants: China, Hong Kong, Thailand, UAE, Saudi Arabia
\item Wholesale CBDC for cross-border payments
\item DLT platform (custom blockchain)
\item Live transactions (2023): \$22B+ in pilots
\item Goals: 24/7 settlement, FX vs payment, reduce correspondent banking
\end{itemize}

\textbf{Project Jura (2021):}
\begin{itemize}
\item Switzerland (SNB), France (Banque de France), BIS
\item Wholesale CBDC for CHF-EUR FX settlement
\item R3 Corda platform
\item Proof-of-concept completed, no production plan
\end{itemize}

\textbf{Project Aber (2020):}
\begin{itemize}
\item Saudi Arabia, UAE
\item Dual-issued wholesale CBDC (SAR-AED)
\item Hyperledger Fabric
\item Successful POC, exploring expansion
\end{itemize}

\column{0.48\textwidth}
\textbf{Dunbar (Australia, Malaysia, Singapore, South Africa):}
\begin{itemize}
\item Multi-currency shared platform
\item Atomic settlement (DVP, PVP)
\item Quorum blockchain (2022 POC)
\end{itemize}

\vspace{2mm}
\textbf{Benefits of mCBDC:}
\begin{itemize}
\item \textbf{Speed:} Instant cross-border settlement (vs 2-5 days SWIFT)
\item \textbf{Cost:} Lower fees (no correspondent chains)
\item \textbf{Transparency:} Real-time tracking and finality
\item \textbf{FX Integration:} Atomic swap capabilities
\end{itemize}

\vspace{2mm}
\textbf{Challenges:}
\begin{itemize}
\item Governance (who controls shared platform?)
\item Legal harmonization (different jurisdictions)
\item Capital controls (easier to bypass with instant settlement)
\item Technology standards (interoperability)
\item Geopolitical tensions (China-led vs US-led initiatives)
\end{itemize}
\end{columns}
\bottomnote{CBDCs represent the digitization of central bank money.}
\end{frame}

\begin{frame}[t]{CBDC Status by Major Economies (2024)}
\begin{columns}[T]
\scriptsize
\column{0.48\textwidth}
\textbf{United States - Digital Dollar:}
\begin{itemize}
\item \textbf{Status:} Research phase, no pilot
\item \textbf{Fed Position:} Cautious, awaiting Congressional authorization
\item \textbf{Concerns:} Privacy, need unclear (stablecoins, private innovation)
\item \textbf{Political:} Divided (Republicans skeptical, Democrats open)
\item \textbf{Alternatives:} FedNow instant payments (2023 launch)
\end{itemize}

\textbf{United Kingdom - Britcoin:}
\begin{itemize}
\item Consultation phase (2021-2023)
\item Design work ongoing (BoE + HM Treasury)
\item Potential launch: 2028-2030 (if approved)
\item Two-tier model similar to Digital Euro
\end{itemize}

\textbf{Japan - Digital Yen:}
\begin{itemize}
\item Pilot phase (2023-2024)
\item BoJ testing with banks and retailers
\item No launch decision, monitoring global developments
\end{itemize}

\column{0.48\textwidth}
\textbf{Canada - Digital Canadian Dollar:}
\begin{itemize}
\item Research and consultation (ongoing)
\item Contingency planning (if cash declines or stablecoins dominate)
\item No immediate need identified
\end{itemize}

\textbf{Brazil - Digital Real (DREX):}
\begin{itemize}
\item Pilot launched 2024
\item Focus: Programmable payments, tokenized assets
\item DLT-based, targeting 2025 launch
\end{itemize}

\vspace{2mm}
\textbf{Global Statistics (BIS, IMF, Atlantic Council 2024):}
\begin{itemize}
\item \textbf{130+ countries:} Exploring CBDCs (98\% of global GDP)
\item \textbf{11 countries:} Fully launched CBDCs
\item \textbf{20+ countries:} Pilot phase (including China, India)
\item \textbf{G7 Principles:} Endorsed CBDC principles (2021) - privacy, no programmability for surveillance
\end{itemize}
\end{columns}
\bottomnote{CBDCs represent the digitization of central bank money.}
\end{frame}

\section{Programmable Money and Smart Contracts}

\begin{frame}[t]{Programmable CBDC Features}
\begin{columns}[T]
\scriptsize
\column{0.48\textwidth}
\textbf{Programmability Concept:}

\textit{Ability to attach conditions and logic to money itself, enforced automatically.}

\vspace{2mm}
\textbf{Use Cases:}

\textbf{1. Conditional Payments:}
\begin{itemize}
\item Pay only if delivery confirmed (IoT integration)
\item Escrow automatically released on milestone
\item Salary paid only to authorized accounts
\end{itemize}

\textbf{2. Time-Locked Funds:}
\begin{itemize}
\item Stimulus funds spendable only at local businesses
\item Budget allocations released monthly (automatic)
\item Pension payments on specific dates
\end{itemize}

\textbf{3. Targeted Fiscal Policy:}
\begin{itemize}
\item COVID relief usable only for essentials (food, rent)
\item Expiring money to encourage spending (negative interest)
\item Carbon credits embedded in payments
\end{itemize}

\column{0.48\textwidth}
\textbf{4. Atomic Transactions (DVP/PVP):}
\begin{itemize}
\item Securities delivery only if payment received (simultaneous)
\item Cross-currency swaps (no settlement risk)
\item Supply chain: Pay on confirmed delivery (IoT sensors)
\end{itemize}

\vspace{2mm}
\textbf{Implementation Approaches:}

\textbf{Smart Contract Layer:}
\begin{itemize}
\item CBDC on blockchain with smart contract capability
\item E.g., Ethereum-like execution environment
\item Challenges: Complexity, security audits
\end{itemize}

\textbf{API-Based Programmability:}
\begin{itemize}
\item Centralized ledger with programmable payment rules
\item Simpler, more controllable
\item Example: Brazil DREX design
\end{itemize}

\vspace{2mm}
\textbf{Concerns:}
\begin{itemize}
\item Privacy erosion (surveillance capitalism)
\item Government overreach (social credit systems)
\item Complexity and bugs (financial system risk)
\item Financial exclusion (conditions disadvantage some)
\end{itemize}
\end{columns}
\bottomnote{CBDCs represent the digitization of central bank money.}
\end{frame}

\begin{frame}[t]{Tokenized Deposits and Synthetic CBDCs}
\begin{columns}[T]
\scriptsize
\column{0.48\textwidth}
\textbf{Tokenized Deposits (Private Sector):}

\textbf{Concept:}
\begin{itemize}
\item Commercial bank deposits on blockchain
\item Programmable, interoperable with smart contracts
\item Bank liability (not central bank money)
\end{itemize}

\vspace{2mm}
\textbf{Examples:}
\begin{itemize}
\item \textbf{JPMorgan JPM Coin:} Tokenized USD deposits for institutional clients (\$1B+ daily volume, 2024)
\item \textbf{SocieteGenerale EURCV:} Euro-denominated stablecoin
\item \textbf{HSBC, Standard Chartered:} Pilots for trade finance
\end{itemize}

\vspace{2mm}
\textbf{Advantages over CBDC:}
\begin{itemize}
\item Faster to market (no central bank coordination)
\item Programmability without government control
\item Integrate with DeFi and tokenized assets
\end{itemize}

\textbf{Disadvantages:}
\begin{itemize}
\item Not risk-free (bank default risk)
\item Fragmented (each bank's token different)
\item Regulatory uncertainty
\end{itemize}

\column{0.48\textwidth}
\textbf{Synthetic CBDC (sCBDC):}

\textbf{Concept (BIS Model):}
\begin{itemize}
\item Private stablecoins fully backed by central bank reserves
\item Central bank provides infrastructure and oversight
\item No public CBDC issuance needed
\end{itemize}

\vspace{2mm}
\textbf{Architecture:}
\begin{enumerate}
\item Private entities issue stablecoins
\item 100\% reserve backing at central bank
\item Real-time auditability by central bank
\item Interoperability via central bank platform
\end{enumerate}

\vspace{2mm}
\textbf{Benefits:}
\begin{itemize}
\item Leverages private sector innovation
\item Central bank avoids operational burden
\item Competition drives better user experience
\item Faster deployment than full CBDC
\end{itemize}

\vspace{2mm}
\textbf{Example: Project Aurum (Hong Kong):}
\begin{itemize}
\item Two-tier sCBDC architecture
\item Banks issue e-HKD backed by reserves at HKMA
\item POC completed 2022
\end{itemize}
\end{columns}
\bottomnote{Tokens represent digital assets and enable new business models on blockchain.}
\end{frame}

\section{Future of Traditional Digital Finance}

\begin{frame}[t]{Convergence: TradFi, DeFi, and CBDCs}
\begin{columns}[T]
\scriptsize
\column{0.48\textwidth}
\textbf{Interoperability Visions:}

\textbf{Unified Ledger Concept (BIS):}
\begin{itemize}
\item Single platform for CBDC, tokenized deposits, tokenized assets
\item Atomic settlement across money and assets
\item Programmable financial contracts
\item Central bank oversight with private innovation
\end{itemize}

\vspace{2mm}
\textbf{Project Guardian (Singapore MAS):}
\begin{itemize}
\item Industry collaboration (DBS, JPMorgan, SBI)
\item DeFi protocols on institutional blockchain
\item Asset tokenization (bonds, funds) + CBDC settlement
\item Live pilots: FX swaps, fixed income trading (2023-2024)
\end{itemize}

\textbf{Project Agorá (BIS, Central Banks, Banks):}
\begin{itemize}
\item Cross-border wholesale CBDC platform
\item Focus: FX trading and settlement
\item 7 central banks (Fed, ECB, BoJ, BoE, BoK, SNB, Banque de France)
\item Target: 2025-2026 proof-of-concept
\end{itemize}

\column{0.48\textwidth}
\textbf{Instant Payments vs CBDCs:}

\textbf{FedNow (US, 2023):}
\begin{itemize}
\item 24/7/365 instant payments
\item Real-time settlement
\item Competes with need for retail CBDC
\item Over 300 banks participating (2024)
\end{itemize}

\textbf{EU Instant Payments Regulation (2025):}
\begin{itemize}
\item Mandatory instant SEPA credit transfers
\item 10-second settlement
\item May reduce urgency for Digital Euro (retail)
\end{itemize}

\vspace{2mm}
\textbf{Scenario: Coexistence}
\begin{itemize}
\item \textbf{Cash:} Declines but persists (privacy, resilience)
\item \textbf{Bank Deposits:} Remain dominant for savings
\item \textbf{Instant Payments:} Everyday transactions (P2P, bills)
\item \textbf{CBDC:} Niche use (cross-border, programmable, government payments)
\item \textbf{Stablecoins:} DeFi and crypto ecosystems
\end{itemize}
\end{columns}
\bottomnote{DeFi recreates traditional financial services in a permissionless, programmable way.}
\end{frame}

\begin{frame}[t]{Technology Trends: Next 5-10 Years}
\begin{columns}[T]
\scriptsize
\column{0.48\textwidth}
\textbf{1. Tokenization of Real-World Assets:}
\begin{itemize}
\item Real estate, private equity, commodities on blockchain
\item Fractional ownership, 24/7 trading
\item Settlement in CBDC or tokenized deposits
\item Market size projection: \$10T+ by 2030 (BCG estimate)
\end{itemize}

\vspace{2mm}
\textbf{2. AI in Finance:}
\begin{itemize}
\item \textbf{Trading:} Reinforcement learning, alternative data
\item \textbf{Risk:} Real-time credit scoring, fraud detection
\item \textbf{Compliance:} Automated regulatory reporting, AML
\item \textbf{Advisory:} Personalized wealth management (robo 2.0)
\item \textbf{Operations:} Document processing (NLP), reconciliation
\end{itemize}

\textbf{GenAI Applications (2024+):}
\begin{itemize}
\item Code generation for trading strategies
\item Natural language interfaces to financial systems
\item Synthetic data for model training (privacy-preserving)
\item Regulatory document interpretation
\end{itemize}

\column{0.48\textwidth}
\textbf{3. Quantum Computing Threat and Opportunity:}

\textbf{Threats:}
\begin{itemize}
\item Break current encryption (RSA, ECC) in 10-15 years
\item Risk to digital signatures, payment security
\item CBDC and blockchain vulnerable
\end{itemize}

\textbf{Response:}
\begin{itemize}
\item Post-quantum cryptography (NIST standards 2024)
\item Migration timeline: 5-10 years
\item CBDC designs incorporating quantum-resistant algorithms
\end{itemize}

\textbf{Opportunities:}
\begin{itemize}
\item Portfolio optimization (quadratic speedup)
\item Derivatives pricing (Monte Carlo acceleration)
\item Machine learning (quantum ML models)
\end{itemize}

\vspace{2mm}
\textbf{4. Decentralized Identity (DID):}
\begin{itemize}
\item Self-sovereign identity (users control data)
\item Verifiable credentials (KYC once, use anywhere)
\item Integration with CBDC wallets
\item Standards: W3C DID, EU Digital Identity Wallet (eIDAS 2.0)
\end{itemize}
\end{columns}
\bottomnote{Future trends inform strategic planning and investment decisions.}
\end{frame}

\begin{frame}[t]{Regulatory Evolution and Global Coordination}
\begin{columns}[T]
\scriptsize
\column{0.48\textwidth}
\textbf{Global Standards and Coordination:}

\textbf{BIS Innovation Hub:}
\begin{itemize}
\item Coordinate CBDC research (mBridge, Dunbar, etc.)
\item Technology experimentation and best practices
\item Hubs: Switzerland, Hong Kong, Singapore, London, Toronto
\end{itemize}

\textbf{IMF and World Bank:}
\begin{itemize}
\item CBDC guidance for developing economies
\item Financial inclusion focus
\item Technical assistance programs
\end{itemize}

\textbf{FSB (Financial Stability Board):}
\begin{itemize}
\item Stablecoin regulation (2023 framework)
\item Cross-border payment roadmap
\item Crypto-asset regulatory standards
\end{itemize}

\textbf{G20 Priorities:}
\begin{itemize}
\item Cross-border payment efficiency (target: 5x faster, 50\% cheaper by 2027)
\item CBDC interoperability principles
\item Stablecoin oversight harmonization
\end{itemize}

\column{0.48\textwidth}
\textbf{Emerging Regulatory Themes:}

\textbf{1. Embedded Finance Regulation:}
\begin{itemize}
\item Non-banks offering financial services (BigTech)
\item Same activity, same risk, same regulation
\item Data privacy and competition concerns
\end{itemize}

\textbf{2. AI Governance in Finance:}
\begin{itemize}
\item Model risk management (SR 11-7 updates)
\item Explainability requirements
\item Bias and fairness testing
\item EU AI Act implications (2024)
\end{itemize}

\textbf{3. Climate Finance and ESG:}
\begin{itemize}
\item Mandatory climate risk disclosure (SEC, ISSB)
\item Green CBDC concepts (incentivize low-carbon spending)
\item Carbon credit tokenization
\end{itemize}

\textbf{4. Open Banking and Data Portability:}
\begin{itemize}
\item PSD3 (EU, 2024): Expand open banking scope
\item US: CFPB Open Banking Rule (2024)
\item Data sharing standards (FAPI, CIBA)
\end{itemize}
\end{columns}
\bottomnote{Understanding history helps predict future developments in the technology.}
\end{frame}

\begin{frame}[t]{Strategic Implications for Financial Institutions}
\begin{columns}[T]
\scriptsize
\column{0.48\textwidth}
\textbf{Banks: Adapt or Disintermediate}

\textbf{Threats:}
\begin{itemize}
\item CBDC reduces need for commercial bank money
\item BigTech and fintechs capture customer relationships
\item Margin compression (instant payments, CBDC)
\item Regulatory burden increases (AI, climate, crypto)
\end{itemize}

\textbf{Opportunities:}
\begin{itemize}
\item Distribution partners for CBDC (two-tier model)
\item Tokenized deposit issuers (programmable banking)
\item Custodians for digital assets
\item Data analytics and AI services
\item Embedded finance (BaaS - Banking as a Service)
\end{itemize}

\vspace{2mm}
\textbf{Strategic Responses:}
\begin{itemize}
\item Invest in technology (cloud, APIs, blockchain)
\item Partner with fintechs (acquire or collaborate)
\item Focus on high-value services (advisory, complex products)
\item Regulatory technology (RegTech, SupTech)
\end{itemize}

\column{0.48\textwidth}
\textbf{Asset Managers and Investors:}

\textbf{New Asset Classes:}
\begin{itemize}
\item Tokenized real assets (real estate, art, private equity)
\item CBDC and stablecoin strategies (yield farming)
\item Digital securities (tokenized bonds, equities)
\end{itemize}

\textbf{Operational Changes:}
\begin{itemize}
\item 24/7 markets (tokenized assets never close)
\item Instant settlement (T+0 becomes standard)
\item Fractional ownership (lower minimums, broader access)
\item Programmable compliance (automatic reporting)
\end{itemize}

\vspace{2mm}
\textbf{Fintechs and Innovators:}

\textbf{Opportunities:}
\begin{itemize}
\item Build on CBDC infrastructure (wallets, apps)
\item DeFi bridges to traditional finance
\item Niche services (cross-border, remittances)
\item AI-driven personalized finance
\end{itemize}

\textbf{Risks:}
\begin{itemize}
\item Regulatory capture (licensing barriers)
\item Competition from incumbent banks
\item Funding challenges (higher interest rates)
\end{itemize}
\end{columns}
\bottomnote{Key concepts from this slide inform practical applications in finance.}
\end{frame}

\begin{frame}[t]{Scenarios for 2030-2035}
\begin{columns}[T]
\scriptsize
\column{0.48\textwidth}
\textbf{Scenario 1: CBDC-Dominant World}

\textbf{Characteristics:}
\begin{itemize}
\item Major economies launch retail CBDCs (Digital Euro, e-CNY, Digital Pound)
\item Cash usage < 5\% of transactions
\item Cross-border mCBDC platforms replace SWIFT for most B2B
\item Commercial banks focus on lending, wealth management
\end{itemize}

\textbf{Implications:}
\begin{itemize}
\item Central banks gain powerful monetary policy tools (negative rates, direct stimulus)
\item Privacy concerns intensify (calls for regulation)
\item Financial inclusion improves (unbanked access CBDC)
\item Geopolitical fragmentation (CBDC blocs: US-led, China-led)
\end{itemize}

\column{0.48\textwidth}
\textbf{Scenario 2: Private Stablecoin Dominance}

\textbf{Characteristics:}
\begin{itemize}
\item Regulated stablecoins (USDC, EURC) become standard
\item CBDCs limited to wholesale use
\item BigTech wallets (Apple Pay, Google Pay) integrate stablecoins
\item DeFi grows to \$5T+ TVL (vs \$100B 2024)
\end{itemize}

\textbf{Implications:}
\begin{itemize}
\item Central banks lose monetary policy leverage
\item Regulatory arbitrage (offshore stablecoin issuers)
\item Innovation thrives (private sector experimentation)
\item Financial stability risks (stablecoin runs)
\end{itemize}

\vspace{2mm}
\textbf{Scenario 3: Hybrid Coexistence (Most Likely)}

\textbf{Characteristics:}
\begin{itemize}
\item Mix of CBDCs (some countries), stablecoins, instant bank payments
\item Interoperability via standards (ISO 20022, mCBDC protocols)
\item Tokenization mainstream for assets, niche for money
\item Traditional finance absorbs blockchain selectively
\end{itemize}
\end{columns}
\bottomnote{Key concepts from this slide inform practical applications in finance.}
\end{frame}

\begin{frame}[t]{Summary and Key Takeaways}
\begin{columns}[T]
\scriptsize
\column{0.48\textwidth}
\textbf{CBDC Fundamentals:}
\begin{itemize}
\item Digital central bank money, distinct from deposits and crypto
\item Retail (public access) vs wholesale (banks only)
\item Goals: Payment efficiency, financial inclusion, monetary sovereignty
\item Risks: Bank disintermediation, privacy, cybersecurity
\end{itemize}

\vspace{2mm}
\textbf{Digital Euro:}
\begin{itemize}
\item Preparation phase (2023-2025), potential launch 2025-2027
\item Two-tier distribution (ECB + intermediaries)
\item Privacy-preserving (offline anonymity, online pseudonymity)
\item Holding limits (EUR 3-4k) to prevent bank runs
\item Centralized ledger (DLT explored but not chosen)
\end{itemize}

\vspace{2mm}
\textbf{Global Landscape:}
\begin{itemize}
\item 130+ countries exploring (98\% global GDP)
\item 11 live (Bahamas, Nigeria, Jamaica, etc.)
\item China e-CNY: 260M+ wallets, \$250B+ cumulative
\item mCBDC: mBridge (\$22B+ pilots), Jura, Dunbar
\item US cautious (FedNow alternative), UK/Japan in design
\end{itemize}

\column{0.48\textwidth}
\textbf{Programmable Money:}
\begin{itemize}
\item Conditional payments, time locks, atomic DVP/PVP
\item Use cases: Targeted stimulus, supply chain, fiscal policy
\item Concerns: Privacy erosion, government overreach
\item Tokenized deposits (JPM Coin \$1B+ daily) vs synthetic CBDC
\end{itemize}

\vspace{2mm}
\textbf{Future Trends (2025-2035):}
\begin{itemize}
\item Tokenization: \$10T+ real-world assets by 2030
\item AI in finance: Trading, risk, compliance, advisory
\item Quantum threat: Post-quantum crypto migration (5-10 years)
\item Decentralized identity: Self-sovereign KYC
\item Regulatory evolution: AI governance, climate finance, open banking
\end{itemize}

\vspace{2mm}
\textbf{Strategic Implications:}
\begin{itemize}
\item Banks: CBDC distributors, tokenized deposit issuers, BaaS
\item Asset managers: 24/7 markets, T+0 settlement, fractional ownership
\item Fintechs: Build on CBDC infrastructure, DeFi bridges
\item Likely outcome: Hybrid coexistence (CBDCs + stablecoins + instant payments)
\end{itemize}
\end{columns}
\end{frame}

\begin{frame}[t]{Course Conclusion: Traditional Digital Finance}
\begin{columns}[T]
\scriptsize
\column{0.48\textwidth}
\textbf{Module 4 Journey:}

\textbf{Lessons 37-39: Foundations}
\begin{itemize}
\item Financial markets infrastructure (exchanges, clearinghouses)
\item Core banking systems (CBS, digital banking)
\item Payment rails (ACH, wires, real-time, cards)
\end{itemize}

\textbf{Lessons 40-42: Trading and Risk}
\begin{itemize}
\item Electronic trading and HFT (market microstructure)
\item Risk management systems (VaR, stress testing, model risk)
\item RegTech and compliance (Basel III, IFRS 9, EMIR)
\end{itemize}

\textbf{Lessons 43-45: Markets and Derivatives}
\begin{itemize}
\item Capital markets technology (OMS, EMS, PMS, T+1)
\item Derivatives technology (pricing, CCPs, EMIR, SOFR)
\item Wealth management systems (robo, hybrid, direct indexing)
\end{itemize}

\column{0.48\textwidth}
\textbf{Lessons 46-48: Data and Future}
\begin{itemize}
\item Financial data vendors (Bloomberg, LSEG, FactSet, alternative data)
\item CBDCs and future (Digital Euro, e-CNY, programmable money, scenarios)
\end{itemize}

\vspace{3mm}
\textbf{Key Themes Across Module:}
\begin{enumerate}
\item \textbf{Digitalization:} Paper $\rightarrow$ Electronic $\rightarrow$ Real-time $\rightarrow$ Programmable
\item \textbf{Automation:} Manual $\rightarrow$ STP $\rightarrow$ AI/ML $\rightarrow$ Autonomous
\item \textbf{Integration:} Siloed systems $\rightarrow$ APIs $\rightarrow$ Platforms $\rightarrow$ Ecosystems
\item \textbf{Regulation:} Reactive $\rightarrow$ Proactive $\rightarrow$ RegTech $\rightarrow$ Embedded compliance
\item \textbf{Convergence:} TradFi $\leftrightarrow$ Fintech $\leftrightarrow$ DeFi $\leftrightarrow$ CBDCs
\end{enumerate}

\vspace{3mm}
\textbf{Looking Ahead:}

The future of digital finance is not a replacement of traditional systems but an evolution—integrating the stability and trust of central banks with the innovation of technology, creating a hybrid ecosystem where efficiency, inclusion, and sovereignty coexist.
\end{columns}
\bottomnote{Key concepts from this slide inform practical applications in finance.}
\end{frame}

\end{document}
