\documentclass[8pt,aspectratio=169]{beamer}
\usetheme{Madrid}
\usepackage{graphicx,booktabs,adjustbox,multicol,amsmath,amssymb}
\definecolor{mlblue}{RGB}{0,102,204}
\definecolor{mlpurple}{RGB}{51,51,178}
\definecolor{mllavender}{RGB}{173,173,224}
\definecolor{mllavender2}{RGB}{193,193,232}
\definecolor{mllavender3}{RGB}{204,204,235}
\definecolor{mllavender4}{RGB}{214,214,239}
\definecolor{mlorange}{RGB}{255,127,14}
\definecolor{mlgreen}{RGB}{44,160,44}
\definecolor{mlred}{RGB}{214,39,40}
\setbeamercolor{palette primary}{bg=mllavender3,fg=mlpurple}
\setbeamercolor{palette secondary}{bg=mllavender2,fg=mlpurple}
\setbeamercolor{palette tertiary}{bg=mllavender,fg=white}
\setbeamercolor{structure}{fg=mlpurple}
\setbeamercolor{frametitle}{fg=mlpurple,bg=mllavender3}
\setbeamertemplate{navigation symbols}{}
\setbeamertemplate{itemize items}[circle]
\setbeamersize{text margin left=5mm,text margin right=5mm}

% Bottom note command for key takeaways
\newcommand{\bottomnote}[1]{%
\vfill
\vspace{-2mm}
\textcolor{mllavender2}{\rule{\textwidth}{0.4pt}}
\vspace{1mm}
\footnotesize
\textbf{#1}
}
\title{Lesson 40: Electronic Trading and Orders}
\subtitle{Module 4: Traditional Digital Finance}
\author{Digital Finance Course}
\date{2025}

\begin{document}

\begin{frame}
\titlepage
\end{frame}

\begin{frame}[t]{Learning Objectives}
\begin{itemize}
\item Understand electronic order types and routing mechanisms
\item Analyze order book dynamics and matching algorithms
\item Examine price-time priority and market structure
\item Evaluate dark pools and alternative trading venues
\item Assess regulatory frameworks for electronic trading
\end{itemize}
\end{frame}

\section{Electronic Trading Infrastructure}

\begin{frame}[t]{Evolution of Trading Technology}
\begin{columns}[T]
\column{0.48\textwidth}
\textbf{Historical Progression:}
\begin{itemize}
\item \textbf{Open Outcry (pre-1990s):} Physical trading floors
\item \textbf{Screen-Based (1990s):} SETS, XETRA introduction
\item \textbf{Direct Market Access (2000s):} Broker bypass
\item \textbf{Algorithmic Trading (2010s):} Automated execution
\item \textbf{Low Latency (2020s):} Microsecond competition
\end{itemize}

\column{0.48\textwidth}
\textbf{Technology Drivers:}
\begin{itemize}
\item Electronic communication networks (ECNs)
\item FIX protocol standardization (1992)
\item Co-location and proximity hosting
\item FPGA and custom hardware acceleration
\item Market fragmentation across venues
\end{itemize}
\end{columns}

\vspace{3mm}
\small\textit{Key Milestone: NASDAQ becomes fully electronic in 1994, NYSE follows with Hybrid Market in 2006}
\bottomnote{Understanding history helps predict future developments in the technology.}
\end{frame}

\begin{frame}[t]{Electronic Trading System Architecture}
\begin{columns}[T]
\column{0.48\textwidth}
\textbf{Core Components:}
\begin{itemize}
\item \textbf{Order Management System (OMS):} Portfolio-level order creation
\item \textbf{Execution Management System (EMS):} Routing and execution
\item \textbf{Smart Order Router (SOR):} Venue selection logic
\item \textbf{Market Data Handler:} Real-time price feeds
\item \textbf{Risk Manager:} Pre-trade compliance
\end{itemize}

\column{0.48\textwidth}
\textbf{Latency Benchmarks:}
\begin{itemize}
\item Order entry to exchange: 100-500 microseconds
\item Matching engine processing: 10-50 microseconds
\item Market data dissemination: 50-200 microseconds
\item Round-trip execution: 200-1000 microseconds
\item Co-located vs remote: 10x latency difference
\end{itemize}
\end{columns}

\vspace{3mm}
\footnotesize\textit{Modern systems process millions of orders per second with 99.999\% availability}
\bottomnote{Electronic trading has transformed market structure and efficiency.}
\end{frame}

\begin{frame}[t]{Trading Latency Benchmarks}
\begin{center}
\includegraphics[width=0.60\textwidth]{figures/trading_latency/trading_latency.pdf}
\end{center}
\bottomnote{Latency optimization is critical for competitive advantage in electronic markets.}
\end{frame}

\section{Order Types and Mechanics}

\begin{frame}[t]{Basic Order Types}
\begin{columns}[T]
\column{0.48\textwidth}
\textbf{Market Orders:}
\begin{itemize}
\item Execute immediately at best available price
\item Guarantee execution, not price
\item Consume liquidity (aggressive)
\item Pay taker fee (typically 0.003-0.005\%)
\item Risk: slippage in volatile markets
\end{itemize}

\vspace{2mm}
\textbf{Limit Orders:}
\begin{itemize}
\item Execute only at specified price or better
\item Provide liquidity (passive)
\item Receive maker rebate (0.001-0.002\%)
\item Risk: non-execution
\end{itemize}

\column{0.48\textwidth}
\textbf{Stop Orders:}
\begin{itemize}
\item Become market/limit order when trigger hit
\item Stop-loss: sell below current price
\item Stop-buy: buy above current price
\item Used for risk management and breakout strategies
\end{itemize}

\vspace{2mm}
\textbf{Execution Example:}
\begin{itemize}
\item Stock trading at \$100.00
\item Limit buy at \$99.50: waits for price drop
\item Stop-loss at \$98.00: sells if price falls
\item Market sell: executes immediately at best bid
\end{itemize}
\end{columns}
\bottomnote{Key concepts from this slide inform practical applications in finance.}
\end{frame}

\begin{frame}[t]{Advanced Order Types}
\begin{columns}[T]
\column{0.48\textwidth}
\textbf{Time-in-Force Instructions:}
\begin{itemize}
\item \textbf{GTC (Good Till Cancelled):} Active until filled or cancelled
\item \textbf{DAY:} Expires at market close
\item \textbf{IOC (Immediate or Cancel):} Execute available, cancel rest
\item \textbf{FOK (Fill or Kill):} Complete fill or cancel entire order
\item \textbf{GTD (Good Till Date):} Active until specified date
\end{itemize}

\column{0.48\textwidth}
\textbf{Conditional Orders:}
\begin{itemize}
\item \textbf{Iceberg/Hidden:} Display portion, hide balance
\item \textbf{Pegged:} Price tracks market (mid-point peg)
\item \textbf{Discretionary:} Price improvement range
\item \textbf{All-or-None:} Execute full size or nothing
\item \textbf{Minimum Quantity:} Require minimum fill size
\end{itemize}
\end{columns}

\vspace{3mm}
\small\textit{Iceberg orders typically display 10-20\% of total size to minimize market impact}
\bottomnote{Key concepts from this slide inform practical applications in finance.}
\end{frame}

\begin{frame}[t]{Algorithmic Order Types}
\begin{columns}[T]
\column{0.48\textwidth}
\textbf{VWAP (Volume-Weighted Average Price):}
\begin{itemize}
\item Target: match daily volume profile
\item Slices order across trading day
\item Benchmark: VWAP = $\sum (P_i \times V_i) / \sum V_i$
\item Typical duration: full trading session
\item Use case: large passive orders
\end{itemize}

\vspace{2mm}
\textbf{TWAP (Time-Weighted Average Price):}
\begin{itemize}
\item Equal-sized slices over time
\item Ignores volume patterns
\item Simpler, more predictable
\item Risk: adverse selection if volume spikes
\end{itemize}

\column{0.48\textwidth}
\textbf{Implementation Shortfall:}
\begin{itemize}
\item Minimize difference vs decision price
\item Balances market impact and timing risk
\item Aggressive when price favorable
\item Passive when price adverse
\end{itemize}

\vspace{2mm}
\textbf{Participation Rate (POV):}
\begin{itemize}
\item Target: fixed \% of market volume
\item Typical range: 5-20\% participation
\item Adapts to volume fluctuations
\item Risk: extended execution in low volume
\end{itemize}
\end{columns}
\bottomnote{Key concepts from this slide inform practical applications in finance.}
\end{frame}

\section{Order Book Dynamics}

\begin{frame}[t]{Order Book Visualization}
\begin{center}
\includegraphics[width=0.60\textwidth]{figures/order_book/order_book.pdf}
\end{center}
\bottomnote{Order book depth reveals supply and demand dynamics at each price level.}
\end{frame}

\begin{frame}[t]{Electronic Order Book Structure}
\begin{columns}[T]
\column{0.48\textwidth}
\textbf{Order Book Components:}
\begin{itemize}
\item \textbf{Bid Side:} Buy orders ranked by price (descending)
\item \textbf{Ask Side:} Sell orders ranked by price (ascending)
\item \textbf{Spread:} Difference between best bid and ask
\item \textbf{Depth:} Quantity at each price level
\item \textbf{Mid-Price:} (Best Bid + Best Ask) / 2
\end{itemize}

\vspace{2mm}
\textbf{Example Order Book:}
\begin{center}
\small
\begin{tabular}{rr|lr}
\multicolumn{2}{c|}{\textbf{Bids}} & \multicolumn{2}{c}{\textbf{Asks}} \\
Size & Price & Price & Size \\
\midrule
500 & 99.98 & 100.02 & 300 \\
800 & 99.97 & 100.03 & 600 \\
1200 & 99.96 & 100.04 & 400 \\
\end{tabular}
\end{center}
Spread = \$0.04 (4 cents or 4 bps)

\column{0.48\textwidth}
\textbf{Order Book Metrics:}
\begin{itemize}
\item \textbf{Quoted Spread:} Ask - Bid
\item \textbf{Effective Spread:} 2 $\times$ |Trade Price - Mid|
\item \textbf{Realized Spread:} Effective spread minus adverse selection
\item \textbf{Order Book Imbalance:} (Bid Vol - Ask Vol) / Total
\item \textbf{Depth Imbalance:} Predictive signal for price movement
\end{itemize}

\vspace{2mm}
\textbf{Liquidity Indicators:}
\begin{itemize}
\item Volume at top 5 levels
\item Average quoted spread (daily)
\item Order arrival/cancellation rates
\end{itemize}
\end{columns}
\bottomnote{Key concepts from this slide inform practical applications in finance.}
\end{frame}

\begin{frame}[t]{Price-Time Priority Matching}
\begin{columns}[T]
\column{0.48\textwidth}
\textbf{Matching Algorithm:}
\begin{enumerate}
\item \textbf{Price Priority:} Best prices matched first
\item \textbf{Time Priority:} Within price level, oldest order first
\item \textbf{Display Priority:} Visible orders before hidden (some venues)
\item \textbf{Size Priority:} Rare, used in some Asian markets
\end{enumerate}

\vspace{2mm}
\textbf{Example Execution:}
\begin{itemize}
\item Time 09:00:00 - Limit buy 500 at \$100.00
\item Time 09:00:05 - Limit buy 300 at \$100.00
\item Time 09:00:10 - Market sell 400
\item \textit{Result:} First 400 from 09:00:00 order filled, 100 remains
\end{itemize}

\column{0.48\textwidth}
\textbf{Alternative Mechanisms:}
\begin{itemize}
\item \textbf{Pro-Rata:} Size-proportional allocation (Eurex)
\item \textbf{FIFO with LMM:} Lead market maker priority (some options)
\item \textbf{Random Selection:} Anti-latency mechanism (IEX D-Peg)
\item \textbf{Batch Auction:} Periodic clearing (opening/closing)
\end{itemize}

\vspace{2mm}
\textbf{Queue Position Value:}
\begin{itemize}
\item Early position in queue = higher fill probability
\item Incentivizes speed competition
\item Queue jumping via price improvement
\item Drives investment in low-latency infrastructure
\end{itemize}
\end{columns}
\bottomnote{Key concepts from this slide inform practical applications in finance.}
\end{frame}

\begin{frame}[t]{Order Book Dynamics and Toxicity}
\begin{columns}[T]
\column{0.48\textwidth}
\textbf{Order Flow Toxicity:}
\begin{itemize}
\item \textbf{Informed Trading:} Orders contain price-relevant information
\item \textbf{Adverse Selection:} Market makers lose to informed traders
\item \textbf{VPIN (Volume-Synchronized PIN):} Toxicity measure
\item \textbf{Order Imbalance:} Predictive of short-term price moves
\end{itemize}

\vspace{2mm}
\textbf{VPIN Calculation:}
$$\text{VPIN} = \frac{|\sum \text{Buy Volume} - \sum \text{Sell Volume}|}{\text{Total Volume}}$$
High VPIN (>0.8) signals elevated informed trading risk

\column{0.48\textwidth}
\textbf{Market Maker Response:}
\begin{itemize}
\item Widen spreads when toxicity increases
\item Reduce quoted depth
\item Increase order cancellation rates
\item Temporarily withdraw liquidity
\end{itemize}

\vspace{2mm}
\textbf{Resilience Metrics:}
\begin{itemize}
\item Order-to-trade ratio: 20:1 to 100:1 typical
\item Cancellation rate: 95-98\% of orders cancelled
\item Book depth recovery time: seconds to minutes
\item Flash crash (2010): VPIN spiked to 0.98
\end{itemize}
\end{columns}
\bottomnote{Key concepts from this slide inform practical applications in finance.}
\end{frame}

\section{Smart Order Routing}

\begin{frame}[t]{Smart Order Router (SOR) Architecture}
\begin{columns}[T]
\column{0.48\textwidth}
\textbf{SOR Decision Factors:}
\begin{itemize}
\item \textbf{Displayed Liquidity:} Visible order book depth
\item \textbf{Fill Probability:} Historical fill rates by venue
\item \textbf{Latency:} Network and execution speed
\item \textbf{Fees/Rebates:} Maker/taker economics
\item \textbf{Price Improvement:} Mid-point or better execution
\item \textbf{Market Impact:} Venue size and anonymity
\end{itemize}

\column{0.48\textwidth}
\textbf{Routing Strategies:}
\begin{itemize}
\item \textbf{Top-of-Book:} Route to best displayed price
\item \textbf{Sweep:} Send to multiple venues simultaneously
\item \textbf{Sequential:} Try venues in priority order
\item \textbf{Hidden Liquidity Seeking:} Ping dark pools first
\item \textbf{Smart Posting:} Become passive liquidity provider
\end{itemize}
\end{columns}

\vspace{3mm}
\footnotesize\textit{Typical SOR evaluates 20-50 venues in under 100 microseconds, routing to top 3-5 destinations}
\bottomnote{Key concepts from this slide inform practical applications in finance.}
\end{frame}

\begin{frame}[t]{Venue Fragmentation and Best Execution}
\begin{columns}[T]
\column{0.48\textwidth}
\textbf{US Equity Market Structure (2024):}
\begin{itemize}
\item 16 public exchanges (Nasdaq, NYSE, CBOE BZX, etc.)
\item 30+ dark pools (40-50\% of volume)
\item Wholesalers (Citadel, Virtu): 40-45\% retail flow
\item Market share: Nasdaq 15\%, NYSE 12\%, off-exchange 50\%
\end{itemize}

\vspace{2mm}
\textbf{European Market (MiFID II):}
\begin{itemize}
\item Dark pool volume cap: 8\% per venue, 4\% instrument-level
\item Systematic Internalizers (SIs) disclosure
\item Consolidated tape still fragmented (no official CTP)
\end{itemize}

\column{0.48\textwidth}
\textbf{Best Execution Criteria:}
\begin{itemize}
\item \textbf{Price:} Most important for large institutions
\item \textbf{Cost:} Total explicit and implicit costs
\item \textbf{Speed:} Critical for time-sensitive strategies
\item \textbf{Likelihood of Execution:} Fill rate priority
\item \textbf{Settlement:} Certainty and timing
\end{itemize}

\vspace{2mm}
\textbf{Regulatory Requirements:}
\begin{itemize}
\item Reg NMS (US): National best bid/offer protection
\item MiFID II (EU): Best execution reporting
\item Quarterly 606 reports: routing practices disclosure
\end{itemize}
\end{columns}
\bottomnote{Key concepts from this slide inform practical applications in finance.}
\end{frame}

\begin{frame}[t]{HFT Market Share Evolution}
\begin{center}
\includegraphics[width=0.60\textwidth]{figures/hft_market_share/hft_market_share.pdf}
\end{center}
\bottomnote{HFT now represents a significant portion of market activity in major venues.}
\end{frame}

\begin{frame}[t]{Market Maker Economics}
\begin{center}
\includegraphics[width=0.60\textwidth]{figures/market_makers/market_makers.pdf}
\end{center}
\bottomnote{Market makers provide continuous liquidity through bid-ask quotes and rebates.}
\end{frame}

\section{Dark Pools and Alternative Venues}

\begin{frame}[t]{Dark Pool Types and Mechanics}
\begin{columns}[T]
\column{0.48\textwidth}
\textbf{Dark Pool Categories:}
\begin{itemize}
\item \textbf{Broker-Dealer Owned:} UBS ATS, MS Pool, GS Sigma X
\item \textbf{Agency Broker:} ITG POSIT, Liquidnet (block focus)
\item \textbf{Exchange-Owned:} NYSE Midpoint, Nasdaq TRF
\item \textbf{Electronic Market Makers:} Citadel Connect, Virtu
\end{itemize}

\vspace{2mm}
\textbf{Execution Mechanisms:}
\begin{itemize}
\item \textbf{Mid-Point Peg:} Execute at NBBO mid-point
\item \textbf{Periodic Auction:} Batch matching (IEX, Cboe LIS)
\item \textbf{Conditional Orders:} Minimum size, IOC only
\item \textbf{No Pre-Trade Transparency:} Orders invisible
\end{itemize}

\column{0.48\textwidth}
\textbf{Advantages:}
\begin{itemize}
\item Reduced market impact for large orders
\item Price improvement (mid-point execution)
\item Information leakage protection
\item Lower adverse selection vs lit markets
\end{itemize}

\vspace{2mm}
\textbf{Disadvantages:}
\begin{itemize}
\item Lower fill rates (10-30\% typical)
\item Information asymmetry concerns
\item Potential for predatory strategies
\item Regulatory scrutiny (MiFID II caps)
\end{itemize}

\vspace{2mm}
\textbf{Market Share:}
US dark pool volume: 12-15\% (down from 18\% pre-2018)
\end{columns}
\bottomnote{Key concepts from this slide inform practical applications in finance.}
\end{frame}

\begin{frame}[t]{Alternative Trading Systems (ATS)}
\begin{columns}[T]
\column{0.48\textwidth}
\textbf{ATS Regulatory Framework:}
\begin{itemize}
\item \textbf{SEC Regulation ATS (1998):} Registration and reporting
\item \textbf{Form ATS:} Operational details disclosure
\item \textbf{Form ATS-N (2018):} Enhanced transparency
\item \textbf{Reg SCI:} Systems compliance and integrity
\item \textbf{CAT (Consolidated Audit Trail):} Order tracking
\end{itemize}

\vspace{2mm}
\textbf{ATS vs Exchange:}
\begin{itemize}
\item No self-regulatory organization (SRO) status
\item Broker-dealer registration required
\item Less stringent listing requirements
\item More flexible fee structures
\end{itemize}

\column{0.48\textwidth}
\textbf{Specialized ATS Models:}
\begin{itemize}
\item \textbf{Block Trading:} Liquidnet (min 10k shares)
\item \textbf{Retail-Focused:} Off-exchange wholesalers
\item \textbf{Speed Bumps:} IEX 350-microsecond delay
\item \textbf{Frequent Batch Auctions:} Anti-HFT design
\end{itemize}

\vspace{2mm}
\textbf{IEX Innovations (2013):}
\begin{itemize}
\item Coiled fiber delay (350 microseconds)
\item Crumbling quote indicator (prevents stale pricing)
\item Discretionary peg (hide spread capture)
\item Exchange status granted 2016
\end{itemize}
\end{columns}
\bottomnote{Electronic trading has transformed market structure and efficiency.}
\end{frame}

\section{Regulatory and Risk Frameworks}

\begin{frame}[t]{Trading Regulation and Surveillance}
\begin{columns}[T]
\column{0.48\textwidth}
\textbf{Key Regulations:}
\begin{itemize}
\item \textbf{Reg NMS (2005):} Order protection, access, sub-penny rules
\item \textbf{MiFID II (2018):} Transparency, best execution, algo regulation
\item \textbf{MAR (2016):} Market abuse detection and reporting
\item \textbf{Circuit Breakers:} Single-stock (LULD) and market-wide
\item \textbf{Position Limits:} Derivatives and commodity futures
\end{itemize}

\column{0.48\textwidth}
\textbf{Surveillance Systems:}
\begin{itemize}
\item \textbf{Pattern Detection:} Layering, spoofing, wash trades
\item \textbf{Cross-Market:} Equity-derivatives manipulation
\item \textbf{CAT (Consolidated Audit Trail):} Full order lifecycle
\item \textbf{Machine Learning:} Anomaly detection algorithms
\item \textbf{Real-Time Alerts:} Sub-second violation flagging
\end{itemize}
\end{columns}

\vspace{3mm}
\textbf{LULD (Limit Up-Limit Down):} Trading halts if price moves exceed \% band (5-10\% depending on tier and time)
\bottomnote{Regulatory frameworks shape adoption patterns and industry structure.}
\end{frame}

\begin{frame}[t]{Pre-Trade and Post-Trade Risk Controls}
\begin{columns}[T]
\column{0.48\textwidth}
\textbf{Pre-Trade Risk Checks:}
\begin{itemize}
\item \textbf{Order Price Limits:} \% deviation from reference price
\item \textbf{Order Size Limits:} Max shares/contracts per order
\item \textbf{Duplicate Orders:} Detect unintended resubmissions
\item \textbf{Position Limits:} Net and gross exposure caps
\item \textbf{Notional Limits:} Dollar value thresholds
\item \textbf{Fat Finger:} Size and price reasonableness checks
\end{itemize}

\vspace{2mm}
\textbf{Latency Impact:}
Pre-trade checks add 10-50 microseconds per order

\column{0.48\textwidth}
\textbf{Post-Trade Monitoring:}
\begin{itemize}
\item \textbf{Wash Trade Detection:} Self-matching prevention
\item \textbf{Marking the Close:} Unusual end-of-day activity
\item \textbf{Momentum Ignition:} Rapid price manipulation
\item \textbf{Layering/Spoofing:} Non-bona fide order patterns
\end{itemize}

\vspace{2mm}
\textbf{Kill Switches:}
\begin{itemize}
\item Exchange-level: immediate market shutdown
\item Broker-level: cancel all active orders
\item Client-level: terminate specific trading IDs
\item Activation time: under 1 second
\end{itemize}
\end{columns}
\bottomnote{Risk management is essential for financial stability and profitability.}
\end{frame}

\begin{frame}[t]{Notable Trading Incidents}
\begin{columns}[T]
\column{0.48\textwidth}
\textbf{Flash Crash (May 6, 2010):}
\begin{itemize}
\item Dow Jones drops 1000 points in minutes
\item E-Mini S\&P futures selling algorithm trigger
\item HFT withdrawal exacerbates liquidity vacuum
\item Led to LULD circuit breakers (2012)
\end{itemize}

\vspace{2mm}
\textbf{Knight Capital (August 1, 2012):}
\begin{itemize}
\item Software deployment error
\item \$440 million loss in 45 minutes
\item 4 million trades across 154 stocks
\item Company near-bankruptcy, acquired by Getco
\end{itemize}

\column{0.48\textwidth}
\textbf{BATS IPO (March 23, 2012):}
\begin{itemize}
\item BATS exchange attempts self-listing
\item Software bug causes shares to drop from \$16 to \$0.0002
\item IPO withdrawn, exchange embarrassed
\item Highlighted exchange technology risks
\end{itemize}

\vspace{2mm}
\textbf{Lessons Learned:}
\begin{itemize}
\item Mandatory kill switches (MiFID II/Reg SCI)
\item Enhanced testing for algorithm changes
\item Circuit breakers at multiple levels
\item Real-time position and P\&L monitoring
\end{itemize}
\end{columns}
\bottomnote{Electronic trading has transformed market structure and efficiency.}
\end{frame}

\begin{frame}[t]{Summary and Key Takeaways}
\begin{columns}[T]
\column{0.48\textwidth}
\textbf{Core Concepts:}
\begin{itemize}
\item Electronic trading systems enable microsecond-latency execution
\item Order types range from basic (market/limit) to complex (iceberg/algorithmic)
\item Price-time priority matching dominates most markets
\item Smart order routers optimize across fragmented venues
\end{itemize}

\vspace{2mm}
\textbf{Market Structure:}
\begin{itemize}
\item 16 US exchanges + 30+ dark pools
\item Dark pools provide 12-15\% of equity volume
\item Mid-point execution reduces impact for large orders
\item Regulatory caps limit dark trading (MiFID II)
\end{itemize}

\column{0.48\textwidth}
\textbf{Risk and Regulation:}
\begin{itemize}
\item Pre-trade risk checks prevent erroneous orders
\item Circuit breakers halt extreme volatility
\item Surveillance systems detect manipulation
\item Incidents (Flash Crash, Knight) drive enhanced controls
\end{itemize}

\vspace{2mm}
\textbf{Technology Trends:}
\begin{itemize}
\item Latency competition drives infrastructure investment
\item Co-location and FPGA acceleration
\item Machine learning in routing and surveillance
\item Blockchain exploration for settlement (T+0)
\end{itemize}
\end{columns}
\end{frame}

\end{document}
