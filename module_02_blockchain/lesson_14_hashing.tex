\documentclass[8pt,aspectratio=169]{beamer}
\usetheme{Madrid}
\usepackage{graphicx}
\usepackage{booktabs}
\usepackage{adjustbox}
\usepackage{multicol}
\usepackage{amsmath}
\usepackage{amssymb}

\definecolor{mlblue}{RGB}{0,102,204}
\definecolor{mlpurple}{RGB}{51,51,178}
\definecolor{mllavender}{RGB}{173,173,224}
\definecolor{mllavender2}{RGB}{193,193,232}
\definecolor{mllavender3}{RGB}{204,204,235}
\definecolor{mllavender4}{RGB}{214,214,239}
\definecolor{mlorange}{RGB}{255, 127, 14}
\definecolor{mlgreen}{RGB}{44, 160, 44}
\definecolor{mlred}{RGB}{214, 39, 40}

\setbeamercolor{palette primary}{bg=mllavender3,fg=mlpurple}
\setbeamercolor{palette secondary}{bg=mllavender2,fg=mlpurple}
\setbeamercolor{palette tertiary}{bg=mllavender,fg=white}
\setbeamercolor{structure}{fg=mlpurple}
\setbeamercolor{frametitle}{fg=mlpurple,bg=mllavender3}
\setbeamertemplate{navigation symbols}{}
\setbeamertemplate{itemize items}[circle]
\setbeamersize{text margin left=5mm,text margin right=5mm}

% Bottom note command for key takeaways
\newcommand{\bottomnote}[1]{%
\vfill
\vspace{-2mm}
\textcolor{mllavender2}{\rule{\textwidth}{0.4pt}}
\vspace{1mm}
\footnotesize
\textbf{#1}
}
\title{Lesson 14: Blocks and Cryptographic Hashing}
\subtitle{Module 2: Blockchain and Cryptocurrencies}
\author{Digital Finance}
\date{\today}

\begin{document}

\begin{frame}
\titlepage
\end{frame}

\begin{frame}{Block Structure: Anatomy of a Bitcoin Block}
\vspace{-2mm}
\textbf{Every Bitcoin block contains two main parts:}

\vspace{3mm}
\begin{columns}[T]
\column{0.48\textwidth}
\textbf{Block Header (80 bytes):}
\begin{itemize}
\item \textcolor{mlpurple}{\textbf{Version}} (4 bytes): Protocol version
\item \textcolor{mlpurple}{\textbf{Previous Hash}} (32 bytes): Link to parent
\item \textcolor{mlpurple}{\textbf{Merkle Root}} (32 bytes): Transaction summary
\item \textcolor{mlpurple}{\textbf{Timestamp}} (4 bytes): Block creation time
\item \textcolor{mlpurple}{\textbf{Difficulty Target}} (4 bytes): Mining threshold
\item \textcolor{mlpurple}{\textbf{Nonce}} (4 bytes): Mining puzzle solution
\end{itemize}

\column{0.48\textwidth}
\textbf{Block Body (Variable):}
\begin{itemize}
\item \textcolor{mlpurple}{\textbf{Transaction Counter}}: Number of transactions
\item \textcolor{mlpurple}{\textbf{Transactions}}: Actual transaction data
\item Typical: 2000-3000 transactions
\item Max size: 4 MB (with SegWit)
\item Average: 1.5-2 MB
\end{itemize}

\vspace{3mm}
\textbf{Block Hash:}
\begin{itemize}
\item SHA-256 hash of the header
\item Uniquely identifies block
\item Must meet difficulty target
\end{itemize}
\end{columns}
\end{frame}

\begin{frame}{Example: Bitcoin Block 800,000}
\vspace{-2mm}
\footnotesize
\begin{table}[h]
\centering
\begin{tabular}{ll}
\toprule
\textbf{Field} & \textbf{Value} \\
\midrule
\textbf{Block Hash} & \texttt{00000000000000000002a7c4c1e48d76c5a37902165a270156b7a8d72728a054} \\
\textbf{Previous Hash} & \texttt{000000000000000000012117ad9f72c1c0e42329492e8c18f9e945a5d0f1b9a4} \\
\textbf{Merkle Root} & \texttt{7e1c6b0f5e9c8d9e3a2f1b4c5d6e7f8a9b0c1d2e3f4a5b6c7d8e9f0a1b2c3d4} \\
\textbf{Timestamp} & 2023-07-13 20:42:05 UTC \\
\textbf{Difficulty} & 53,911,173,001,054 \\
\textbf{Nonce} & 1,868,822,685 \\
\textbf{Transactions} & 3,285 \\
\textbf{Block Size} & 1,582,419 bytes (1.58 MB) \\
\textbf{Block Reward} & 6.25 BTC \\
\textbf{Total Fees} & 0.183 BTC \\
\midrule
\textbf{Height} & 800,000 \\
\textbf{Confirmations} & 50,000+ (as of Dec 2024) \\
\bottomrule
\end{tabular}
\end{table}

\vspace{2mm}
\textbf{Note:} Hash starts with 19 leading zeros - probability of finding this: $1/2^{76}$
\bottomnote{Case studies provide concrete evidence of technology impact and adoption patterns.}
\end{frame}

\begin{frame}{What is a Hash Function?}
\vspace{-2mm}
\textbf{Hash Function:} Mathematical algorithm that maps arbitrary data to fixed-size output

\vspace{3mm}
\begin{columns}[T]
\column{0.48\textwidth}
\textbf{Properties:}
\begin{enumerate}
\item \textbf{Deterministic:} Same input always produces same output
\item \textbf{Fast:} Quick to compute
\item \textbf{One-way:} Cannot reverse (pre-image resistance)
\item \textbf{Collision-resistant:} Hard to find two inputs with same hash
\item \textbf{Avalanche effect:} Tiny input change drastically changes output
\end{enumerate}

\column{0.48\textwidth}
\textbf{Examples:}
\begin{itemize}
\item MD5: 128 bits (broken, not secure)
\item SHA-1: 160 bits (deprecated)
\item SHA-256: 256 bits (Bitcoin)
\item SHA-3: Variable (latest standard)
\item BLAKE2: 256/512 bits (faster)
\end{itemize}

\vspace{3mm}
\textbf{Output Size:}
\begin{itemize}
\item SHA-256: 64 hex characters
\item SHA-256: $2^{256}$ possible outputs
\item More atoms in universe: $\approx 2^{265}$
\end{itemize}
\end{columns}
\bottomnote{Clear definitions are essential for understanding complex technical concepts.}
\end{frame}

\begin{frame}{SHA-256: The Bitcoin Hash Function}
\vspace{-2mm}
\textbf{SHA-256} (Secure Hash Algorithm 256-bit) - designed by NSA, published 2001

\vspace{3mm}
\textbf{Input/Output:}
\begin{itemize}
\item Input: Any data (message, file, block header)
\item Output: 256-bit hash (64 hexadecimal characters)
\item Example: \texttt{SHA256("Hello") = 185f8db32271fe25f561a6fc938b2e264306ec304eda518007d1764826381969}
\end{itemize}

\vspace{3mm}
\textbf{Algorithm Steps (Simplified):}
\begin{enumerate}
\item \textbf{Padding:} Add bits to make length $\equiv 448 \pmod{512}$
\item \textbf{Append Length:} Add 64-bit message length
\item \textbf{Initialize:} Set 8 hash values (constants)
\item \textbf{Process:} 64 rounds of bitwise operations per 512-bit chunk
\item \textbf{Output:} Concatenate final 8 values = 256-bit hash
\end{enumerate}

\vspace{3mm}
\textbf{Security:} No known practical attacks - best attack requires $2^{128}$ operations (infeasible)
\bottomnote{Bitcoin remains the largest cryptocurrency by market cap and network security.}
\end{frame}

\begin{frame}{The Avalanche Effect: Demonstration}
\vspace{-2mm}
\textbf{Small input change causes completely different hash:}

\vspace{3mm}
\footnotesize
\begin{table}[h]
\centering
\begin{tabular}{lp{9cm}}
\toprule
\textbf{Input} & \textbf{SHA-256 Hash} \\
\midrule
\texttt{Hello World} & \texttt{a591a6d40bf420404a011733cfb7b190d62c65bf0bcda32b57b277d9ad9f146e} \\
\texttt{Hello World!} & \texttt{7f83b1657ff1fc53b92dc18148a1d65dfc2d4b1fa3d677284addd200126d9069} \\
\texttt{hello world} & \texttt{b94d27b9934d3e08a52e52d7da7dabfac484efe37a5380ee9088f7ace2efcde9} \\
\texttt{Hello Worle} & \texttt{44896a8f3e6ec4f93c8d3c92d72c8df3e3c74b5d6e9e8f3a5b6c7d8e9f0a1b2c} \\
\bottomrule
\end{tabular}
\end{table}

\vspace{3mm}
\textbf{Analysis:}
\begin{itemize}
\item Adding single character (\texttt{!}) changes 100\% of hash bits
\item Changing case (\texttt{H} to \texttt{h}) changes entire hash
\item Changing one letter (\texttt{d} to \texttt{e}) produces unrelated hash
\item No pattern or correlation between similar inputs
\end{itemize}

\vspace{3mm}
\textbf{Implication:} Cannot predict hash output - must compute to verify
\bottomnote{Key concepts from this slide inform practical applications in finance.}
\end{frame}

\begin{frame}{Hash Space and Collision Resistance}
\vspace{-2mm}
\textbf{SHA-256 produces $2^{256}$ possible hashes - how big is that?}

\vspace{3mm}
\begin{itemize}
\item $2^{256} \approx 1.16 \times 10^{77}$ different hashes
\item Estimated atoms in observable universe: $10^{80}$
\item Estimated grains of sand on Earth: $10^{24}$
\item Age of universe in seconds: $4.3 \times 10^{17}$
\end{itemize}

\vspace{3mm}
\textbf{Birthday Paradox and Collision Probability:}
\begin{itemize}
\item To find collision with 50\% probability: Need to hash $2^{128}$ inputs
\item At 1 trillion hashes/second: Would take $10^{25}$ years
\item For comparison: Sun will die in $5 \times 10^9$ years
\end{itemize}

\vspace{3mm}
\textbf{Practical Conclusion:} SHA-256 collisions are computationally infeasible
\bottomnote{Cryptographic primitives provide the security foundation for blockchain systems.}
\end{frame}

\begin{frame}{Pre-image Resistance: The One-Way Property}
\vspace{-2mm}
\textbf{Given a hash, can you find the original input?}

\vspace{3mm}
\begin{columns}[T]
\column{0.48\textwidth}
\textbf{Forward (Easy):}
\[
\text{Input} \xrightarrow{\text{SHA-256}} \text{Hash}
\]
\begin{itemize}
\item Instant computation
\item Example: \texttt{SHA256("Bitcoin")} = \texttt{b4056df6...}
\item Deterministic and fast
\end{itemize}

\column{0.48\textwidth}
\textbf{Reverse (Impossible):}
\[
\text{Hash} \xrightarrow{\text{???}} \text{Input}
\]
\begin{itemize}
\item No mathematical inverse
\item Only option: Brute force
\item Try all possible inputs
\item Computationally infeasible
\end{itemize}
\end{columns}

\vspace{5mm}
\textbf{Example Attack:} Find input for hash \texttt{000000000019d6689c085ae165831e93...}
\begin{itemize}
\item Average attempts needed: $2^{255}$ (half the hash space)
\item At 1 quadrillion attempts/second: $10^{61}$ years
\item Even with all computers on Earth: Still infeasible
\end{itemize}
\bottomnote{Key concepts from this slide inform practical applications in finance.}
\end{frame}

\begin{frame}{Hash Pointers: Linking Blocks Together}
\vspace{-2mm}
\textbf{Hash Pointer:} Data structure that contains both location AND cryptographic hash of data

\vspace{3mm}
\begin{columns}[T]
\column{0.48\textwidth}
\textbf{Regular Pointer:}
\begin{itemize}
\item Points to memory address
\item Can retrieve data
\item No integrity check
\item Data can be tampered
\end{itemize}

\vspace{3mm}
\textbf{Example:}
\begin{itemize}
\item Pointer: \texttt{0x7ffe5367e044}
\item Data at address: \texttt{"Alice pays Bob 5 BTC"}
\item If data changes: No detection
\end{itemize}

\column{0.48\textwidth}
\textbf{Hash Pointer:}
\begin{itemize}
\item Points to data location
\item Contains hash of data
\item Can retrieve AND verify
\item Tamper-evident
\end{itemize}

\vspace{3mm}
\textbf{Example:}
\begin{itemize}
\item Location: Block 100
\item Hash: \texttt{00000000000080b66c911bd5ba14a74db...}
\item If data changes: Hash mismatch detected
\end{itemize}
\end{columns}

\vspace{5mm}
\textbf{Blockchain Application:} Each block header contains hash pointer to previous block
\bottomnote{Cryptographic primitives provide the security foundation for blockchain systems.}
\end{frame}

\begin{frame}{Why Blockchain is Immutable}
\vspace{-2mm}
\textbf{Changing a single transaction breaks the entire chain:}

\vspace{3mm}
\textbf{Scenario:} Attacker tries to modify transaction in Block 100

\vspace{2mm}
\begin{enumerate}
\item Modify transaction data in Block 100
\item Merkle root changes (transaction hash changed)
\item Block 100 hash changes (header data changed)
\item Block 101's ``Previous Hash'' field no longer matches
\item Block 101 becomes invalid
\item Must recalculate Block 101 hash (requires solving PoW puzzle)
\item Block 102's ``Previous Hash'' no longer matches
\item Must recalculate Block 102, 103, 104... all subsequent blocks
\item While attacker recalculates, honest miners add new blocks
\item Attacker must outpace entire network (51\% attack)
\end{enumerate}

\vspace{3mm}
\textbf{Conclusion:} Older blocks are exponentially harder to modify - after 6 confirmations (1 hour), considered practically irreversible
\bottomnote{AI and ML are transforming financial services through automation and prediction.}
\end{frame}

\begin{frame}{Merkle Trees: Efficient Transaction Verification}
\vspace{-2mm}
\textbf{Merkle Tree:} Binary tree of hashes allowing efficient verification of large datasets

\vspace{3mm}
\textbf{Structure (Bottom-up):}
\begin{enumerate}
\item \textbf{Leaf Nodes:} Hash of each transaction
\item \textbf{Parent Nodes:} Hash of concatenated child hashes
\item \textbf{Root:} Final hash (Merkle root) in block header
\end{enumerate}

\vspace{3mm}
\begin{columns}[T]
\column{0.48\textwidth}
\textbf{Benefits:}
\begin{itemize}
\item Compact representation of all transactions
\item Verify transaction inclusion with $O(\log n)$ hashes
\item SPV (Simplified Payment Verification) for light clients
\item Only need block headers + Merkle proof
\end{itemize}

\column{0.48\textwidth}
\textbf{Example (8 transactions):}
\begin{itemize}
\item Full verification: Download all 8 transactions
\item Merkle proof: Download 3 hashes only
\item Proof size: $\log_2(8) = 3$ hashes
\item For 2000 transactions: Only 11 hashes needed
\end{itemize}
\end{columns}

\vspace{3mm}
\textbf{SPV Use Case:} Mobile Bitcoin wallet verifies payment without downloading full blockchain (500+ GB)
\bottomnote{Key concepts from this slide inform practical applications in finance.}
\end{frame}

\begin{frame}{Merkle Tree Example: 4 Transactions}
\vspace{-2mm}
\textbf{Constructing Merkle Root:}

\vspace{3mm}
\footnotesize
\begin{align*}
\text{Transactions:} \quad & T_1 = \text{``Alice to Bob 1 BTC''} \\
& T_2 = \text{``Carol to Dave 2 BTC''} \\
& T_3 = \text{``Eve to Frank 0.5 BTC''} \\
& T_4 = \text{``Grace to Heidi 3 BTC''}
\end{align*}

\vspace{2mm}
\textbf{Step 1 - Leaf Hashes:}
\begin{align*}
H_1 &= \text{SHA256}(T_1) = \texttt{a3c8...} \\
H_2 &= \text{SHA256}(T_2) = \texttt{7f4b...} \\
H_3 &= \text{SHA256}(T_3) = \texttt{9e2d...} \\
H_4 &= \text{SHA256}(T_4) = \texttt{1c5a...}
\end{align*}

\vspace{2mm}
\textbf{Step 2 - Parent Hashes:}
\begin{align*}
H_{12} &= \text{SHA256}(H_1 \| H_2) = \texttt{d8f3...} \\
H_{34} &= \text{SHA256}(H_3 \| H_4) = \texttt{6b9e...}
\end{align*}

\vspace{2mm}
\textbf{Step 3 - Merkle Root:}
\[
\text{MerkleRoot} = \text{SHA256}(H_{12} \| H_{34}) = \texttt{4e7a...}
\]
\bottomnote{Case studies provide concrete evidence of technology impact and adoption patterns.}
\end{frame}

\begin{frame}{Merkle Proof: Verifying Transaction Inclusion}
\vspace{-2mm}
\textbf{Problem:} Light client wants to verify $T_2$ is in block, without downloading all transactions

\vspace{3mm}
\textbf{Solution:} Full node provides Merkle proof (3 hashes):
\begin{itemize}
\item $H_1$ (sibling of $T_2$)
\item $H_{34}$ (sibling of $H_{12}$)
\item Merkle root from block header
\end{itemize}

\vspace{3mm}
\textbf{Verification Steps:}
\begin{enumerate}
\item Client has $T_2$ and computes $H_2 = \text{SHA256}(T_2)$
\item Receives $H_1$ from proof, computes $H_{12} = \text{SHA256}(H_1 \| H_2)$
\item Receives $H_{34}$ from proof, computes $\text{MerkleRoot} = \text{SHA256}(H_{12} \| H_{34})$
\item Compares computed Merkle root with block header
\item If match: Transaction confirmed in block
\end{enumerate}

\vspace{3mm}
\textbf{Efficiency:} For $n$ transactions, proof size = $O(\log_2 n)$ hashes
\begin{itemize}
\item 1,000 transactions: 10 hashes (320 bytes)
\item 1,000,000 transactions: 20 hashes (640 bytes)
\end{itemize}
\bottomnote{Key concepts from this slide inform practical applications in finance.}
\end{frame}

\begin{frame}{Difficulty Target and Leading Zeros}
\vspace{-2mm}
\textbf{Bitcoin Mining Puzzle:} Find nonce such that block hash < target

\vspace{3mm}
\textbf{Target Representation:}
\begin{itemize}
\item Difficulty target: Maximum allowed hash value
\item Higher difficulty = Lower target = More leading zeros required
\item Current (Dec 2024): ~19 leading zeros
\end{itemize}

\vspace{3mm}
\begin{table}[h]
\centering
\footnotesize
\begin{tabular}{lll}
\toprule
\textbf{Leading Zeros} & \textbf{Probability} & \textbf{Average Attempts} \\
\midrule
1 & $1/16$ & 16 \\
4 & $1/65,536$ & 65,536 \\
8 & $1/4.3 \times 10^9$ & 4.3 billion \\
16 & $1/1.8 \times 10^{19}$ & 18 quintillion \\
19 & $1/1.5 \times 10^{23}$ & 150 sextillion \\
\bottomrule
\end{tabular}
\end{table}

\vspace{3mm}
\textbf{Example Valid Hash:}
\texttt{00000000000000000002a7c4c1e48d76c5a37902165a270156b7a8d72728a054}
\bottomnote{Key concepts from this slide inform practical applications in finance.}
\end{frame}

\begin{frame}{Difficulty Adjustment Mechanism}
\vspace{-2mm}
\textbf{Goal:} Maintain ~10 minute average block time despite changing network hash rate

\vspace{3mm}
\textbf{Algorithm:}
\begin{itemize}
\item Every 2,016 blocks (~2 weeks)
\item Calculate actual time taken vs. expected time (20,160 minutes)
\item Adjust difficulty: $\text{NewDifficulty} = \text{OldDifficulty} \times \frac{\text{Expected Time}}{\text{Actual Time}}$
\item Maximum adjustment: 4x increase or 1/4 decrease per period
\end{itemize}

\vspace{3mm}
\textbf{Example (2023):}
\begin{itemize}
\item Period: Blocks 780,000 - 782,015
\item Expected: 20,160 minutes (14 days)
\item Actual: 18,900 minutes (13.125 days) - blocks came faster
\item Adjustment: Difficulty increased by 6.3\%
\end{itemize}

\vspace{3mm}
\textbf{Historical Trend:}
\begin{itemize}
\item 2009: Difficulty = 1 (CPU mining)
\item 2024: Difficulty = 73 trillion (ASIC mining)
\item $10^{13}$ times harder than genesis block
\end{itemize}
\bottomnote{Key concepts from this slide inform practical applications in finance.}
\end{frame}

\begin{frame}{Hash Rate: Network Computing Power}
\vspace{-2mm}
\textbf{Hash Rate:} Total number of hash computations per second across network

\vspace{3mm}
\begin{table}[h]
\centering
\footnotesize
\begin{tabular}{lll}
\toprule
\textbf{Period} & \textbf{Hash Rate} & \textbf{Mining Technology} \\
\midrule
2009 & ~5 MH/s & CPU (Satoshi solo mining) \\
2010 & ~200 GH/s & GPU mining begins \\
2013 & ~20 TH/s & ASIC mining begins \\
2017 & ~10 EH/s & Industrial mining farms \\
2021 & ~180 EH/s & Peak before China ban \\
2024 & ~600 EH/s & Post-halving recovery \\
\bottomrule
\end{tabular}
\end{table}

\vspace{3mm}
\textbf{Scale:}
\begin{itemize}
\item 600 EH/s = 600,000,000,000,000,000,000 hashes per second
\item More computing power than top 500 supercomputers combined
\item Each hash attempt: $\approx 1/600$ quintillion chance of success
\end{itemize}
\bottomnote{Network metrics provide objective measures of adoption and ecosystem health.}
\end{frame}

\begin{frame}{Block Propagation and Orphan Blocks}
\vspace{-2mm}
\textbf{Problem:} Network latency - two miners find valid blocks simultaneously

\vspace{3mm}
\textbf{Scenario:}
\begin{enumerate}
\item Miner A finds Block 700,000 at 12:00:00
\item Miner B finds different Block 700,000 at 12:00:01
\item Both blocks are valid and propagate through network
\item Network temporarily splits (some nodes see A, others see B)
\item Miners start working on both competing chains
\item Miner C finds Block 700,001 building on Block A
\item Longest chain rule: Block A chain wins
\item Block B becomes ``orphan block'' - discarded
\item Miner B loses 6.25 BTC reward
\end{enumerate}

\vspace{3mm}
\textbf{Statistics:}
\begin{itemize}
\item Orphan rate: ~0.5\% of blocks (rare but happens)
\item Average propagation time: 1-2 seconds globally
\item Why 10 minutes?: Minimize orphans while maintaining decentralization
\end{itemize}
\bottomnote{Key concepts from this slide inform practical applications in finance.}
\end{frame}

\begin{frame}{Blockchain Data Structures Summary}
\vspace{-2mm}
\begin{table}[h]
\centering
\footnotesize
\begin{tabular}{p{3cm}p{5cm}p{5cm}}
\toprule
\textbf{Structure} & \textbf{Purpose} & \textbf{Properties} \\
\midrule
Hash Function & Fingerprint data & Deterministic, one-way, collision-resistant, avalanche effect \\
Hash Pointer & Link + verify & Points to data + contains hash for integrity \\
Merkle Tree & Efficient verification & $O(\log n)$ proof size, enables SPV \\
Block Header & Minimal representation & 80 bytes contains all block metadata \\
Blockchain & Tamper-evident ledger & Chain of hash pointers, immutable \\
Difficulty Target & Mining puzzle & Adjusts to maintain 10-min block time \\
\bottomrule
\end{tabular}
\end{table}

\vspace{3mm}
\textbf{Key Insight:} All blockchain security stems from cryptographic hash functions - if SHA-256 breaks, Bitcoin breaks
\end{frame}

\begin{frame}{Security Analysis: What Could Go Wrong?}
\vspace{-2mm}
\begin{columns}[T]
\column{0.48\textwidth}
\textbf{Hash Function Attacks:}
\begin{itemize}
\item \textcolor{mlred}{Pre-image:} Find input from hash
\item \textcolor{mlred}{Collision:} Find two inputs with same hash
\item \textcolor{mlred}{Length extension:} Exploit hash construction
\end{itemize}

\vspace{3mm}
\textbf{SHA-256 Status:}
\begin{itemize}
\item \textcolor{mlgreen}{Pre-image:} $2^{256}$ security (safe)
\item \textcolor{mlgreen}{Collision:} $2^{128}$ security (safe)
\item \textcolor{mlgreen}{Length extension:} Not applicable to Bitcoin (double hashing)
\end{itemize}

\column{0.48\textwidth}
\textbf{Quantum Computing Threat:}
\begin{itemize}
\item Grover's algorithm: $\sqrt{2^{256}} = 2^{128}$ speedup
\item Still computationally infeasible
\item Bigger threat: Public key crypto (covered next lesson)
\end{itemize}

\vspace{3mm}
\textbf{Mitigation Strategies:}
\begin{itemize}
\item SHA-3 ready as backup
\item Post-quantum hash functions (SPHINCS+)
\item Bitcoin can hard fork if needed
\end{itemize}
\end{columns}

\vspace{3mm}
\centering
\textbf{Current Assessment:} No practical threat to SHA-256 in foreseeable future
\end{frame}

\begin{frame}{Practical Exercise: Computing Block Hash}
\vspace{-2mm}
\textbf{Simplified Example:} Calculate hash for mini-block

\vspace{3mm}
\textbf{Block Header (simplified):}
\begin{itemize}
\item Previous Hash: \texttt{00000000000000000001}
\item Merkle Root: \texttt{abcdef123456}
\item Timestamp: \texttt{1702000000}
\item Nonce: \texttt{?}
\end{itemize}

\vspace{3mm}
\textbf{Task:} Find nonce such that hash starts with 4 zeros (\texttt{0000...})

\vspace{3mm}
\textbf{Brute Force Approach:}
\begin{enumerate}
\item Concatenate header: \texttt{00000000000000000001|abcdef123456|1702000000|0}
\item Compute SHA-256: \texttt{7a3b2c1d...} (no leading zeros)
\item Increment nonce: Try \texttt{1}, then \texttt{2}, then \texttt{3}...
\item Keep trying until hash starts with \texttt{0000}
\item Expected attempts: $2^{16} = 65,536$
\end{enumerate}

\vspace{3mm}
\textbf{Solution:} Nonce \texttt{45,892} produces \texttt{0000f7a3b2c1d...}
\bottomnote{Cryptographic primitives provide the security foundation for blockchain systems.}
\end{frame}

\begin{frame}{Block Finality: How Many Confirmations?}
\vspace{-2mm}
\textbf{Confirmations:} Number of blocks built on top of transaction's block

\vspace{3mm}
\begin{table}[h]
\centering
\footnotesize
\begin{tabular}{clp{6cm}}
\toprule
\textbf{Confirmations} & \textbf{Time} & \textbf{Security Level} \\
\midrule
0 & 0 min & Unconfirmed - in mempool only, easily reversible \\
1 & ~10 min & Low - vulnerable to accidental orphaning \\
3 & ~30 min & Medium - coffee purchase acceptable \\
6 & ~60 min & High - exchanges require this, 99.9\% final \\
100 & ~17 hours & Very High - coinbase maturity (mining reward) \\
\bottomrule
\end{tabular}
\end{table}

\vspace{3mm}
\textbf{Attack Cost Analysis (6 confirmations):}
\begin{itemize}
\item Attacker must mine 7 blocks faster than honest network
\item Requires 51\% hash rate for extended period
\item Cost: ~\$500,000/hour in electricity + equipment
\item Only profitable for large transaction values
\end{itemize}

\vspace{3mm}
\textbf{Rule of Thumb:} Value < \$1,000: 1-3 confirmations; Value > \$10,000: 6+ confirmations
\bottomnote{Understanding the process flow is key to identifying optimization opportunities.}
\end{frame}

\begin{frame}{Next Lesson Preview}
\vspace{-2mm}
\begin{center}
\Large\textbf{Lesson 15: Public Key Cryptography and Signatures}
\end{center}

\vspace{5mm}
\textbf{What We'll Cover:}
\begin{itemize}
\item Symmetric vs asymmetric encryption
\item Elliptic Curve Cryptography (ECC) fundamentals
\item Public/private key pairs in Bitcoin
\item Digital signatures (ECDSA)
\item Address generation and wallet structure
\item Security best practices
\end{itemize}

\vspace{5mm}
\textbf{Prepare:}
\begin{itemize}
\item Review basic modular arithmetic ($a \bmod n$)
\item Understand concept of mathematical groups
\item Install Bitcoin wallet for hands-on key generation
\end{itemize}
\end{frame}

\begin{frame}{Summary: Key Takeaways}
\vspace{-2mm}
\begin{enumerate}
\item \textbf{Block Structure:} 80-byte header + variable transaction body
\item \textbf{SHA-256:} One-way, collision-resistant hash with $2^{256}$ output space
\item \textbf{Avalanche Effect:} Tiny input change completely alters hash
\item \textbf{Hash Pointers:} Enable tamper-evident data structures
\item \textbf{Merkle Trees:} Efficient verification with $O(\log n)$ proof size
\item \textbf{Immutability:} Changing old block requires recalculating all subsequent blocks
\item \textbf{Difficulty:} Adjusts every 2016 blocks to maintain 10-min average
\item \textbf{Finality:} 6 confirmations (~1 hour) considered irreversible
\end{enumerate}

\vspace{3mm}
\centering
\textit{``Blockchain security is reducible to hash function security - SHA-256 is the foundation.''}
\end{frame}

\end{document}
