\documentclass[8pt,aspectratio=169]{beamer}
\usetheme{Madrid}
\usepackage{graphicx}
\usepackage{booktabs}
\usepackage{adjustbox}
\usepackage{multicol}
\usepackage{amsmath}
\usepackage{amssymb}

\definecolor{mlblue}{RGB}{0,102,204}
\definecolor{mlpurple}{RGB}{51,51,178}
\definecolor{mllavender}{RGB}{173,173,224}
\definecolor{mllavender2}{RGB}{193,193,232}
\definecolor{mllavender3}{RGB}{204,204,235}
\definecolor{mllavender4}{RGB}{214,214,239}
\definecolor{mlorange}{RGB}{255, 127, 14}
\definecolor{mlgreen}{RGB}{44, 160, 44}
\definecolor{mlred}{RGB}{214, 39, 40}

\setbeamercolor{palette primary}{bg=mllavender3,fg=mlpurple}
\setbeamercolor{palette secondary}{bg=mllavender2,fg=mlpurple}
\setbeamercolor{palette tertiary}{bg=mllavender,fg=white}
\setbeamercolor{structure}{fg=mlpurple}
\setbeamercolor{frametitle}{fg=mlpurple,bg=mllavender3}
\setbeamertemplate{navigation symbols}{}
\setbeamertemplate{itemize items}[circle]
\setbeamersize{text margin left=5mm,text margin right=5mm}

% Bottom note command for key takeaways
\newcommand{\bottomnote}[1]{%
\vfill
\vspace{-2mm}
\textcolor{mllavender2}{\rule{\textwidth}{0.4pt}}
\vspace{1mm}
\footnotesize
\textbf{#1}
}
\title{Lesson 14: Blocks and Cryptographic Hashing}
\subtitle{Module 2: Blockchain and Cryptocurrencies}
\author{Digital Finance}
\date{\today}

\begin{document}

\begin{frame}
\titlepage
\end{frame}

\begin{frame}[t]{Block Structure: Anatomy of a Bitcoin Block}
\vspace{-2mm}
\begin{center}
\includegraphics[width=0.60\textwidth]{figures/block_structure_detailed/block_structure_detailed.pdf}
\end{center}
\bottomnote{The 80-byte header contains all essential metadata; body holds actual transactions.}
\end{frame}

\begin{frame}[t]{Example: Bitcoin Block 800,000}
\vspace{-2mm}
\footnotesize
\begin{table}[h]
\centering
\begin{tabular}{ll}
\toprule
\textbf{Field} & \textbf{Value} \\
\midrule
\textbf{Block Hash} & \texttt{00000000000000000002a7c4c1e48d76c5a37902165a270156b7a8d72728a054} \\
\textbf{Previous Hash} & \texttt{000000000000000000012117ad9f72c1c0e42329492e8c18f9e945a5d0f1b9a4} \\
\textbf{Merkle Root} & \texttt{7e1c6b0f5e9c8d9e3a2f1b4c5d6e7f8a9b0c1d2e3f4a5b6c7d8e9f0a1b2c3d4} \\
\textbf{Timestamp} & 2023-07-13 20:42:05 UTC \\
\textbf{Difficulty} & 53,911,173,001,054 \\
\textbf{Nonce} & 1,868,822,685 \\
\textbf{Transactions} & 3,285 \\
\textbf{Block Size} & 1,582,419 bytes (1.58 MB) \\
\textbf{Block Reward} & 6.25 BTC \\
\textbf{Total Fees} & 0.183 BTC \\
\midrule
\textbf{Height} & 800,000 \\
\textbf{Confirmations} & 50,000+ (as of Dec 2024) \\
\bottomrule
\end{tabular}
\end{table}

\vspace{2mm}
\textbf{Note:} Hash starts with 19 leading zeros - probability of finding this: $1/2^{76}$
\bottomnote{Case studies provide concrete evidence of technology impact and adoption patterns.}
\end{frame}

\begin{frame}[t]{What is a Hash Function?}
\vspace{-2mm}
\textbf{Hash Function:} Mathematical algorithm that maps arbitrary data to fixed-size output

\vspace{3mm}
\begin{columns}[T]
\column{0.48\textwidth}
\textbf{Properties:}
\begin{enumerate}
\item \textbf{Deterministic:} Same input always produces same output
\item \textbf{Fast:} Quick to compute
\item \textbf{One-way:} Cannot reverse (pre-image resistance)
\item \textbf{Collision-resistant:} Hard to find two inputs with same hash
\item \textbf{Avalanche effect:} Tiny input change drastically changes output
\end{enumerate}

\column{0.48\textwidth}
\textbf{Examples:}
\begin{itemize}
\item MD5: 128 bits (broken, not secure)
\item SHA-1: 160 bits (deprecated)
\item SHA-256: 256 bits (Bitcoin)
\item SHA-3: Variable (latest standard)
\item BLAKE2: 256/512 bits (faster)
\end{itemize}

\vspace{3mm}
\textbf{Output Size:}
\begin{itemize}
\item SHA-256: 64 hex characters
\item SHA-256: $2^{256}$ possible outputs
\item More atoms in universe: $\approx 2^{265}$
\end{itemize}
\end{columns}
\bottomnote{Clear definitions are essential for understanding complex technical concepts.}
\end{frame}

\begin{frame}[t]{SHA-256: The Bitcoin Hash Function}
\vspace{-2mm}
\begin{center}
\includegraphics[width=0.60\textwidth]{figures/sha256_visualization/sha256_visualization.pdf}
\end{center}
\bottomnote{SHA-256 produces 256-bit fixed output regardless of input size - the backbone of Bitcoin security.}
\end{frame}

\begin{frame}[t]{The Avalanche Effect: Demonstration}
\vspace{-2mm}
\begin{center}
\includegraphics[width=0.60\textwidth]{figures/avalanche_effect/avalanche_effect.pdf}
\end{center}
\bottomnote{Even the smallest change produces a completely unpredictable new hash - no pattern correlation.}
\end{frame}

\begin{frame}[t]{Hash Space and Collision Resistance}
\vspace{-2mm}
\begin{center}
\includegraphics[width=0.60\textwidth]{figures/hash_space_scale/hash_space_scale.pdf}
\end{center}
\bottomnote{SHA-256 collisions are computationally infeasible - the space is incomprehensibly large.}
\end{frame}

\begin{frame}[t]{Pre-image Resistance: The One-Way Property}
\vspace{-2mm}
\textbf{Given a hash, can you find the original input?}

\vspace{3mm}
\begin{columns}[T]
\column{0.48\textwidth}
\textbf{Forward (Easy):}
\[
\text{Input} \xrightarrow{\text{SHA-256}} \text{Hash}
\]
\begin{itemize}
\item Instant computation
\item Example: \texttt{SHA256("Bitcoin")} = \texttt{b4056df6...}
\item Deterministic and fast
\end{itemize}

\column{0.48\textwidth}
\textbf{Reverse (Impossible):}
\[
\text{Hash} \xrightarrow{\text{???}} \text{Input}
\]
\begin{itemize}
\item No mathematical inverse
\item Only option: Brute force
\item Try all possible inputs
\item Computationally infeasible
\end{itemize}
\end{columns}

\vspace{5mm}
\textbf{Example Attack:} Find input for hash \texttt{000000000019d6689c085ae165831e93...}
\begin{itemize}
\item Average attempts needed: $2^{255}$ (half the hash space)
\item At 1 quadrillion attempts/second: $10^{61}$ years
\item Even with all computers on Earth: Still infeasible
\end{itemize}
\bottomnote{Key concepts from this slide inform practical applications in finance.}
\end{frame}

\begin{frame}[t]{Hash Pointers: Linking Blocks Together}
\vspace{-2mm}
\textbf{Hash Pointer:} Data structure that contains both location AND cryptographic hash of data

\vspace{3mm}
\begin{columns}[T]
\column{0.48\textwidth}
\textbf{Regular Pointer:}
\begin{itemize}
\item Points to memory address
\item Can retrieve data
\item No integrity check
\item Data can be tampered
\end{itemize}

\vspace{3mm}
\textbf{Example:}
\begin{itemize}
\item Pointer: \texttt{0x7ffe5367e044}
\item Data at address: \texttt{"Alice pays Bob 5 BTC"}
\item If data changes: No detection
\end{itemize}

\column{0.48\textwidth}
\textbf{Hash Pointer:}
\begin{itemize}
\item Points to data location
\item Contains hash of data
\item Can retrieve AND verify
\item Tamper-evident
\end{itemize}

\vspace{3mm}
\textbf{Example:}
\begin{itemize}
\item Location: Block 100
\item Hash: \texttt{00000000000080b66c911bd5ba14a74db...}
\item If data changes: Hash mismatch detected
\end{itemize}
\end{columns}

\vspace{5mm}
\textbf{Blockchain Application:} Each block header contains hash pointer to previous block
\bottomnote{Cryptographic primitives provide the security foundation for blockchain systems.}
\end{frame}

\begin{frame}[t]{Why Blockchain is Immutable}
\vspace{-2mm}
\begin{center}
\includegraphics[width=0.60\textwidth]{figures/immutability_cascade/immutability_cascade.pdf}
\end{center}
\bottomnote{Changing one block requires re-hashing all subsequent blocks - impractical against the whole network.}
\end{frame}

\begin{frame}[t]{Merkle Trees: Efficient Transaction Verification}
\vspace{-2mm}
\begin{center}
\includegraphics[width=0.60\textwidth]{figures/merkle_tree_structure/merkle_tree_structure.pdf}
\end{center}
\bottomnote{Merkle root summarizes all transactions; changing any transaction changes the root.}
\end{frame}

\begin{frame}[t]{Merkle Tree Example: 4 Transactions}
\vspace{-2mm}
\textbf{Constructing Merkle Root:}

\vspace{3mm}
\footnotesize
\begin{align*}
\text{Transactions:} \quad & T_1 = \text{``Alice to Bob 1 BTC''} \\
& T_2 = \text{``Carol to Dave 2 BTC''} \\
& T_3 = \text{``Eve to Frank 0.5 BTC''} \\
& T_4 = \text{``Grace to Heidi 3 BTC''}
\end{align*}

\vspace{2mm}
\textbf{Step 1 - Leaf Hashes:}
\begin{align*}
H_1 &= \text{SHA256}(T_1) = \texttt{a3c8...} \\
H_2 &= \text{SHA256}(T_2) = \texttt{7f4b...} \\
H_3 &= \text{SHA256}(T_3) = \texttt{9e2d...} \\
H_4 &= \text{SHA256}(T_4) = \texttt{1c5a...}
\end{align*}

\vspace{2mm}
\textbf{Step 2 - Parent Hashes:}
\begin{align*}
H_{12} &= \text{SHA256}(H_1 \| H_2) = \texttt{d8f3...} \\
H_{34} &= \text{SHA256}(H_3 \| H_4) = \texttt{6b9e...}
\end{align*}

\vspace{2mm}
\textbf{Step 3 - Merkle Root:}
\[
\text{MerkleRoot} = \text{SHA256}(H_{12} \| H_{34}) = \texttt{4e7a...}
\]
\bottomnote{Case studies provide concrete evidence of technology impact and adoption patterns.}
\end{frame}

\begin{frame}[t]{Merkle Proof: Verifying Transaction Inclusion}
\vspace{-2mm}
\begin{center}
\includegraphics[width=0.60\textwidth]{figures/merkle_proof/merkle_proof.pdf}
\end{center}
\bottomnote{Proof size O(log n): verify 1 of 1000 transactions with only 10 hashes.}
\end{frame}

\begin{frame}[t]{Difficulty Target and Leading Zeros}
\vspace{-2mm}
\begin{center}
\includegraphics[width=0.60\textwidth]{figures/difficulty_target/difficulty_target.pdf}
\end{center}
\bottomnote{Finding a valid hash requires enormous computational effort - like finding a needle in a haystack.}
\end{frame}

\begin{frame}[t]{Difficulty Adjustment Mechanism}
\vspace{-2mm}
\textbf{Goal:} Maintain ~10 minute average block time despite changing network hash rate

\vspace{3mm}
\textbf{Algorithm:}
\begin{itemize}
\item Every 2,016 blocks (~2 weeks)
\item Calculate actual time taken vs. expected time (20,160 minutes)
\item Adjust difficulty: $\text{NewDifficulty} = \text{OldDifficulty} \times \frac{\text{Expected Time}}{\text{Actual Time}}$
\item Maximum adjustment: 4x increase or 1/4 decrease per period
\end{itemize}

\vspace{3mm}
\textbf{Example (2023):}
\begin{itemize}
\item Period: Blocks 780,000 - 782,015
\item Expected: 20,160 minutes (14 days)
\item Actual: 18,900 minutes (13.125 days) - blocks came faster
\item Adjustment: Difficulty increased by 6.3\%
\end{itemize}

\vspace{3mm}
\textbf{Historical Trend:}
\begin{itemize}
\item 2009: Difficulty = 1 (CPU mining)
\item 2024: Difficulty = 73 trillion (ASIC mining)
\item $10^{13}$ times harder than genesis block
\end{itemize}
\bottomnote{Key concepts from this slide inform practical applications in finance.}
\end{frame}

\begin{frame}[t]{Hash Rate: Network Computing Power}
\vspace{-2mm}
\begin{center}
\includegraphics[width=0.60\textwidth]{figures/hashrate_evolution/hashrate_evolution.pdf}
\end{center}
\bottomnote{Network hashrate has grown exponentially: from CPUs to industrial ASIC farms.}
\end{frame}

\begin{frame}[t]{Block Propagation and Orphan Blocks}
\vspace{-2mm}
\textbf{Problem:} Network latency - two miners find valid blocks simultaneously

\vspace{3mm}
\textbf{Scenario:}
\begin{enumerate}
\item Miner A finds Block 700,000 at 12:00:00
\item Miner B finds different Block 700,000 at 12:00:01
\item Both blocks are valid and propagate through network
\item Network temporarily splits (some nodes see A, others see B)
\item Miners start working on both competing chains
\item Miner C finds Block 700,001 building on Block A
\item Longest chain rule: Block A chain wins
\item Block B becomes ``orphan block'' - discarded
\item Miner B loses 6.25 BTC reward
\end{enumerate}

\vspace{3mm}
\textbf{Statistics:}
\begin{itemize}
\item Orphan rate: ~0.5\% of blocks (rare but happens)
\item Average propagation time: 1-2 seconds globally
\item Why 10 minutes?: Minimize orphans while maintaining decentralization
\end{itemize}
\bottomnote{Key concepts from this slide inform practical applications in finance.}
\end{frame}

\begin{frame}[t]{Blockchain Data Structures Summary}
\vspace{-2mm}
\begin{table}[h]
\centering
\footnotesize
\begin{tabular}{p{3cm}p{5cm}p{5cm}}
\toprule
\textbf{Structure} & \textbf{Purpose} & \textbf{Properties} \\
\midrule
Hash Function & Fingerprint data & Deterministic, one-way, collision-resistant, avalanche effect \\
Hash Pointer & Link + verify & Points to data + contains hash for integrity \\
Merkle Tree & Efficient verification & $O(\log n)$ proof size, enables SPV \\
Block Header & Minimal representation & 80 bytes contains all block metadata \\
Blockchain & Tamper-evident ledger & Chain of hash pointers, immutable \\
Difficulty Target & Mining puzzle & Adjusts to maintain 10-min block time \\
\bottomrule
\end{tabular}
\end{table}

\vspace{3mm}
\textbf{Key Insight:} All blockchain security stems from cryptographic hash functions - if SHA-256 breaks, Bitcoin breaks
\end{frame}

\begin{frame}[t]{Security Analysis: What Could Go Wrong?}
\vspace{-2mm}
\begin{columns}[T]
\column{0.48\textwidth}
\textbf{Hash Function Attacks:}
\begin{itemize}
\item \textcolor{mlred}{Pre-image:} Find input from hash
\item \textcolor{mlred}{Collision:} Find two inputs with same hash
\item \textcolor{mlred}{Length extension:} Exploit hash construction
\end{itemize}

\vspace{3mm}
\textbf{SHA-256 Status:}
\begin{itemize}
\item \textcolor{mlgreen}{Pre-image:} $2^{256}$ security (safe)
\item \textcolor{mlgreen}{Collision:} $2^{128}$ security (safe)
\item \textcolor{mlgreen}{Length extension:} Not applicable to Bitcoin (double hashing)
\end{itemize}

\column{0.48\textwidth}
\textbf{Quantum Computing Threat:}
\begin{itemize}
\item Grover's algorithm: $\sqrt{2^{256}} = 2^{128}$ speedup
\item Still computationally infeasible
\item Bigger threat: Public key crypto (covered next lesson)
\end{itemize}

\vspace{3mm}
\textbf{Mitigation Strategies:}
\begin{itemize}
\item SHA-3 ready as backup
\item Post-quantum hash functions (SPHINCS+)
\item Bitcoin can hard fork if needed
\end{itemize}
\end{columns}

\vspace{3mm}
\centering
\textbf{Current Assessment:} No practical threat to SHA-256 in foreseeable future
\end{frame}

\begin{frame}[t]{Practical Exercise: Computing Block Hash}
\vspace{-2mm}
\textbf{Simplified Example:} Calculate hash for mini-block

\vspace{3mm}
\textbf{Block Header (simplified):}
\begin{itemize}
\item Previous Hash: \texttt{00000000000000000001}
\item Merkle Root: \texttt{abcdef123456}
\item Timestamp: \texttt{1702000000}
\item Nonce: \texttt{?}
\end{itemize}

\vspace{3mm}
\textbf{Task:} Find nonce such that hash starts with 4 zeros (\texttt{0000...})

\vspace{3mm}
\textbf{Brute Force Approach:}
\begin{enumerate}
\item Concatenate header: \texttt{00000000000000000001|abcdef123456|1702000000|0}
\item Compute SHA-256: \texttt{7a3b2c1d...} (no leading zeros)
\item Increment nonce: Try \texttt{1}, then \texttt{2}, then \texttt{3}...
\item Keep trying until hash starts with \texttt{0000}
\item Expected attempts: $2^{16} = 65,536$
\end{enumerate}

\vspace{3mm}
\textbf{Solution:} Nonce \texttt{45,892} produces \texttt{0000f7a3b2c1d...}
\bottomnote{Cryptographic primitives provide the security foundation for blockchain systems.}
\end{frame}

\begin{frame}[t]{Block Finality: How Many Confirmations?}
\vspace{-2mm}
\begin{center}
\includegraphics[width=0.60\textwidth]{figures/confirmation_depth/confirmation_depth.pdf}
\end{center}
\bottomnote{More confirmations = harder to reverse. 6 confirmations is the gold standard for Bitcoin.}
\end{frame}

\begin{frame}[t]{Next Lesson Preview}
\vspace{-2mm}
\begin{center}
\Large\textbf{Lesson 15: Public Key Cryptography and Signatures}
\end{center}

\vspace{5mm}
\textbf{What We'll Cover:}
\begin{itemize}
\item Symmetric vs asymmetric encryption
\item Elliptic Curve Cryptography (ECC) fundamentals
\item Public/private key pairs in Bitcoin
\item Digital signatures (ECDSA)
\item Address generation and wallet structure
\item Security best practices
\end{itemize}

\vspace{5mm}
\textbf{Prepare:}
\begin{itemize}
\item Review basic modular arithmetic ($a \bmod n$)
\item Understand concept of mathematical groups
\item Install Bitcoin wallet for hands-on key generation
\end{itemize}
\end{frame}

\begin{frame}[t]{Summary: Key Takeaways}
\vspace{-2mm}
\begin{enumerate}
\item \textbf{Block Structure:} 80-byte header + variable transaction body
\item \textbf{SHA-256:} One-way, collision-resistant hash with $2^{256}$ output space
\item \textbf{Avalanche Effect:} Tiny input change completely alters hash
\item \textbf{Hash Pointers:} Enable tamper-evident data structures
\item \textbf{Merkle Trees:} Efficient verification with $O(\log n)$ proof size
\item \textbf{Immutability:} Changing old block requires recalculating all subsequent blocks
\item \textbf{Difficulty:} Adjusts every 2016 blocks to maintain 10-min average
\item \textbf{Finality:} 6 confirmations (~1 hour) considered irreversible
\end{enumerate}

\vspace{3mm}
\centering
\textit{``Blockchain security is reducible to hash function security - SHA-256 is the foundation.''}
\end{frame}

\end{document}
