\documentclass[8pt,aspectratio=169]{beamer}
\usetheme{Madrid}
\usepackage{graphicx,booktabs,adjustbox,multicol,amsmath,amssymb}
\definecolor{mlblue}{RGB}{0,102,204}
\definecolor{mlpurple}{RGB}{51,51,178}
\definecolor{mllavender}{RGB}{173,173,224}
\definecolor{mllavender2}{RGB}{193,193,232}
\definecolor{mllavender3}{RGB}{204,204,235}
\definecolor{mllavender4}{RGB}{214,214,239}
\definecolor{mlorange}{RGB}{255,127,14}
\definecolor{mlgreen}{RGB}{44,160,44}
\definecolor{mlred}{RGB}{214,39,40}
\setbeamercolor{palette primary}{bg=mllavender3,fg=mlpurple}
\setbeamercolor{palette secondary}{bg=mllavender2,fg=mlpurple}
\setbeamercolor{palette tertiary}{bg=mllavender,fg=white}
\setbeamercolor{structure}{fg=mlpurple}
\setbeamercolor{frametitle}{fg=mlpurple,bg=mllavender3}
\setbeamertemplate{navigation symbols}{}
\setbeamertemplate{itemize items}[circle]
\setbeamersize{text margin left=5mm,text margin right=5mm}

% Bottom note command for key takeaways
\newcommand{\bottomnote}[1]{%
\vfill
\vspace{-2mm}
\textcolor{mllavender2}{\rule{\textwidth}{0.4pt}}
\vspace{1mm}
\footnotesize
\textbf{#1}
}
\title{Lesson 15: Public Key Cryptography \& Digital Signatures}
\subtitle{Module 2: Blockchain Fundamentals}
\author{Digital Finance}
\date{}

\begin{document}

\begin{frame}
\titlepage
\end{frame}

\begin{frame}[t]{The Problem: Secure Communication Over Insecure Channels}
\vspace{-2mm}
\begin{columns}[T]
\scriptsize
\column{0.48\textwidth}
\textbf{Challenge:}
\begin{itemize}
\item How do two parties communicate securely without meeting?
\item How do you verify someone's identity online?
\item How do you prove authorship of a message?
\end{itemize}

\vspace{3mm}
\textbf{Traditional Solution:}
\begin{itemize}
\item Symmetric cryptography (shared secret)
\item Problem: Key distribution
\item Requires secure channel to share key
\end{itemize}

\column{0.48\textwidth}
\includegraphics[width=\textwidth]{figures/symmetric_vs_asymmetric/symmetric_vs_asymmetric.pdf}
\end{columns}
\bottomnote{Key concepts from this slide inform practical applications in finance.}
\end{frame}

\begin{frame}[t]{Symmetric vs Asymmetric Cryptography}
\vspace{-2mm}
\begin{center}
\includegraphics[width=0.48\textwidth]{figures/crypto_comparison/crypto_comparison.pdf}
\end{center}
\vspace{-3mm}
\begin{itemize}
\item \textbf{Symmetric}: Same key for encryption and decryption (AES, DES)
\item \textbf{Asymmetric}: Key pair -- public key encrypts, private key decrypts
\item \textbf{Blockchain Use}: Asymmetric for identity, symmetric for bulk data
\end{itemize}
\bottomnote{Comparative analysis helps identify the right tool for specific requirements.}
\end{frame}

\begin{frame}[t]{Public Key Cryptography: Revolutionary Idea}
\vspace{-2mm}
\begin{columns}[T]
\scriptsize
\column{0.48\textwidth}
\textbf{Key Pair Structure:}
\begin{itemize}
\item \textcolor{mlgreen}{Public Key}: Shared openly
\item \textcolor{mlred}{Private Key}: Kept secret
\item Mathematical relationship
\item One-way function (easy to compute, hard to reverse)
\end{itemize}

\vspace{3mm}
\textbf{Properties:}
\begin{itemize}
\item Encrypt with public $\rightarrow$ decrypt with private
\item Sign with private $\rightarrow$ verify with public
\item Cannot derive private from public
\end{itemize}

\column{0.48\textwidth}
\includegraphics[width=\textwidth]{figures/key_pair_generation/key_pair_generation.pdf}
\end{columns}
\bottomnote{Understanding history helps predict future developments in the technology.}
\end{frame}

\begin{frame}[t]{Mathematical Foundation: Trapdoor Functions}
\vspace{-2mm}
\begin{columns}[T]
\scriptsize
\column{0.48\textwidth}
\textbf{One-Way Function:}
\[
y = f(x) \quad \text{(easy)}
\]
\[
x = f^{-1}(y) \quad \text{(hard)}
\]

\vspace{3mm}
\textbf{Examples:}
\begin{itemize}
\item Factoring large primes (RSA)
\item Discrete logarithm (Diffie-Hellman)
\item Elliptic curve discrete log (ECDSA)
\end{itemize}

\column{0.48\textwidth}
\includegraphics[width=\textwidth]{figures/trapdoor_function/trapdoor_function.pdf}

\vspace{2mm}
\small
\textbf{Trapdoor:} Secret information (private key) makes inverse easy
\end{columns}
\bottomnote{Key concepts from this slide inform practical applications in finance.}
\end{frame}

\begin{frame}[t]{RSA Cryptography (Classic Approach)}
\vspace{-2mm}
\textbf{Key Generation:}
\begin{enumerate}
\item Choose two large primes: $p, q$
\item Compute $n = p \times q$ (modulus)
\item Compute $\phi(n) = (p-1)(q-1)$
\item Choose public exponent $e$ (commonly 65537)
\item Compute private exponent $d \equiv e^{-1} \pmod{\phi(n)}$
\end{enumerate}

\vspace{2mm}
\textbf{Encryption/Decryption:}
\[
\text{Ciphertext: } c = m^e \bmod n \quad |\quad \text{Plaintext: } m = c^d \bmod n
\]

\vspace{2mm}
\textbf{Example:} $p=61, q=53, n=3233, e=17, d=2753$
\begin{itemize}
\item Message $m=123$: $c = 123^{17} \bmod 3233 = 855$
\item Decrypt: $m = 855^{2753} \bmod 3233 = 123$
\end{itemize}
\bottomnote{Cryptographic primitives provide the security foundation for blockchain systems.}
\end{frame}

\begin{frame}[t]{Elliptic Curve Cryptography (ECC)}
\vspace{-2mm}
\begin{columns}[T]
\scriptsize
\column{0.48\textwidth}
\textbf{Why ECC?}
\begin{itemize}
\item Smaller key sizes (256-bit ECC $\approx$ 3072-bit RSA)
\item Faster computations
\item Lower bandwidth
\item Standard in Bitcoin/Ethereum
\end{itemize}

\vspace{3mm}
\textbf{Curve Equation:}
\[
y^2 = x^3 + ax + b
\]

\vspace{2mm}
\textbf{Bitcoin uses:} secp256k1
\[
y^2 = x^3 + 7
\]

\column{0.48\textwidth}
\includegraphics[width=\textwidth]{figures/elliptic_curve/elliptic_curve.pdf}
\end{columns}
\bottomnote{Cryptographic primitives provide the security foundation for blockchain systems.}
\end{frame}

\begin{frame}[t]{ECC Point Addition: The Core Operation}
\vspace{-2mm}
\begin{center}
\includegraphics[width=0.48\textwidth]{figures/ecc_point_addition/ecc_point_addition.pdf}
\end{center}
\vspace{-3mm}
\textbf{Operations:}
\begin{itemize}
\item \textbf{Point Addition}: $P + Q = R$ (draw line through $P$ and $Q$, reflect third intersection)
\item \textbf{Point Doubling}: $P + P = 2P$ (tangent line at $P$)
\item \textbf{Scalar Multiplication}: $nP = P + P + \ldots + P$ (n times)
\end{itemize}
\bottomnote{Key concepts from this slide inform practical applications in finance.}
\end{frame}

\begin{frame}[t]{ECC Security: Discrete Logarithm Problem}
\vspace{-2mm}
\begin{columns}[T]
\scriptsize
\column{0.48\textwidth}
\textbf{Easy Problem:}
\begin{itemize}
\item Given point $G$ and scalar $k$
\item Compute $P = k \cdot G$
\item Fast using double-and-add
\end{itemize}

\vspace{3mm}
\textbf{Hard Problem (ECDLP):}
\begin{itemize}
\item Given points $G$ and $P$
\item Find scalar $k$ such that $P = k \cdot G$
\item No efficient algorithm known
\item This is the \textcolor{mlred}{private key}
\end{itemize}

\column{0.48\textwidth}
\includegraphics[width=\textwidth]{figures/ecdlp_visualization/ecdlp_visualization.pdf}

\vspace{2mm}
\small
\textbf{Security:} Best attack takes $\mathcal{O}(\sqrt{n})$ operations for $n$-bit key
\end{columns}
\bottomnote{Security analysis identifies vulnerabilities and helps design robust systems.}
\end{frame}

\begin{frame}[t]{Digital Signatures: Proving Authorship}
\vspace{-2mm}
\begin{columns}[T]
\scriptsize
\column{0.48\textwidth}
\textbf{Purpose:}
\begin{itemize}
\item Prove message was created by you
\item Ensure message wasn't altered
\item Non-repudiation (can't deny)
\end{itemize}

\vspace{3mm}
\textbf{Process:}
\begin{enumerate}
\item Hash the message
\item Encrypt hash with private key
\item Attach signature to message
\item Verify: Decrypt with public key, compare hash
\end{enumerate}

\column{0.48\textwidth}
\includegraphics[width=\textwidth]{figures/digital_signature_flow/digital_signature_flow.pdf}
\end{columns}
\bottomnote{Key concepts from this slide inform practical applications in finance.}
\end{frame}

\begin{frame}[t]{ECDSA: Elliptic Curve Digital Signature Algorithm}
\vspace{-2mm}
\textbf{Signature Generation:}
\begin{enumerate}
\item Message $m$, private key $d$, public key $Q = d \cdot G$
\item Hash message: $z = \text{hash}(m)$
\item Choose random $k$, compute $R = k \cdot G = (x_R, y_R)$
\item Compute $r = x_R \bmod n$
\item Compute $s = k^{-1}(z + rd) \bmod n$
\item Signature: $(r, s)$
\end{enumerate}

\vspace{2mm}
\textbf{Signature Verification:}
\begin{enumerate}
\item Compute $w = s^{-1} \bmod n$
\item Compute $u_1 = zw \bmod n$, $u_2 = rw \bmod n$
\item Compute $P = u_1 \cdot G + u_2 \cdot Q$
\item Valid if $x_P \equiv r \pmod{n}$
\end{enumerate}
\bottomnote{Key concepts from this slide inform practical applications in finance.}
\end{frame}

\begin{frame}[t]{ECDSA Visualization}
\vspace{-2mm}
\begin{center}
\includegraphics[width=0.48\textwidth]{figures/ecdsa_process/ecdsa_process.pdf}
\end{center}
\vspace{-3mm}
\textbf{Key Properties:}
\begin{itemize}
\item Signature size: 64 bytes (256-bit $r$ + 256-bit $s$)
\item Cannot forge without private key
\item Each signature requires unique random $k$ (reuse breaks security!)
\end{itemize}
\bottomnote{Key concepts from this slide inform practical applications in finance.}
\end{frame}

\begin{frame}[t]{Cryptocurrency Wallets: Key Management}
\vspace{-2mm}
\begin{columns}[T]
\scriptsize
\column{0.48\textwidth}
\textbf{Wallet Components:}
\begin{itemize}
\item Private key (spend authority)
\item Public key (derived from private)
\item Address (hash of public key)
\end{itemize}

\vspace{3mm}
\textbf{Key Derivation:}
\[
\text{Private Key} \xrightarrow{\text{ECC}} \text{Public Key} \xrightarrow{\text{Hash}} \text{Address}
\]

\vspace{3mm}
\textbf{One-way:} Cannot derive private from address

\column{0.48\textwidth}
\includegraphics[width=\textwidth]{figures/wallet_key_hierarchy/wallet_key_hierarchy.pdf}
\end{columns}
\bottomnote{Cryptographic primitives provide the security foundation for blockchain systems.}
\end{frame}

\begin{frame}[t]{Bitcoin Address Generation}
\vspace{-2mm}
\begin{center}
\includegraphics[width=0.48\textwidth]{figures/bitcoin_address_generation/bitcoin_address_generation.pdf}
\end{center}
\vspace{-3mm}
\textbf{Steps:}
\begin{enumerate}
\item Generate 256-bit private key (random number)
\item Compute public key: $\text{PubKey} = \text{PrivKey} \times G$ (secp256k1)
\item Hash public key: SHA256, then RIPEMD160
\item Add version byte, compute checksum
\item Base58 encode $\rightarrow$ Address (e.g., 1A1zP1eP5QGefi2DMPTfTL5SLmv7DivfNa)
\end{enumerate}
\bottomnote{Bitcoin remains the largest cryptocurrency by market cap and network security.}
\end{frame}

\begin{frame}[t]{Hierarchical Deterministic (HD) Wallets}
\vspace{-2mm}
\begin{columns}[T]
\scriptsize
\column{0.48\textwidth}
\textbf{Problem:}
\begin{itemize}
\item Managing multiple random keys
\item Backup complexity
\item Privacy (address reuse)
\end{itemize}

\vspace{3mm}
\textbf{Solution (BIP-32/44):}
\begin{itemize}
\item Master seed (12/24 words)
\item Derive infinite keys deterministically
\item Hierarchical tree structure
\item One backup for all keys
\end{itemize}

\column{0.48\textwidth}
\includegraphics[width=\textwidth]{figures/hd_wallet_tree/hd_wallet_tree.pdf}
\end{columns}
\bottomnote{Key concepts from this slide inform practical applications in finance.}
\end{frame}

\begin{frame}[t]{Mnemonic Seed Phrases (BIP-39)}
\vspace{-2mm}
\begin{center}
\includegraphics[width=0.48\textwidth]{figures/mnemonic_generation/mnemonic_generation.pdf}
\end{center}
\vspace{-3mm}
\textbf{12-Word Example:}
\begin{center}
\small\texttt{witch collapse practice feed shame open despair creek road again ice least}
\end{center}

\textbf{Properties:} 128-bit entropy = 12 words, 256-bit = 24 words; 2048-word BIP-39 dictionary; checksum ensures validity
\bottomnote{BIP-39 mnemonic phrases enable human-readable backup of cryptographic keys.}
\end{frame}

\begin{frame}[t]{Wallet Types: Hot vs Cold}
\vspace{-2mm}
\begin{center}
\includegraphics[width=0.48\textwidth]{figures/wallet_types_comparison/wallet_types_comparison.pdf}
\end{center}
\vspace{-3mm}
\begin{columns}[T]
\scriptsize
\column{0.48\textwidth}
\textbf{Hot Wallets:}
\begin{itemize}
\item Connected to internet
\item Software/mobile wallets
\item Convenient but vulnerable
\end{itemize}

\column{0.48\textwidth}
\textbf{Cold Wallets:}
\begin{itemize}
\item Offline storage
\item Hardware wallets, paper wallets
\item Secure but less convenient
\end{itemize}
\end{columns}
\bottomnote{Comparative analysis helps identify the right tool for specific requirements.}
\end{frame}

\begin{frame}[t]{Security Best Practices}
\vspace{-2mm}
\textbf{Never Share:}
\begin{itemize}
\item Private keys
\item Seed phrases
\item Wallet files without encryption
\end{itemize}

\vspace{3mm}
\textbf{Recommendations:}
\begin{itemize}
\item Use hardware wallets for large amounts (Ledger, Trezor)
\item Backup seed phrase offline (metal, paper in safe)
\item Multi-signature for institutional custody
\item Never store keys on cloud services
\item Verify addresses carefully (malware can swap addresses)
\end{itemize}

\vspace{3mm}
\textbf{Warning:} \textcolor{mlred}{Lost private key = Lost funds permanently}
\bottomnote{Security analysis identifies vulnerabilities and helps design robust systems.}
\end{frame}

\begin{frame}[t]{Real-World Implications}
\vspace{-2mm}
\begin{columns}[T]
\scriptsize
\column{0.48\textwidth}
\textbf{Benefits:}
\begin{itemize}
\item Self-custody (be your own bank)
\item No intermediaries
\item Censorship resistance
\item Programmable ownership
\end{itemize}

\vspace{3mm}
\textbf{Challenges:}
\begin{itemize}
\item User error (lost keys)
\item No password reset
\item Irreversible transactions
\item Phishing attacks
\end{itemize}

\column{0.48\textwidth}
\includegraphics[width=\textwidth]{figures/key_security_threats/key_security_threats.pdf}
\end{columns}
\bottomnote{Key concepts from this slide inform practical applications in finance.}
\end{frame}

\begin{frame}[t]{Summary}
\vspace{-2mm}
\begin{itemize}
\item \textbf{Public Key Cryptography}: Two keys (public/private), trapdoor functions
\item \textbf{ECC}: Efficient, smaller keys, basis for Bitcoin/Ethereum signatures
\item \textbf{ECDSA}: Digital signatures prove transaction authorship
\item \textbf{Wallets}: Manage private keys, addresses derived from public keys
\item \textbf{HD Wallets}: One seed generates infinite keys (BIP-32/39/44)
\item \textbf{Security}: Private key = ownership, loss is permanent
\end{itemize}

\vspace{4mm}
\textbf{Next Lesson:} Proof of Work -- how cryptography secures the blockchain
\end{frame}

\end{document}
