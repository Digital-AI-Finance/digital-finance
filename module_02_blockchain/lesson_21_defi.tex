\documentclass[8pt,aspectratio=169]{beamer}
\usetheme{Madrid}
\usepackage{graphicx,booktabs,adjustbox,multicol,amsmath,amssymb}
\definecolor{mlblue}{RGB}{0,102,204}
\definecolor{mlpurple}{RGB}{51,51,178}
\definecolor{mllavender}{RGB}{173,173,224}
\definecolor{mllavender2}{RGB}{193,193,232}
\definecolor{mllavender3}{RGB}{204,204,235}
\definecolor{mllavender4}{RGB}{214,214,239}
\definecolor{mlorange}{RGB}{255,127,14}
\definecolor{mlgreen}{RGB}{44,160,44}
\definecolor{mlred}{RGB}{214,39,40}
\setbeamercolor{palette primary}{bg=mllavender3,fg=mlpurple}
\setbeamercolor{palette secondary}{bg=mllavender2,fg=mlpurple}
\setbeamercolor{palette tertiary}{bg=mllavender,fg=white}
\setbeamercolor{structure}{fg=mlpurple}
\setbeamercolor{frametitle}{fg=mlpurple,bg=mllavender3}
\setbeamertemplate{navigation symbols}{}
\setbeamertemplate{itemize items}[circle]
\setbeamersize{text margin left=5mm,text margin right=5mm}

\title{Lesson 21: DeFi Fundamentals}
\subtitle{Module 2: Blockchain Fundamentals}
\author{Digital Finance}
\date{}

\begin{document}

\begin{frame}
\titlepage
\end{frame}

\begin{frame}{DeFi: Decentralized Finance Revolution}
\begin{columns}[T]
\scriptsize
\column{0.48\textwidth}
\textbf{Traditional Finance (TradFi):}
\begin{itemize}
\item Banks, brokers, exchanges
\item Intermediaries control access
\item Centralized custody
\item Limited hours, geographic restrictions
\item KYC/AML required
\end{itemize}

\vspace{3mm}
\textbf{DeFi:}
\begin{itemize}
\item Smart contracts replace intermediaries
\item Permissionless access (anyone with wallet)
\item Self-custody
\item 24/7 global access
\item Pseudonymous (no KYC)
\end{itemize}

\column{0.48\textwidth}
\includegraphics[width=\textwidth]{charts/lesson_21/tradfi_vs_defi.pdf}
\end{columns}
\end{frame}

\begin{frame}{DeFi Primitives}
\begin{center}
\includegraphics[width=0.85\textwidth]{charts/lesson_21/defi_stack.pdf}
\end{center}
\vspace{-3mm}
\textbf{Core Building Blocks:}
\begin{itemize}
\item \textbf{Asset Layer:} Tokens (ERC-20, stablecoins)
\item \textbf{Protocol Layer:} Lending, DEXs, derivatives
\item \textbf{Application Layer:} Aggregators, wallets, dashboards
\item \textbf{Composability:} ``Money Legos'' -- protocols interact seamlessly
\end{itemize}
\end{frame}

\begin{frame}{Decentralized Exchanges (DEXs): The Problem}
\begin{columns}[T]
\scriptsize
\column{0.48\textwidth}
\textbf{Centralized Exchanges (CEXs):}
\begin{itemize}
\item Order book model
\item Custodial (exchange holds funds)
\item Counterparty risk (FTX collapse)
\item KYC requirements
\item Single point of failure
\end{itemize}

\vspace{3mm}
\textbf{Challenges for DEX:}
\begin{itemize}
\item On-chain order book too expensive
\item Liquidity fragmentation
\item Constant price updates
\end{itemize}

\column{0.48\textwidth}
\includegraphics[width=\textwidth]{charts/lesson_21/orderbook_gas_costs.pdf}
\end{columns}
\end{frame}

\begin{frame}{AMM: Automated Market Maker}
\begin{center}
\includegraphics[width=0.85\textwidth]{charts/lesson_21/amm_concept.pdf}
\end{center}
\vspace{-3mm}
\textbf{Key Idea:}
\begin{itemize}
\item Liquidity pools replace order books
\item Algorithmic pricing via bonding curve
\item Anyone can provide liquidity (LP)
\item Passive market making
\end{itemize}
\end{frame}

\begin{frame}{Constant Product Formula: $x \times y = k$}
\textbf{Uniswap V2 Model:}
\[
x \times y = k
\]
where $x$ = reserve of token A, $y$ = reserve of token B, $k$ = constant

\vspace{3mm}
\textbf{Example:} ETH/USDC pool with 100 ETH and 200,000 USDC
\[
k = 100 \times 200{,}000 = 20{,}000{,}000
\]

\vspace{3mm}
\textbf{Trade:} Buy 10 ETH
\begin{itemize}
\item New ETH reserve: $x' = 100 - 10 = 90$
\item New USDC reserve: $y' = k / x' = 20{,}000{,}000 / 90 = 222{,}222$
\item USDC paid: $222{,}222 - 200{,}000 = 22{,}222$ (effective price $\sim$\$2,222/ETH)
\end{itemize}
\end{frame}

\begin{frame}{AMM Bonding Curve Visualization}
\begin{center}
\includegraphics[width=0.85\textwidth]{charts/lesson_21/bonding_curve.pdf}
\end{center}
\vspace{-3mm}
\textbf{Properties:}
\begin{itemize}
\item Price = slope of curve at current point: $P = y/x$
\item Larger trades move price more (slippage)
\item Infinite liquidity (asymptotic, but expensive for large trades)
\item No order book needed
\end{itemize}
\end{frame}

\begin{frame}{Price Impact and Slippage}
\begin{columns}[T]
\scriptsize
\column{0.48\textwidth}
\textbf{Price Impact:}
\begin{itemize}
\item How much your trade moves the price
\item Larger trade $\rightarrow$ worse price
\item Function of trade size relative to pool depth
\end{itemize}

\vspace{3mm}
\textbf{Slippage:}
\begin{itemize}
\item Difference between expected and executed price
\item Set slippage tolerance (e.g., 0.5\%)
\item Trade reverts if exceeded
\end{itemize}

\column{0.48\textwidth}
\includegraphics[width=\textwidth]{charts/lesson_21/slippage_curve.pdf}
\end{columns}

\vspace{3mm}
\textbf{Formula:} Price impact $\approx \Delta x / (x + \Delta x)$
\end{frame}

\begin{frame}{Liquidity Providers: Earning Fees}
\begin{columns}[T]
\scriptsize
\column{0.48\textwidth}
\textbf{How it Works:}
\begin{enumerate}
\item Deposit equal value of both tokens
\item Receive LP tokens (claim on pool share)
\item Earn 0.3\% of all trades (Uniswap V2)
\item Withdraw anytime (burn LP tokens)
\end{enumerate}

\vspace{3mm}
\textbf{Example:}
\begin{itemize}
\item Pool: 1,000 ETH + 2M USDC
\item You deposit: 10 ETH + 20K USDC (1\%)
\item Daily volume: 500K USDC
\item Your daily fees: $500{,}000 \times 0.003 \times 0.01 = \$15$
\end{itemize}

\column{0.48\textwidth}
\includegraphics[width=\textwidth]{charts/lesson_21/liquidity_provision.pdf}
\end{columns}
\end{frame}

\begin{frame}{Impermanent Loss: The Hidden Cost}
\begin{center}
\includegraphics[width=0.85\textwidth]{charts/lesson_21/impermanent_loss.pdf}
\end{center}
\vspace{-3mm}
\textbf{Definition:} Loss compared to holding tokens vs providing liquidity

\vspace{2mm}
\textbf{Example:}
\begin{itemize}
\item Deposit 1 ETH (\$2000) + 2000 USDC when ETH = \$2000
\item ETH rises to \$3000
\item Pool rebalances: 0.816 ETH + 2449 USDC = \$4449 total
\item Holding: 1 ETH + 2000 USDC = \$5000
\item Impermanent loss: \$551 (11\%)
\end{itemize}
\end{frame}

\begin{frame}{Impermanent Loss Formula}
\textbf{Exact Formula:}
\[
\text{IL} = \frac{2\sqrt{r}}{1 + r} - 1
\]
where $r$ = price ratio change (final price / initial price)

\vspace{3mm}
\begin{center}
\begin{tabular}{@{}p{4cm}p{4cm}p{4cm}@{}}
\toprule
\textbf{Price Change} & \textbf{Ratio ($r$)} & \textbf{Impermanent Loss} \\
\midrule
+25\% & 1.25 & -0.6\% \\
+50\% & 1.5 & -2.0\% \\
+100\% (2x) & 2.0 & -5.7\% \\
+400\% (5x) & 5.0 & -25.5\% \\
-50\% & 0.5 & -5.7\% \\
\bottomrule
\end{tabular}
\end{center}

\vspace{2mm}
\textbf{Mitigation:} Fees earned over time can offset IL (especially in high-volume pairs)
\end{frame}

\begin{frame}{Uniswap V3: Concentrated Liquidity}
\begin{columns}[T]
\scriptsize
\column{0.48\textwidth}
\textbf{V2 Limitation:}
\begin{itemize}
\item Liquidity spread across entire price range
\item Capital inefficient
\item Most liquidity never used
\end{itemize}

\vspace{3mm}
\textbf{V3 Innovation:}
\begin{itemize}
\item Concentrated liquidity in price ranges
\item LPs choose custom ranges
\item Up to 4000x capital efficiency
\item Active management required
\end{itemize}

\column{0.48\textwidth}
\includegraphics[width=\textwidth]{charts/lesson_21/uniswap_v3_ranges.pdf}
\end{columns}
\end{frame}

\begin{frame}{Lending Protocols: Aave and Compound}
\begin{center}
\includegraphics[width=0.85\textwidth]{charts/lesson_21/lending_protocol_architecture.pdf}
\end{center}
\vspace{-3mm}
\textbf{Mechanism:}
\begin{itemize}
\item \textbf{Lenders:} Deposit assets, earn interest (aTokens/cTokens)
\item \textbf{Borrowers:} Post collateral, borrow assets, pay interest
\item \textbf{Interest Rates:} Algorithmically determined by utilization
\item \textbf{Liquidation:} If collateral drops below threshold, liquidators repay loan, seize collateral at discount
\end{itemize}
\end{frame}

\begin{frame}{Over-Collateralization Requirement}
\begin{columns}[T]
\scriptsize
\column{0.48\textwidth}
\textbf{Why Over-Collateralize?}
\begin{itemize}
\item No credit checks (permissionless)
\item Price volatility protection
\item Liquidation buffer
\end{itemize}

\vspace{3mm}
\textbf{Example (Aave):}
\begin{itemize}
\item Deposit: 10 ETH (\$20K)
\item Max LTV: 80\%
\item Borrow: 16K USDC
\item Liquidation threshold: 85\%
\end{itemize}

\column{0.48\textwidth}
\includegraphics[width=\textwidth]{charts/lesson_21/collateral_ratio.pdf}
\end{columns}

\vspace{3mm}
\textbf{Risk:} If ETH drops $>$15\%, position liquidated
\end{frame}

\begin{frame}{Interest Rate Model}
\begin{center}
\includegraphics[width=0.85\textwidth]{charts/lesson_21/interest_rate_curve.pdf}
\end{center}
\vspace{-3mm}
\textbf{Utilization Rate:}
\[
U = \frac{\text{Total Borrowed}}{\text{Total Supplied}}
\]

\textbf{Interest Rates:}
\begin{itemize}
\item Low utilization: Low rates (encourage borrowing)
\item High utilization: High rates (incentivize deposits, discourage borrowing)
\item Kinked model: Sharp increase near optimal utilization (e.g., 80\%)
\end{itemize}
\end{frame}

\begin{frame}{Flash Loans: Zero-Collateral Instant Loans}
\begin{columns}[T]
\scriptsize
\column{0.48\textwidth}
\textbf{Concept:}
\begin{itemize}
\item Borrow any amount
\item No collateral required
\item Must repay within same transaction
\item If not repaid, entire transaction reverts
\end{itemize}

\vspace{3mm}
\textbf{Use Cases:}
\begin{itemize}
\item Arbitrage (exploit price differences)
\item Collateral swaps
\item Liquidations
\item Refinancing
\end{itemize}

\column{0.48\textwidth}
\includegraphics[width=\textwidth]{charts/lesson_21/flash_loan_flow.pdf}
\end{columns}
\end{frame}

\begin{frame}{Flash Loan Attack Example}
\textbf{Scenario: Exploit Price Oracle Manipulation (2020, bZx attack)}
\begin{enumerate}
\item Flash loan 10,000 ETH from dYdX
\item Swap 5,000 ETH for WBTC on Uniswap (moves price)
\item Use manipulated WBTC price to borrow over-collateralized assets on bZx
\item Repay flash loan, keep profit
\item Exploit: \$350K profit
\end{enumerate}

\vspace{3mm}
\textbf{Defense:}
\begin{itemize}
\item Time-weighted average price (TWAP) oracles
\item Multiple oracle sources (Chainlink)
\item Borrow caps
\end{itemize}

\vspace{3mm}
\textbf{Note:} Flash loans are not inherently bad, but enable rapid exploitation of vulnerabilities
\end{frame}

\begin{frame}{Yield Farming: Chasing Returns}
\begin{columns}[T]
\scriptsize
\column{0.48\textwidth}
\textbf{Strategy:}
\begin{itemize}
\item Provide liquidity or lend assets
\item Earn fees + protocol token rewards
\item Compound yields (reinvest)
\item Move capital to highest APY
\end{itemize}

\vspace{3mm}
\textbf{Yield Sources:}
\begin{itemize}
\item Trading fees (0.3\% on Uniswap)
\item Borrow interest (Aave, Compound)
\item Token incentives (governance tokens)
\item Staking rewards
\end{itemize}

\column{0.48\textwidth}
\includegraphics[width=\textwidth]{charts/lesson_21/yield_farming_stack.pdf}
\end{columns}
\end{frame}

\begin{frame}{DeFi Summer 2020: Liquidity Mining Explosion}
\begin{center}
\includegraphics[width=0.9\textwidth]{charts/lesson_21/defi_tvl_history.pdf}
\end{center}
\vspace{-3mm}
\textbf{Catalyst:} Compound launches COMP token distribution (June 2020)

\vspace{2mm}
\textbf{Result:}
\begin{itemize}
\item Total Value Locked (TVL): \$1B $\rightarrow$ \$100B+ (2021 peak)
\item APYs: 100--1000\%+ (unsustainable, token inflation)
\item Copycat protocols, ``DeFi Degen'' culture
\end{itemize}
\end{frame}

\begin{frame}{Risks in DeFi}
\begin{itemize}
\item \textbf{Smart Contract Risk:} Bugs, exploits (e.g., The DAO, bZx)
\item \textbf{Impermanent Loss:} Price divergence reduces LP returns
\item \textbf{Oracle Manipulation:} Flash loan attacks on price feeds
\item \textbf{Liquidation Risk:} Volatile collateral leads to forced sales
\item \textbf{Rug Pulls:} Developers drain liquidity (unaudited projects)
\item \textbf{Regulatory Risk:} Securities classification, AML/KYC future requirements
\item \textbf{Composability Risk:} Cascading failures (one protocol exploited affects others)
\end{itemize}

\vspace{3mm}
\textbf{Mitigation:} Use audited protocols, diversify, understand risks, start small
\end{frame}

\begin{frame}{DeFi vs CeFi: Trade-offs}
\begin{center}
\begin{tabular}{@{}p{3.5cm}p{5cm}p{5cm}@{}}
\toprule
\textbf{Aspect} & \textbf{DeFi} & \textbf{CeFi} \\
\midrule
Custody & Self-custody (your keys) & Custodial (exchange holds) \\
\midrule
Access & Permissionless (anyone) & KYC/AML required \\
\midrule
Transparency & Open-source, on-chain & Opaque (trust exchange) \\
\midrule
Execution & Slower (block time), higher fees & Instant, low fees \\
\midrule
Risk & Smart contract risk & Counterparty risk (FTX) \\
\midrule
Liquidity & Fragmented across DEXs & Concentrated on CEX \\
\bottomrule
\end{tabular}
\end{center}
\end{frame}

\begin{frame}{Summary}
\begin{itemize}
\item \textbf{DeFi:} Permissionless financial services via smart contracts
\item \textbf{AMMs:} Constant product formula ($x \times y = k$), liquidity pools replace order books
\item \textbf{Impermanent Loss:} LPs lose vs holding when prices diverge
\item \textbf{Lending:} Over-collateralized loans, algorithmic interest rates, liquidations
\item \textbf{Flash Loans:} Uncollateralized loans repaid in same transaction
\item \textbf{Risks:} Smart contract bugs, IL, oracle manipulation, liquidations
\end{itemize}

\vspace{4mm}
\textbf{Next Lesson:} Stablecoins and Terra/Luna collapse case study
\end{frame}

\end{document}
