\documentclass[8pt,aspectratio=169]{beamer}
\usetheme{Madrid}
\usepackage{graphicx,booktabs,adjustbox,multicol,amsmath,amssymb}
\definecolor{mlblue}{RGB}{0,102,204}
\definecolor{mlpurple}{RGB}{51,51,178}
\definecolor{mllavender}{RGB}{173,173,224}
\definecolor{mllavender2}{RGB}{193,193,232}
\definecolor{mllavender3}{RGB}{204,204,235}
\definecolor{mllavender4}{RGB}{214,214,239}
\definecolor{mlorange}{RGB}{255,127,14}
\definecolor{mlgreen}{RGB}{44,160,44}
\definecolor{mlred}{RGB}{214,39,40}
\setbeamercolor{palette primary}{bg=mllavender3,fg=mlpurple}
\setbeamercolor{palette secondary}{bg=mllavender2,fg=mlpurple}
\setbeamercolor{palette tertiary}{bg=mllavender,fg=white}
\setbeamercolor{structure}{fg=mlpurple}
\setbeamercolor{frametitle}{fg=mlpurple,bg=mllavender3}
\setbeamertemplate{navigation symbols}{}
\setbeamertemplate{itemize items}[circle]
\setbeamersize{text margin left=5mm,text margin right=5mm}

% Bottom note command for key takeaways
\newcommand{\bottomnote}[1]{%
\vfill
\vspace{-2mm}
\textcolor{mllavender2}{\rule{\textwidth}{0.4pt}}
\vspace{1mm}
\footnotesize
\textbf{#1}
}
\title{Lesson 20: Tokens -- ERC-20 and NFTs}
\subtitle{Module 2: Blockchain Fundamentals}
\author{Digital Finance}
\date{}

\begin{document}

\begin{frame}
\titlepage
\end{frame}

\begin{frame}[t]{What is a Token?}
\vspace{-2mm}
\begin{columns}[T]
\scriptsize
\column{0.48\textwidth}
\textbf{Definition:}
\begin{itemize}
\item Digital asset on blockchain
\item Smart contract manages ownership
\item Not native currency (e.g., not ETH)
\item Programmable properties
\end{itemize}

\vspace{3mm}
\textbf{Types:}
\begin{itemize}
\item \textbf{Fungible:} Interchangeable (e.g., USDC, UNI)
\item \textbf{Non-Fungible:} Unique (e.g., NFTs, real estate)
\item \textbf{Semi-Fungible:} Hybrid (e.g., gaming items)
\end{itemize}

\column{0.48\textwidth}
\includegraphics[width=\textwidth]{figures/token_types/token_types.pdf}
\end{columns}
\bottomnote{Clear definitions are essential for understanding complex technical concepts.}
\end{frame}

\begin{frame}[t]{ERC-20: Fungible Token Standard}
\vspace{-2mm}
\begin{center}
\includegraphics[width=0.48\textwidth]{figures/erc20_overview/erc20_overview.pdf}
\end{center}
\vspace{-3mm}
\textbf{Standard Functions:}
\begin{itemize}
\item \texttt{totalSupply()}: Returns total token supply
\item \texttt{balanceOf(address)}: Returns balance of account
\item \texttt{transfer(to, amount)}: Send tokens
\item \texttt{approve(spender, amount)}: Allow third party to spend
\item \texttt{transferFrom(from, to, amount)}: Third-party transfer (after approval)
\end{itemize}
\bottomnote{Tokens represent digital assets and enable new business models on blockchain.}
\end{frame}

\begin{frame}[t]{ERC-20 Implementation Example}
\vspace{-2mm}
\begin{center}
\includegraphics[width=0.55\textwidth]{figures/erc20_code_structure/erc20_code_structure.pdf}
\end{center}
\vspace{-3mm}
\textbf{Key Components:}
\begin{itemize}
\item \textbf{State Variables:} \texttt{balances}, \texttt{allowances}, \texttt{totalSupply}
\item \textbf{Events:} \texttt{Transfer(from, to, amount)}, \texttt{Approval(owner, spender, amount)}
\item \textbf{Metadata:} Name, symbol, decimals (usually 18)
\end{itemize}
\bottomnote{Case studies provide concrete evidence of technology impact and adoption patterns.}
\end{frame}

\begin{frame}[t]{ERC-20 Transfer Flow}
\vspace{-2mm}
\begin{center}
\includegraphics[width=0.48\textwidth]{figures/erc20_transfer_flow/erc20_transfer_flow.pdf}
\end{center}
\vspace{-3mm}
\textbf{Direct Transfer:}
\begin{enumerate}
\item User calls \texttt{transfer(recipient, 100)}
\item Contract checks balance
\item Updates \texttt{balances[sender] -= 100}, \texttt{balances[recipient] += 100}
\item Emits \texttt{Transfer} event
\end{enumerate}
\bottomnote{Understanding the process flow is key to identifying optimization opportunities.}
\end{frame}

\begin{frame}[t]{Approval and TransferFrom Pattern}
\vspace{-2mm}
\begin{columns}[T]
\scriptsize
\column{0.48\textwidth}
\textbf{Use Case:}
\begin{itemize}
\item Allow smart contract to spend your tokens
\item Required for DeFi (DEXs, lending)
\item Two-step process
\end{itemize}

\vspace{3mm}
\textbf{Steps:}
\begin{enumerate}
\item User approves DEX: \texttt{approve(DEX, 1000)}
\item DEX calls: \texttt{transferFrom(user, pool, 500)}
\item Check allowance, transfer tokens
\end{enumerate}

\column{0.48\textwidth}
\includegraphics[width=\textwidth]{figures/approve_transferfrom/approve_transferfrom.pdf}
\end{columns}

\vspace{3mm}
\textbf{Security Note:} Approve 0 before changing allowance to prevent race conditions
\bottomnote{Key concepts from this slide inform practical applications in finance.}
\end{frame}

\begin{frame}[t]{Popular ERC-20 Tokens (2024)}
\vspace{-2mm}
\begin{center}
\begin{tabular}{@{}p{3cm}p{2.5cm}p{3cm}p{5cm}@{}}
\toprule
\textbf{Token} & \textbf{Symbol} & \textbf{Market Cap} & \textbf{Use Case} \\
\midrule
Tether & USDT & \$100B+ & Stablecoin (pegged to USD) \\
\midrule
USD Coin & USDC & \$30B+ & Regulated stablecoin \\
\midrule
Uniswap & UNI & \$5B+ & DEX governance token \\
\midrule
Chainlink & LINK & \$8B+ & Oracle network payments \\
\midrule
Wrapped Bitcoin & WBTC & \$10B+ & Bitcoin on Ethereum \\
\bottomrule
\end{tabular}
\end{center}

\vspace{3mm}
\textbf{Total ERC-20 Tokens:} $>$500,000 deployed on Ethereum
\bottomnote{Tokens represent digital assets and enable new business models on blockchain.}
\end{frame}

\begin{frame}[t]{Token Utility: Why Create Tokens?}
\vspace{-2mm}
\begin{columns}[T]
\scriptsize
\column{0.48\textwidth}
\textbf{Governance:}
\begin{itemize}
\item DAO voting rights (e.g., UNI, COMP)
\item Protocol parameter changes
\item Treasury allocation
\end{itemize}

\vspace{3mm}
\textbf{Access:}
\begin{itemize}
\item Platform access (e.g., Filecoin storage)
\item Fee discounts (e.g., BNB on Binance)
\item Staking for rewards
\end{itemize}

\column{0.48\textwidth}
\textbf{Incentives:}
\begin{itemize}
\item Liquidity mining rewards
\item Early adopter benefits
\item Network effects
\end{itemize}

\vspace{3mm}
\textbf{Speculation:}
\begin{itemize}
\item Investment asset
\item Price appreciation
\item Trading on exchanges
\end{itemize}
\end{columns}
\bottomnote{Tokens represent digital assets and enable new business models on blockchain.}
\end{frame}

\begin{frame}[t]{NFTs: Non-Fungible Tokens}
\vspace{-2mm}
\begin{center}
\includegraphics[width=0.48\textwidth]{figures/nft_concept/nft_concept.pdf}
\end{center}
\vspace{-3mm}
\textbf{Key Properties:}
\begin{itemize}
\item \textbf{Unique:} Each token has distinct identifier (tokenId)
\item \textbf{Indivisible:} Cannot split (unlike ERC-20)
\item \textbf{Provably Scarce:} Limited supply enforced by contract
\item \textbf{Metadata:} Points to off-chain data (image, video, properties)
\end{itemize}
\bottomnote{Tokens represent digital assets and enable new business models on blockchain.}
\end{frame}

\begin{frame}[t]{ERC-721: NFT Standard}
\vspace{-2mm}
\textbf{Required Functions:}
\begin{itemize}
\item \texttt{balanceOf(owner)}: Number of NFTs owned
\item \texttt{ownerOf(tokenId)}: Who owns specific token
\item \texttt{transferFrom(from, to, tokenId)}: Transfer NFT
\item \texttt{approve(to, tokenId)}: Approve transfer
\item \texttt{safeTransferFrom(...)}: Safe transfer (checks recipient can receive)
\end{itemize}

\vspace{3mm}
\textbf{Optional Metadata Extension:}
\begin{itemize}
\item \texttt{tokenURI(tokenId)}: Returns URL to metadata JSON
\item Metadata typically stored on IPFS or centralized server
\end{itemize}

\vspace{3mm}
\textbf{Example:} CryptoPunks, Bored Ape Yacht Club, Azuki
\bottomnote{Key concepts from this slide inform practical applications in finance.}
\end{frame}

\begin{frame}[t]{NFT Metadata Structure}
\vspace{-2mm}
\begin{columns}[T]
\scriptsize
\column{0.48\textwidth}
\textbf{On-Chain (Contract):}
\begin{itemize}
\item Token ID
\item Owner address
\item Approval state
\item Pointer to metadata URI
\end{itemize}

\vspace{3mm}
\textbf{Off-Chain (IPFS/Server):}
\begin{itemize}
\item Name, description
\item Image/video URL
\item Attributes (rarity traits)
\item Properties
\end{itemize}

\column{0.48\textwidth}
\includegraphics[width=\textwidth]{figures/nft_metadata_flow/nft_metadata_flow.pdf}
\end{columns}

\vspace{3mm}
\textbf{Centralization Risk:} If server hosting image goes down, NFT becomes broken link
\bottomnote{Quality data is the foundation for effective machine learning models.}
\end{frame}

\begin{frame}[t]{NFT Use Cases}
\vspace{-2mm}
\begin{columns}[T]
\scriptsize
\column{0.48\textwidth}
\textbf{Digital Art:}
\begin{itemize}
\item Provable ownership
\item Royalties on resales (via marketplaces)
\item Scarcity enforcement
\end{itemize}

\vspace{3mm}
\textbf{Gaming:}
\begin{itemize}
\item In-game items (weapons, skins)
\item Land ownership (Decentraland)
\item Cross-game interoperability
\end{itemize}

\column{0.48\textwidth}
\textbf{Identity/Credentials:}
\begin{itemize}
\item Digital diplomas
\item Membership badges
\item Soulbound tokens (non-transferable)
\end{itemize}

\vspace{3mm}
\textbf{Real-World Assets:}
\begin{itemize}
\item Real estate deeds
\item Luxury goods authentication
\item Tickets/event access
\end{itemize}
\end{columns}
\bottomnote{Real-world applications demonstrate the practical value of blockchain technology.}
\end{frame}

\begin{frame}[t]{NFT Mania: 2021 Boom and Bust}
\vspace{-2mm}
\begin{center}
\includegraphics[width=0.48\textwidth]{figures/nft_market_volume/nft_market_volume.pdf}
\end{center}
\vspace{-3mm}
\textbf{Peak (Aug 2021 -- Jan 2022):}
\begin{itemize}
\item Monthly volume: \$5B+ on OpenSea
\item Bored Ape floor price: 150 ETH (\$600K)
\item Celebrity endorsements, mainstream media
\end{itemize}

\textbf{Crash (2022--2024):} 90\%+ decline in volume, floor prices collapsed
\bottomnote{Key concepts from this slide inform practical applications in finance.}
\end{frame}

\begin{frame}[t]{Criticisms of NFTs}
\vspace{-2mm}
\begin{itemize}
\item \textbf{Ownership Confusion:} You own token, not copyright or image itself
\item \textbf{Environmental (Pre-Merge):} High energy usage on PoW Ethereum
\item \textbf{Speculation Bubble:} Most projects have no utility, pure speculation
\item \textbf{Centralization:} Metadata often on centralized servers, not fully decentralized
\item \textbf{Money Laundering:} Wash trading, inflated sale prices
\item \textbf{IP Issues:} Plagiarism, unauthorized minting of others' art
\item \textbf{Market Manipulation:} Pump-and-dump schemes, insider trading
\end{itemize}

\vspace{3mm}
\textbf{Counterargument:} Technology has legitimate use cases beyond JPEGs (credentials, gaming, ticketing)
\bottomnote{Key concepts from this slide inform practical applications in finance.}
\end{frame}

\begin{frame}[t]{Tokenomics: Designing Token Economics}
\vspace{-2mm}
\begin{columns}[T]
\scriptsize
\column{0.48\textwidth}
\textbf{Supply Mechanics:}
\begin{itemize}
\item \textbf{Fixed Supply:} Bitcoin (21M cap)
\item \textbf{Inflationary:} Continuous issuance (older Ethereum)
\item \textbf{Deflationary:} Burn mechanisms (EIP-1559)
\item \textbf{Elastic:} Rebasing (Ampleforth)
\end{itemize}

\vspace{3mm}
\textbf{Distribution:}
\begin{itemize}
\item Team allocation (with vesting)
\item Investors/VCs (lockup periods)
\item Community rewards (airdrops, mining)
\item Treasury for governance
\end{itemize}

\column{0.48\textwidth}
\includegraphics[width=\textwidth]{figures/tokenomics_distribution/tokenomics_distribution.pdf}
\end{columns}
\bottomnote{Tokens represent digital assets and enable new business models on blockchain.}
\end{frame}

\begin{frame}[t]{Vesting and Lock-Ups}
\vspace{-2mm}
\begin{center}
\includegraphics[width=0.48\textwidth]{figures/vesting_schedule/vesting_schedule.pdf}
\end{center}
\vspace{-3mm}
\textbf{Purpose:} Prevent early investors/team from immediate sell-off (dump)

\vspace{2mm}
\textbf{Typical Schedule:}
\begin{itemize}
\item \textbf{Cliff:} 6--12 months (no tokens released)
\item \textbf{Linear Vesting:} Monthly releases over 2--4 years
\item \textbf{Example:} 1-year cliff, then 25\% per year for 4 years
\end{itemize}
\bottomnote{Key concepts from this slide inform practical applications in finance.}
\end{frame}

\begin{frame}[t]{Airdrops: Free Token Distribution}
\vspace{-2mm}
\begin{columns}[T]
\scriptsize
\column{0.48\textwidth}
\textbf{Reasons:}
\begin{itemize}
\item Reward early users
\item Decentralize governance
\item Marketing/awareness
\item Avoid securities regulations (gift, not sale)
\end{itemize}

\vspace{3mm}
\textbf{Famous Examples:}
\begin{itemize}
\item Uniswap: 400 UNI to all users (\$1200+)
\item Ethereum Name Service: Retroactive airdrop
\item Arbitrum: Governance token to users
\end{itemize}

\column{0.48\textwidth}
\includegraphics[width=\textwidth]{figures/airdrop_strategy/airdrop_strategy.pdf}
\end{columns}
\bottomnote{Tokens represent digital assets and enable new business models on blockchain.}
\end{frame}

\begin{frame}[t]{Token Launch Models}
\vspace{-2mm}
\begin{center}
\begin{tabular}{@{}p{3cm}p{5cm}p{5cm}@{}}
\toprule
\textbf{Model} & \textbf{Mechanism} & \textbf{Pros/Cons} \\
\midrule
ICO (2017--18) & Public sale at fixed price & Simple, but many scams, regulatory issues \\
\midrule
IEO (2019) & Sale on exchange (e.g., Binance Launchpad) & Vetted, but centralized \\
\midrule
Fair Launch (2020) & No pre-sale, everyone equal & Community-driven, but vulnerable to bots \\
\midrule
LBP (2021) & Liquidity Bootstrapping Pool (declining price) & Price discovery, less FOMO \\
\bottomrule
\end{tabular}
\end{center}

\vspace{3mm}
\textbf{Trend:} Moving away from ICOs toward fairer, community-first models
\bottomnote{Tokens represent digital assets and enable new business models on blockchain.}
\end{frame}

\begin{frame}[t]{ERC-1155: Multi-Token Standard}
\vspace{-2mm}
\begin{columns}[T]
\scriptsize
\column{0.48\textwidth}
\textbf{Innovation:}
\begin{itemize}
\item Single contract manages multiple token types
\item Fungible, non-fungible, semi-fungible
\item Batch operations (gas efficient)
\end{itemize}

\vspace{3mm}
\textbf{Use Case: Gaming}
\begin{itemize}
\item Gold (fungible)
\item Sword \#123 (non-fungible)
\item Health potion (semi-fungible, limited)
\item All in one contract
\end{itemize}

\column{0.48\textwidth}
\includegraphics[width=\textwidth]{figures/erc1155_structure/erc1155_structure.pdf}
\end{columns}
\bottomnote{Tokens represent digital assets and enable new business models on blockchain.}
\end{frame}

\begin{frame}[t]{Token Security: Common Vulnerabilities}
\vspace{-2mm}
\begin{itemize}
\item \textbf{Reentrancy:} External calls before state updates (use checks-effects-interactions)
\item \textbf{Integer Overflow/Underflow:} Fixed in Solidity 0.8+ (automatic checks)
\item \textbf{Approval Race Condition:} Approve 0 before changing allowance
\item \textbf{Unchecked Return Values:} ERC-20 transfer may silently fail
\item \textbf{Front-Running:} Miners/bots see pending transactions, exploit
\item \textbf{Centralization:} Owner has mint/burn/pause powers (rug pull risk)
\end{itemize}

\vspace{3mm}
\textbf{Best Practice:} Use OpenZeppelin audited contracts, multiple audits, time-locks on admin functions
\bottomnote{Tokens represent digital assets and enable new business models on blockchain.}
\end{frame}

\begin{frame}[t]{Summary}
\vspace{-2mm}
\begin{itemize}
\item \textbf{ERC-20:} Fungible token standard, balances, transfer, approve/transferFrom
\item \textbf{ERC-721 (NFTs):} Unique tokens, digital art/collectibles/gaming
\item \textbf{NFT Metadata:} On-chain ownership, off-chain images (IPFS/centralized)
\item \textbf{Tokenomics:} Supply, distribution, vesting, airdrops
\item \textbf{ERC-1155:} Multi-token standard, efficient for gaming
\item \textbf{Security:} Reentrancy, overflows, centralization risks
\end{itemize}

\vspace{4mm}
\textbf{Next Lesson:} DeFi Fundamentals -- AMMs, liquidity pools, lending protocols
\end{frame}

\end{document}
