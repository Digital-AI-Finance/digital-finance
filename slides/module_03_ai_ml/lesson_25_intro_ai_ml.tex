\documentclass[8pt,aspectratio=169]{beamer}
\usetheme{Madrid}
\usepackage{graphicx,booktabs,adjustbox,multicol,amsmath,amssymb}
\definecolor{mlblue}{RGB}{0,102,204}
\definecolor{mlpurple}{RGB}{51,51,178}
\definecolor{mllavender}{RGB}{173,173,224}
\definecolor{mllavender2}{RGB}{193,193,232}
\definecolor{mllavender3}{RGB}{204,204,235}
\definecolor{mllavender4}{RGB}{214,214,239}
\definecolor{mlorange}{RGB}{255,127,14}
\definecolor{mlgreen}{RGB}{44,160,44}
\definecolor{mlred}{RGB}{214,39,40}
\setbeamercolor{palette primary}{bg=mllavender3,fg=mlpurple}
\setbeamercolor{palette secondary}{bg=mllavender2,fg=mlpurple}
\setbeamercolor{palette tertiary}{bg=mllavender,fg=white}
\setbeamercolor{structure}{fg=mlpurple}
\setbeamercolor{frametitle}{fg=mlpurple,bg=mllavender3}
\setbeamertemplate{navigation symbols}{}
\setbeamertemplate{itemize items}[circle]
\setbeamersize{text margin left=5mm,text margin right=5mm}

\title{Digital Finance 3: Technology in Finance}
\subtitle{Lesson 25: Introduction to AI/ML in Finance}
\author{FHGR}
\date{\today}

\begin{document}

\begin{frame}
\titlepage
\end{frame}

\begin{frame}{Learning Objectives}
By the end of this lesson, you will be able to:
\begin{itemize}
\item Define artificial intelligence, machine learning, and deep learning
\item Understand the hierarchy and relationships between AI concepts
\item Identify key applications of AI/ML in finance
\item Distinguish between realistic capabilities and overhype
\item Recognize the evolution of AI in financial services
\end{itemize}
\end{frame}

\begin{frame}{What is Artificial Intelligence?}
\begin{columns}[T]
\column{0.48\textwidth}
\textbf{Broad Definition:}
\begin{itemize}
\item Simulation of human intelligence by machines
\item Systems that can reason, learn, and act autonomously
\item Originated in 1956 at Dartmouth Conference
\item Multiple ``AI winters'' and resurgences
\end{itemize}

\column{0.48\textwidth}
\textbf{Key Characteristics:}
\begin{itemize}
\item Perception (vision, speech)
\item Reasoning (logic, planning)
\item Learning (from data, experience)
\item Natural language processing
\item Problem-solving
\end{itemize}
\end{columns}
\end{frame}

\begin{frame}{The AI Hierarchy: From Broad to Narrow}
\begin{columns}[T]
\column{0.48\textwidth}
\textbf{Three Nested Concepts:}
\begin{enumerate}
\item \textcolor{mlpurple}{\textbf{Artificial Intelligence}} (broadest)\\
Any technique enabling computers to mimic human intelligence

\item \textcolor{mlblue}{\textbf{Machine Learning}} (subset)\\
Systems that learn from data without explicit programming

\item \textcolor{mlorange}{\textbf{Deep Learning}} (subset of ML)\\
Neural networks with multiple layers
\end{enumerate}

\column{0.48\textwidth}
\textbf{Analogy:}
\begin{itemize}
\item AI = Transportation
\item ML = Automobiles
\item Deep Learning = Electric cars
\end{itemize}

\vspace{1em}
\textbf{Modern Reality:}\\
Most ``AI in finance'' today is actually machine learning, specifically supervised learning algorithms.
\end{columns}
\end{frame}

\begin{frame}{Machine Learning: The Core Idea}
\begin{columns}[T]
\column{0.48\textwidth}
\textbf{Traditional Programming:}
\begin{itemize}
\item Humans write explicit rules
\item Input + Rules = Output
\item Example: ``IF credit score < 600 THEN reject''
\item Hard to scale for complex patterns
\end{itemize}

\column{0.48\textwidth}
\textbf{Machine Learning:}
\begin{itemize}
\item Algorithm learns rules from data
\item Input + Output = Rules (learned)
\item Example: Discover credit patterns from 1M loan histories
\item Scales to high-dimensional problems
\end{itemize}
\end{columns}

\vspace{1em}
\textbf{Key Insight:}\\
ML excels when:
\begin{itemize}
\item Patterns are complex and non-obvious
\item Large amounts of data are available
\item Rules are difficult to articulate explicitly
\end{itemize}
\end{frame}

\begin{frame}{Three Types of Machine Learning}
\begin{columns}[T]
\column{0.31\textwidth}
\textbf{Supervised Learning}
\begin{itemize}
\item Labeled data (X, Y)
\item Learn mapping: X → Y
\item Examples:
  \begin{itemize}
  \item Credit scoring
  \item Fraud detection
  \item Stock prediction
  \end{itemize}
\item \textcolor{mlgreen}{Most common in finance}
\end{itemize}

\column{0.31\textwidth}
\textbf{Unsupervised Learning}
\begin{itemize}
\item Unlabeled data (X only)
\item Find hidden structure
\item Examples:
  \begin{itemize}
  \item Customer segmentation
  \item Anomaly detection
  \item Portfolio clustering
  \end{itemize}
\item Discovery-oriented
\end{itemize}

\column{0.31\textwidth}
\textbf{Reinforcement Learning}
\begin{itemize}
\item Agent learns via trial/error
\item Maximize cumulative reward
\item Examples:
  \begin{itemize}
  \item Algorithmic trading
  \item Dynamic hedging
  \item Game playing (chess, Go)
  \end{itemize}
\item Still research-heavy
\end{itemize}
\end{columns}
\end{frame}

\begin{frame}{Deep Learning: Neural Networks at Scale}
\begin{columns}[T]
\column{0.48\textwidth}
\textbf{What Makes It ``Deep''?}
\begin{itemize}
\item Multiple hidden layers (10s to 100s)
\item Automatic feature learning
\item Inspired by brain neurons (loosely)
\item Requires massive data and compute
\end{itemize}

\vspace{0.5em}
\textbf{Breakthroughs (2012-present):}
\begin{itemize}
\item Image recognition (ImageNet 2012)
\item Speech recognition (Google, Apple)
\item Language models (GPT, BERT)
\item Game mastery (AlphaGo 2016)
\end{itemize}

\column{0.48\textwidth}
\textbf{Finance Applications:}
\begin{itemize}
\item Document processing (OCR, contracts)
\item Sentiment analysis (news, social media)
\item Time series forecasting (limited success)
\item Alternative data (satellite, text)
\end{itemize}

\vspace{0.5em}
\textbf{Reality Check:}\\
Deep learning excels with:
\begin{itemize}
\item Unstructured data (text, images)
\item Millions of training examples
\item Pattern recognition tasks
\end{itemize}
Not always superior for structured financial data (tabular).
\end{columns}
\end{frame}

\begin{frame}{AI/ML Applications in Finance: Overview}
\begin{columns}[T]
\column{0.48\textwidth}
\textbf{Risk Management:}
\begin{itemize}
\item Credit scoring and underwriting
\item Fraud detection
\item Anti-money laundering (AML)
\item Market risk modeling
\item Stress testing
\end{itemize}

\vspace{0.5em}
\textbf{Trading and Investment:}
\begin{itemize}
\item Algorithmic trading strategies
\item Portfolio optimization
\item Market prediction (limited)
\item Robo-advisors
\item Alternative data analysis
\end{itemize}

\column{0.48\textwidth}
\textbf{Customer Service:}
\begin{itemize}
\item Chatbots and virtual assistants
\item Personalized recommendations
\item Customer segmentation
\item Churn prediction
\end{itemize}

\vspace{0.5em}
\textbf{Operations:}
\begin{itemize}
\item Document processing (OCR, NLP)
\item Regulatory compliance automation
\item Process optimization
\item Cybersecurity threat detection
\end{itemize}
\end{columns}

\vspace{0.5em}
\textbf{Common Thread:} Automation of pattern recognition tasks previously requiring human expertise.
\end{frame}

\begin{frame}{Case Study: Credit Scoring Evolution}
\begin{columns}[T]
\column{0.48\textwidth}
\textbf{Traditional Approach (1960s-2000s):}
\begin{itemize}
\item FICO score (5 factors, fixed weights)
\item Linear scorecards
\item Based on credit bureau data only
\item Transparent, regulated
\item Limited predictive power
\end{itemize}

\vspace{0.5em}
\textbf{Limitations:}
\begin{itemize}
\item Misses non-linear relationships
\item Cannot handle alternative data
\item One-size-fits-all model
\end{itemize}

\column{0.48\textwidth}
\textbf{ML Approach (2010s-present):}
\begin{itemize}
\item Gradient boosting (XGBoost, LightGBM)
\item 100s to 1000s of features
\item Alternative data (mobile, social, payments)
\item Dynamic model updates
\item Higher accuracy (10-30\% improvement)
\end{itemize}

\vspace{0.5em}
\textbf{New Challenges:}
\begin{itemize}
\item Explainability (``black box'')
\item Fairness and bias
\item Regulatory acceptance
\item Data privacy
\end{itemize}
\end{columns}

\vspace{0.5em}
\textbf{Key Lesson:} Technology enables better predictions but introduces new risks and ethical questions.
\end{frame}

\begin{frame}{The Hype Cycle: Expectations vs Reality}
\begin{columns}[T]
\column{0.48\textwidth}
\textbf{Gartner Hype Cycle Phases:}
\begin{enumerate}
\item Innovation Trigger
\item Peak of Inflated Expectations
\item Trough of Disillusionment
\item Slope of Enlightenment
\item Plateau of Productivity
\end{enumerate}

\vspace{0.5em}
\textbf{Where is AI/ML in Finance?}
\begin{itemize}
\item Overall: Slope of Enlightenment
\item Deep Learning: Still some hype
\item Traditional ML: Plateau (established)
\item Generative AI: Peak (2023-2024)
\end{itemize}

\column{0.48\textwidth}
\textbf{Common Misconceptions:}
\begin{itemize}
\item ``AI will replace all analysts'' (No)
\item ``ML always outperforms rules'' (No)
\item ``More data always helps'' (Diminishing returns)
\item ``Black boxes are always better'' (Transparency matters)
\end{itemize}

\vspace{0.5em}
\textbf{Realistic Expectations:}
\begin{itemize}
\item AI augments, not replaces, humans
\item ML excels at narrow, repetitive tasks
\item Domain expertise still critical
\item Hybrid approaches often best
\end{itemize}
\end{columns}
\end{frame}

\begin{frame}{What AI/ML Can and Cannot Do in Finance}
\begin{columns}[T]
\column{0.48\textwidth}
\textbf{Can Do Well:}
\begin{itemize}
\item Pattern recognition (fraud, anomalies)
\item Classification (credit risk, default)
\item Prediction with stable patterns (short-term)
\item Data processing at scale (NLP, OCR)
\item Optimization (portfolio, pricing)
\item Personalization (recommendations)
\end{itemize}

\vspace{0.5em}
\textbf{Success Factors:}
\begin{itemize}
\item Large, high-quality datasets
\item Stable underlying patterns
\item Clear objective function
\item Ability to validate and test
\end{itemize}

\column{0.48\textwidth}
\textbf{Cannot Do (or Struggles):}
\begin{itemize}
\item Predict regime changes (crashes, crises)
\item Explain ``why'' without human input
\item Handle novel situations (out-of-sample)
\item Replace human judgment entirely
\item Guarantee fairness or ethics
\end{itemize}

\vspace{0.5em}
\textbf{Fundamental Limits:}
\begin{itemize}
\item No free lunch (NFL theorem)
\item Efficient Market Hypothesis constraints
\item Overfitting to historical noise
\item Adversarial dynamics (arms race)
\end{itemize}
\end{columns}

\vspace{0.5em}
\textbf{Bottom Line:} AI/ML is a powerful tool, not magic. Success requires proper problem framing, quality data, and realistic expectations.
\end{frame}

\begin{frame}{Historical Timeline: AI in Finance}
\begin{columns}[T]
\column{0.48\textwidth}
\textbf{1980s-1990s: Expert Systems}
\begin{itemize}
\item Rule-based systems (if-then)
\item Limited success, brittle
\item Example: MYCIN for medical diagnosis
\end{itemize}

\vspace{0.3em}
\textbf{2000s: First ML Wave}
\begin{itemize}
\item Support Vector Machines (SVM)
\item Random Forests
\item Credit scoring improvements
\end{itemize}

\vspace{0.3em}
\textbf{2010s: Deep Learning Era}
\begin{itemize}
\item Neural networks for NLP
\item Algorithmic trading explosion
\item Robo-advisors launched
\end{itemize}

\column{0.48\textwidth}
\textbf{2015-2020: Maturation}
\begin{itemize}
\item Gradient boosting dominance (XGBoost)
\item Regulatory frameworks emerge
\item Focus on explainability
\end{itemize}

\vspace{0.3em}
\textbf{2020-present: Generative AI}
\begin{itemize}
\item Large Language Models (GPT-3/4)
\item Financial document analysis
\item Code generation for analysts
\item New regulatory challenges
\end{itemize}

\vspace{0.3em}
\textbf{Future Trends:}
\begin{itemize}
\item Federated learning (privacy)
\item Causal inference integration
\item Hybrid human-AI systems
\end{itemize}
\end{columns}
\end{frame}

\begin{frame}{Industry Adoption: Survey Data}
\begin{columns}[T]
\column{0.48\textwidth}
\textbf{Adoption Rates (2023 surveys):}
\begin{itemize}
\item Large banks: 80-90\% have AI initiatives
\item Asset managers: 60-70\% use ML
\item Fintechs: 90\%+ (core to business)
\item Regional banks: 30-50\% (growing)
\end{itemize}

\vspace{0.5em}
\textbf{Top Use Cases:}
\begin{enumerate}
\item Fraud detection (85\%)
\item Customer service chatbots (70\%)
\item Credit risk modeling (65\%)
\item AML/KYC automation (60\%)
\item Algorithmic trading (50\%)
\end{enumerate}

\column{0.48\textwidth}
\textbf{Barriers to Adoption:}
\begin{itemize}
\item Data quality/availability (65\%)
\item Lack of skilled talent (60\%)
\item Regulatory uncertainty (55\%)
\item Integration with legacy systems (50\%)
\item Explainability requirements (45\%)
\end{itemize}

\vspace{0.5em}
\textbf{Investment Trends:}
\begin{itemize}
\item Global AI in finance market: \$10B (2023)
\item Projected: \$35B by 2030 (CAGR 20\%)
\item Focus shifting from experimentation to production scaling
\end{itemize}
\end{columns}
\end{frame}

\begin{frame}{Key Players and Ecosystem}
\begin{columns}[T]
\column{0.48\textwidth}
\textbf{Large Tech Companies:}
\begin{itemize}
\item Google Cloud (AI Platform, AutoML)
\item Amazon Web Services (SageMaker)
\item Microsoft Azure (ML Studio)
\item IBM (Watson Financial Services)
\end{itemize}

\vspace{0.5em}
\textbf{Specialized Fintechs:}
\begin{itemize}
\item Upstart (AI lending)
\item Kasisto (chatbots)
\item Kensho (analytics, acquired by S\&P)
\item Ayasdi (AML)
\end{itemize}

\column{0.48\textwidth}
\textbf{Traditional Finance + AI:}
\begin{itemize}
\item JPMorgan Chase (COiN, IndexGPT)
\item Goldman Sachs (Marcus, Marquee)
\item BlackRock (Aladdin platform)
\item Capital One (credit models)
\end{itemize}

\vspace{0.5em}
\textbf{Open Source Community:}
\begin{itemize}
\item scikit-learn (ML library)
\item TensorFlow, PyTorch (deep learning)
\item Hugging Face (NLP models)
\item Kaggle (competitions, datasets)
\end{itemize}
\end{columns}

\vspace{0.5em}
\textbf{Trend:} Increasing collaboration between big tech, fintechs, and traditional banks.
\end{frame}

\begin{frame}{Data: The Fuel for AI/ML}
\begin{columns}[T]
\column{0.48\textwidth}
\textbf{Why Data Matters:}
\begin{itemize}
\item ML models are only as good as training data
\item More data often beats better algorithms
\item Quality > Quantity (garbage in, garbage out)
\end{itemize}

\vspace{0.5em}
\textbf{Types of Financial Data:}
\begin{itemize}
\item Structured: Prices, returns, accounting
\item Unstructured: News, reports, social media
\item Alternative: Satellite, mobile, web scraping
\item Real-time: Tick data, order books
\end{itemize}

\column{0.48\textwidth}
\textbf{Data Challenges:}
\begin{itemize}
\item Availability (proprietary, expensive)
\item Quality (errors, missing values)
\item Bias (survivorship, selection)
\item Privacy (GDPR, regulations)
\item Stationarity (patterns change over time)
\end{itemize}

\vspace{0.5em}
\textbf{Best Practices:}
\begin{itemize}
\item Rigorous data cleaning
\item Train/validation/test splits
\item Cross-validation
\item Out-of-sample testing
\item Monitor data drift
\end{itemize}
\end{columns}

\vspace{0.5em}
\textbf{Next Lesson:} Deep dive into financial data types and preparation.
\end{frame}

\begin{frame}{Ethical Considerations}
\begin{columns}[T]
\column{0.48\textwidth}
\textbf{Fairness and Bias:}
\begin{itemize}
\item Models can perpetuate historical discrimination
\item Protected attributes (race, gender, age)
\item Proxy variables (zip code = race)
\item Disparate impact vs. disparate treatment
\end{itemize}

\vspace{0.5em}
\textbf{Transparency:}
\begin{itemize}
\item Right to explanation (GDPR Article 22)
\item Black box models vs. interpretability
\item Trade-off: accuracy vs. explainability
\end{itemize}

\column{0.48\textwidth}
\textbf{Accountability:}
\begin{itemize}
\item Who is responsible for AI decisions?
\item Human-in-the-loop vs. full automation
\item Audit trails and governance
\end{itemize}

\vspace{0.5em}
\textbf{Privacy:}
\begin{itemize}
\item Data minimization principle
\item Consent and purpose limitation
\item Anonymization challenges
\item Model inversion attacks
\end{itemize}
\end{columns}

\vspace{0.5em}
\textbf{Regulatory Response:} EU AI Act, algorithmic accountability laws, model risk management frameworks.
\end{frame}

\begin{frame}{Skills for AI/ML in Finance}
\begin{columns}[T]
\column{0.48\textwidth}
\textbf{Technical Skills:}
\begin{itemize}
\item Programming (Python, R)
\item Statistics and probability
\item Linear algebra and calculus
\item ML algorithms and frameworks
\item Data manipulation (SQL, pandas)
\item Cloud platforms (AWS, Azure, GCP)
\end{itemize}

\vspace{0.5em}
\textbf{Finance Domain Knowledge:}
\begin{itemize}
\item Financial markets and instruments
\item Risk management principles
\item Regulatory environment
\item Business context
\end{itemize}

\column{0.48\textwidth}
\textbf{Soft Skills:}
\begin{itemize}
\item Problem framing
\item Critical thinking (avoiding overfitting)
\item Communication (explaining models)
\item Ethics and responsibility
\item Collaboration (cross-functional teams)
\end{itemize}

\vspace{0.5em}
\textbf{Career Paths:}
\begin{itemize}
\item Quantitative Analyst
\item Data Scientist (Finance)
\item ML Engineer
\item Risk Modeler
\item AI Product Manager
\end{itemize}
\end{columns}

\vspace{0.5em}
\textbf{Key Insight:} Success requires combination of technical skills, domain expertise, and ethical awareness.
\end{frame}

\begin{frame}{Summary and Key Takeaways}
\begin{columns}[T]
\column{0.48\textwidth}
\textbf{Core Concepts:}
\begin{itemize}
\item AI > ML > Deep Learning (hierarchy)
\item ML learns patterns from data
\item Supervised learning dominates finance
\item Deep learning for unstructured data
\end{itemize}

\vspace{0.5em}
\textbf{Finance Applications:}
\begin{itemize}
\item Risk: Credit, fraud, AML
\item Trading: Algorithms, robo-advisors
\item Operations: NLP, automation
\item Customer: Chatbots, personalization
\end{itemize}

\column{0.48\textwidth}
\textbf{Reality Check:}
\begin{itemize}
\item AI augments, not replaces humans
\item Data quality is paramount
\item Hype vs. realistic capabilities
\item Ethical and regulatory challenges
\end{itemize}

\vspace{0.5em}
\textbf{Looking Ahead:}
\begin{itemize}
\item Next 11 lessons: Detailed exploration
\item Hands-on understanding (conceptual)
\item Critical evaluation skills
\item Practical applications
\end{itemize}
\end{columns}
\end{frame}

\begin{frame}{Next Lesson Preview}
\textbf{Lesson 26: Financial Data for AI/ML}

\vspace{0.5em}
Topics to be covered:
\begin{itemize}
\item Structured vs. unstructured data
\item Data sources and vendors
\item Alternative data revolution
\item Data quality and preprocessing
\item GDPR and privacy considerations
\item Feature engineering basics
\end{itemize}

\vspace{1em}
\textbf{Preparation:}
\begin{itemize}
\item Review basic statistics (mean, variance, correlation)
\item Think about data quality issues in your own experience
\item Consider: What makes financial data unique?
\end{itemize}
\end{frame}

\end{document}
