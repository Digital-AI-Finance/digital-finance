\documentclass[8pt,aspectratio=169]{beamer}
\usetheme{Madrid}
\usepackage{graphicx,booktabs,adjustbox,multicol,amsmath,amssymb}
\definecolor{mlblue}{RGB}{0,102,204}
\definecolor{mlpurple}{RGB}{51,51,178}
\definecolor{mllavender}{RGB}{173,173,224}
\definecolor{mllavender2}{RGB}{193,193,232}
\definecolor{mllavender3}{RGB}{204,204,235}
\definecolor{mllavender4}{RGB}{214,214,239}
\definecolor{mlorange}{RGB}{255,127,14}
\definecolor{mlgreen}{RGB}{44,160,44}
\definecolor{mlred}{RGB}{214,39,40}
\setbeamercolor{palette primary}{bg=mllavender3,fg=mlpurple}
\setbeamercolor{palette secondary}{bg=mllavender2,fg=mlpurple}
\setbeamercolor{palette tertiary}{bg=mllavender,fg=white}
\setbeamercolor{structure}{fg=mlpurple}
\setbeamercolor{frametitle}{fg=mlpurple,bg=mllavender3}
\setbeamertemplate{navigation symbols}{}
\setbeamertemplate{itemize items}[circle]
\setbeamersize{text margin left=5mm,text margin right=5mm}

% Bottom note command for key takeaways
\newcommand{\bottomnote}[1]{%
\vfill
\vspace{-2mm}
\textcolor{mllavender2}{\rule{\textwidth}{0.4pt}}
\vspace{1mm}
\footnotesize
\textbf{#1}
}
\title{Digital Finance 3: Technology in Finance}
\subtitle{Lesson 29: Algorithmic Trading Concepts}
\author{FHGR}
\date{\today}

\begin{document}

\begin{frame}
\titlepage
\end{frame}

\begin{frame}{Learning Objectives}
By the end of this lesson, you will be able to:
\begin{itemize}
\item Classify different types of algorithmic trading strategies
\item Design and execute backtesting frameworks
\item Identify and avoid common backtesting pitfalls
\item Understand overfitting in trading models
\item Account for transaction costs and market impact
\item Set realistic performance expectations
\end{itemize}
\end{frame}

\begin{frame}{What is Algorithmic Trading?}
\begin{columns}[T]
\column{0.48\textwidth}
\textbf{Definition:}
\begin{itemize}
\item Automated execution based on rules
\item No human intervention
\item Computer algorithms make decisions
\item Processes data faster than humans
\end{itemize}

\vspace{0.5em}
\textbf{Market Share:}
\begin{itemize}
\item US equities: 70-80\% of volume
\item Futures: 60-70\%
\item FX: 50-60\%
\end{itemize}

\column{0.48\textwidth}
\textbf{Key Advantages:}
\begin{itemize}
\item Speed (microseconds)
\item Consistency (no emotions)
\item Backtesting capability
\item Scalability
\end{itemize}

\vspace{0.5em}
\textbf{Challenges:}
\begin{itemize}
\item Overfitting to historical data
\item Model decay (regime changes)
\item Technology costs
\item Regulatory scrutiny
\end{itemize}
\end{columns}
\bottomnote{Clear definitions are essential for understanding complex technical concepts.}
\end{frame}

\begin{frame}{Types of Strategies}
\begin{columns}[T]
\column{0.31\textwidth}
\textbf{Execution Algorithms:}
\begin{itemize}
\item VWAP, TWAP
\item Minimize market impact
\item Cost minimization
\end{itemize}

\vspace{0.5em}
\textbf{Market Making:}
\begin{itemize}
\item Provide liquidity
\item Profit from spread
\item High-frequency trading
\end{itemize}

\column{0.31\textwidth}
\textbf{Statistical Arbitrage:}
\begin{itemize}
\item Mean reversion
\item Pair trading
\item Market-neutral
\end{itemize}

\vspace{0.5em}
\textbf{Momentum:}
\begin{itemize}
\item Follow trends
\item Breakout strategies
\item Moving averages
\end{itemize}

\column{0.31\textwidth}
\textbf{ML-Based:}
\begin{itemize}
\item Prediction models
\item Alternative data
\item Classification/regression
\end{itemize}

\vspace{0.5em}
\textbf{HFT:}
\begin{itemize}
\item Ultra-short holding
\item Latency arbitrage
\item Co-location required
\end{itemize}
\end{columns}
\bottomnote{Key concepts from this slide inform practical applications in finance.}
\end{frame}

\begin{frame}{Backtesting Framework}
\begin{columns}[T]
\column{0.48\textwidth}
\textbf{Steps:}
\begin{enumerate}
\item Define strategy rules
\item Acquire historical data
\item Simulate trades
\item Calculate returns (net of costs)
\item Evaluate metrics
\item Iterate and refine
\end{enumerate}

\vspace{0.5em}
\textbf{Key Metrics:}
\begin{itemize}
\item Total return
\item Sharpe ratio (risk-adjusted)
\item Maximum drawdown
\item Win rate
\item Profit factor
\end{itemize}

\column{0.48\textwidth}
\textbf{Realistic Targets:}
\begin{itemize}
\item Sharpe > 1.5: Good
\item Sharpe > 2.0: Very good
\item Sharpe > 3.0: Exceptional (or overfitting?)
\end{itemize}

\vspace{0.5em}
\textbf{Common Pitfalls:}
\begin{itemize}
\item Look-ahead bias
\item Survivorship bias
\item Data snooping
\item Ignoring costs
\item Market impact
\item Overfitting
\end{itemize}

\vspace{0.5em}
\textbf{Warning:} Backtest performance usually overstates live performance.
\end{columns}
\bottomnote{Key concepts from this slide inform practical applications in finance.}
\end{frame}

\begin{frame}{Summary}
\textbf{Key Takeaways:}
\begin{itemize}
\item Algorithmic trading dominates modern markets
\item Many strategy types (execution, market making, stat arb, ML)
\item Backtesting essential but has pitfalls
\item Overfitting is the central danger
\item Transaction costs matter (0.2-0.5\% per round-trip)
\item Realistic expectations: Alpha is scarce
\end{itemize}

\vspace{1em}
\textbf{Next Lesson:} Credit Scoring and Risk Models
\end{frame}

\end{document}
