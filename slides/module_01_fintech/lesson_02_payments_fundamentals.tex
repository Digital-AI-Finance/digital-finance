\documentclass[8pt,aspectratio=169]{beamer}
\usetheme{Madrid}
\usepackage{graphicx}
\usepackage{booktabs}
\usepackage{adjustbox}
\usepackage{multicol}
\usepackage{amsmath}
\usepackage{amssymb}
\usepackage{eurosym}

\definecolor{mlblue}{RGB}{0,102,204}
\definecolor{mlpurple}{RGB}{51,51,178}
\definecolor{mllavender}{RGB}{173,173,224}
\definecolor{mllavender2}{RGB}{193,193,232}
\definecolor{mllavender3}{RGB}{204,204,235}
\definecolor{mllavender4}{RGB}{214,214,239}
\definecolor{mlorange}{RGB}{255, 127, 14}
\definecolor{mlgreen}{RGB}{44, 160, 44}
\definecolor{mlred}{RGB}{214, 39, 40}

\setbeamercolor{palette primary}{bg=mllavender3,fg=mlpurple}
\setbeamercolor{palette secondary}{bg=mllavender2,fg=mlpurple}
\setbeamercolor{palette tertiary}{bg=mllavender,fg=white}
\setbeamercolor{structure}{fg=mlpurple}
\setbeamercolor{frametitle}{fg=mlpurple,bg=mllavender3}
\setbeamertemplate{navigation symbols}{}
\setbeamertemplate{itemize items}[circle]
\setbeamersize{text margin left=5mm,text margin right=5mm}

% Bottom note command for key takeaways
\newcommand{\bottomnote}[1]{%
\vfill
\vspace{-2mm}
\textcolor{mllavender2}{\rule{\textwidth}{0.4pt}}
\vspace{1mm}
\footnotesize
\textbf{#1}
}
\title{Lesson 2: Digital Payments I -- Fundamentals}
\subtitle{Module 1: Fintech and Innovation}
\author{Digital Finance Course}
\date{\today}

\begin{document}

\begin{frame}
\titlepage
\end{frame}

% =============================================================================
\section{Payment Fundamentals}
% =============================================================================

\begin{frame}[t]{Payment Fundamentals: Overview}
\begin{columns}[T]
\column{0.48\textwidth}
\textbf{What is a Payment?}
\begin{itemize}
\item Transfer of monetary value from payer to payee
\item Settlement of obligation or exchange
\item Requires: Authorization, clearing, settlement
\item Involves multiple intermediaries
\end{itemize}

\vspace{5mm}
\textbf{Payment Ecosystem Participants:}
\begin{itemize}
\item Payer (consumer/business)
\item Payee (merchant/recipient)
\item Issuing bank (payer's bank)
\item Acquiring bank (merchant's bank)
\item Payment network (Visa, Mastercard)
\item Payment processors (Stripe, Adyen)
\end{itemize}

\column{0.48\textwidth}
\textbf{Key Payment Characteristics:}
\begin{itemize}
\item Speed (real-time vs batch)
\item Cost (fees and FX spread)
\item Security (fraud prevention)
\item Finality (irrevocable vs reversible)
\item Reach (domestic vs cross-border)
\end{itemize}

\vspace{5mm}
\textbf{Payment Types:}
\begin{itemize}
\item Push: Payer initiates (credit transfer)
\item Pull: Payee initiates (direct debit)
\item Card: Network-based (Visa/MC)
\item A2A: Account-to-account (open banking)
\end{itemize}
\end{columns}
\bottomnote{Every payment involves authorization, clearing, and settlement across multiple intermediaries.}
\end{frame}

\begin{frame}[t]{Payment Lifecycle: Four Stages}
\begin{columns}[T]
\column{0.48\textwidth}
\textbf{1. Authorization:}
\begin{itemize}
\item Merchant requests payment approval
\item Issuer checks funds/credit availability
\item Response: Approve or decline
\item Duration: 1-3 seconds
\item Holds funds on card (authorization hold)
\end{itemize}

\vspace{3mm}
\textbf{2. Authentication:}
\begin{itemize}
\item Verify cardholder identity
\item 3D Secure (3DS): Password/OTP
\item Strong Customer Authentication (SCA, PSD2)
\item Biometrics (fingerprint, face ID)
\item Reduces fraud, increases approval rates
\end{itemize}

\column{0.48\textwidth}
\textbf{3. Clearing:}
\begin{itemize}
\item Batch processing of transactions
\item Exchange of transaction details
\item Calculation of net positions
\item Typically occurs daily (end-of-day)
\end{itemize}

\vspace{3mm}
\textbf{4. Settlement:}
\begin{itemize}
\item Actual movement of funds
\item Issuer pays acquirer (via network)
\item Merchant receives funds (minus fees)
\item T+1 to T+3 for card payments
\item Instant for real-time payment rails
\end{itemize}
\end{columns}
\bottomnote{Settlement is when funds actually move---clearing calculates who owes what.}
\end{frame}

\begin{frame}[t]{Card Payment Flow: Step-by-Step}
\begin{columns}[T]
\column{0.48\textwidth}
\textbf{Transaction Steps:}
\begin{enumerate}
\item Customer presents card at merchant POS/online
\item Merchant sends authorization request to acquirer
\item Acquirer routes to card network (Visa/MC)
\item Network routes to issuing bank
\item Issuer checks account and responds (approve/decline)
\item Response travels back through chain
\item Merchant receives approval, completes sale
\item Clearing batch sent at end of day
\item Settlement occurs (funds transferred)
\item Merchant receives payment (T+1 to T+3)
\end{enumerate}

\column{0.48\textwidth}
\textbf{Messaging Standards:}
\begin{itemize}
\item ISO 8583: Card transaction messaging
\item EMV: Chip card standards
\item 3D Secure: Online authentication protocol
\item PCI DSS: Security standards for card data
\end{itemize}

\vspace{3mm}
\textbf{Approval Rate Factors:}
\begin{itemize}
\item Sufficient funds/credit limit
\item Card not blocked/expired
\item Fraud screening passed
\item Correct CVV and address
\item Industry average: 85\%-90\%
\end{itemize}
\end{columns}
\bottomnote{Card authorization takes 1-3 seconds; merchants wait 1-3 days for settlement.}
\end{frame}

% =============================================================================
\section{Card Networks and Economics}
% =============================================================================

\begin{frame}[t]{Card Networks: Visa and Mastercard}
\begin{columns}[T]
\column{0.48\textwidth}
\textbf{Visa Inc:}
\begin{itemize}
\item Founded: 1958 (BankAmericard)
\item Market cap: \$530B (2024)
\item Cards in circulation: 4.2B globally
\item Transaction volume: \$14.8 trillion (2023)
\item Revenue: \$32.7B (FY2023)
\item Net margin: 51\%
\end{itemize}

\vspace{3mm}
\textbf{Mastercard Inc:}
\begin{itemize}
\item Founded: 1966 (Interbank Card)
\item Market cap: \$410B (2024)
\item Cards in circulation: 3.1B globally
\item Transaction volume: \$9.0 trillion (2023)
\item Revenue: \$25.1B (FY2023)
\item Net margin: 46\%
\end{itemize}

\column{0.48\textwidth}
\textbf{Business Model:}
\begin{itemize}
\item Four-party scheme (issuer, acquirer, network, cardholder)
\item Do NOT issue cards or acquire merchants
\item Revenue from transaction fees
\item License brand to financial institutions
\item Operate global payment networks
\end{itemize}

\vspace{3mm}
\textbf{Revenue Streams:}
\begin{itemize}
\item Service fees: 0.13\%-0.15\% per transaction
\item Cross-border assessment fees: 1.0\%-1.1\%
\item Volume-based rebates and incentives
\item Value-added services (fraud, analytics)
\item Licensing and certification
\end{itemize}
\end{columns}
\bottomnote{Visa and Mastercard are technology networks, not card issuers---they license the brand.}
\end{frame}

% Card Network Volumes Chart
\begin{frame}[t]{Card Network Payment Volumes (2023)}
\begin{center}
\includegraphics[width=0.65\textwidth]{figures/card_network_volumes/card_network_volumes.pdf}
\end{center}
\bottomnote{Visa + Mastercard process 92\% of global card volume; four-party model creates network effects.}
\end{frame}

\begin{frame}[t]{Interchange Fees: The Hidden Economy}
\begin{columns}[T]
\column{0.48\textwidth}
\textbf{What is Interchange?}
\begin{itemize}
\item Fee paid by acquirer to issuer
\item Compensation for fraud risk, funding, processing
\item Set by card networks (Visa, Mastercard)
\item Largest component of merchant fees
\item Controversial and heavily regulated
\end{itemize}

\vspace{3mm}
\textbf{Interchange Rate Factors:}
\begin{itemize}
\item Card type (debit vs credit, premium)
\item Merchant category code (MCC)
\item Transaction type (card-present vs online)
\item Region (EU capped, US market-driven)
\item Card acceptance method (chip vs swipe)
\end{itemize}

\column{0.48\textwidth}
\textbf{Typical Interchange Rates (EU):}
\begin{itemize}
\item Consumer debit: 0.2\% (capped by regulation)
\item Consumer credit: 0.3\% (capped by regulation)
\item Commercial cards: Uncapped (1.5\%-2.5\%)
\item Cross-border: Additional 0.4\%-0.8\%
\end{itemize}

\vspace{3mm}
\textbf{Typical Rates (USA, unregulated):}
\begin{itemize}
\item Debit card: 1.0\%-1.5\%
\item Credit card: 1.5\%-3.0\%
\item Premium rewards cards: 2.5\%-3.5\%
\item Example: \$100 purchase = \$2-\$3 interchange
\end{itemize}
\end{columns}
\bottomnote{Interchange is the largest component of merchant fees---funding bank rewards programs.}
\end{frame}

% Interchange Rates Comparison Chart
\begin{frame}[t]{Interchange Fee Rates: EU vs USA}
\begin{center}
\includegraphics[width=0.65\textwidth]{figures/interchange_rates_comparison/interchange_rates_comparison.pdf}
\end{center}
\bottomnote{EU regulation caps consumer interchange at 0.2-0.3\%; US credit cards remain unregulated.}
\end{frame}

\begin{frame}[t]{Merchant Discount Rate (MDR): Total Cost}
\begin{columns}[T]
\column{0.48\textwidth}
\textbf{MDR Components:}
\begin{itemize}
\item Interchange fee (70\%-80\% of MDR)
\item Card network fee (0.13\%-0.15\%)
\item Acquirer markup (0.10\%-0.50\%)
\item Payment processor fee (if applicable)
\end{itemize}

\vspace{3mm}
\textbf{Total MDR Examples (EU):}
\begin{itemize}
\item Debit card: 0.5\%-1.0\%
\item Credit card: 0.8\%-1.5\%
\item American Express: 2.0\%-3.5\% (three-party)
\item Large merchants: Negotiate lower rates
\item Small merchants: Pay higher rates
\end{itemize}

\column{0.48\textwidth}
\textbf{Merchant Cost Analysis:}
\begin{itemize}
\item \euro{}100 credit card purchase (EU)
\item Interchange: \euro{}0.30 (0.3\% cap)
\item Network fee: \euro{}0.14 (0.14\%)
\item Acquirer margin: \euro{}0.25 (0.25\%)
\item Total MDR: \euro{}0.69 (0.69\%)
\item Merchant receives: \euro{}99.31
\end{itemize}

\vspace{3mm}
\textbf{Impact on Business:}
\begin{itemize}
\item Low-margin sectors (supermarkets: 2\%-3\%)
\item Payment fees = significant cost
\item Incentive for cash or A2A payments
\item Volume discounts for large merchants
\end{itemize}
\end{columns}
\bottomnote{Low-margin retailers (supermarkets: 2-3\% margin) see payment fees as significant costs.}
\end{frame}

\begin{frame}[t]{Interchange Fee Regulation}
\begin{columns}[T]
\column{0.48\textwidth}
\textbf{European Union (IFR 2015):}
\begin{itemize}
\item Consumer debit cap: 0.2\%
\item Consumer credit cap: 0.3\%
\item Commercial cards: Exempt
\item Cross-border treated as domestic
\item Saved merchants \euro{}1.2B annually
\end{itemize}

\vspace{3mm}
\textbf{United Kingdom (Post-Brexit):}
\begin{itemize}
\item Maintained EU caps for domestic
\item Cross-border EEA: Caps removed (2021)
\item Interchange tripled on EU cards (1.5\%)
\item PSR reviewing commercial card fees
\end{itemize}

\column{0.48\textwidth}
\textbf{United States:}
\begin{itemize}
\item Durbin Amendment (2010): Debit cap \$0.21 + 0.05\%
\item Applies only to large banks (\$10B+ assets)
\item Credit cards: Unregulated
\item Ongoing litigation (merchant class actions)
\end{itemize}

\vspace{3mm}
\textbf{Australia:}
\begin{itemize}
\item RBA regulation since 2003
\item Weighted average caps
\item Credit: 0.5\%, Debit: 0.2\% (2017)
\item Allowed surcharging (pass cost to consumer)
\end{itemize}
\end{columns}
\bottomnote{EU IFR 2015 saved merchants EUR 1.2B annually; post-Brexit UK saw interchange triple on EU cards.}
\end{frame}

% =============================================================================
\section{Bank Payment Systems}
% =============================================================================

\begin{frame}[t]{ACH: Automated Clearing House}
\begin{columns}[T]
\column{0.48\textwidth}
\textbf{What is ACH?}
\begin{itemize}
\item Electronic batch payment system (US)
\item Operated by Nacha (formerly NACHA)
\item Bank-to-bank transfers
\item Low cost, high volume, batch processing
\item 31B transactions, \$76 trillion (2023)
\end{itemize}

\vspace{3mm}
\textbf{ACH Transaction Types:}
\begin{itemize}
\item Direct deposit (payroll, benefits)
\item Direct debit (bill payments, subscriptions)
\item B2B payments (vendor payments)
\item P2P transfers (Venmo, Cash App backend)
\item Tax payments and refunds
\end{itemize}

\column{0.48\textwidth}
\textbf{ACH Processing:}
\begin{itemize}
\item Batch processing (4 times daily)
\item Standard: T+1 settlement
\item Same-Day ACH: Settlement same day (since 2016)
\item Cost: \$0.20-\$1.50 per transaction
\item Pull (debit) or Push (credit)
\end{itemize}

\vspace{3mm}
\textbf{ACH vs Wire Transfer:}
\begin{itemize}
\item ACH: Batch, low cost, T+1, reversible
\item Wire: Real-time, high cost (\$15-\$30), irrevocable
\item ACH volume: 100x higher than wire
\item Wires for large/urgent payments
\end{itemize}
\end{columns}
\bottomnote{ACH processes 31B transactions/year at \$0.20-\$1.50 each---100x cheaper than wire transfers.}
\end{frame}

\begin{frame}[t]{SEPA: Single Euro Payments Area}
\begin{columns}[T]
\column{0.48\textwidth}
\textbf{What is SEPA?}
\begin{itemize}
\item Harmonized euro payment system
\item Covers 36 countries (EU + EEA + others)
\item Launched: 2008 (credit transfer), 2009 (direct debit)
\item Treats cross-border like domestic
\item 500M+ citizens, 20M+ businesses
\end{itemize}

\vspace{3mm}
\textbf{SEPA Instruments:}
\begin{itemize}
\item SEPA Credit Transfer (SCT): Push payments
\item SEPA Instant Credit Transfer (SCT Inst): Real-time
\item SEPA Direct Debit (SDD): Pull payments
\item SEPA Card Framework (less successful)
\end{itemize}

\column{0.48\textwidth}
\textbf{SEPA Credit Transfer (SCT):}
\begin{itemize}
\item Settlement: T+1 (next business day)
\item No amount limit
\item IBAN and BIC required
\item ISO 20022 XML messaging
\item Cost: Same as domestic
\end{itemize}

\vspace{3mm}
\textbf{SEPA Instant (SCT Inst):}
\begin{itemize}
\item Launched: November 2017
\item Settlement: 10 seconds maximum
\item Limit: \euro{}100,000 per transaction (raised from \euro{}15k in 2024)
\item Available 24/7/365
\item Adoption: 70\%+ of EU banks (2024)
\item Mandatory for banks by 2025 (EU regulation)
\end{itemize}
\end{columns}
\bottomnote{SEPA Instant settles in 10 seconds, 24/7/365---mandatory for all EU banks by 2025.}
\end{frame}

\begin{frame}[t]{SEPA Direct Debit}
\begin{columns}[T]
\column{0.48\textwidth}
\textbf{SDD Core Scheme:}
\begin{itemize}
\item Consumer pull payments
\item Requires signed mandate
\item Pre-notification required (14 days default)
\item Refund rights: 8 weeks (authorized), 13 months (unauthorized)
\item Settlement: T+1 or T+2
\end{itemize}

\vspace{3mm}
\textbf{SDD B2B Scheme:}
\begin{itemize}
\item Business-to-business only
\item No refund right (authorized mandate)
\item Bank verification of mandate
\item Lower fraud risk
\item Separate scheme identifier
\end{itemize}

\column{0.48\textwidth}
\textbf{Mandate Requirements:}
\begin{itemize}
\item Unique Mandate Reference (UMR)
\item Creditor Identifier (CI)
\item Debtor IBAN
\item Debtor signature (physical or electronic)
\item Electronic mandates: PSD2 SCA required
\end{itemize}

\vspace{3mm}
\textbf{Use Cases:}
\begin{itemize}
\item Recurring bills (utilities, telecom)
\item Subscriptions (streaming, gym)
\item Loan repayments
\item Insurance premiums
\item Variable amount collections
\end{itemize}
\end{columns}
\bottomnote{Direct debit mandates allow 8-week refunds (Core) or no refunds (B2B)---know your scheme.}
\end{frame}

% =============================================================================
\section{Payment Processors and Security}
% =============================================================================

\begin{frame}[t]{Payment Processing Value Chain}
\begin{columns}[T]
\column{0.48\textwidth}
\textbf{Frontend Processors:}
\begin{itemize}
\item Gateway providers (Stripe, Adyen, Checkout.com)
\item Authorization routing
\item Fraud screening
\item 3D Secure authentication
\item Merchant dashboard and APIs
\item Revenue: Payment processing fees
\end{itemize}

\vspace{3mm}
\textbf{Backend Processors:}
\begin{itemize}
\item Clearing and settlement
\item Reconciliation services
\item Chargeback handling
\item FIS, Fiserv, TSYS (now Fiserv)
\end{itemize}

\column{0.48\textwidth}
\textbf{Acquirer vs Gateway:}
\begin{itemize}
\item Acquirer: Licensed bank, holds funds, settles
\item Gateway: Technology layer, routes transactions
\item Full-stack processors: Combine both (Adyen, Stripe)
\item Traditional: Separate acquirer + gateway
\end{itemize}

\vspace{3mm}
\textbf{Modern Payment Stack:}
\begin{itemize}
\item Merchant integration: API/SDK
\item Payment orchestration: Routing logic
\item Multiple acquirers: Redundancy, optimization
\item Network tokens: Enhanced security
\item Real-time reporting: Analytics
\end{itemize}
\end{columns}
\bottomnote{Full-stack processors (Adyen, Stripe) combine gateway and acquiring---simplifying integration.}
\end{frame}

\begin{frame}[t]{Payment Service Providers (PSPs)}
\begin{columns}[T]
\column{0.48\textwidth}
\textbf{Leading PSPs:}
\begin{itemize}
\item Stripe: \$1 trillion TPV (2023), 100+ countries
\item Adyen: \euro{}786B TPV (2023), unified platform
\item PayPal/Braintree: \$1.5 trillion TPV (2023)
\item Worldpay (FIS): \$2.2 trillion TPV
\item Square/Block: \$211B TPV (2023)
\end{itemize}

\vspace{3mm}
\textbf{PSP Value Proposition:}
\begin{itemize}
\item Single API for multiple payment methods
\item Global reach with local acquiring
\item Fraud prevention tools
\item PCI compliance management
\item Fast onboarding (hours vs months)
\end{itemize}

\column{0.48\textwidth}
\textbf{Pricing Models:}
\begin{itemize}
\item Blended rate: e.g., 2.9\% + \$0.30 per transaction
\item Interchange plus: IC + 0.5\% + \$0.10
\item Custom enterprise pricing (volume discounts)
\item Monthly subscription fees (optional)
\end{itemize}

\vspace{3mm}
\textbf{Feature Comparison:}
\begin{itemize}
\item Stripe: Developer-first, excellent APIs
\item Adyen: Enterprise focus, single platform
\item PayPal: Consumer brand, checkout conversion
\item Square: SMB focus, integrated POS
\item Checkout.com: High customization, flexible
\end{itemize}
\end{columns}
\bottomnote{Stripe pioneered developer-first APIs---onboarding in hours vs months with traditional acquirers.}
\end{frame}

\begin{frame}[t]{Payment Security: PCI DSS}
\begin{columns}[T]
\column{0.48\textwidth}
\textbf{PCI DSS Overview:}
\begin{itemize}
\item Payment Card Industry Data Security Standard
\item Managed by PCI Security Standards Council
\item Mandatory for all card payment entities
\item Current version: PCI DSS 4.0 (March 2022)
\item Compliance deadline: March 2025
\end{itemize}

\vspace{3mm}
\textbf{12 Requirements (6 Goals):}
\begin{enumerate}
\item Build and maintain secure network
\item Protect cardholder data
\item Maintain vulnerability management
\item Implement strong access controls
\item Monitor and test networks
\item Maintain information security policy
\end{enumerate}

\column{0.48\textwidth}
\textbf{Compliance Levels (Visa):}
\begin{itemize}
\item Level 1: 6M+ transactions/year (annual audit)
\item Level 2: 1M-6M transactions (annual SAQ)
\item Level 3: 20k-1M e-commerce (annual SAQ)
\item Level 4: <20k e-commerce (annual SAQ)
\end{itemize}

\vspace{3mm}
\textbf{Key Technical Controls:}
\begin{itemize}
\item Encryption of cardholder data
\item Tokenization (replace PAN with token)
\item Firewall and network segmentation
\item Vulnerability scanning
\item Penetration testing
\item Log monitoring and SIEM
\end{itemize}
\end{columns}
\bottomnote{PCI DSS 4.0 deadline is March 2025---12 requirements, 6 goals, mandatory for all card processors.}
\end{frame}

\begin{frame}[t]{Payment Fraud: Types and Prevention}
\begin{columns}[T]
\column{0.48\textwidth}
\textbf{Fraud Types:}
\begin{itemize}
\item Card-not-present (CNP): Online fraud
\item Card-present: Counterfeit, stolen cards
\item Account takeover (ATO): Credential theft
\item Friendly fraud: Chargeback abuse
\item Refund fraud: Return scams
\end{itemize}

\vspace{3mm}
\textbf{Fraud Statistics (Global):}
\begin{itemize}
\item Total card fraud: \$32B annually (2023)
\item CNP fraud: 75\% of total fraud
\item Fraud rate: 0.15\%-0.20\% of volume
\item Chargebacks: 0.6\%-1.0\% of transactions
\item Cost to merchants: 3.6x transaction value
\end{itemize}

\column{0.48\textwidth}
\textbf{Prevention Technologies:}
\begin{itemize}
\item 3D Secure 2.0: Frictionless authentication
\item Machine learning fraud scoring
\item Device fingerprinting
\item Behavioral biometrics
\item Address Verification System (AVS)
\item CVV/CVC verification
\end{itemize}

\vspace{3mm}
\textbf{Liability Shift:}
\begin{itemize}
\item EMV chip: Shift to non-compliant party
\item 3DS: Shift to issuer (if authenticated)
\item Merchants incentivized to adopt
\item Reduces fraud and chargebacks
\end{itemize}
\end{columns}
\bottomnote{Card fraud costs \$32B annually; 3D Secure 2.0 shifts liability to issuers on authenticated transactions.}
\end{frame}

\begin{frame}[t]{Chargebacks: Process and Impact}
\begin{columns}[T]
\column{0.48\textwidth}
\textbf{Chargeback Reasons:}
\begin{itemize}
\item Fraud: Card used without authorization
\item Processing error: Duplicate charge, wrong amount
\item Consumer dispute: Product not received/as described
\item Authorization issue: Expired authorization
\end{itemize}

\vspace{3mm}
\textbf{Chargeback Process:}
\begin{enumerate}
\item Cardholder disputes with issuer (120 days)
\item Issuer initiates chargeback
\item Merchant notified via acquirer
\item Merchant provides evidence (10-45 days)
\item Issuer reviews and decides
\item Pre-arbitration or arbitration (if disputed)
\item Final decision (issuer or network)
\end{enumerate}

\column{0.48\textwidth}
\textbf{Merchant Impact:}
\begin{itemize}
\item Chargeback fee: \$20-\$100 per case
\item Lost revenue from sale
\item Lost merchandise (if shipped)
\item Administrative time
\item High ratios: Monitoring programs
\end{itemize}

\vspace{3mm}
\textbf{Excessive Chargeback Programs:}
\begin{itemize}
\item Visa: >0.9\% ratio or 100/month
\item Mastercard: >1.5\% ratio or 100/month
\item Penalties: Fines, higher MDR, termination
\item Prevention: Fraud tools, clear policies
\end{itemize}
\end{columns}
\bottomnote{Chargeback ratio >0.9\% triggers Visa monitoring program---\$20-\$100 fee per dispute.}
\end{frame}

% =============================================================================
\section{Real-Time Payments}
% =============================================================================

\begin{frame}[t]{Real-Time Payments: Global Overview}
\begin{columns}[T]
\column{0.48\textwidth}
\textbf{Key Characteristics:}
\begin{itemize}
\item Settlement in seconds (<10 seconds)
\item 24/7/365 availability
\item Immediate funds availability
\item Irrevocable payment
\item ISO 20022 messaging standard
\end{itemize}

\vspace{3mm}
\textbf{Global RTP Systems:}
\begin{itemize}
\item UK: Faster Payments Service (2008)
\item India: UPI (2016, 12B txns/month)
\item Brazil: PIX (2020, 3.6B txns/month)
\item USA: FedNow (2023), RTP Network (2017)
\item EU: SEPA Instant (2017)
\item China: Alipay/WeChat (proprietary)
\end{itemize}

\column{0.48\textwidth}
\textbf{UPI Success (India):}
\begin{itemize}
\item Volume: 12.8B transactions/month (Oct 2024)
\item Value: \$240B/month
\item Zero transaction fees (government subsidy)
\item QR code-based payments
\item 500M+ active users
\end{itemize}

\vspace{3mm}
\textbf{Benefits:}
\begin{itemize}
\item Improved cash flow
\item Reduced fraud (account verified)
\item Lower cost than cards
\item Enhanced customer experience
\item Financial inclusion (no card needed)
\end{itemize}
\end{columns}
\bottomnote{India's UPI processes 16.58B transactions/month at zero fees---government-subsidized success story.}
\end{frame}

% Real-Time Payments Adoption Chart
\begin{frame}[t]{Real-Time Payment Systems: Global Adoption}
\begin{center}
\includegraphics[width=0.65\textwidth]{figures/realtime_payments_adoption/realtime_payments_adoption.pdf}
\end{center}
\bottomnote{UPI and PIX demonstrate RTP potential: 16.58B and 5.5B monthly transactions respectively.}
\end{frame}

\begin{frame}[t]{FedNow: US Real-Time Payments}
\begin{columns}[T]
\column{0.48\textwidth}
\textbf{FedNow Service:}
\begin{itemize}
\item Launched: July 2023
\item Operated by Federal Reserve
\item Competing with RTP Network (Clearing House)
\item Instant settlement (seconds)
\item No transaction amount limit (initially \$500k)
\item Available to all US depository institutions
\end{itemize}

\vspace{3mm}
\textbf{Adoption Challenges:}
\begin{itemize}
\item Requires bank system upgrades
\item Competition from existing RTP
\item Zelle dominance in P2P
\item Limited initial use cases
\item 700+ financial institutions signed up (2024)
\end{itemize}

\column{0.48\textwidth}
\textbf{RTP Network (Clearing House):}
\begin{itemize}
\item Launched: 2017
\item Private sector alternative
\item Owned by 22 major banks
\item 400+ participants
\item Similar functionality to FedNow
\end{itemize}

\vspace{3mm}
\textbf{Use Cases:}
\begin{itemize}
\item Emergency disbursements (disaster relief)
\item Instant refunds and rebates
\item Just-in-time payroll
\item Gig economy payments
\item Bill pay with immediate credit
\item Business liquidity management
\end{itemize}
\end{columns}
\bottomnote{FedNow launched July 2023---competes with private RTP Network for US real-time dominance.}
\end{frame}

% =============================================================================
\section{Summary}
% =============================================================================

\begin{frame}[t]{Payment Method Comparison}
\begin{columns}[T]
\column{0.48\textwidth}
\textbf{Cards:}
\begin{itemize}
\item Speed: 2-4 seconds authorization, T+1-3 settlement
\item Cost: 1\%-3\% MDR
\item Reach: Global
\item Security: Strong (EMV, 3DS)
\item Consumer protection: Chargebacks
\item Best for: E-commerce, POS
\end{itemize}

\vspace{3mm}
\textbf{ACH/SEPA:}
\begin{itemize}
\item Speed: T+1 (standard), same-day available
\item Cost: \$0.20-\$1.50 or 0.2\%-0.5\%
\item Reach: Domestic/regional
\item Security: Bank-level
\item Best for: Recurring, B2B, large amounts
\end{itemize}

\column{0.48\textwidth}
\textbf{Real-Time Payments:}
\begin{itemize}
\item Speed: <10 seconds
\item Cost: Low (often free to consumer)
\item Reach: Domestic
\item Security: Bank-verified accounts
\item Irrevocable: Limited consumer protection
\item Best for: P2P, urgent payments
\end{itemize}

\vspace{3mm}
\textbf{Wire Transfer:}
\begin{itemize}
\item Speed: Same-day
\item Cost: \$15-\$50 per transaction
\item Reach: Global (SWIFT)
\item Security: High, irrevocable
\item Best for: Large, urgent, cross-border
\end{itemize}
\end{columns}
\bottomnote{Choose payment method based on trade-offs: speed vs cost vs security vs consumer protection.}
\end{frame}

\begin{frame}[t]{Summary and Key Takeaways}
\begin{columns}[T]
\column{0.48\textwidth}
\textbf{Payment Lifecycle:}
\begin{itemize}
\item Authorization → Authentication → Clearing → Settlement
\item Multiple intermediaries (issuer, acquirer, network)
\item Card payments: 2-3 days for merchant settlement
\item Real-time rails: Seconds
\end{itemize}

\vspace{3mm}
\textbf{Card Networks:}
\begin{itemize}
\item Visa and Mastercard dominate globally
\item Four-party model (efficient network effects)
\item Interchange = largest merchant cost component
\item EU regulation caps interchange (0.2\%-0.3\%)
\end{itemize}

\column{0.48\textwidth}
\textbf{Payment Systems:}
\begin{itemize}
\item ACH: US batch system, low cost, T+1
\item SEPA: EU harmonized system, instant available
\item Real-time: Growing globally, instant settlement
\item Trade-offs: Speed vs cost vs security
\end{itemize}

\vspace{3mm}
\textbf{Emerging Trends:}
\begin{itemize}
\item Real-time payment adoption accelerating
\item Account-to-account (A2A) challenging cards
\item ISO 20022 standardization
\item Open banking enabling payment initiation
\end{itemize}
\end{columns}
\bottomnote{Real-time A2A payments are disrupting cards---lower cost, instant settlement, no intermediaries.}
\end{frame}

\begin{frame}[t]{Next Lesson Preview}
\begin{center}
\Large \textbf{Lesson 3: Digital Payments II -- Mobile Wallets}

\vspace{5mm}

\normalsize
We will explore:
\begin{itemize}
\item NFC technology and contactless payments
\item Tokenization and security
\item Apple Pay, Google Pay architecture
\item M-Pesa and mobile money in emerging markets
\item Alipay, WeChat Pay, and super-app models
\end{itemize}
\end{center}
\end{frame}

\end{document}
