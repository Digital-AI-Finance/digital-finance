\documentclass[8pt,aspectratio=169]{beamer}
\usetheme{Madrid}
\usepackage{graphicx}
\usepackage{booktabs}
\usepackage{adjustbox}
\usepackage{multicol}
\usepackage{amsmath}
\usepackage{amssymb}

\definecolor{mlblue}{RGB}{0,102,204}
\definecolor{mlpurple}{RGB}{51,51,178}
\definecolor{mllavender}{RGB}{173,173,224}
\definecolor{mllavender2}{RGB}{193,193,232}
\definecolor{mllavender3}{RGB}{204,204,235}
\definecolor{mllavender4}{RGB}{214,214,239}
\definecolor{mlorange}{RGB}{255, 127, 14}
\definecolor{mlgreen}{RGB}{44, 160, 44}
\definecolor{mlred}{RGB}{214, 39, 40}

\setbeamercolor{palette primary}{bg=mllavender3,fg=mlpurple}
\setbeamercolor{palette secondary}{bg=mllavender2,fg=mlpurple}
\setbeamercolor{palette tertiary}{bg=mllavender,fg=white}
\setbeamercolor{structure}{fg=mlpurple}
\setbeamercolor{frametitle}{fg=mlpurple,bg=mllavender3}
\setbeamertemplate{navigation symbols}{}
\setbeamertemplate{itemize items}[circle]
\setbeamersize{text margin left=5mm,text margin right=5mm}

% Bottom annotation command
\newcommand{\bottomnote}[1]{%
\vfill
\vspace{-2mm}
\textcolor{mllavender2}{\rule{\textwidth}{0.4pt}}
\vspace{1mm}
\footnotesize
\textbf{#1}
}

\title{Lesson 1: Introduction to Fintech}
\subtitle{Module 1: Fintech and Innovation}
\author{Digital Finance Course}
\date{\today}

\begin{document}

\begin{frame}
\titlepage
\end{frame}

%============================================================
\section{What is Fintech?}
%============================================================

\begin{frame}[t]{What is Fintech?}
\begin{columns}[T]
\column{0.48\textwidth}
\textbf{Definition:}
\begin{itemize}
\item Financial Technology (``Fintech'')
\item Technology-enabled financial services innovation
\item Intersection of finance, technology, and business models
\end{itemize}

\vspace{2mm}
\textbf{Core Characteristics:}
\begin{itemize}
\item Customer-centric design
\item Digital-first approach
\item Data-driven decision making
\end{itemize}

\column{0.48\textwidth}
\textbf{Not Just About Technology:}
\begin{itemize}
\item Business model innovation
\item User experience transformation
\item Market structure disruption
\end{itemize}

\vspace{2mm}
\textbf{Evolution of Term:}
\begin{itemize}
\item Pre-2000s: Back-office technology
\item 2010s: Startup-driven disruption
\item 2020s: Embedded finance, DeFi
\end{itemize}
\end{columns}
\bottomnote{Key insight: Fintech is about reimagining financial services, not just digitizing existing processes.}
\end{frame}

\begin{frame}[t]{Historical Evolution of Fintech}
\begin{columns}[T]
\column{0.48\textwidth}
\textbf{Era 1 (1866-1967): Foundation}
\begin{itemize}
\item 1866: First transatlantic cable
\item 1918: Fedwire launch
\item 1950: First credit card (Diners Club)
\item 1967: First ATM (Barclays, London)
\end{itemize}

\vspace{2mm}
\textbf{Era 2 (1967-2008): Digitization}
\begin{itemize}
\item 1973: SWIFT established
\item 1994: First online banking
\item 1998: PayPal founded
\item 2007: First iPhone launched
\end{itemize}

\column{0.48\textwidth}
\textbf{Era 3 (2008-Present): Disruption}
\begin{itemize}
\item 2008: Global Financial Crisis
\item 2009: Bitcoin whitepaper
\item 2015: Neobank explosion
\item 2020: Embedded finance rise
\item 2023: Generative AI in finance
\end{itemize}

\vspace{2mm}
\textbf{Key Inflection Point:}
\begin{itemize}
\item Post-2008: Trust erosion in banks
\item Smartphone enabling mobile-first services
\end{itemize}
\end{columns}
\bottomnote{The 2008 financial crisis was the catalyst for the modern fintech revolution.}
\end{frame}

%============================================================
\section{Drivers and Ecosystem}
%============================================================

\begin{frame}[t]{Key Drivers of Fintech Growth}
\begin{columns}[T]
\column{0.48\textwidth}
\textbf{Technology Enablers:}
\begin{itemize}
\item Cloud computing infrastructure
\item Mobile ubiquity (5B+ smartphones)
\item API economy and microservices
\item AI and machine learning
\end{itemize}

\vspace{2mm}
\textbf{Regulatory Drivers:}
\begin{itemize}
\item Open banking mandates (PSD2)
\item Sandbox frameworks (FCA, MAS)
\item Digital identity standards (eIDAS)
\end{itemize}

\column{0.48\textwidth}
\textbf{Market Forces:}
\begin{itemize}
\item Millennial and Gen Z preferences
\item Underbanked populations (1.4B globally)
\item COVID-19 digital acceleration
\item Network effects and platform economics
\end{itemize}

\vspace{2mm}
\textbf{Economic Factors:}
\begin{itemize}
\item VC investment driving innovation
\item Cost pressure on incumbents
\item Declining transaction costs
\end{itemize}
\end{columns}
\bottomnote{Multiple forces converged post-2008 to create the perfect environment for fintech disruption.}
\end{frame}

\begin{frame}[t]{The Fintech Ecosystem}
\begin{columns}[T]
\column{0.48\textwidth}
\textbf{Core Players:}
\begin{itemize}
\item Fintech startups (Stripe, Revolut, Wise)
\item Incumbent banks (JPMorgan, DBS)
\item Big Tech (Apple, Google, Ant Group)
\item Payment networks (Visa, Mastercard)
\end{itemize}

\vspace{2mm}
\textbf{Enablers:}
\begin{itemize}
\item Infrastructure (Plaid, Adyen)
\item Core banking (Mambu, Thought Machine)
\item Cloud providers (AWS, Azure, GCP)
\end{itemize}

\column{0.48\textwidth}
\textbf{Capital Providers:}
\begin{itemize}
\item Venture capital (Sequoia, a16z)
\item Private equity (KKR, Blackstone)
\item Corporate venture arms
\end{itemize}

\vspace{2mm}
\textbf{Support Infrastructure:}
\begin{itemize}
\item Regulators (ECB, Fed, FCA, BaFin)
\item Accelerators (Y Combinator, Techstars)
\item Professional services
\end{itemize}
\end{columns}
\bottomnote{Fintech success depends on a complex ecosystem of players, enablers, and capital.}
\end{frame}

%============================================================
\section{Market Size and Investment}
%============================================================

\begin{frame}[t]{Global Fintech Investment Trends}
\begin{columns}[T]
\column{0.45\textwidth}
\textbf{Investment Trends (2024):}
\begin{itemize}
\item 2024: \$95.6B global investment
\item Down from \$119.8B in 2023
\item Peak 2021: \$238B (crypto boom)
\item 7-year low reflects market correction
\end{itemize}

\vspace{2mm}
\textbf{Regional Distribution (2024):}
\begin{itemize}
\item Americas: 52\% of investment
\item EMEA: 27\% of investment
\item APAC: 21\% of investment
\end{itemize}

\column{0.50\textwidth}
\begin{center}
\includegraphics[width=\textwidth]{figures/fintech_investment_trends/fintech_investment_trends.pdf}
\end{center}
\end{columns}
\bottomnote{2024 marks a return to fundamentals: profitability over growth at any cost.}
\end{frame}

\begin{frame}[t]{Top Fintech Valuations (2024)}
\begin{columns}[T]
\column{0.45\textwidth}
\textbf{Global Leaders:}
\begin{itemize}
\item Ant Group: \$78B (China)
\item Stripe: \$70B (USA)
\item Revolut: \$45B (UK)
\item Chime: \$25B (USA)
\item Nubank: \$60B (Brazil, public)
\end{itemize}

\vspace{2mm}
\textbf{Notable Changes:}
\begin{itemize}
\item Klarna: \$6.7B (down from \$46B)
\item Many 50-70\% valuation cuts
\item Shift to profitability metrics
\end{itemize}

\column{0.50\textwidth}
\begin{center}
\includegraphics[width=\textwidth]{figures/fintech_unicorns/fintech_unicorns.pdf}
\end{center}
\end{columns}
\bottomnote{Valuations have corrected significantly from 2021 peaks as investors demand profitability.}
\end{frame}

\begin{frame}[t]{Global Fintech Hubs}
\begin{center}
\includegraphics[width=0.60\textwidth]{figures/global_fintech_hubs/global_fintech_hubs.pdf}
\end{center}
\bottomnote{London, New York, and Singapore lead as global fintech hubs---emerging markets show rapid growth.}
\end{frame}

\begin{frame}[t]{Emerging Market Fintech Growth}
\begin{center}
\includegraphics[width=0.60\textwidth]{figures/emerging_market_fintech/emerging_market_fintech.pdf}
\end{center}
\bottomnote{Emerging markets drive fintech adoption---mobile-first solutions address financial exclusion.}
\end{frame}

\begin{frame}[t]{Venture Funding Cycles}
\begin{center}
\includegraphics[width=0.60\textwidth]{figures/venture_funding_cycles/venture_funding_cycles.pdf}
\end{center}
\bottomnote{VC funding cycles show boom-bust pattern---2021 peak followed by 2022-2024 correction.}
\end{frame}

\begin{frame}[t]{Fintech IPO Performance}
\begin{center}
\includegraphics[width=0.60\textwidth]{figures/fintech_ipo_performance/fintech_ipo_performance.pdf}
\end{center}
\bottomnote{Many fintech IPOs underperformed initial valuations---profitability concerns dominate investor sentiment.}
\end{frame}

\begin{frame}[t]{Fintech M\&A Activity}
\begin{center}
\includegraphics[width=0.60\textwidth]{figures/fintech_m_and_a/fintech_m_and_a.pdf}
\end{center}
\bottomnote{M\&A activity increases as market consolidates---banks acquire fintech capabilities, fintechs merge for scale.}
\end{frame}

%============================================================
\section{Fintech Categories}
%============================================================

\begin{frame}[t]{Fintech Categories: Payments}
\begin{columns}[T]
\column{0.48\textwidth}
\textbf{Digital Payments:}
\begin{itemize}
\item Payment processors (Stripe, Adyen, Square)
\item Mobile wallets (Apple Pay, Google Pay, Alipay)
\item Cross-border (Wise, Remitly)
\item BNPL (Klarna, Affirm, Afterpay)
\end{itemize}

\vspace{2mm}
\textbf{Market Dynamics:}
\begin{itemize}
\item Shift from cash to digital (80\% by 2025)
\item Real-time payments (FedNow, SEPA Instant)
\item Embedded payments growing
\end{itemize}

\column{0.48\textwidth}
\textbf{Key Innovations:}
\begin{itemize}
\item QR code payments (WeChat Pay, Paytm)
\item Contactless NFC technology
\item Account-to-account (A2A) payments
\end{itemize}

\vspace{2mm}
\textbf{European Context:}
\begin{itemize}
\item PSD2 enabling third-party access
\item Strong Customer Authentication (SCA)
\item Digital euro development (ECB)
\end{itemize}
\end{columns}
\bottomnote{Payments is the largest and most mature fintech category, driving financial inclusion globally.}
\end{frame}

\begin{frame}[t]{Fintech Categories: Banking and Lending}
\begin{columns}[T]
\column{0.48\textwidth}
\textbf{Neobanks/Challenger Banks:}
\begin{itemize}
\item Mobile-first current accounts
\item No physical branches
\item Examples: Revolut, N26, Monzo, Chime
\item EU: 35M+ neobank customers (2024)
\end{itemize}

\vspace{2mm}
\textbf{Alternative Lending:}
\begin{itemize}
\item P2P lending (Funding Circle)
\item SME financing (Kabbage)
\item Invoice financing (BlueVine)
\end{itemize}

\column{0.48\textwidth}
\textbf{Business Models:}
\begin{itemize}
\item Freemium basic accounts
\item Premium subscriptions (5-15 EUR/month)
\item Interchange from card spending
\item FX margin on currency exchange
\end{itemize}

\vspace{2mm}
\textbf{Technology Advantages:}
\begin{itemize}
\item Lower CAC (\$10-50 vs \$200-400)
\item Instant account opening (3-5 min)
\item AI-driven underwriting
\end{itemize}
\end{columns}
\bottomnote{Neobanks achieve 10x lower customer acquisition costs through digital channels.}
\end{frame}

\begin{frame}[t]{Fintech Categories: Wealth and Investment}
\begin{columns}[T]
\column{0.48\textwidth}
\textbf{Robo-Advisors:}
\begin{itemize}
\item Automated portfolio management
\item Algorithm-based asset allocation
\item Betterment, Wealthfront, Scalable Capital
\item AUM: \$2.5 trillion globally (2024)
\end{itemize}

\vspace{2mm}
\textbf{Commission-Free Trading:}
\begin{itemize}
\item Zero-fee stock trading (Robinhood)
\item Fractional shares investing
\item Revenue from PFOF and interest
\end{itemize}

\column{0.48\textwidth}
\textbf{Fee Comparison:}
\begin{itemize}
\item Traditional advisor: 1.0\%-2.0\% AUM
\item Robo-advisor: 0.25\%-0.50\% AUM
\item Hybrid model: 0.50\%-0.75\% AUM
\end{itemize}

\vspace{2mm}
\textbf{Investment Products:}
\begin{itemize}
\item ETF-based portfolios
\item ESG/thematic investing
\item Tax-loss harvesting
\item Cryptocurrency access
\end{itemize}
\end{columns}
\bottomnote{Robo-advisors democratize wealth management with fees 75\% lower than traditional advisors.}
\end{frame}

\begin{frame}[t]{Fintech Categories: Insurance (InsurTech)}
\begin{columns}[T]
\column{0.48\textwidth}
\textbf{Business Model Innovations:}
\begin{itemize}
\item P2P insurance (Lemonade, Friendsurance)
\item Usage-based insurance (Root, Metromile)
\item On-demand insurance (Cuvva, Trov)
\item Embedded insurance (Qover)
\end{itemize}

\vspace{2mm}
\textbf{Technology Applications:}
\begin{itemize}
\item AI claims processing (90\% automation)
\item IoT risk monitoring (telematics)
\item Parametric insurance (auto payouts)
\end{itemize}

\column{0.48\textwidth}
\textbf{Market Impact:}
\begin{itemize}
\item Global InsurTech investment: \$8B (2024)
\item Digital insurance penetration: 25\%
\item Incumbents partnering vs competing
\end{itemize}

\vspace{2mm}
\textbf{Product Examples:}
\begin{itemize}
\item Pay-per-mile auto insurance
\item Cyber insurance for SMEs
\item Crop insurance with satellite data
\item Flight delay parametric coverage
\end{itemize}
\end{columns}
\bottomnote{InsurTech uses IoT and AI to shift from reactive claims to proactive risk prevention.}
\end{frame}

\begin{frame}[t]{Fintech Categories: RegTech and Infrastructure}
\begin{columns}[T]
\column{0.48\textwidth}
\textbf{RegTech Solutions:}
\begin{itemize}
\item KYC/AML automation (Onfido)
\item Transaction monitoring (Featurespace)
\item Regulatory reporting (Suade)
\item Risk management (Quantexa)
\end{itemize}

\vspace{2mm}
\textbf{Cost Savings:}
\begin{itemize}
\item Compliance: 4\%-10\% of revenue
\item RegTech reduces by 30\%-50\%
\item Manual KYC: \$50-100/customer
\item Automated KYC: \$5-15/customer
\end{itemize}

\column{0.48\textwidth}
\textbf{Infrastructure/BaaS:}
\begin{itemize}
\item Core banking (Mambu, Thought Machine)
\item Banking-as-a-Service (Solarisbank)
\item Payment infrastructure (Plaid)
\item Identity verification (Jumio)
\end{itemize}

\vspace{2mm}
\textbf{Enabling Technologies:}
\begin{itemize}
\item NLP for regulation parsing
\item ML for anomaly detection
\item Graph analytics for networks
\end{itemize}
\end{columns}
\bottomnote{RegTech enables 80\% cost reduction in compliance while improving accuracy.}
\end{frame}

%============================================================
\section{Competitive Landscape}
%============================================================

\begin{frame}[t]{Fintech vs Traditional Banks: Comparison}
\begin{columns}[T]
\column{0.48\textwidth}
\textbf{Cost Structure:}
\begin{itemize}
\item Bank cost-to-income: 55\%-65\%
\item Fintech cost-to-income: 35\%-50\%
\item Branch network vs cloud
\item Legacy IT vs modern stack
\end{itemize}

\vspace{2mm}
\textbf{Customer Acquisition:}
\begin{itemize}
\item Bank CAC: \$200-\$400
\item Fintech CAC: \$10-\$50
\item Viral growth and referrals
\end{itemize}

\column{0.48\textwidth}
\textbf{Fintech Advantages:}
\begin{itemize}
\item Speed: Account in minutes vs days
\item UX: Mobile-first, intuitive design
\item Pricing: Lower fees, better FX
\item Innovation: Rapid iteration
\end{itemize}

\vspace{2mm}
\textbf{Bank Advantages:}
\begin{itemize}
\item Trust and brand recognition
\item Full license and deposit insurance
\item Diversified revenue streams
\item Physical presence for complex needs
\end{itemize}
\end{columns}
\bottomnote{Fintechs win on cost and UX; banks retain trust and regulatory advantages.}
\end{frame}

\begin{frame}[t]{Big Tech in Finance}
\begin{columns}[T]
\column{0.45\textwidth}
\textbf{Apple:}
\begin{itemize}
\item Apple Pay (500M+ users globally)
\item Apple Card (Goldman partnership)
\item Apple Pay Later (BNPL)
\item Savings account (4.15\% APY)
\end{itemize}

\vspace{2mm}
\textbf{Google:}
\begin{itemize}
\item Google Pay (150M+ users)
\item Google Wallet integration
\item Limited banking ambitions
\end{itemize}

\column{0.50\textwidth}
\begin{center}
\includegraphics[width=\textwidth]{figures/big_tech_finance/big_tech_finance.pdf}
\end{center}
\end{columns}
\bottomnote{Big Tech leverages existing user bases and trust to enter financial services.}
\end{frame}

\begin{frame}[t]{Chinese Tech Giants: Super-App Model}
\begin{columns}[T]
\column{0.48\textwidth}
\textbf{Ant Group (Alipay):}
\begin{itemize}
\item 1.3 billion users globally
\item Payments + commerce + finance
\item Yu'e Bao money market fund
\item Regulatory crackdown 2021-2023
\end{itemize}

\vspace{2mm}
\textbf{Tencent (WeChat Pay):}
\begin{itemize}
\item 900 million users
\item Embedded in WeChat ecosystem
\item Social + payments integration
\end{itemize}

\column{0.48\textwidth}
\textbf{Super-App Model:}
\begin{itemize}
\item Single app for all financial needs
\item Payments, lending, insurance, wealth
\item Social commerce integration
\item Data advantage from ecosystem
\end{itemize}

\vspace{2mm}
\textbf{Global Adoption:}
\begin{itemize}
\item Grab (Southeast Asia)
\item Paytm (India)
\item Revolut pursuing similar model
\end{itemize}
\end{columns}
\bottomnote{The super-app model from China is being adapted globally, combining finance with commerce.}
\end{frame}

%============================================================
\section{Challenges and Outlook}
%============================================================

\begin{frame}[t]{Challenges and Risks}
\begin{columns}[T]
\column{0.48\textwidth}
\textbf{Business Challenges:}
\begin{itemize}
\item Path to profitability (most still loss-making)
\item Customer retention and churn
\item Scaling while maintaining economics
\item Competition from Big Tech
\end{itemize}

\vspace{2mm}
\textbf{Technology Risks:}
\begin{itemize}
\item Cybersecurity threats
\item System outages and downtime
\item Third-party vendor dependency
\item AI bias and fairness
\end{itemize}

\column{0.48\textwidth}
\textbf{Regulatory Risks:}
\begin{itemize}
\item Licensing requirements
\item Capital adequacy rules
\item Consumer protection mandates
\item Cross-border complexity
\end{itemize}

\vspace{2mm}
\textbf{Market Risks:}
\begin{itemize}
\item Interest rate sensitivity
\item Credit risk in lending
\item Valuation corrections (2022-2024)
\item Funding winter and VC pullback
\end{itemize}
\end{columns}
\bottomnote{Profitability remains the key challenge: only 5\% of neobanks are profitable globally.}
\end{frame}

\begin{frame}[t]{European Fintech Landscape}
\begin{columns}[T]
\column{0.48\textwidth}
\textbf{Leading Markets (2024):}
\begin{itemize}
\item UK: \$9B investment, 2,500+ fintechs
\item Germany: \$2.5B, focus on B2B
\item France: \$2.2B, strong InsurTech
\item Netherlands: \$1.8B, payments hub
\item Sweden: \$1.5B, Klarna home
\end{itemize}

\vspace{2mm}
\textbf{European Champions:}
\begin{itemize}
\item Revolut (UK, \$45B valuation)
\item Adyen (Netherlands, \$40B market cap)
\item N26 (Germany, \$9B)
\item Wise (UK, \$8B market cap)
\end{itemize}

\column{0.48\textwidth}
\textbf{Regulatory Environment:}
\begin{itemize}
\item PSD2: Open banking mandate
\item MiFID II: Investment services
\item GDPR: Data protection
\item MiCA: Crypto-asset regulation
\item DORA: Digital operational resilience
\end{itemize}

\vspace{2mm}
\textbf{Competitive Dynamics:}
\begin{itemize}
\item Fragmented market (27 countries)
\item Cross-border passporting
\item Emerging hubs (Estonia, Lithuania)
\end{itemize}
\end{columns}
\bottomnote{The UK remains Europe's fintech capital, but Brexit complicates EU market access.}
\end{frame}

\begin{frame}[t]{Future Trends and Outlook}
\begin{columns}[T]
\column{0.48\textwidth}
\textbf{Emerging Trends:}
\begin{itemize}
\item Embedded finance: Finance in non-financial apps
\item Open finance: Beyond banking to all products
\item Real-time everything: Payments, credit, FX
\item Generative AI: Personalized advice, fraud
\end{itemize}

\vspace{2mm}
\textbf{Technology Evolution:}
\begin{itemize}
\item Central Bank Digital Currencies (CBDCs)
\item Quantum-resistant cryptography
\item Biometric authentication standards
\end{itemize}

\column{0.48\textwidth}
\textbf{Market Consolidation:}
\begin{itemize}
\item M\&A activity increasing
\item Bank acquisitions of fintechs
\item Platform aggregation (super-apps)
\item Vertical integration strategies
\end{itemize}

\vspace{2mm}
\textbf{Regulatory Evolution:}
\begin{itemize}
\item Harmonized EU fintech regulation
\item AI governance frameworks
\item ESG integration in fintech
\item Consumer protection enhancements
\end{itemize}
\end{columns}
\bottomnote{Embedded finance is projected to be a \$7 trillion market by 2030.}
\end{frame}

%============================================================
\section{Summary}
%============================================================

\begin{frame}[t]{Summary and Key Takeaways}
\begin{columns}[T]
\column{0.48\textwidth}
\textbf{Core Concepts:}
\begin{itemize}
\item Fintech = Technology-enabled financial innovation
\item Driven by mobile, cloud, AI, and regulation
\item Ecosystem of startups, banks, Big Tech
\item Multiple categories: Payments, banking, wealth, insurance
\end{itemize}

\vspace{2mm}
\textbf{Market Reality (2024):}
\begin{itemize}
\item \$95.6B investment (down from peak)
\item European leaders: Revolut, Adyen, Wise
\item Path to profitability remains challenging
\end{itemize}

\column{0.48\textwidth}
\textbf{Critical Success Factors:}
\begin{itemize}
\item Customer experience and trust
\item Unit economics and scalability
\item Regulatory compliance
\item Sustainable business models
\end{itemize}

\vspace{2mm}
\textbf{Looking Ahead:}
\begin{itemize}
\item Embedded finance as dominant model
\item Consolidation and maturation
\item Collaboration over disruption
\item AI-driven personalization
\end{itemize}
\end{columns}
\bottomnote{Fintech has moved from disruption to collaboration; the winners will combine innovation with profitability.}
\end{frame}

\begin{frame}[t]{Next Lesson Preview}
\begin{center}
\Large \textbf{Lesson 2: Digital Payments I -- Fundamentals}

\vspace{8mm}

\normalsize
We will explore:
\begin{itemize}
\item Payment lifecycle and infrastructure
\item Card networks (Visa, Mastercard)
\item Interchange fees and economics
\item ACH and SEPA payment systems
\item Payment processing value chain
\end{itemize}
\end{center}
\end{frame}

\end{document}
