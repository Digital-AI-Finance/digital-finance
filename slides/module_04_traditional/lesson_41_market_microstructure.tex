\documentclass[8pt,aspectratio=169]{beamer}
\usetheme{Madrid}
\usepackage{graphicx,booktabs,adjustbox,multicol,amsmath,amssymb}
\definecolor{mlblue}{RGB}{0,102,204}
\definecolor{mlpurple}{RGB}{51,51,178}
\definecolor{mllavender}{RGB}{173,173,224}
\definecolor{mllavender2}{RGB}{193,193,232}
\definecolor{mllavender3}{RGB}{204,204,235}
\definecolor{mllavender4}{RGB}{214,214,239}
\definecolor{mlorange}{RGB}{255,127,14}
\definecolor{mlgreen}{RGB}{44,160,44}
\definecolor{mlred}{RGB}{214,39,40}
\setbeamercolor{palette primary}{bg=mllavender3,fg=mlpurple}
\setbeamercolor{palette secondary}{bg=mllavender2,fg=mlpurple}
\setbeamercolor{palette tertiary}{bg=mllavender,fg=white}
\setbeamercolor{structure}{fg=mlpurple}
\setbeamercolor{frametitle}{fg=mlpurple,bg=mllavender3}
\setbeamertemplate{navigation symbols}{}
\setbeamertemplate{itemize items}[circle]
\setbeamersize{text margin left=5mm,text margin right=5mm}

% Bottom note command for key takeaways
\newcommand{\bottomnote}[1]{%
\vfill
\vspace{-2mm}
\textcolor{mllavender2}{\rule{\textwidth}{0.4pt}}
\vspace{1mm}
\footnotesize
\textbf{#1}
}
\title{Lesson 41: Market Microstructure and HFT}
\subtitle{Module 4: Traditional Digital Finance}
\author{Digital Finance Course}
\date{2025}

\begin{document}

\begin{frame}
\titlepage
\end{frame}

\begin{frame}[t]{Learning Objectives}
\begin{itemize}
\item Understand market microstructure theory and empirical regularities
\item Analyze bid-ask spreads and their components
\item Examine market maker economics and inventory management
\item Evaluate high-frequency trading strategies and market impact
\item Assess flash crashes and systemic stability concerns
\end{itemize}
\end{frame}

\section{Market Microstructure Foundations}

\begin{frame}[t]{Microstructure Theory Overview}
\begin{columns}[T]
\column{0.48\textwidth}
\textbf{Core Questions:}
\begin{itemize}
\item How are prices formed in continuous trading?
\item What determines bid-ask spreads?
\item How does information get incorporated into prices?
\item What is the role of market makers and liquidity providers?
\item How do trading protocols affect efficiency?
\end{itemize}

\vspace{2mm}
\textbf{Key Concepts:}
\begin{itemize}
\item \textbf{Price Discovery:} Aggregating dispersed information
\item \textbf{Liquidity:} Ability to trade without price impact
\item \textbf{Market Depth:} Volume available at various prices
\item \textbf{Resilience:} Speed of price recovery after shocks
\end{itemize}

\column{0.48\textwidth}
\textbf{Trading Costs Framework:}
\begin{itemize}
\item \textbf{Explicit Costs:} Commissions, fees, taxes
\item \textbf{Implicit Costs:} Spread, impact, opportunity
\item \textbf{Total Cost:} $TC = \text{Spread} + \text{Impact} + \text{Delay}$
\end{itemize}

\vspace{2mm}
\textbf{Market Quality Metrics:}
\begin{itemize}
\item \textbf{Efficiency:} Prices reflect available information
\item \textbf{Liquidity:} Low cost, high volume capacity
\item \textbf{Transparency:} Order flow and trade visibility
\item \textbf{Stability:} Resistance to manipulation and crashes
\item \textbf{Fairness:} Equal access and opportunity
\end{itemize}
\end{columns}
\end{frame}

\begin{frame}[t]{Bid-Ask Spread Fundamentals}
\begin{columns}[T]
\column{0.48\textwidth}
\textbf{Spread Definitions:}
\begin{itemize}
\item \textbf{Quoted Spread:} $S_Q = P_{ask} - P_{bid}$
\item \textbf{Percent Spread:} $S_{\%} = \frac{P_{ask} - P_{bid}}{P_{mid}} \times 100$
\item \textbf{Effective Spread:} $S_E = 2|P_{trade} - P_{mid}|$
\item \textbf{Realized Spread:} $S_R = 2(P_{trade} - P_{mid+5min}) \times D$
\end{itemize}
where $D = +1$ for buyer-initiated, $-1$ for seller-initiated

\vspace{2mm}
\textbf{Example:}
\begin{itemize}
\item Bid: \$99.98, Ask: \$100.02
\item Quoted spread: \$0.04 (4 cents)
\item Percent spread: 0.04\%
\item Trade at \$100.01 (price improvement)
\item Effective spread: 2 $\times$ (\$100.01 - \$100.00) = \$0.02
\end{itemize}

\column{0.48\textwidth}
\textbf{Spread Components (Stoll 1989):}
\begin{itemize}
\item \textbf{Order Processing:} Fixed costs (40-50\%)
\item \textbf{Inventory Holding:} Risk aversion costs (10-20\%)
\item \textbf{Adverse Selection:} Informed trading (30-50\%)
\end{itemize}

\vspace{2mm}
\textbf{Determinants of Spreads:}
\begin{itemize}
\item \textbf{Volume:} Higher volume $\rightarrow$ tighter spreads
\item \textbf{Volatility:} Higher volatility $\rightarrow$ wider spreads
\item \textbf{Competition:} More market makers $\rightarrow$ tighter
\item \textbf{Tick Size:} Minimum increment constraint
\item \textbf{Information Asymmetry:} More informed trading $\rightarrow$ wider
\end{itemize}

\vspace{2mm}
\textit{US large-cap spreads: 1-3 bps; small-cap: 10-50 bps (2024)}
\end{columns}
\bottomnote{Key concepts from this slide inform practical applications in finance.}
\end{frame}

\begin{frame}[t]{Bid-Ask Spread Analysis}
\begin{center}
\includegraphics[width=0.60\textwidth]{figures/bid_ask_spread/bid_ask_spread.pdf}
\end{center}
\bottomnote{Bid-ask spread measures liquidity cost and information asymmetry.}
\end{frame}

\begin{frame}[t]{Liquidity Metrics Across Markets}
\begin{center}
\includegraphics[width=0.60\textwidth]{figures/liquidity_metrics/liquidity_metrics.pdf}
\end{center}
\bottomnote{Multiple dimensions capture different aspects of market liquidity.}
\end{frame}

\begin{frame}[t]{Volatility Term Structure}
\begin{center}
\includegraphics[width=0.60\textwidth]{figures/volatility_term_structure/volatility_term_structure.pdf}
\end{center}
\bottomnote{Volatility term structure reveals market expectations across time horizons.}
\end{frame}

\begin{frame}[t]{Asset Correlation Matrix}
\begin{center}
\includegraphics[width=0.55\textwidth]{figures/correlation_matrix/correlation_matrix.pdf}
\end{center}
\bottomnote{Asset correlations drive portfolio diversification and risk management.}
\end{frame}

\begin{frame}[t]{Adverse Selection and Information Models}
\begin{columns}[T]
\column{0.48\textwidth}
\textbf{Glosten-Milgrom Model (1985):}
\begin{itemize}
\item Sequential trade model with information asymmetry
\item Informed traders know true value $V$
\item Uninformed traders are noise/liquidity traders
\item Market maker sets bid-ask to break even
\item Spread compensates for adverse selection
\end{itemize}

\vspace{2mm}
\textbf{Key Implications:}
\begin{itemize}
\item Spread widens with more informed trading
\item Price update after each trade (Bayesian learning)
\item Bid-ask bounce causes negative autocorrelation
\item Larger trades signal more information
\end{itemize}

\column{0.48\textwidth}
\textbf{Kyle Model (1985):}
\begin{itemize}
\item Batch auction with strategic informed trader
\item Informed trader optimizes profit vs price impact
\item Market depth (lambda): $\lambda = \frac{dP}{dQ}$
\item Price impact linear in order flow
\end{itemize}

\vspace{2mm}
\textbf{Kyle's Lambda:}
$$\lambda = \frac{\sigma_v}{2\sigma_u}$$
where $\sigma_v$ = value volatility, $\sigma_u$ = noise trading volatility

\vspace{2mm}
\textbf{Empirical Evidence:}
\begin{itemize}
\item Larger trades move prices more (concave impact)
\item Block trades have 5-10x impact vs VWAP execution
\item Price impact persists 30-60 minutes post-trade
\end{itemize}
\end{columns}
\bottomnote{Key concepts from this slide inform practical applications in finance.}
\end{frame}

\section{Market Makers and Liquidity Provision}

\begin{frame}[t]{Market Maker Role and Economics}
\begin{columns}[T]
\column{0.48\textwidth}
\textbf{Market Maker Functions:}
\begin{itemize}
\item Continuous two-sided quotes (bid and ask)
\item Absorb temporary order imbalances
\item Facilitate price discovery
\item Reduce search costs for traders
\item Profit from bid-ask spread capture
\end{itemize}

\vspace{2mm}
\textbf{Designated Market Makers (DMM):}
\begin{itemize}
\item \textbf{NYSE:} DMM obligations for assigned stocks
\item \textbf{Nasdaq:} Competitive market makers (no exclusivity)
\item \textbf{Obligations:} Maintain fair and orderly markets
\item \textbf{Benefits:} Informational advantages, rebates
\end{itemize}

\column{0.48\textwidth}
\textbf{Profitability Sources:}
\begin{itemize}
\item \textbf{Spread Capture:} Buy bid, sell ask
\item \textbf{Maker Rebates:} 0.15-0.30 cents/share (US equities)
\item \textbf{Order Flow Internalization:} Payment for order flow (PFOF)
\item \textbf{Statistical Arbitrage:} Short-term mean reversion
\end{itemize}

\vspace{2mm}
\textbf{Risks:}
\begin{itemize}
\item \textbf{Inventory Risk:} Directional exposure
\item \textbf{Adverse Selection:} Losing to informed traders
\item \textbf{Volatility Spikes:} Widening spreads, reduced depth
\item \textbf{Technology Failures:} Latency, connectivity issues
\end{itemize}

\vspace{2mm}
\textit{Top HFT market makers: Citadel Securities, Virtu, Jane Street, Jump Trading}
\end{columns}
\bottomnote{Key concepts from this slide inform practical applications in finance.}
\end{frame}

\begin{frame}[t]{Inventory Management Models}
\begin{columns}[T]
\column{0.48\textwidth}
\textbf{Stoll (1978) Inventory Model:}
\begin{itemize}
\item Market maker adjusts quotes based on inventory
\item High long inventory $\rightarrow$ lower ask, lower bid
\item High short inventory $\rightarrow$ higher bid, higher ask
\item Inventory mean-reversion via asymmetric quotes
\end{itemize}

\vspace{2mm}
\textbf{Quote Adjustment:}
$$P_{bid} = P_{mid} - \frac{S}{2} - \alpha \cdot I$$
$$P_{ask} = P_{mid} + \frac{S}{2} - \alpha \cdot I$$
where $I$ = inventory position, $\alpha$ = inventory aversion

\vspace{2mm}
\textbf{Example:}
\begin{itemize}
\item Neutral: Bid \$99.98, Ask \$100.02
\item Long 10k shares: Bid \$99.96, Ask \$100.00 (skewed to sell)
\end{itemize}

\column{0.48\textwidth}
\textbf{Avellaneda-Stoikov Model (2008):}
\begin{itemize}
\item Optimal market making with risk aversion
\item Maximizes expected utility of terminal wealth
\item Incorporates arrival rates and fill probabilities
\item Dynamic spread and mid-price adjustments
\end{itemize}

\vspace{2mm}
\textbf{Optimal Quotes:}
$$\delta_{bid}^* = \frac{1}{\gamma} \ln\left(1 + \frac{\gamma}{\kappa}\right) + \frac{\gamma \sigma^2 (T-t)}{2} q$$
$$\delta_{ask}^* = \frac{1}{\gamma} \ln\left(1 + \frac{\gamma}{\kappa}\right) - \frac{\gamma \sigma^2 (T-t)}{2} q$$
where $\gamma$ = risk aversion, $\kappa$ = order arrival intensity, $q$ = inventory

\end{columns}
\bottomnote{Avellaneda-Stoikov model optimizes market making with risk aversion and dynamic spread adjustments.}
\end{frame}

\begin{frame}[t]{Payment for Order Flow (PFOF)}
\begin{columns}[T]
\column{0.48\textwidth}
\textbf{PFOF Mechanics:}
\begin{itemize}
\item Retail broker routes orders to wholesaler
\item Wholesaler internalizes (does not send to exchange)
\item Wholesaler pays broker 0.1-0.5 cents/share
\item Retail order receives NBBO or better (price improvement)
\item Wholesaler profits from spread capture + rebates
\end{itemize}

\vspace{2mm}
\textbf{Major Wholesalers (2024):}
\begin{itemize}
\item Citadel Securities: 40-45\% retail market share
\item Virtu Financial: 25-30\%
\item Jane Street: 10-15\%
\item Two Sigma Securities: 5-10\%
\end{itemize}

\vspace{2mm}
\textbf{Volume:}
40-45\% of US equity volume is off-exchange (largely PFOF)

\column{0.48\textwidth}
\textbf{Controversies:}
\begin{itemize}
\item \textbf{Conflict of Interest:} Broker incentive vs best execution
\item \textbf{Market Segmentation:} Uninformed (retail) vs informed (institutional)
\item \textbf{Reduced Exchange Volume:} Less price discovery on lit markets
\item \textbf{GameStop (2021):} PFOF scrutiny after meme stock frenzy
\end{itemize}

\vspace{2mm}
\textbf{Regulatory Responses:}
\begin{itemize}
\item SEC considering PFOF ban (2022-2024 discussions)
\item MiFID II (EU): PFOF banned for equities (2018)
\item Enhanced Rule 606 disclosures (quarterly routing reports)
\item Best execution analysis requirements
\end{itemize}

\vspace{2mm}
\textit{Retail traders receive average 0.5-1 cent price improvement per share vs NBBO}
\end{columns}
\bottomnote{Understanding the process flow is key to identifying optimization opportunities.}
\end{frame}

\section{High-Frequency Trading}

\begin{frame}[t]{HFT Definition and Characteristics}
\begin{columns}[T]
\column{0.48\textwidth}
\textbf{HFT Defining Features:}
\begin{itemize}
\item \textbf{Ultra-Low Latency:} Microsecond execution speeds
\item \textbf{High Order-to-Trade Ratio:} 100:1 to 1000:1
\item \textbf{High Daily Volume:} Millions of shares/contracts
\item \textbf{Flat Overnight Positions:} Minimal directional risk
\item \textbf{Co-Location:} Servers at exchange data centers
\item \textbf{Direct Market Access:} Sponsored access or memberships
\end{itemize}

\vspace{2mm}
\textbf{Market Share:}
\begin{itemize}
\item US equities: 50-55\% of volume (down from 60-65\% in 2010)
\item Futures: 60-70\% of volume
\item FX: 20-30\% of spot volume
\item European equities: 30-40\% of volume
\end{itemize}

\column{0.48\textwidth}
\textbf{Technology Infrastructure:}
\begin{itemize}
\item \textbf{Co-Location Costs:} \$10k-50k/month per exchange
\item \textbf{Market Data Feeds:} Direct feeds vs consolidated (SIP)
\item \textbf{FPGA Acceleration:} Hardware-based order processing
\item \textbf{Microwave Networks:} Chicago-NYC in 4.1 milliseconds
\item \textbf{Kernel Bypass:} User-space networking (10-100x faster)
\end{itemize}

\vspace{2mm}
\textbf{Latency Benchmarks:}
\begin{itemize}
\item Matching engine: 10-50 microseconds
\item Co-located order entry: 50-200 microseconds
\item Cross-venue arbitrage window: 100-500 microseconds
\item Human reaction time: 200,000 microseconds (200 ms)
\end{itemize}
\end{columns}
\bottomnote{Clear definitions are essential for understanding complex technical concepts.}
\end{frame}

\begin{frame}[t]{HFT Strategies}
\begin{columns}[T]
\column{0.48\textwidth}
\textbf{Market Making:}
\begin{itemize}
\item Provide liquidity via limit orders
\item Profit from spread capture + maker rebates
\item Inventory risk management critical
\item Accounts for 40-50\% of HFT volume
\end{itemize}

\vspace{2mm}
\textbf{Statistical Arbitrage:}
\begin{itemize}
\item Mean-reversion strategies (tick-level)
\item Pairs trading at microsecond horizons
\item Order book imbalance signals
\item Typical hold time: seconds to minutes
\end{itemize}

\vspace{2mm}
\textbf{Latency Arbitrage:}
\begin{itemize}
\item Exploit slow price updates across venues
\item React to information faster than competitors
\item Requires fastest infrastructure
\item Controversial (``arms race'' criticism)
\end{itemize}

\column{0.48\textwidth}
\textbf{Event Arbitrage:}
\begin{itemize}
\item News parsing and trading (microseconds)
\item Economic data releases (e.g., NFP, FOMC)
\item Earnings announcements
\item Machine-readable news (MRN) feeds
\end{itemize}

\vspace{2mm}
\textbf{Order Anticipation:}
\begin{itemize}
\item Detect large institutional orders
\item Trade ahead of expected price impact
\item Legal gray area (vs illegal front-running)
\item Detects via order book patterns
\end{itemize}

\vspace{2mm}
\textbf{Index Arbitrage:}
\begin{itemize}
\item ETF vs underlying basket discrepancies
\item Futures vs cash basis trades
\item Dividend arbitrage around ex-dates
\item Holds positions minutes to hours
\end{itemize}
\end{columns}
\bottomnote{Key concepts from this slide inform practical applications in finance.}
\end{frame}

\begin{frame}[t]{HFT Market Impact Debate}
\begin{columns}[T]
\column{0.48\textwidth}
\textbf{Positive Contributions:}
\begin{itemize}
\item \textbf{Tighter Spreads:} Increased competition narrows bid-ask
\item \textbf{Increased Liquidity:} Higher quoted depth
\item \textbf{Faster Price Discovery:} Information incorporated quicker
\item \textbf{Lower Trading Costs:} Spreads down 50-70\% since 2000
\item \textbf{Cross-Market Integration:} Arbitrage keeps markets aligned
\end{itemize}

\vspace{2mm}
\textbf{Empirical Evidence (Brogaard et al., 2014):}
\begin{itemize}
\item HFT trades align with permanent price changes
\item Net provision of liquidity (market making)
\item Faster incorporation of public information
\item No evidence of systematic predatory behavior
\end{itemize}

\column{0.48\textwidth}
\textbf{Criticisms and Concerns:}
\begin{itemize}
\item \textbf{Adverse Selection:} Institutional orders picked off
\item \textbf{Phantom Liquidity:} Quotes vanish in stress periods
\item \textbf{Flash Crashes:} Amplify volatility (e.g., 2010)
\item \textbf{Unfair Advantage:} Speed and co-location benefits
\item \textbf{Socially Wasteful:} Arms race with low social value
\end{itemize}

\vspace{2mm}
\textbf{Market Stability Risks:}
\begin{itemize}
\item Simultaneous withdrawal during stress
\item Correlated algorithmic behavior
\item Feedback loops and cascades
\item Fragility due to speed dependencies
\end{itemize}

\vspace{2mm}
\textit{``HFT is like GPS: improves efficiency but introduces new failure modes''} -- SEC Commissioner (2014)
\end{columns}
\bottomnote{Key concepts from this slide inform practical applications in finance.}
\end{frame}

\section{Flash Crashes and Market Disruptions}

\begin{frame}[t]{May 6, 2010 Flash Crash}
\begin{columns}[T]
\column{0.48\textwidth}
\textbf{Event Timeline:}
\begin{itemize}
\item \textbf{14:32 ET:} Large mutual fund executes \$4.1B E-Mini sell program
\item \textbf{14:41-14:45:} Dow drops 600 points in 5 minutes
\item \textbf{14:45:} Markets recover 70\% of losses in minutes
\item \textbf{End of Day:} Dow closes down 348 points
\end{itemize}

\vspace{2mm}
\textbf{Trigger Mechanism:}
\begin{itemize}
\item Aggressive VWAP algorithm floods market
\item HFT firms initially absorb selling
\item Hot potato: HFTs trade among themselves
\item Liquidity withdrawal as inventory limits hit
\item Prices collapse due to lack of buyers
\end{itemize}

\column{0.48\textwidth}
\textbf{Key Findings (CFTC-SEC Report):}
\begin{itemize}
\item E-Mini futures led equities down
\item Cross-market arbitrage transmitted stress
\item Stub quotes executed (e.g., Accenture to \$0.01)
\item 20,000+ trades broken (clearly erroneous)
\item HFT exacerbated but did not cause crash
\end{itemize}

\vspace{2mm}
\textbf{Regulatory Responses:}
\begin{itemize}
\item \textbf{Limit Up-Limit Down (2012):} Single-stock circuit breakers
\item \textbf{Clearly Erroneous Trades:} Standardized break criteria
\item \textbf{Reg SCI (2015):} Systems compliance for critical infrastructure
\item \textbf{Market-Wide Breakers:} 7\%, 13\%, 20\% thresholds
\end{itemize}
\end{columns}
\bottomnote{Key concepts from this slide inform practical applications in finance.}
\end{frame}

\begin{frame}[t]{Other Notable Flash Events}
\begin{columns}[T]
\column{0.48\textwidth}
\textbf{October 15, 2014 Treasury Flash Rally:}
\begin{itemize}
\item 10-year yield drops 37 bps in 12 minutes
\item Largest intraday move in decades
\item No fundamental news trigger
\item Joint Staff Report: HFT amplified volatility
\item Highlighted fragility in Treasury market structure
\end{itemize}

\vspace{2mm}
\textbf{August 24, 2015 ETF Flash Crash:}
\begin{itemize}
\item 1100+ trading halts in first 36 minutes
\item 20\% of ETFs trade 10\%+ away from NAV
\item LULD breakers overwhelmed by volume
\item Exposing ETF market making fragility
\end{itemize}

\column{0.48\textwidth}
\textbf{GBP Flash Crash (October 7, 2016):}
\begin{itemize}
\item Sterling drops 9\% vs USD in minutes (Asian hours)
\item Thin liquidity + algorithmic selling spiral
\item Recovered within 30 minutes
\item Highlighted FX market vulnerabilities
\end{itemize}

\vspace{2mm}
\textbf{Common Patterns:}
\begin{itemize}
\item Initial shock (fundamental or algorithmic)
\item Liquidity provider withdrawal
\item Cascading sell orders (stop losses, algos)
\item Feedback loop amplification
\item Recovery once human oversight intervenes
\end{itemize}

\vspace{2mm}
\textit{VPIN (toxicity indicator) spiked to 0.98 before 2010 Flash Crash (normal < 0.5)}
\end{columns}
\bottomnote{Key concepts from this slide inform practical applications in finance.}
\end{frame}

\begin{frame}[t]{Market Stability Mechanisms}
\begin{columns}[T]
\column{0.48\textwidth}
\textbf{Circuit Breakers (US Equities):}
\begin{itemize}
\item \textbf{LULD Bands:} 5-20\% depending on tier and time
\item \textbf{Tier 1:} S\&P 500, Russell 1000 (5\% bands)
\item \textbf{Tier 2:} Other NMS stocks (10\% bands)
\item \textbf{Trading Halt:} 5-minute pause if limit breached
\item \textbf{Market-Wide:} 7\%, 13\%, 20\% S\&P 500 declines
\end{itemize}

\vspace{2mm}
\textbf{Clearly Erroneous Trades:}
\begin{itemize}
\item Numerical thresholds for breaking trades
\item \$0-\$25 stocks: 10\% from reference price
\item \$25-\$50: 5\%, Over \$50: 3\%
\item Must be reported within 30 minutes
\item Exchange decision (not automatic)
\end{itemize}

\column{0.48\textwidth}
\textbf{Kill Switch Requirements (MiFID II):}
\begin{itemize}
\item Ability to cancel all orders in under 2 seconds
\item Mandatory for algorithmic traders
\item Pre-trade risk controls on parameters
\item Post-trade monitoring and alerts
\end{itemize}

\vspace{2mm}
\textbf{Volatility Auctions:}
\begin{itemize}
\item European markets use volatility interruptions
\item 2-5 minute random auction when thresholds hit
\item Allows human intervention and price discovery
\item Less disruptive than hard halts
\end{itemize}

\vspace{2mm}
\textbf{Speed Bumps:}
\begin{itemize}
\item IEX: 350-microsecond delay on all orders
\item Prevents latency arbitrage strategies
\item Controversial: reduces efficiency vs stability
\end{itemize}
\end{columns}
\bottomnote{Key concepts from this slide inform practical applications in finance.}
\end{frame}

\section{Empirical Market Microstructure}

\begin{frame}[t]{Intraday Patterns and Anomalies}
\begin{columns}[T]
\column{0.48\textwidth}
\textbf{U-Shaped Volume Pattern:}
\begin{itemize}
\item High volume at open (09:30-10:00 ET)
\item Low volume midday (11:00-14:00)
\item High volume at close (15:30-16:00)
\item Opening 30 min: 15-20\% of daily volume
\item Closing 30 min: 20-25\% of daily volume
\end{itemize}

\vspace{2mm}
\textbf{U-Shaped Volatility:}
\begin{itemize}
\item Spreads widest at open (information asymmetry)
\item Tightest spreads midday (steady state)
\item Widening into close (position squaring)
\end{itemize}

\column{0.48\textwidth}
\textbf{Bid-Ask Bounce:}
\begin{itemize}
\item Trades alternate between bid and ask
\item Induces negative autocorrelation in returns
\item Roll (1984) model: $\text{Spread} = 2\sqrt{-\text{Cov}(r_t, r_{t-1})}$
\item Empirically: first-order autocorr = -0.05 to -0.15
\end{itemize}

\vspace{2mm}
\textbf{Price Impact Asymmetry:}
\begin{itemize}
\item Buy orders have larger impact than sells
\item More pronounced for small-cap stocks
\item Attributed to short-sale constraints
\item Temporary component decays exponentially
\end{itemize}

\vspace{2mm}
\textit{Opening cross (NYSE, Nasdaq): 5-10\% of daily volume in single batch auction}
\end{columns}
\bottomnote{Key concepts from this slide inform practical applications in finance.}
\end{frame}

\begin{frame}[t]{Tick Size and Market Quality}
\begin{columns}[T]
\column{0.48\textwidth}
\textbf{US Tick Size Evolution:}
\begin{itemize}
\item Pre-2001: 1/16 dollar (\$0.0625)
\item 2001: Decimalization to \$0.01
\item Result: Spreads compressed 30-50\%
\item But: Depth per price level declined
\item Trade-off: Tighter spreads vs lower depth
\end{itemize}

\vspace{2mm}
\textbf{Tick Size Pilot (2016-2018):}
\begin{itemize}
\item SEC mandated \$0.05 tick for 1200 small-cap stocks
\item Goal: Improve liquidity and market making economics
\item Results: Wider spreads, lower volume, minimal IPO impact
\item Pilot ended; no permanent changes adopted
\end{itemize}

\column{0.48\textwidth}
\textbf{Optimal Tick Size Theory:}
\begin{itemize}
\item Too small: Excessive queue jumping, minimal profit
\item Too large: Constrained price discovery, wide spreads
\item Optimal tick proportional to stock price
\item Harris (1994): Tick-to-price ratio = 0.1-0.5\%
\end{itemize}

\vspace{2mm}
\textbf{Empirical Regularities:}
\begin{itemize}
\item Quotes cluster at round numbers
\item Spreads often exactly 1 tick (minimum)
\item 60-70\% of stocks trade at 1-cent spread
\item Sub-penny trading banned for most stocks (Reg NMS)
\end{itemize}

\vspace{2mm}
\textit{European markets: varying tick sizes by price range (MiFID II tick size regime)}
\end{columns}
\bottomnote{Key concepts from this slide inform practical applications in finance.}
\end{frame}

\begin{frame}[t]{Summary and Key Takeaways}
\begin{columns}[T]
\column{0.48\textwidth}
\textbf{Microstructure Foundations:}
\begin{itemize}
\item Bid-ask spreads compensate for order processing, inventory, and adverse selection
\item Market makers provide liquidity and earn spread via inventory management
\item Information asymmetry drives adverse selection costs
\item PFOF internalizes retail flow (40-45\% of US volume)
\end{itemize}

\vspace{2mm}
\textbf{High-Frequency Trading:}
\begin{itemize}
\item 50-55\% of US equity volume (microsecond speeds)
\item Strategies: market making, stat arb, latency arb
\item Tighter spreads but phantom liquidity concerns
\item Regulatory focus on stability and fairness
\end{itemize}

\column{0.48\textwidth}
\textbf{Flash Crashes and Stability:}
\begin{itemize}
\item May 6, 2010: 600-point drop in 5 minutes
\item Liquidity withdrawal amplifies shocks
\item LULD circuit breakers now standard
\item Kill switches and risk controls mandatory (MiFID II, Reg SCI)
\end{itemize}

\vspace{2mm}
\textbf{Market Quality:}
\begin{itemize}
\item U-shaped volume and volatility patterns
\item Decimalization tightened spreads (2001)
\item Tick size pilot failed to improve small-cap liquidity
\item Trade-off: efficiency vs stability in design
\end{itemize}
\end{columns}
\end{frame}

\end{document}
