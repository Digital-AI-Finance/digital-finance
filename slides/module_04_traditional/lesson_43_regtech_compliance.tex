\documentclass[8pt,aspectratio=169]{beamer}
\usetheme{Madrid}
\usepackage{graphicx,booktabs,adjustbox,multicol,amsmath,amssymb}
\definecolor{mlblue}{RGB}{0,102,204}
\definecolor{mlpurple}{RGB}{51,51,178}
\definecolor{mllavender}{RGB}{173,173,224}
\definecolor{mllavender2}{RGB}{193,193,232}
\definecolor{mllavender3}{RGB}{204,204,235}
\definecolor{mllavender4}{RGB}{214,214,239}
\definecolor{mlorange}{RGB}{255,127,14}
\definecolor{mlgreen}{RGB}{44,160,44}
\definecolor{mlred}{RGB}{214,39,40}
\setbeamercolor{palette primary}{bg=mllavender3,fg=mlpurple}
\setbeamercolor{palette secondary}{bg=mllavender2,fg=mlpurple}
\setbeamercolor{palette tertiary}{bg=mllavender,fg=white}
\setbeamercolor{structure}{fg=mlpurple}
\setbeamercolor{frametitle}{fg=mlpurple,bg=mllavender3}
\setbeamertemplate{navigation symbols}{}
\setbeamertemplate{itemize items}[circle]
\setbeamersize{text margin left=5mm,text margin right=5mm}

% Bottom note command for key takeaways
\newcommand{\bottomnote}[1]{%
\vfill
\vspace{-2mm}
\textcolor{mllavender2}{\rule{\textwidth}{0.4pt}}
\vspace{1mm}
\footnotesize
\textbf{#1}
}
\title{Lesson 43: RegTech and Compliance}
\subtitle{Module 4: Traditional Digital Finance}
\author{Digital Finance Course}
\date{2025}

\begin{document}

\begin{frame}
\titlepage
\end{frame}

\begin{frame}[t]{Learning Objectives}
\begin{itemize}
\item Understand RegTech scope and technology applications
\item Analyze Basel III capital and liquidity requirements
\item Examine IFRS 9 expected credit loss implementation
\item Evaluate regulatory reporting automation (EMIR, MiFID II)
\item Assess AML/KYC technology and transaction monitoring
\end{itemize}
\end{frame}

\section{RegTech Overview}

\begin{frame}[t]{RegTech Definition and Scope}
\begin{columns}[T]
\scriptsize
\column{0.48\textwidth}
\textbf{RegTech Definition:}

\textit{Use of technology (especially information technology) to enhance regulatory processes, compliance, and reporting.}

\vspace{2mm}
\textbf{Key Drivers:}
\begin{itemize}
\item Post-2008 regulatory explosion (Dodd-Frank, MiFID II, EMIR)
\item Rising compliance costs (10-15\% of bank operating costs)
\item Manual processes prone to errors and delays
\item Regulatory demands for real-time reporting
\item Availability of AI/ML, cloud, and big data tools
\end{itemize}

\column{0.48\textwidth}
\textbf{RegTech Applications:}
\begin{enumerate}
\item \textbf{Regulatory Reporting:} Automated data extraction and submission
\item \textbf{Risk Management:} Real-time risk analytics and stress testing
\item \textbf{Compliance:} Policy enforcement and monitoring
\item \textbf{Identity Management:} KYC/AML automation
\item \textbf{Transaction Monitoring:} Fraud and market abuse detection
\item \textbf{Regulatory Intelligence:} Track and interpret rule changes
\end{enumerate}

\vspace{2mm}
\textbf{Market Size:}
\begin{itemize}
\item Global RegTech market: \$12B (2023), projected \$50B+ by 2030
\item 20-30\% CAGR driven by regulatory complexity
\end{itemize}
\end{columns}
\bottomnote{Clear definitions are essential for understanding complex technical concepts.}
\end{frame}

\begin{frame}[t]{RegTech Technology Stack}
\begin{columns}[T]
\scriptsize
\column{0.48\textwidth}
\textbf{Core Technologies:}
\begin{itemize}
\item \textbf{Machine Learning:} AML pattern detection, risk scoring
\item \textbf{Natural Language Processing:} Regulatory text parsing, contract analysis
\item \textbf{Robotic Process Automation (RPA):} Data extraction from legacy systems
\item \textbf{Cloud Computing:} Scalable infrastructure, vendor solutions
\item \textbf{Blockchain:} Immutable audit trails, shared KYC utilities
\item \textbf{Big Data:} Transaction monitoring across terabytes
\end{itemize}

\column{0.48\textwidth}
\textbf{Leading RegTech Vendors:}
\begin{itemize}
\item \textbf{Compliance:} ComplyAdvantage, Chainalysis, Elliptic
\item \textbf{Regulatory Reporting:} Wolters Kluwer, Moody's Analytics, ABIDE Financial
\item \textbf{KYC/AML:} Refinitiv World-Check, LexisNexis, Trulioo
\item \textbf{Transaction Monitoring:} NICE Actimize, SAS, Feedzai
\item \textbf{Risk Analytics:} Axiom, Quantexa, Ayasdi
\end{itemize}

\vspace{2mm}
\textbf{Build vs Buy:}
\begin{itemize}
\item Large banks: Hybrid (build core, buy specialized)
\item Regional banks: Primarily vendor solutions
\item Fintechs: Cloud-native RegTech-as-a-Service
\end{itemize}
\end{columns}
\bottomnote{Key concepts from this slide inform practical applications in finance.}
\end{frame}

\section{Basel III Capital Framework}

\begin{frame}[t]{Basel III Overview}
\begin{columns}[T]
\scriptsize
\column{0.48\textwidth}
\textbf{Historical Context:}
\begin{itemize}
\item \textbf{Basel I (1988):} Simple risk weights by asset class
\item \textbf{Basel II (2004):} Internal models, three pillars
\item \textbf{Basel III (2010):} Post-crisis reforms
\item \textbf{Basel III Finalization (2017):} Output floor, standardized approach revisions
\end{itemize}

\vspace{2mm}
\textbf{Three Pillars:}
\begin{enumerate}
\item \textbf{Minimum Capital:} CET1, Tier 1, Total Capital ratios
\item \textbf{Supervisory Review:} Stress testing, Pillar 2 add-ons
\item \textbf{Market Discipline:} Public disclosure requirements
\end{enumerate}

\column{0.48\textwidth}
\textbf{Capital Adequacy Ratios:}
\begin{itemize}
\item \textbf{CET1 Ratio:} $\geq$ 4.5\% (Core equity / RWA)
\item \textbf{Tier 1 Ratio:} $\geq$ 6\% (CET1 + AT1 / RWA)
\item \textbf{Total Capital Ratio:} $\geq$ 8\% (Tier 1 + Tier 2 / RWA)
\item \textbf{Capital Conservation Buffer:} 2.5\% above minimums
\item \textbf{Countercyclical Buffer:} 0-2.5\% (jurisdiction-specific)
\item \textbf{G-SIB Surcharge:} 1-3.5\% for systemically important banks
\end{itemize}

\vspace{2mm}
\textbf{Effective CET1 Requirement (G-SIB):}
$$4.5\% + 2.5\% + 1\% + 2.5\% = 10.5\% \text{ CET1}$$
(Minimum + Conservation + Countercyclical + G-SIB)
\end{columns}
\end{frame}

\begin{frame}[t]{Risk-Weighted Assets (RWA) Calculation}
\begin{columns}[T]
\scriptsize
\column{0.48\textwidth}
\textbf{Credit Risk RWA:}

\textbf{Standardized Approach:}
\begin{itemize}
\item Fixed risk weights by exposure class
\item AAA-AA: 20\%, A+/A: 50\%, BBB+/BBB-: 100\%
\item Residential mortgages: 35\% (low LTV)
\item Corporate: 100\% (unrated)
\item Sovereign: 0\% (OECD), 100\% (others)
\end{itemize}

\textbf{IRB Approach (Internal Ratings-Based):}
$$\text{RWA} = K \times 12.5 \times \text{EAD}$$
where $K$ = capital requirement function of PD, LGD, M

{\scriptsize
$K = [\text{LGD} \times N(\sqrt{\frac{1}{1-R}} N^{-1}(\text{PD}) + \sqrt{\frac{R}{1-R}} N^{-1}(0.999)) - \text{PD} \times \text{LGD}] \times (1 + (M-2.5)b)$
}

\vspace{1mm}
{\tiny where $R$ = correlation, $M$ = maturity}

\column{0.48\textwidth}
\textbf{Market Risk RWA:}
\begin{itemize}
\item Standardized Approach (SA): Sensitivity-based
\item Internal Models Approach (IMA): Expected Shortfall
\item FRTB (Fundamental Review): $\text{RWA} = 12.5 \times \text{Capital}$
\end{itemize}

\vspace{2mm}
\textbf{Operational Risk RWA (Standardized):}
$$\text{OR Capital} = \text{BIC} \times \text{ILM}$$
where BIC = Business Indicator Component, ILM = Internal Loss Multiplier

\vspace{2mm}
\textbf{Output Floor (Basel III Finalization):}
$$\text{RWA}_{IRB} \geq 72.5\% \times \text{RWA}_{Standardized}$$

\textit{Limits internal model benefit, effective January 2023 (phased to 2028)}
\end{columns}
\end{frame}

\begin{frame}[t]{Liquidity Requirements: LCR and NSFR}
\begin{columns}[T]
\scriptsize
\column{0.48\textwidth}
\textbf{Liquidity Coverage Ratio (LCR):}

$$\text{LCR} = \frac{\text{High-Quality Liquid Assets}}{\text{Net Cash Outflows (30-day stress)}} \geq 100\%$$

\vspace{2mm}
\textbf{HQLA Categories:}
\begin{itemize}
\item \textbf{Level 1:} Cash, central bank reserves, sovereign debt (0\% haircut)
\item \textbf{Level 2A:} High-quality corporate/covered bonds (15\% haircut)
\item \textbf{Level 2B:} Lower-rated corporates, equities (50\% haircut)
\item \textbf{Cap:} Level 2 max 40\% of HQLA, Level 2B max 15\%
\end{itemize}

\vspace{2mm}
\textbf{Net Cash Outflows:}
\begin{itemize}
\item Retail deposits: 3-10\% runoff (stable to less stable)
\item Wholesale deposits: 25-100\% runoff
\item Committed facilities: 30-100\% drawdown
\item Derivatives collateral calls
\end{itemize}

\column{0.48\textwidth}
\textbf{Net Stable Funding Ratio (NSFR):}

$$\text{NSFR} = \frac{\text{Available Stable Funding}}{\text{Required Stable Funding}} \geq 100\%$$

\vspace{2mm}
\textbf{ASF Factors (by liability type):}
\begin{itemize}
\item Equity, long-term debt (>1 year): 100\%
\item Stable retail deposits: 95\%
\item Less stable retail, SME deposits: 90\%
\item Wholesale deposits (<1 year): 50\%
\item Short-term wholesale (<6 months): 0\%
\end{itemize}

\vspace{2mm}
\textbf{RSF Factors (by asset type):}
\begin{itemize}
\item Cash, reserves: 0\%
\item Sovereign bonds (< 6m maturity): 5\%
\item High-quality bonds: 10-15\%
\item Residential mortgages: 65\%
\item Corporate loans, other assets: 85-100\%
\end{itemize}
\end{columns}
\bottomnote{Key concepts from this slide inform practical applications in finance.}
\end{frame}

\section{IFRS 9 Expected Credit Loss}

\begin{frame}[t]{IFRS 9 Overview and ECL Model}
\begin{columns}[T]
\scriptsize
\column{0.48\textwidth}
\textbf{Key Changes from IAS 39:}
\begin{itemize}
\item \textbf{Incurred Loss $\rightarrow$ Expected Loss:} Forward-looking provisioning
\item \textbf{3-Stage Model:} Based on credit deterioration
\item \textbf{Lifetime ECL:} For Stage 2 and 3 assets
\item \textbf{Effective Date:} January 1, 2018 (EU/IFRS jurisdictions)
\end{itemize}

\vspace{2mm}
\textbf{Three-Stage Classification:}
\begin{enumerate}
\item \textbf{Stage 1 (Performing):} 12-month ECL
\begin{itemize}
\item No significant credit deterioration since origination
\item Interest revenue on gross carrying amount
\end{itemize}

\item \textbf{Stage 2 (Underperforming):} Lifetime ECL
\begin{itemize}
\item Significant increase in credit risk (SICR)
\item Not yet credit-impaired
\item Interest on gross carrying amount
\end{itemize}
\end{enumerate}

\column{0.48\textwidth}
\textbf{3. Stage 3 (Non-Performing):} Lifetime ECL
\begin{itemize}
\item Credit-impaired (objective evidence of default)
\item Typically 90+ days past due
\item Interest on net carrying amount (after provisions)
\end{itemize}

\vspace{2mm}
\textbf{ECL Formula:}
$$\text{ECL} = \text{PD} \times \text{LGD} \times \text{EAD}$$

\textbf{Stage 1:} 12-month PD \\
\textbf{Stage 2/3:} Lifetime PD (sum over maturity)

$$\text{Lifetime ECL} = \sum_{t=1}^{T} \text{PD}_t \times \text{LGD}_t \times \text{EAD}_t \times \text{DF}_t$$

where $\text{DF}_t$ = discount factor, $T$ = contractual maturity
\end{columns}
\bottomnote{Key concepts from this slide inform practical applications in finance.}
\end{frame}

\begin{frame}[t]{SICR Criteria and Implementation}
\begin{columns}[T]
\scriptsize
\column{0.48\textwidth}
\textbf{Significant Increase in Credit Risk (SICR):}

\textbf{Quantitative Triggers:}
\begin{itemize}
\item Absolute change in PD (e.g., +200 bps)
\item Relative change in PD (e.g., 2x origination PD)
\item 30+ days past due (rebuttable backstop)
\item Internal rating downgrade (3+ notches)
\end{itemize}

\vspace{2mm}
\textbf{Qualitative Indicators:}
\begin{itemize}
\item Forbearance or restructuring
\item Watchlist/early warning flags
\item Significant financial difficulty
\item Covenant breaches
\item Macroeconomic deterioration in sector
\end{itemize}

\column{0.48\textwidth}
\textbf{Implementation Challenges:}
\begin{itemize}
\item \textbf{Data Requirements:} Origination PD, lifetime PD curves
\item \textbf{Model Development:} PD, LGD, EAD models for each portfolio
\item \textbf{Forward-Looking Information:} Macroeconomic scenarios
\item \textbf{Systems:} Calculate ECL for millions of exposures
\item \textbf{Governance:} SICR criteria approval and monitoring
\end{itemize}

\vspace{2mm}
\textbf{Technology Solutions:}
\begin{itemize}
\item \textbf{Moody's Analytics:} CreditLens, RiskCalc
\item \textbf{SAS:} Expected Credit Loss solution
\item \textbf{Wolters Kluwer:} OneSumX for Finance, Risk \& Regulatory Reporting
\item \textbf{Oracle:} IFRS 9 ECL module (OFSAA)
\end{itemize}
\end{columns}
\bottomnote{Key concepts from this slide inform practical applications in finance.}
\end{frame}

\begin{frame}[t]{Macroeconomic Scenarios and Probability Weighting}
\begin{columns}[T]
\scriptsize
\column{0.48\textwidth}
\textbf{Forward-Looking Scenarios:}

IFRS 9 requires incorporating reasonable and supportable macroeconomic forecasts.

\vspace{2mm}
\textbf{Typical Scenario Framework:}
\begin{itemize}
\item \textbf{Base Case (50-60\% weight):} Consensus forecast (GDP, unemployment, property prices)
\item \textbf{Upside (15-25\% weight):} Optimistic economic conditions
\item \textbf{Downside (15-30\% weight):} Recession or stress scenario
\end{itemize}

\vspace{2mm}
\textbf{Key Macroeconomic Variables:}
\begin{itemize}
\item GDP growth
\item Unemployment rate
\item Interest rates (policy rate, term structure)
\item Property prices (residential, commercial)
\item Equity indices
\item Commodity prices (for relevant sectors)
\end{itemize}

\column{0.48\textwidth}
\textbf{Probability-Weighted ECL:}
$$\text{ECL} = \sum_{i=1}^{n} p_i \times \text{ECL}_i$$
where $p_i$ = scenario probability, $\text{ECL}_i$ = ECL under scenario $i$

\vspace{2mm}
\textbf{Example (Mortgage Portfolio):}
\begin{center}
\footnotesize
\begin{tabular}{lrr}
\textbf{Scenario} & \textbf{Prob.} & \textbf{ECL (bps)} \\
\midrule
Upside & 20\% & 15 \\
Base & 60\% & 30 \\
Downside & 20\% & 80 \\
\midrule
\textbf{Weighted ECL} & & \textbf{36 bps} \\
\end{tabular}
\end{center}

\vspace{2mm}
\textbf{COVID-19 Impact (2020):}
\begin{itemize}
\item Banks increased downside weights to 30-50\%
\item ECL provisions doubled or tripled
\item Subsequent releases as economies recovered (2021-2022)
\end{itemize}
\end{columns}
\bottomnote{Key concepts from this slide inform practical applications in finance.}
\end{frame}

\section{Regulatory Reporting Automation}

\begin{frame}[t]{EMIR Trade Reporting}
\begin{columns}[T]
\scriptsize
\column{0.48\textwidth}
\textbf{European Market Infrastructure Regulation (EMIR):}

\textbf{Objectives:}
\begin{itemize}
\item Increase derivatives market transparency
\item Reduce systemic risk via central clearing
\item Standardize OTC derivatives
\end{itemize}

\vspace{2mm}
\textbf{Reporting Obligations:}
\begin{itemize}
\item \textbf{Scope:} All derivatives (OTC and exchange-traded)
\item \textbf{Counterparties:} Both sides report (or delegate to Trade Repository)
\item \textbf{Timing:} T+1 (next working day)
\item \textbf{Trade Repositories:} DTCC, Regis-TR, UnaVista, etc.
\end{itemize}

\column{0.48\textwidth}
\textbf{Reporting Fields (EMIR REFIT):}
\begin{itemize}
\item 203 fields per trade (effective 2024)
\item Counterparty identifiers (LEI mandatory)
\item Trade economics (notional, price, maturity)
\item Collateral and margin details
\item Valuation and lifecycle events (novations, compressions)
\end{itemize}

\vspace{2mm}
\textbf{Technology Challenges:}
\begin{itemize}
\item Data extraction from multiple systems
\item LEI management and validation
\item Reconciliation between counterparties (UTI matching)
\item Error handling and resubmissions
\item Regulatory feedback and breaks
\end{itemize}

\vspace{2mm}
\textit{Industry-wide reconciliation break rate: 5-15\% of trades (falling over time)}
\end{columns}
\bottomnote{Key concepts from this slide inform practical applications in finance.}
\end{frame}

\begin{frame}[t]{MiFID II Reporting and Transaction Monitoring}
\begin{columns}[T]
\scriptsize
\column{0.48\textwidth}
\textbf{MiFID II Transaction Reporting:}

\textbf{Scope:}
\begin{itemize}
\item All transactions in financial instruments (equities, bonds, derivatives)
\item Executed on EU venues or by EU firms
\item Reported to national competent authorities (NCAs)
\end{itemize}

\vspace{2mm}
\textbf{Reporting Timeline:}
\begin{itemize}
\item \textbf{T+1:} Next working day by end-of-day
\item \textbf{Real-Time:} Pre/post-trade transparency (lit venues)
\end{itemize}

\vspace{2mm}
\textbf{Data Fields (65+ fields):}
\begin{itemize}
\item Instrument identifiers (ISIN, MIC, CFI)
\item Client identifiers (national ID, LEI)
\item Execution details (price, quantity, timestamp)
\item Flags (waiver, algo, short selling, commodities derivative)
\end{itemize}

\column{0.48\textwidth}
\textbf{Best Execution Reporting (RTS 27/28):}
\begin{itemize}
\item \textbf{RTS 27:} Venues publish execution quality statistics
\item \textbf{RTS 28:} Firms disclose top 5 venues by asset class
\item Quarterly publication requirement
\end{itemize}

\vspace{2mm}
\textbf{Market Abuse Detection:}
\begin{itemize}
\item Suspicious Transaction and Order Reports (STORs)
\item Algorithmic trading flags and identifiers
\item High-frequency trading (HFT) identification
\item Order book reconstruction for surveillance
\end{itemize}

\vspace{2mm}
\textbf{Vendor Solutions:}
\begin{itemize}
\item FCA's GRID (regulatory data platform)
\item Cappitech, ABIDE Financial, Primatics
\item Managed reporting services (outsourced compliance)
\end{itemize}
\end{columns}
\bottomnote{Key concepts from this slide inform practical applications in finance.}
\end{frame}

\begin{frame}[t]{US Regulatory Reporting (Dodd-Frank)}
\begin{columns}[T]
\scriptsize
\column{0.48\textwidth}
\textbf{Dodd-Frank Swap Data Reporting:}

\textbf{Reporting Entities:}
\begin{itemize}
\item Swap Dealers (SDs) and Major Swap Participants (MSPs)
\item Report to Swap Data Repositories (SDRs)
\item DTCC (credit, equity, rates), CME (commodities, FX)
\end{itemize}

\vspace{2mm}
\textbf{Real-Time Public Dissemination:}
\begin{itemize}
\item Block trades: 15-minute delay
\item Non-block: Real-time (as soon as technologically practicable)
\item Capped dissemination to protect counterparty identity
\end{itemize}

\vspace{2mm}
\textbf{CFTC Part 45 Reporting:}
\begin{itemize}
\item Primary Economic Terms (PET)
\item Continuation data (valuation, collateral)
\item Lifecycle events (assignments, terminations)
\end{itemize}

\column{0.48\textwidth}
\textbf{CAT (Consolidated Audit Trail):}

\textbf{Objectives:}
\begin{itemize}
\item Track all equity and options orders across US markets
\item Reconstruct market events for surveillance
\item Detect market manipulation and insider trading
\end{itemize}

\vspace{2mm}
\textbf{Reporting Requirements:}
\begin{itemize}
\item All SRO members and broker-dealers
\item Report customer and proprietary orders
\item Lifecycle: receipt, routing, execution, allocation
\item Timestamps: Millisecond granularity
\end{itemize}

\vspace{2mm}
\textbf{Implementation Challenges:}
\begin{itemize}
\item 58 billion records per day (estimated)
\item Data privacy concerns (customer PII)
\item Cybersecurity of centralized database
\item Delayed go-live (originally 2017, phased 2020-2024)
\end{itemize}
\end{columns}
\bottomnote{Key concepts from this slide inform practical applications in finance.}
\end{frame}

\section{AML and KYC Technology}

\begin{frame}[t]{KYC and Customer Due Diligence}
\begin{columns}[T]
\scriptsize
\column{0.48\textwidth}
\textbf{KYC Requirements (FATF, EU 4AMLD/5AMLD):}

\textbf{Customer Identification:}
\begin{itemize}
\item Full name, date of birth, address
\item Government-issued ID verification
\item Beneficial ownership (25\%+ threshold)
\item Source of funds and wealth
\end{itemize}

\vspace{2mm}
\textbf{Risk-Based Approach:}
\begin{itemize}
\item \textbf{Simplified Due Diligence (SDD):} Low-risk customers (e.g., domestic individuals, small balances)
\item \textbf{Customer Due Diligence (CDD):} Standard risk
\item \textbf{Enhanced Due Diligence (EDD):} High-risk (PEPs, high-risk jurisdictions, correspondent banking)
\end{itemize}

\column{0.48\textwidth}
\textbf{Technology Solutions:}
\begin{itemize}
\item \textbf{Digital Identity Verification:} Jumio, Onfido, Trulioo
\begin{itemize}
\item Document scanning and OCR
\item Facial recognition and liveness detection
\item Biometric authentication
\end{itemize}

\item \textbf{Screening Databases:}
\begin{itemize}
\item Sanctions lists (OFAC, UN, EU)
\item PEP databases (Politically Exposed Persons)
\item Adverse media screening
\item Refinitiv World-Check, Dow Jones Risk \& Compliance
\end{itemize}

\item \textbf{Utility Models:}
\begin{itemize}
\item Shared KYC platforms (e.g., SWIFT KYC Registry)
\item Reduce duplication across institutions
\item Blockchain-based KYC consortia (pilot stage)
\end{itemize}
\end{itemize}
\end{columns}
\bottomnote{Key concepts from this slide inform practical applications in finance.}
\end{frame}

\begin{frame}[t]{Transaction Monitoring and AML}
\begin{columns}[T]
\scriptsize
\column{0.48\textwidth}
\textbf{AML Transaction Monitoring:}

\textbf{Typologies Detected:}
\begin{itemize}
\item \textbf{Structuring (Smurfing):} Multiple small deposits under reporting threshold
\item \textbf{Rapid Movement:} Funds in and out within short period
\item \textbf{Round-Tripping:} Circular fund flows
\item \textbf{Layering:} Complex web of transactions to obscure origin
\item \textbf{High-Risk Jurisdictions:} Transfers to/from sanctioned countries
\item \textbf{Unusual Patterns:} Deviations from customer profile
\end{itemize}

\vspace{2mm}
\textbf{Rule-Based Systems:}
\begin{itemize}
\item Threshold-based alerts (e.g., cash deposit > \$10k)
\item Velocity rules (e.g., 5+ wire transfers in 24 hours)
\item High false positive rates: 95-99\% of alerts
\end{itemize}

\column{0.48\textwidth}
\textbf{Machine Learning Approaches:}
\begin{itemize}
\item \textbf{Supervised Learning:} Train on historical SAR filings
\item \textbf{Unsupervised Learning:} Anomaly detection (clustering, autoencoders)
\item \textbf{Network Analysis:} Graph algorithms to detect suspicious networks
\item \textbf{NLP:} Adverse media and document analysis
\end{itemize}

\vspace{2mm}
\textbf{Benefits of ML:}
\begin{itemize}
\item Reduce false positives by 30-70\%
\item Detect novel typologies (zero-day AML)
\item Prioritize high-risk alerts for investigators
\item Adapt to evolving criminal tactics
\end{itemize}

\vspace{2mm}
\textbf{Regulatory Acceptance:}
\begin{itemize}
\item UK FCA: Supportive but requires explainability
\item US FinCEN: Pilot programs encouraged
\item Model validation and governance critical
\end{itemize}
\end{columns}
\bottomnote{AI and ML are transforming financial services through automation and prediction.}
\end{frame}

\begin{frame}[t]{Sanctions Screening and Trade Surveillance}
\begin{columns}[T]
\scriptsize
\column{0.48\textwidth}
\textbf{Sanctions Screening:}

\textbf{Screening Points:}
\begin{itemize}
\item \textbf{Onboarding:} Customer and beneficiary names
\item \textbf{Real-Time Payments:} SWIFT messages, wire transfers
\item \textbf{Trade Finance:} Parties in LC and documentary collections
\item \textbf{Securities:} Issuer and counterparty screening
\end{itemize}

\vspace{2mm}
\textbf{Sanctions Lists:}
\begin{itemize}
\item \textbf{OFAC (US):} SDN list (6000+ entities), sectoral sanctions
\item \textbf{UN:} Security Council consolidated list
\item \textbf{EU:} CFSP sanctions, national lists
\item \textbf{Others:} UK HMT, Canada, Australia
\end{itemize}

\vspace{2mm}
\textbf{Fuzzy Matching:}
\begin{itemize}
\item Name variations, transliterations, typos
\item Phonetic algorithms (Soundex, Metaphone)
\item Machine learning name-matching engines
\end{itemize}

\column{0.48\textwidth}
\textbf{Trade Surveillance (Market Abuse):}

\textbf{Surveillance Patterns:}
\begin{itemize}
\item \textbf{Insider Trading:} Abnormal trading before announcements
\item \textbf{Market Manipulation:} Spoofing, layering, wash trades
\item \textbf{Front-Running:} Broker trades ahead of client orders
\item \textbf{Pump and Dump:} Artificially inflate price, sell at peak
\end{itemize}

\vspace{2mm}
\textbf{Technology Solutions:}
\begin{itemize}
\item \textbf{NICE Actimize:} Cross-asset surveillance
\item \textbf{SAS:} AML and trade surveillance
\item \textbf{IPC Connexus:} Voice and eComms surveillance
\item \textbf{Behavox:} AI-driven conduct risk monitoring
\end{itemize}

\vspace{2mm}
\textbf{Data Inputs:}
\begin{itemize}
\item Order and trade data (CAT, blue sheets)
\item Communications (emails, chats, voice)
\item Employee trading (personal account dealing)
\item Market announcements and news
\end{itemize}
\end{columns}
\bottomnote{Key concepts from this slide inform practical applications in finance.}
\end{frame}

\section{Emerging RegTech Trends}

\begin{frame}[t]{SupTech and Machine-Readable Regulation}
\begin{columns}[T]
\scriptsize
\column{0.48\textwidth}
\textbf{SupTech (Supervisory Technology):}

\textit{Use of technology by regulators to enhance supervision and surveillance.}

\vspace{2mm}
\textbf{Applications:}
\begin{itemize}
\item \textbf{Data Analytics:} Identify outliers and systemic risks
\item \textbf{Virtual Assistants:} Chatbots for regulated entity queries
\item \textbf{Real-Time Monitoring:} Dashboards of market activity
\item \textbf{Network Analysis:} Systemic risk mapping (inter-bank exposures)
\item \textbf{NLP:} Analyze disclosure documents at scale
\end{itemize}

\vspace{2mm}
\textbf{Examples:}
\begin{itemize}
\item \textbf{FCA (UK):} Data and Analytics Hub for market surveillance
\item \textbf{MAS (Singapore):} SupTech initiatives (API-based data collection)
\item \textbf{ESMA (EU):} FIRDS (Financial Instruments Reference Data System)
\end{itemize}

\column{0.48\textwidth}
\textbf{Machine-Readable Regulation:}

\textbf{Concept:}
\begin{itemize}
\item Translate regulatory rules into code
\item Automate compliance checks via smart contracts or rule engines
\item Reduce interpretation ambiguity
\end{itemize}

\vspace{2mm}
\textbf{Initiatives:}
\begin{itemize}
\item \textbf{BIS Innovation Hub:} Project Genesis (machine-executable regulations)
\item \textbf{FCA Digital Regulatory Reporting:} Pilot with 5 banks (2021-2023)
\item \textbf{ACPR (France):} Regulatory data dictionary
\end{itemize}

\vspace{2mm}
\textbf{Challenges:}
\begin{itemize}
\item Principles-based regulation hard to codify
\item Legal liability for automated decisions
\item Maintenance as regulations evolve
\item Standardization across jurisdictions
\end{itemize}
\end{columns}
\bottomnote{Regulatory frameworks shape adoption patterns and industry structure.}
\end{frame}

\begin{frame}[t]{Summary and Key Takeaways}
\begin{columns}[T]
\scriptsize
\column{0.48\textwidth}
\textbf{RegTech Overview:}
\begin{itemize}
\item Technology-driven regulatory compliance
\item Global market: \$12B (2023) to \$50B+ (2030)
\item Core tech: ML, NLP, RPA, cloud, blockchain
\end{itemize}

\vspace{2mm}
\textbf{Basel III:}
\begin{itemize}
\item CET1 minimum 4.5\% + buffers (effective 10-13\%)
\item RWA calculation: Standardized vs IRB
\item Liquidity: LCR (30-day) and NSFR (1-year) both $\geq$ 100\%
\item FRTB: Expected Shortfall replaces VaR
\end{itemize}

\vspace{2mm}
\textbf{IFRS 9:}
\begin{itemize}
\item 3-stage model: 12-month ECL (Stage 1), Lifetime ECL (Stage 2/3)
\item SICR triggers: PD changes, 30 DPD, rating downgrades
\item Probability-weighted macroeconomic scenarios
\end{itemize}

\column{0.48\textwidth}
\textbf{Regulatory Reporting:}
\begin{itemize}
\item EMIR: Derivatives reporting (203 fields, T+1)
\item MiFID II: Transaction reporting (65+ fields, T+1)
\item Dodd-Frank: Swap data to SDRs, CAT for equities
\end{itemize}

\vspace{2mm}
\textbf{AML/KYC:}
\begin{itemize}
\item Digital ID verification (Jumio, Onfido)
\item Sanctions screening: OFAC, UN, EU lists
\item Transaction monitoring: ML reduces false positives 30-70\%
\item Trade surveillance: Detect spoofing, insider trading
\end{itemize}

\vspace{2mm}
\textbf{Emerging Trends:}
\begin{itemize}
\item SupTech: Regulators using tech for supervision
\item Machine-readable regulation: Rules as code
\item Cloud-native RegTech-as-a-Service
\end{itemize}
\end{columns}
\end{frame}

\end{document}
