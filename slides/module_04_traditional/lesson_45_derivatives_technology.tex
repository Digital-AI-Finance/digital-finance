\documentclass[8pt,aspectratio=169]{beamer}
\usetheme{Madrid}
\usepackage{graphicx,booktabs,adjustbox,multicol,amsmath,amssymb}
\definecolor{mlblue}{RGB}{0,102,204}
\definecolor{mlpurple}{RGB}{51,51,178}
\definecolor{mllavender}{RGB}{173,173,224}
\definecolor{mllavender2}{RGB}{193,193,232}
\definecolor{mllavender3}{RGB}{204,204,235}
\definecolor{mllavender4}{RGB}{214,214,239}
\definecolor{mlorange}{RGB}{255,127,14}
\definecolor{mlgreen}{RGB}{44,160,44}
\definecolor{mlred}{RGB}{214,39,40}
\setbeamercolor{palette primary}{bg=mllavender3,fg=mlpurple}
\setbeamercolor{palette secondary}{bg=mllavender2,fg=mlpurple}
\setbeamercolor{palette tertiary}{bg=mllavender,fg=white}
\setbeamercolor{structure}{fg=mlpurple}
\setbeamercolor{frametitle}{fg=mlpurple,bg=mllavender3}
\setbeamertemplate{navigation symbols}{}
\setbeamertemplate{itemize items}[circle]
\setbeamersize{text margin left=5mm,text margin right=5mm}

% Bottom note command for key takeaways
\newcommand{\bottomnote}[1]{%
\vfill
\vspace{-2mm}
\textcolor{mllavender2}{\rule{\textwidth}{0.4pt}}
\vspace{1mm}
\footnotesize
\textbf{#1}
}
\title{Lesson 45: Derivatives Technology}
\subtitle{Module 4: Traditional Digital Finance}
\author{Digital Finance Course}
\date{2025}

\begin{document}

\begin{frame}
\titlepage
\end{frame}

\begin{frame}[t]{Learning Objectives}
\begin{itemize}
\item Understand derivatives pricing and risk systems architecture
\item Analyze futures and options trading platforms
\item Examine OTC derivatives lifecycle and central clearing (CCPs)
\item Evaluate EMIR and Dodd-Frank technology requirements
\item Assess margin systems and collateral optimization
\end{itemize}
\end{frame}

\section{Derivatives Overview and Market Structure}

\begin{frame}[t]{Derivatives Notional Outstanding}
\begin{center}
\includegraphics[width=0.60\textwidth]{figures/derivatives_notional/derivatives_notional.pdf}
\end{center}
\bottomnote{Derivatives markets measured in notional value dwarf underlying assets.}
\end{frame}

\begin{frame}[t]{Derivatives Market Landscape}
\begin{columns}[T]
\scriptsize
\column{0.48\textwidth}
\textbf{Market Size (BIS, June 2024):}
\begin{itemize}
\item \textbf{OTC Derivatives:} \$715 trillion notional outstanding
\begin{itemize}
\item Interest rate derivatives: \$590T (82\%)
\item FX derivatives: \$100T (14\%)
\item Equity derivatives: \$7T (1\%)
\item Credit derivatives: \$9T (1\%)
\item Commodity derivatives: \$2T
\end{itemize}
\item \textbf{Exchange-Traded Derivatives:} 45 billion contracts (2023)
\begin{itemize}
\item Futures: 30B contracts
\item Options: 15B contracts
\end{itemize}
\end{itemize}

\column{0.48\textwidth}
\textbf{Trading Venues:}

\textbf{Exchange-Traded:}
\begin{itemize}
\item \textbf{CME Group:} Rates, equity indices, commodities, FX
\item \textbf{Eurex:} European equity and rates derivatives
\item \textbf{ICE:} Energy, agriculture, financial futures
\item \textbf{Cboe:} Equity options, VIX futures
\end{itemize}

\textbf{OTC:}
\begin{itemize}
\item Bilateral negotiation (historically)
\item SEFs (Swap Execution Facilities) - US
\item MTFs (Multilateral Trading Facilities) - EU
\item Electronic platforms: Bloomberg SEF, Tradeweb, MarketAxess
\end{itemize}

\vspace{2mm}
\textit{Post-2008 trend: 75\%+ of interest rate swaps now cleared via CCPs (vs 20\% pre-crisis)}
\end{columns}
\bottomnote{Derivatives enable risk transfer and price discovery.}
\end{frame}

\begin{frame}[t]{Derivatives Product Types}
\begin{columns}[T]
\scriptsize
\column{0.48\textwidth}
\textbf{Futures:}
\begin{itemize}
\item Standardized exchange-traded contracts
\item Daily mark-to-market and margin
\item \textbf{Financial:} Equity indices (E-Mini S\&P), rates (Eurodollar), FX
\item \textbf{Commodities:} Oil (WTI, Brent), metals (gold), agriculture (corn, wheat)
\end{itemize}

\vspace{2mm}
\textbf{Options:}
\begin{itemize}
\item \textbf{Call:} Right to buy at strike price
\item \textbf{Put:} Right to sell at strike price
\item \textbf{American:} Exercise any time before expiry
\item \textbf{European:} Exercise only at expiry
\item \textbf{Exotic:} Barriers, Asians, lookbacks, digitals
\end{itemize}

\column{0.48\textwidth}
\textbf{Swaps:}
\begin{itemize}
\item \textbf{Interest Rate Swaps (IRS):} Fixed-for-floating (LIBOR/SOFR)
\item \textbf{Cross-Currency Swaps:} Principal and interest in different currencies
\item \textbf{Equity Swaps:} Index return vs floating rate
\item \textbf{Commodity Swaps:} Fixed price vs floating (spot)
\end{itemize}

\vspace{2mm}
\textbf{Credit Derivatives:}
\begin{itemize}
\item \textbf{CDS (Credit Default Swaps):} Insurance against default
\item \textbf{Index CDS:} CDX (US), iTraxx (Europe)
\item \textbf{Tranches:} First-loss, mezzanine, senior
\end{itemize}

\vspace{2mm}
\textit{IRS: Most liquid OTC derivative (80\%+ of notional outstanding)}
\end{columns}
\bottomnote{Derivatives enable risk transfer and price discovery.}
\end{frame}

\begin{frame}[t]{Option Payoff Profiles}
\begin{center}
\includegraphics[width=0.60\textwidth]{figures/option_payoff/option_payoff.pdf}
\end{center}
\bottomnote{Option payoff profiles create asymmetric risk-return characteristics.}
\end{frame}

\begin{frame}[t]{Option Greeks Sensitivity}
\begin{center}
\includegraphics[width=0.60\textwidth]{figures/option_greeks/option_greeks.pdf}
\end{center}
\bottomnote{Greeks quantify option price sensitivity to underlying factors.}
\end{frame}

\section{Pricing and Risk Systems}

\begin{frame}[t]{Derivatives Pricing Models}
\begin{columns}[T]
\scriptsize
\column{0.48\textwidth}
\textbf{Equity Options (Black-Scholes):}

$$C = S_0 N(d_1) - K e^{-rT} N(d_2)$$
$$P = K e^{-rT} N(-d_2) - S_0 N(-d_1)$$

where
$$d_1 = \frac{\ln(S_0/K) + (r + \sigma^2/2)T}{\sigma\sqrt{T}}$$
$$d_2 = d_1 - \sigma\sqrt{T}$$

\textbf{Inputs:} $S_0$ = spot price, $K$ = strike, $r$ = risk-free rate, $\sigma$ = volatility, $T$ = time to expiry

\vspace{2mm}
\textbf{Limitations:}
\begin{itemize}
\item Constant volatility assumption (implied vol smile/skew observed)
\item No early exercise (American options need binomial/FDM)
\item No dividends in simple version
\end{itemize}

\column{0.48\textwidth}
\textbf{Interest Rate Swaps:}

\textbf{Fixed Leg PV:}
$$PV_{fixed} = N \times c \times \sum_{i=1}^{n} \tau_i \times DF(t_i)$$

\textbf{Floating Leg PV:}
$$PV_{float} = N \times \sum_{i=1}^{n} F(t_{i-1}, t_i) \times \tau_i \times DF(t_i)$$

where $N$ = notional, $c$ = fixed coupon, $\tau_i$ = accrual fraction, $DF(t_i)$ = discount factor, $F$ = forward rate

\vspace{2mm}
\textbf{Par Swap Rate:}
$$c = \frac{1 - DF(T_n)}{\sum_{i=1}^{n} \tau_i \times DF(t_i)}$$

\vspace{2mm}
\textbf{Curve Construction:}
\begin{itemize}
\item Bootstrap discount curve from market instruments (deposits, FRAs, swaps)
\item Multi-curve framework post-2008 (OIS discounting vs LIBOR/SOFR projection)
\end{itemize}
\end{columns}
\bottomnote{Derivatives enable risk transfer and price discovery.}
\end{frame}

\begin{frame}[t]{Greeks and Risk Management}
\begin{columns}[T]
\scriptsize
\column{0.48\textwidth}
\textbf{First-Order Greeks:}

\textbf{Delta ($\Delta$):} Sensitivity to underlying price
$$\Delta = \frac{\partial V}{\partial S}$$
Call delta: 0 to 1, Put delta: -1 to 0

\textbf{Vega ($\mathcal{V}$):} Sensitivity to volatility
$$\mathcal{V} = \frac{\partial V}{\partial \sigma}$$
Long options: positive vega (benefit from vol increase)

\textbf{Theta ($\Theta$):} Time decay
$$\Theta = \frac{\partial V}{\partial t}$$
Typically negative (options lose value over time)

\textbf{Rho ($\rho$):} Interest rate sensitivity
$$\rho = \frac{\partial V}{\partial r}$$

\column{0.48\textwidth}
\textbf{Second-Order Greeks:}

\textbf{Gamma ($\Gamma$):} Curvature of delta
$$\Gamma = \frac{\partial^2 V}{\partial S^2} = \frac{\partial \Delta}{\partial S}$$
Maximum at-the-money, near expiry (high gamma risk)

\textbf{Vanna:} Delta sensitivity to volatility
$$\text{Vanna} = \frac{\partial^2 V}{\partial S \partial \sigma}$$

\textbf{Volga (Vomma):} Vega sensitivity to volatility
$$\text{Volga} = \frac{\partial^2 V}{\partial \sigma^2}$$

\vspace{2mm}
\textbf{Portfolio Greeks Aggregation:}
\begin{itemize}
\item Sum across positions for linear Greeks (Delta, Vega)
\item Gamma concentration risk (short gamma in crash)
\item Hedge Greeks dynamically (delta-neutral, vega-neutral)
\end{itemize}
\end{columns}
\bottomnote{Risk management is essential for financial stability and profitability.}
\end{frame}

\begin{frame}[t]{Derivatives Valuation Systems}
\begin{columns}[T]
\scriptsize
\column{0.48\textwidth}
\textbf{System Architecture:}

\textbf{Core Components:}
\begin{itemize}
\item \textbf{Trade Capture:} Front-office booking (Murex, Calypso, Summit)
\item \textbf{Market Data:} Live and historical prices, curves, vols
\item \textbf{Pricing Library:} Quantitative models (C++/Python)
\item \textbf{Risk Engine:} Calculate sensitivities and VaR
\item \textbf{P\&L Attribution:} Explain daily P\&L by risk factor
\item \textbf{Valuation Adjustments:} XVA (CVA, FVA, MVA)
\end{itemize}

\vspace{2mm}
\textbf{Calculation Methods:}
\begin{itemize}
\item \textbf{Closed-Form:} Black-Scholes, bond formulas
\item \textbf{Lattice:} Binomial/trinomial trees for American options
\item \textbf{Finite Difference:} PDE solvers for complex derivatives
\item \textbf{Monte Carlo:} Path-dependent, multi-asset options
\end{itemize}

\column{0.48\textwidth}
\textbf{Leading Platforms:}
\begin{itemize}
\item \textbf{Murex (Global):} Front-to-risk for derivatives
\item \textbf{Calypso:} Multi-asset capital markets
\item \textbf{SunGard (FIS) Adaptiv:} Counterparty risk and CVA
\item \textbf{Numerix:} Pricing analytics and XVA
\item \textbf{Bloomberg DLIB:} Derivatives pricing library
\item \textbf{QuantLib:} Open-source quantitative finance (C++)
\end{itemize}

\vspace{2mm}
\textbf{Performance Optimization:}
\begin{itemize}
\item \textbf{Grid Computing:} Distribute Monte Carlo simulations
\item \textbf{GPU Acceleration:} 10-100x speedup for pricing
\item \textbf{AAD (Adjoint Algorithmic Differentiation):} Fast Greeks
\item \textbf{Approximations:} Least-squares Monte Carlo, proxy models
\end{itemize}

\vspace{2mm}
\textit{Large bank derivatives book: 100k-1M positions, EOD risk calc in 2-4 hours}
\end{columns}
\bottomnote{Derivatives enable risk transfer and price discovery.}
\end{frame}

\section{Exchange-Traded Derivatives}

\begin{frame}[t]{Futures Trading Platforms}
\begin{columns}[T]
\scriptsize
\column{0.48\textwidth}
\textbf{Major Futures Exchanges:}

\textbf{CME Group:}
\begin{itemize}
\item CME Globex electronic platform
\item Products: E-Mini S\&P, Eurodollar, Treasury futures, FX, commodities
\item Avg daily volume: 25-30 million contracts (2024)
\item Latency: 50-100 microseconds (co-located)
\end{itemize}

\textbf{Eurex:}
\begin{itemize}
\item European equity index futures (DAX, Euro Stoxx)
\item Fixed income (Bund, Schatz, Bobl)
\item T7 matching engine
\end{itemize}

\textbf{ICE Futures:}
\begin{itemize}
\item Energy: WTI crude, Brent, natural gas
\item Agriculture: Sugar, coffee, cotton
\item Financial: FTSE, MSCI indices
\end{itemize}

\column{0.48\textwidth}
\textbf{Futures Trading Technology:}

\textbf{Order Types:}
\begin{itemize}
\item Outright: Single contract month
\item Calendar spreads: Long one month, short another
\item Butterfly: 1-2-1 ratio across 3 maturities
\item Inter-commodity spreads: Related contracts (e.g., crack spread)
\end{itemize}

\vspace{2mm}
\textbf{Execution Venues:}
\begin{itemize}
\item Exchange electronic (CME Globex, Eurex T7)
\item Broker platforms (CQG, Trading Technologies, Bloomberg EMSX)
\item Algorithmic execution (VWAP, TWAP for large orders)
\item Direct market access (DMA) for speed
\end{itemize}

\vspace{2mm}
\textbf{Market Making:}
\begin{itemize}
\item HFT firms provide liquidity (Citadel, Virtu, Jump)
\item Capture bid-ask spread (1-2 ticks typically)
\item Delta-hedge with underlying or options
\end{itemize}
\end{columns}
\bottomnote{Future trends inform strategic planning and investment decisions.}
\end{frame}

\begin{frame}[t]{Options Trading and Volatility Markets}
\begin{columns}[T]
\scriptsize
\column{0.48\textwidth}
\textbf{Options Exchanges:}

\textbf{US:}
\begin{itemize}
\item \textbf{Cboe:} Largest options exchange, VIX futures/options
\item \textbf{ISE (Nasdaq):} Electronic equity options
\item \textbf{NYSE Arca:} Equity and ETF options
\item \textbf{MIAX, PHLX, BOX, BZX:} Regional and new entrants
\end{itemize}

\textbf{Europe:}
\begin{itemize}
\item \textbf{Eurex:} Equity index options (ODAX, OESX)
\item \textbf{Euronext:} Individual equity options
\end{itemize}

\vspace{2mm}
\textbf{Options Volume:}
\begin{itemize}
\item US equity options: 10-12 billion contracts/year (2024)
\item 50\%+ volume in SPY, QQQ, AAPL, TSLA
\item 0DTE (zero days to expiry) options: 40-50\% of SPX volume (2024)
\end{itemize}

\column{0.48\textwidth}
\textbf{Volatility Trading:}

\textbf{Implied Volatility Surface:}
\begin{itemize}
\item Volatility varies by strike (smile/skew) and tenor
\item \textbf{At-the-Money (ATM):} Reference volatility
\item \textbf{Out-of-the-Money Puts:} Higher vol (tail risk premium)
\item \textbf{Skew:} Put vol - call vol (negative for equities)
\end{itemize}

\vspace{2mm}
\textbf{VIX (CBOE Volatility Index):}
\begin{itemize}
\item 30-day implied volatility of S\&P 500 options
\item Weighted average of OTM calls and puts
\item Typically 12-20 (calm), spikes to 40-80 (crisis)
\item VIX futures and options tradeable (hedge tail risk)
\end{itemize}

\vspace{2mm}
\textbf{Variance Swaps:}
$$\text{Payoff} = N \times (\sigma_{realized}^2 - K_{var})$$
where $N$ = variance notional, $K_{var}$ = strike variance, $\sigma_{realized}$ = realized volatility
\end{columns}
\bottomnote{Electronic trading has transformed market structure and efficiency.}
\end{frame}

\begin{frame}[t]{Margin Requirements Framework}
\begin{center}
\includegraphics[width=0.60\textwidth]{figures/margin_requirements/margin_requirements.pdf}
\end{center}
\bottomnote{Margin requirements ensure counterparty risk mitigation in derivatives trading.}
\end{frame}

\section{OTC Derivatives and Central Clearing}

\begin{frame}[t]{OTC Derivatives Lifecycle}
\begin{columns}[T]
\scriptsize
\column{0.48\textwidth}
\textbf{Pre-Trade:}
\begin{enumerate}
\item \textbf{Credit Check:} Verify counterparty limits
\item \textbf{Legal Documentation:} ISDA Master Agreement, CSA
\item \textbf{Pricing:} Request quotes from dealers (RFQ)
\item \textbf{Negotiation:} Economic terms (notional, tenor, fixed rate)
\end{enumerate}

\vspace{2mm}
\textbf{Execution:}
\begin{enumerate}
\setcounter{enumi}{4}
\item \textbf{Trade Capture:} Book in derivatives system (Murex, Calypso)
\item \textbf{Trade Confirmation:} Electronic (MarkitServ, AcadiaSoft) or manual
\item \textbf{Trade Reporting:} Regulatory (EMIR, Dodd-Frank) to TR
\end{enumerate}

\column{0.48\textwidth}
\textbf{Post-Trade:}
\begin{enumerate}
\setcounter{enumi}{7}
\item \textbf{Clearing Determination:} Mandatory clearing (standardized) or bilateral
\item \textbf{Novation to CCP:} If cleared, CCP becomes central counterparty
\item \textbf{Margin:} Initial and variation margin posting
\item \textbf{Valuation:} Daily mark-to-market (independent price verification)
\item \textbf{Lifecycle Events:} Resets, coupons, compressions, amendments
\item \textbf{Termination:} Maturity or early unwind
\end{enumerate}

\vspace{3mm}
\textbf{ISDA Documentation:}
\begin{itemize}
\item Master Agreement: Legal framework for all trades
\item Schedule: Customizations (jurisdiction, netting, events of default)
\item CSA (Credit Support Annex): Collateral terms
\item Confirmations: Trade-specific economics
\end{itemize}
\end{columns}
\bottomnote{Technology adoption follows predictable patterns---timing matters for investment decisions.}
\end{frame}

\begin{frame}[t]{Central Counterparty Clearing (CCPs)}
\begin{columns}[T]
\scriptsize
\column{0.48\textwidth}
\textbf{CCP Clearing Mandate (Post-2008):}

\textbf{Dodd-Frank (US):}
\begin{itemize}
\item Mandatory clearing for standardized swaps
\item SEF trading requirement (pre-trade transparency)
\item Trade reporting to Swap Data Repositories (SDRs)
\end{itemize}

\textbf{EMIR (EU):}
\begin{itemize}
\item Clearing obligation for IRS (EUR, GBP, USD), index CDS
\item Reporting to Trade Repositories (TRs)
\item Risk mitigation for non-cleared (margin, dispute resolution)
\end{itemize}

\vspace{2mm}
\textbf{Clearing Rate (2024):}
\begin{itemize}
\item Interest rate swaps: 75-80\% cleared
\item Credit derivatives: 50-60\% (index CDS higher)
\item FX derivatives: 5-10\% (mostly bilateral)
\end{itemize}

\column{0.48\textwidth}
\textbf{Major CCPs:}

\textbf{LCH (LSEG):}
\begin{itemize}
\item SwapClear: \$500T+ notional (IRS, OIS, FRAs)
\item CDSClear: Index and single-name CDS
\item RepoClear: Repo and securities lending
\end{itemize}

\textbf{CME Clearing:}
\begin{itemize}
\item IRS clearing (US and global)
\item Futures and options across all asset classes
\end{itemize}

\textbf{ICE Clear:}
\begin{itemize}
\item ICE Clear Credit: CDS clearing (North America, Europe)
\item ICE Clear Europe: Energy and commodity derivatives
\end{itemize}

\vspace{2mm}
\textbf{CCP Risk Management:}
\begin{itemize}
\item Initial margin (SPAN, VaR-based)
\item Variation margin (daily MTM settlement)
\item Default fund (mutualized loss-sharing)
\item Stress testing and concentration limits
\end{itemize}
\end{columns}
\bottomnote{Key concepts from this slide inform practical applications in finance.}
\end{frame}

\begin{frame}[t]{Margin and Collateral Systems}
\begin{columns}[T]
\scriptsize
\column{0.48\textwidth}
\textbf{Initial Margin (IM):}

\textbf{CCP IM Calculation:}
\begin{itemize}
\item \textbf{SPAN (Standard Portfolio Analysis of Risk):} CME method
\begin{itemize}
\item Scenario-based approach (16+ price/vol scenarios)
\item Accounts for offsets within product families
\end{itemize}
\item \textbf{VaR-Based:} LCH, Eurex (99\% 5-day VaR)
\item \textbf{Expected Shortfall:} Emerging standard (coherent measure)
\end{itemize}

\vspace{2mm}
\textbf{Non-Cleared IM (UMR):}
\begin{itemize}
\item Uncleared Margin Rules (Basel/IOSCO, BCBS-IOSCO 2015)
\item Mandatory for firms with \$50B+ derivatives notional
\item Phased implementation (2016-2023)
\item SIMM (Standard Initial Margin Model): ISDA standard
\item Segregation requirement: Third-party custodian
\end{itemize}

\column{0.48\textwidth}
\textbf{Variation Margin (VM):}
\begin{itemize}
\item Daily exchange of collateral based on MTM
\item Mandatory for all OTC derivatives (Dodd-Frank, EMIR)
\item Cash or high-quality securities (government bonds)
\item Dispute resolution (valuation differences)
\end{itemize}

\vspace{2mm}
\textbf{Collateral Optimization:}
\begin{itemize}
\item \textbf{Cheapest-to-Deliver:} Post lowest-opportunity-cost asset
\item \textbf{Substitution:} Replace collateral to free up better assets
\item \textbf{Transformation:} Borrow higher-quality collateral (upgrade trades)
\item \textbf{Triparty Agents:} BNY Mellon, Euroclear manage pools
\end{itemize}

\vspace{2mm}
\textbf{Technology Platforms:}
\begin{itemize}
\item \textbf{AcadiaSoft:} Margin calculation and optimization
\item \textbf{CloudMargin:} Collateral management platform
\item \textbf{Calypso Margin:} Integrated with trade processing
\item \textbf{TriOptima (CME):} Compression and optimization
\end{itemize}
\end{columns}
\bottomnote{Key concepts from this slide inform practical applications in finance.}
\end{frame}

\section{Regulatory Reporting: EMIR and Dodd-Frank}

\begin{frame}[t]{EMIR Reporting Requirements}
\begin{columns}[T]
\scriptsize
\column{0.48\textwidth}
\textbf{European Market Infrastructure Regulation (EMIR):}

\textbf{Reporting Obligations:}
\begin{itemize}
\item \textbf{Scope:} All derivatives (OTC and exchange-traded)
\item \textbf{Counterparties:} Both sides report (or delegate)
\item \textbf{Timing:} T+1 (next working day)
\item \textbf{Destination:} Trade Repositories (TRs)
\item \textbf{Fields:} 203 fields (EMIR REFIT, 2024)
\end{itemize}

\vspace{2mm}
\textbf{EMIR REFIT (2024):}
\begin{itemize}
\item Simplified reporting for smaller firms
\item Standardized data formats (ISO 20022)
\item UTI (Unique Trade Identifier) mandatory
\item LEI (Legal Entity Identifier) for all counterparties
\item Backloading of historical trades
\end{itemize}

\column{0.48\textwidth}
\textbf{Trade Repositories (TRs):}
\begin{itemize}
\item \textbf{DTCC GTR (Global Trade Repository):} Multi-asset
\item \textbf{UnaVista (LSEG):} EMIR and MiFID II reporting
\item \textbf{Regis-TR:} European focused
\item \textbf{CME Trade Repository:} Commodities and derivatives
\end{itemize}

\vspace{2mm}
\textbf{Reporting Challenges:}
\begin{itemize}
\item Data quality (reconciliation breaks)
\item Lifecycle event reporting (amendments, compressions)
\item Backloading for old trades
\item Cross-border coordination (EMIR vs Dodd-Frank)
\item LEI adoption and validation
\end{itemize}

\vspace{2mm}
\textbf{Vendor Solutions:}
\begin{itemize}
\item Cappitech, ABIDE Financial, Primatics
\item Integrated with derivatives platforms (Murex, Calypso)
\end{itemize}
\end{columns}
\bottomnote{Key concepts from this slide inform practical applications in finance.}
\end{frame}

\begin{frame}[t]{Dodd-Frank Swap Data Reporting}
\begin{columns}[T]
\scriptsize
\column{0.48\textwidth}
\textbf{Dodd-Frank Act (2010):}

\textbf{Title VII - Derivatives:}
\begin{itemize}
\item Mandatory clearing for standardized swaps
\item SEF (Swap Execution Facility) trading
\item Swap Data Repository (SDR) reporting
\item Enhanced capital and margin (Volcker Rule)
\end{itemize}

\vspace{2mm}
\textbf{Reporting Requirements:}
\begin{itemize}
\item \textbf{Part 45:} Real-time reporting to SDRs
\item \textbf{Part 43:} Public dissemination (anonymized)
\item \textbf{Reporting Parties:} Swap dealers (SDs), major participants (MSPs)
\item \textbf{Timing:} As soon as technologically practicable (minutes)
\item \textbf{Block Trades:} 15-minute delay for large notional
\end{itemize}

\column{0.48\textwidth}
\textbf{Swap Data Repositories (SDRs):}
\begin{itemize}
\item \textbf{DTCC SDR:} Credit, equity, rates
\item \textbf{CME SDR:} Commodities, FX
\item \textbf{ICE Trade Vault:} Multi-asset
\end{itemize}

\vspace{2mm}
\textbf{SEF Trading:}
\begin{itemize}
\item Electronic platforms for swap execution
\item Bloomberg SEF, Tradeweb, MarketAxess, ICE Swap Trade
\item Pre-trade transparency (executable quotes)
\item Request-for-quote (RFQ) dominant model
\end{itemize}

\vspace{2mm}
\textbf{Harmonization Efforts:}
\begin{itemize}
\item CFTC-ESMA cooperation (cross-border)
\item Substituted compliance (equivalence)
\item UTI and UPI standards (global identifiers)
\item Critical trade information (CTI) standardization
\end{itemize}
\end{columns}
\bottomnote{Quality data is the foundation for effective machine learning models.}
\end{frame}

\section{Emerging Trends in Derivatives Technology}

\begin{frame}[t]{SOFR Transition and IBOR Reform}
\begin{columns}[T]
\scriptsize
\column{0.48\textwidth}
\textbf{LIBOR Cessation (2021-2023):}
\begin{itemize}
\item USD LIBOR ended June 30, 2023
\item GBP, EUR, CHF, JPY LIBOR ended Dec 31, 2021
\item Legacy contracts: Fallback language (ISDA protocol)
\item Estimated \$300+ trillion notional affected globally
\end{itemize}

\vspace{2mm}
\textbf{Replacement Rates:}
\begin{itemize}
\item \textbf{USD:} SOFR (Secured Overnight Financing Rate)
\item \textbf{GBP:} SONIA (Sterling Overnight Index Average)
\item \textbf{EUR:} ESTR (Euro Short-Term Rate)
\item \textbf{CHF:} SARON (Swiss Average Rate Overnight)
\item \textbf{JPY:} TONAR (Tokyo Overnight Average Rate)
\end{itemize}

\column{0.48\textwidth}
\textbf{SOFR Characteristics:}
\begin{itemize}
\item Overnight rate (vs forward-looking LIBOR tenors)
\item Based on Treasury repo transactions (\$1T+ daily volume)
\item No credit risk premium (secured rate)
\item Requires compounding for term rates (Term SOFR available)
\end{itemize}

\vspace{2mm}
\textbf{Technology Impact:}
\begin{itemize}
\item \textbf{Curve Construction:} Multi-curve framework (OIS discounting, SOFR projection)
\item \textbf{Fallback Implementation:} ISDA fallback spreads, waterfall logic
\item \textbf{Historical Data:} Limited SOFR history (started 2018)
\item \textbf{Systems Upgrade:} Pricing, risk, reporting systems updated
\item \textbf{Vendor Solutions:} Bloomberg, Refinitiv SOFR calculators
\end{itemize}

\vspace{2mm}
\textit{CME SOFR futures: \$10+ trillion notional outstanding (2024), primary hedging instrument}
\end{columns}
\bottomnote{Key concepts from this slide inform practical applications in finance.}
\end{frame}

\begin{frame}[t]{DLT and Smart Derivatives Contracts}
\begin{columns}[T]
\scriptsize
\column{0.48\textwidth}
\textbf{Blockchain in Derivatives:}

\textbf{Use Cases:}
\begin{itemize}
\item \textbf{Trade Confirmation:} Shared ledger eliminates reconciliation
\item \textbf{Lifecycle Events:} Automated coupons, resets, terminations
\item \textbf{Collateral Management:} Real-time margin calls and settlements
\item \textbf{Regulatory Reporting:} Single source of truth for regulators
\end{itemize}

\vspace{2mm}
\textbf{Pilot Projects:}
\begin{itemize}
\item \textbf{ISDA CDM (Common Domain Model):} Standardized data model
\item \textbf{Axoni (AxCore):} Equity swaps post-trade (DTCC partnership)
\item \textbf{Digital Asset (DAML):} Smart contract language for derivatives
\item \textbf{Project Guardian (MAS):} Tokenized bonds and FX swaps
\end{itemize}

\column{0.48\textwidth}
\textbf{Smart Derivatives Contracts:}

\textbf{ISDA Common Domain Model (CDM):}
\begin{itemize}
\item Machine-executable representation of derivatives
\item Event processing (lifecycle automation)
\item Legal certainty (code = contract)
\item Implemented in DAML, Solidity, Java
\end{itemize}

\vspace{2mm}
\textbf{Benefits:}
\begin{itemize}
\item Reduce operational risk (eliminate manual processing)
\item Faster settlement (T+0 for collateral)
\item Lower costs (remove intermediaries)
\item Enhanced transparency (shared audit trail)
\end{itemize}

\vspace{2mm}
\textbf{Challenges:}
\begin{itemize}
\item Legal enforceability (code vs traditional contracts)
\item Scalability (blockchain performance for high volumes)
\item Interoperability (multiple DLT platforms)
\item Regulatory acceptance (still evolving)
\end{itemize}
\end{columns}
\bottomnote{Derivatives enable risk transfer and price discovery.}
\end{frame}

\begin{frame}[t]{Summary and Key Takeaways}
\begin{columns}[T]
\scriptsize
\column{0.48\textwidth}
\textbf{Derivatives Market:}
\begin{itemize}
\item \$715T OTC notional (82\% interest rate swaps)
\item 45B exchange contracts/year (futures + options)
\item Post-2008: 75-80\% of IRS now cleared via CCPs
\end{itemize}

\vspace{2mm}
\textbf{Pricing and Risk:}
\begin{itemize}
\item Black-Scholes for equity options (constant vol limitation)
\item Greeks: Delta, gamma, vega for risk management
\item Platforms: Murex, Calypso, Numerix for valuation
\item GPU acceleration: 10-100x speedup for Monte Carlo
\end{itemize}

\vspace{2mm}
\textbf{Exchange-Traded:}
\begin{itemize}
\item CME Globex: 25-30M contracts/day (50-100 microsecond latency)
\item VIX futures/options for volatility trading
\item 0DTE options: 40-50\% of SPX volume (2024)
\end{itemize}

\column{0.48\textwidth}
\textbf{OTC and Clearing:}
\begin{itemize}
\item ISDA documentation: Master Agreement + CSA
\item CCPs: LCH SwapClear (\$500T+), CME, ICE Clear
\item Initial margin (SPAN, VaR) + variation margin (daily MTM)
\item UMR: Mandatory IM for \$50B+ notional (segregated)
\end{itemize}

\vspace{2mm}
\textbf{Regulatory Reporting:}
\begin{itemize}
\item EMIR: 203 fields, T+1 reporting to TRs
\item Dodd-Frank: Real-time SDR reporting, SEF trading
\item UTI, LEI, UPI standardization for cross-border
\end{itemize}

\vspace{2mm}
\textbf{Emerging Trends:}
\begin{itemize}
\item SOFR transition (LIBOR ended 2023): \$300T+ impacted
\item ISDA CDM: Machine-executable derivatives contracts
\item DLT pilots: Axoni, Digital Asset, Project Guardian
\end{itemize}
\end{columns}
\end{frame}

\end{document}
