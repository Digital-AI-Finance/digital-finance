\documentclass[8pt,aspectratio=169]{beamer}
\usetheme{Madrid}
\usepackage{graphicx,booktabs,adjustbox,multicol,amsmath,amssymb}
\definecolor{mlblue}{RGB}{0,102,204}
\definecolor{mlpurple}{RGB}{51,51,178}
\definecolor{mllavender}{RGB}{173,173,224}
\definecolor{mllavender2}{RGB}{193,193,232}
\definecolor{mllavender3}{RGB}{204,204,235}
\definecolor{mllavender4}{RGB}{214,214,239}
\definecolor{mlorange}{RGB}{255,127,14}
\definecolor{mlgreen}{RGB}{44,160,44}
\definecolor{mlred}{RGB}{214,39,40}
\setbeamercolor{palette primary}{bg=mllavender3,fg=mlpurple}
\setbeamercolor{palette secondary}{bg=mllavender2,fg=mlpurple}
\setbeamercolor{palette tertiary}{bg=mllavender,fg=white}
\setbeamercolor{structure}{fg=mlpurple}
\setbeamercolor{frametitle}{fg=mlpurple,bg=mllavender3}
\setbeamertemplate{navigation symbols}{}
\setbeamertemplate{itemize items}[circle]
\setbeamersize{text margin left=5mm,text margin right=5mm}

\title{Lesson 42: Risk Management Systems}
\subtitle{Module 4: Traditional Digital Finance}
\author{Digital Finance Course}
\date{2025}

\begin{document}

\begin{frame}
\titlepage
\end{frame}

\begin{frame}{Learning Objectives}
\begin{itemize}
\item Understand Value at Risk (VaR) methodologies and implementation
\item Analyze stress testing and scenario analysis frameworks
\item Examine Expected Shortfall (CVaR) and coherent risk measures
\item Evaluate model risk management and validation processes
\item Assess enterprise risk management systems and architecture
\end{itemize}
\end{frame}

\section{Value at Risk (VaR)}

\begin{frame}{VaR Definition and Framework}
\begin{columns}[T]
\column{0.48\textwidth}
\textbf{Formal Definition:}

\textit{Value at Risk (VaR) is the maximum loss over a target horizon at a given confidence level.}

$$\Pr(L > \text{VaR}_\alpha) = 1 - \alpha$$

where $L$ = portfolio loss, $\alpha$ = confidence level (typically 95\% or 99\%)

\vspace{2mm}
\textbf{Standard Parameters:}
\begin{itemize}
\item \textbf{Confidence Level:} 95\% (regulatory: 99\%)
\item \textbf{Time Horizon:} 1 day (trading), 10 days (regulatory)
\item \textbf{Currency:} Reporting currency (USD, EUR)
\item \textbf{Scope:} Individual desk, portfolio, firm-wide
\end{itemize}

\column{0.48\textwidth}
\textbf{Interpretation Example:}
\begin{itemize}
\item 1-day 99\% VaR = \$10 million
\item Interpretation: ``We expect losses to exceed \$10M on 1\% of trading days (2-3 days per year)''
\item Not a worst-case measure (tail risk beyond VaR)
\end{itemize}

\vspace{2mm}
\textbf{Applications:}
\begin{itemize}
\item \textbf{Regulatory Capital:} Basel III market risk
\item \textbf{Risk Limits:} Desk-level VaR limits
\item \textbf{Performance Attribution:} Risk-adjusted returns
\item \textbf{Client Reporting:} UCITS, hedge fund disclosures
\item \textbf{Stress Testing:} Baseline for scenario comparison
\end{itemize}
\end{columns}
\end{frame}

\begin{frame}{VaR Methodologies}
\begin{columns}[T]
\column{0.48\textwidth}
\textbf{1. Parametric VaR (Variance-Covariance):}

$$\text{VaR}_\alpha = \mu + z_\alpha \sigma \sqrt{t}$$

where $\mu$ = expected return, $z_\alpha$ = critical value (2.33 for 99\%), $\sigma$ = volatility, $t$ = horizon

\vspace{2mm}
\textbf{Assumptions:}
\begin{itemize}
\item Normal distribution of returns
\item Linear portfolio exposures
\item Constant volatility and correlations
\end{itemize}

\textbf{Pros:} Fast, simple, transparent \\
\textbf{Cons:} Underestimates tail risk, poor for options

\column{0.48\textwidth}
\textbf{2. Historical Simulation:}
\begin{itemize}
\item Apply past N days returns to current positions
\item Sort simulated P\&L outcomes
\item VaR = $\alpha$-quantile of distribution
\item Typical lookback: 250-500 days
\end{itemize}

\textbf{Pros:} No distributional assumptions, captures fat tails \\
\textbf{Cons:} Backward-looking, sensitive to window choice

\vspace{2mm}
\textbf{3. Monte Carlo Simulation:}
\begin{itemize}
\item Generate 10,000+ random scenarios
\item Price portfolio under each scenario
\item Calculate VaR from simulated distribution
\item Can model path-dependent options
\end{itemize}

\textbf{Pros:} Flexible, handles complex derivatives \\
\textbf{Cons:} Computationally intensive, model risk
\end{columns}
\end{frame}

\begin{frame}{VaR Backtesting and Model Validation}
\begin{columns}[T]
\column{0.48\textwidth}
\textbf{Backtesting Framework:}
\begin{itemize}
\item Compare daily VaR forecasts to realized P\&L
\item Count exceedances (days where loss $>$ VaR)
\item \textbf{Expected Exceedances:} 1\% of days for 99\% VaR
\item Over 250 days: expect 2-3 exceedances
\end{itemize}

\vspace{2mm}
\textbf{Statistical Tests:}
\begin{itemize}
\item \textbf{Kupiec Test (1995):} Likelihood ratio test for correct number of exceedances
\item \textbf{Christoffersen Test (1998):} Independence of exceedances
\item \textbf{Traffic Light Approach:} Green (0-4), Yellow (5-9), Red (10+) zones for 250 days at 99\%
\end{itemize}

\column{0.48\textwidth}
\textbf{Basel III Traffic Lights:}
\begin{center}
\small
\begin{tabular}{ll}
\textbf{Zone} & \textbf{Exceedances (250 days)} \\
\midrule
Green & 0-4 \\
Yellow & 5-9 \\
Red & 10+ \\
\end{tabular}
\end{center}

Yellow/Red zones trigger capital multiplier increases

\vspace{2mm}
\textbf{Common Failures:}
\begin{itemize}
\item Clustered exceedances (volatility regime change)
\item Underestimation during crisis periods
\item Model drift due to changing market conditions
\item Inadequate stress scenario coverage
\end{itemize}

\vspace{2mm}
\textit{2008 Crisis: Most banks had 20-40 VaR exceedances vs expected 2-3}
\end{columns}
\end{frame}

\begin{frame}{Limitations of VaR and Alternatives}
\begin{columns}[T]
\column{0.48\textwidth}
\textbf{VaR Limitations:}
\begin{itemize}
\item \textbf{Not Coherent:} Fails sub-additivity property
\item \textbf{Ignores Tail:} No information beyond VaR level
\item \textbf{Diversification Paradox:} Portfolio VaR can exceed sum of components
\item \textbf{Model Risk:} Sensitive to assumptions (normality, correlation stability)
\item \textbf{Procyclical:} VaR increases during stress, forcing deleveraging
\end{itemize}

\vspace{2mm}
\textbf{Non-Subadditivity Example:}
\begin{itemize}
\item Asset A: 99\% VaR = \$100M
\item Asset B: 99\% VaR = \$100M
\item Portfolio A+B: VaR could be \$220M (if extreme correlation)
\item Violates diversification intuition
\end{itemize}

\column{0.48\textwidth}
\textbf{Expected Shortfall (ES / CVaR):}

$$\text{ES}_\alpha = \mathbb{E}[L \mid L > \text{VaR}_\alpha]$$

\textit{Average loss beyond VaR threshold}

\vspace{2mm}
\textbf{Advantages over VaR:}
\begin{itemize}
\item Coherent risk measure (satisfies all axioms)
\item Captures tail risk beyond VaR cutoff
\item Subadditive: encourages diversification
\item Basel III shift: ES replacing VaR for market risk (2023)
\end{itemize}

\vspace{2mm}
\textbf{Challenges:}
\begin{itemize}
\item Less intuitive for communication
\item Harder to backtest (tail events rare)
\item More sensitive to model assumptions
\item Regulatory adoption still evolving
\end{itemize}
\end{columns}
\end{frame}

\section{Stress Testing and Scenario Analysis}

\begin{frame}{Stress Testing Framework}
\begin{columns}[T]
\column{0.48\textwidth}
\textbf{Types of Stress Tests:}
\begin{itemize}
\item \textbf{Sensitivity Analysis:} Single risk factor shock (e.g., +100 bps rates)
\item \textbf{Scenario Analysis:} Coherent multi-factor scenarios
\item \textbf{Historical Scenarios:} Replay past crises (2008, COVID-19)
\item \textbf{Hypothetical Scenarios:} Forward-looking extreme events
\item \textbf{Reverse Stress Tests:} Find scenarios causing failure
\end{itemize}

\vspace{2mm}
\textbf{Regulatory Stress Tests:}
\begin{itemize}
\item \textbf{CCAR (US):} Comprehensive Capital Analysis and Review
\item \textbf{EBA (EU):} European Banking Authority stress tests
\item \textbf{PRA (UK):} Annual Cyclical Scenario, Biennial Exploratory
\end{itemize}

\column{0.48\textwidth}
\textbf{Scenario Design Principles:}
\begin{itemize}
\item \textbf{Severity:} Plausible but extreme (1-in-30 year events)
\item \textbf{Coherence:} Internally consistent macro narrative
\item \textbf{Coverage:} All material risk factors
\item \textbf{Granularity:} Regional and sectoral detail
\item \textbf{Horizon:} Multi-year path (typically 3-5 years)
\end{itemize}

\vspace{2mm}
\textbf{Example Scenario (CCAR 2024):}
\begin{itemize}
\item Severely Adverse: Unemployment 10\%, GDP -3.5\%
\item Equity markets down 40\%
\item Commercial real estate prices -35\%
\item Corporate bond spreads widen 400 bps
\end{itemize}
\end{columns}
\end{frame}

\begin{frame}{Historical Crisis Scenarios}
\begin{columns}[T]
\column{0.48\textwidth}
\textbf{2008 Global Financial Crisis:}
\begin{itemize}
\item S\&P 500: -38\% (2008)
\item VIX spike: 89 (Nov 2008)
\item Investment-grade spreads: +500 bps
\item High-yield spreads: +1700 bps
\item Lehman bankruptcy: CDS spreads explode
\end{itemize}

\vspace{2mm}
\textbf{COVID-19 Crash (March 2020):}
\begin{itemize}
\item S\&P 500: -34\% in 23 days (fastest bear market)
\item Oil: -50\% (WTI negative pricing April 2020)
\item Treasury volatility: MOVE Index to 165
\item Credit markets freeze (Fed intervention)
\end{itemize}

\column{0.48\textwidth}
\textbf{1998 LTCM Crisis:}
\begin{itemize}
\item Russia default + devaluation
\item Flight to quality (Treasuries rally, spreads widen)
\item Equity vol spike: VIX to 45
\item Liquidity-driven correlations break models
\end{itemize}

\vspace{2mm}
\textbf{European Sovereign Debt Crisis (2011-2012):}
\begin{itemize}
\item Greek, Irish, Portuguese bond yields spike
\item 2-year Greek yields exceed 100\%
\item Bank CDS correlate with sovereign risk
\item EUR/USD volatility and basis blowouts
\end{itemize}

\vspace{2mm}
\textit{Key Lesson: Correlations increase to 1 during crises (diversification fails)}
\end{columns}
\end{frame}

\begin{frame}{Reverse Stress Testing}
\begin{columns}[T]
\column{0.48\textwidth}
\textbf{Methodology:}
\begin{itemize}
\item \textbf{Start with Outcome:} Failure point (insolvency, regulatory breach)
\item \textbf{Work Backwards:} What scenarios lead to this outcome?
\item \textbf{Assess Plausibility:} How realistic are these scenarios?
\item \textbf{Identify Vulnerabilities:} Concentrations and weaknesses
\item \textbf{Mitigants:} Risk limits, hedges, contingency plans
\end{itemize}

\vspace{2mm}
\textbf{Example (Retail Bank):}
\begin{itemize}
\item Failure: Tier 1 capital ratio below 4.5\%
\item Scenario 1: Residential mortgages default rate 15\%
\item Scenario 2: Wholesale funding freeze + deposit run
\item Scenario 3: Major operational loss + cyber incident
\end{itemize}

\column{0.48\textwidth}
\textbf{Regulatory Requirements:}
\begin{itemize}
\item \textbf{PRA (UK):} Annual reverse stress test mandatory
\item \textbf{EBA (EU):} Recovery and Resolution Plans
\item \textbf{Fed (US):} Resolution planning (``living wills'')
\end{itemize}

\vspace{2mm}
\textbf{Use Cases:}
\begin{itemize}
\item Identify tail risks not covered in standard tests
\item Challenge business model assumptions
\item Inform risk appetite and limit frameworks
\item Support recovery and resolution planning
\item Board-level strategic discussions
\end{itemize}

\vspace{2mm}
\textit{``What would it take to break us?'' -- Common reverse stress test framing}
\end{columns}
\end{frame}

\section{Model Risk Management}

\begin{frame}{Model Risk Framework}
\begin{columns}[T]
\column{0.48\textwidth}
\textbf{Model Risk Definition (SR 11-7):}

\textit{``The potential for adverse consequences from decisions based on incorrect or misused model outputs and reports.''}

\vspace{2mm}
\textbf{Sources of Model Risk:}
\begin{itemize}
\item \textbf{Fundamental Error:} Incorrect theory or assumptions
\item \textbf{Implementation Error:} Coding bugs, data errors
\item \textbf{Misuse:} Application outside intended scope
\item \textbf{Parameter Error:} Mis-calibration or estimation error
\item \textbf{Data Quality:} Missing, stale, or incorrect inputs
\end{itemize}

\column{0.48\textwidth}
\textbf{Regulatory Guidance:}
\begin{itemize}
\item \textbf{SR 11-7 (Fed, 2011):} Supervisory Guidance on Model Risk Management
\item \textbf{SS1/23 (PRA, 2023):} Model risk management principles
\item \textbf{BCBS 239:} Risk data aggregation and reporting
\end{itemize}

\vspace{2mm}
\textbf{Model Governance Structure:}
\begin{itemize}
\item \textbf{1st Line:} Model development and ownership
\item \textbf{2nd Line:} Independent validation and risk oversight
\item \textbf{3rd Line:} Internal audit
\item \textbf{Model Risk Committee:} Senior governance forum
\end{itemize}
\end{columns}
\end{frame}

\begin{frame}{Model Validation Process}
\begin{columns}[T]
\column{0.48\textwidth}
\textbf{Validation Components:}
\begin{enumerate}
\item \textbf{Conceptual Soundness:}
\begin{itemize}
\item Review theoretical framework
\item Assess assumptions and limitations
\item Evaluate model design choices
\item Literature review and benchmarking
\end{itemize}

\item \textbf{Ongoing Monitoring:}
\begin{itemize}
\item Backtesting and performance metrics
\item Sensitivity and stability analysis
\item Benchmark comparison
\item Process verification
\end{itemize}

\item \textbf{Outcomes Analysis:}
\begin{itemize}
\item Compare predictions to actual outcomes
\item Statistical tests (e.g., VaR backtests)
\item Analyze forecast errors and bias
\item Identify model drift over time
\end{itemize}
\end{enumerate}

\column{0.48\textwidth}
\textbf{Model Tiering (Risk-Based):}
\begin{center}
\small
\begin{tabular}{lll}
\textbf{Tier} & \textbf{Risk} & \textbf{Validation Frequency} \\
\midrule
1 & High & Annual \\
2 & Medium & Biennial \\
3 & Low & Triennial \\
\end{tabular}
\end{center}

\vspace{2mm}
\textbf{High-Risk Model Examples:}
\begin{itemize}
\item Capital calculation (Basel III, IFRS 9)
\item Pricing of illiquid derivatives
\item Stress testing and scenario models
\item Credit risk scorecards
\item Trading desk VaR models
\end{itemize}

\vspace{2mm}
\textbf{Validation Report Contents:}
\begin{itemize}
\item Executive summary and conclusions
\item Scope and methodology
\item Findings and recommendations
\item Limitations and qualifications
\item Model rating (e.g., satisfactory, needs enhancement, unsatisfactory)
\end{itemize}
\end{columns}
\end{frame}

\begin{frame}{Model Risk in Machine Learning}
\begin{columns}[T]
\column{0.48\textwidth}
\textbf{Additional ML Challenges:}
\begin{itemize}
\item \textbf{Explainability:} Black-box models lack transparency
\item \textbf{Overfitting:} High in-sample, poor out-of-sample
\item \textbf{Regime Changes:} Models trained on past may fail in new regimes
\item \textbf{Data Drift:} Input distributions shift over time
\item \textbf{Adversarial Examples:} Susceptible to manipulation
\end{itemize}

\vspace{2mm}
\textbf{ML-Specific Validation:}
\begin{itemize}
\item Cross-validation and holdout sets
\item Feature importance and SHAP analysis
\item Robustness to input perturbations
\item Comparison to simpler benchmark models
\item Continuous monitoring of performance drift
\end{itemize}

\column{0.48\textwidth}
\textbf{Regulatory Concerns:}
\begin{itemize}
\item ECB (2021): Guide on model risk for AI/ML
\item Fed SR 11-7 applies to all models (including ML)
\item Emphasis on documentation and explainability
\item Need for human oversight and expert judgment
\end{itemize}

\vspace{2mm}
\textbf{Emerging Practices:}
\begin{itemize}
\item \textbf{Model Cards:} Standardized documentation
\item \textbf{Champion-Challenger:} Continuous benchmarking
\item \textbf{Ensemble Methods:} Reduce single-model risk
\item \textbf{Explainable AI:} LIME, SHAP for interpretability
\item \textbf{Fairness Testing:} Detect and mitigate bias
\end{itemize}

\vspace{2mm}
\textit{Industry trend: Hybrid models combining ML predictions with traditional risk frameworks}
\end{columns}
\end{frame}

\section{Enterprise Risk Management Systems}

\begin{frame}{ERM System Architecture}
\begin{columns}[T]
\column{0.48\textwidth}
\textbf{Core Components:}
\begin{enumerate}
\item \textbf{Data Aggregation Layer:}
\begin{itemize}
\item Trade/position feeds from front office
\item Market data (prices, curves, volatilities)
\item Reference data (securities master, counterparties)
\item ETL processes and data quality checks
\end{itemize}

\item \textbf{Risk Calculation Engine:}
\begin{itemize}
\item Sensitivities (Greeks, DV01, duration)
\item VaR and stress tests
\item Counterparty credit risk (CVA, PFE)
\item Aggregation across desks and entities
\end{itemize}

\item \textbf{Limit Monitoring and Alerting:}
\begin{itemize}
\item Real-time limit checks
\item Breach notifications and escalation
\item Approval workflows for exceptions
\end{itemize}
\end{enumerate}

\column{0.48\textwidth}
\textbf{4. Reporting and Analytics:}
\begin{itemize}
\item Interactive dashboards (Tableau, Power BI)
\item Regulatory reports (CCAR, FRTB)
\item Ad-hoc analysis and drill-down
\item Historical P\&L and risk attribution
\end{itemize}

\vspace{2mm}
\textbf{Leading Platforms:}
\begin{itemize}
\item \textbf{Bloomberg AIM:} Multi-asset risk analytics
\item \textbf{MSCI RiskMetrics:} Portfolio risk and performance
\item \textbf{SunGard (FIS) Adaptiv:} Counterparty credit and CVA
\item \textbf{Murex:} Front-to-risk integrated platform
\item \textbf{Calypso:} Cross-asset trading and risk
\end{itemize}

\vspace{2mm}
\textit{Typical latency: Intraday VaR calculated every 15-30 minutes; EOD full suite in 2-4 hours}
\end{columns}
\end{frame}

\begin{frame}{Risk Aggregation Challenges}
\begin{columns}[T]
\column{0.48\textwidth}
\textbf{Technical Challenges:}
\begin{itemize}
\item \textbf{Data Latency:} T+1 positions from legacy systems
\item \textbf{Reconciliation:} Front office vs risk systems breaks
\item \textbf{Market Data:} Missing or stale prices (illiquid securities)
\item \textbf{Calculation Performance:} Monte Carlo for large portfolios
\item \textbf{Infrastructure:} Grid computing and cloud scaling
\end{itemize}

\vspace{2mm}
\textbf{Organizational Challenges:}
\begin{itemize}
\item Siloed data across business units
\item Multiple risk systems (acquisitions)
\item Inconsistent methodologies across desks
\item Manual data adjustments and overrides
\end{itemize}

\column{0.48\textwidth}
\textbf{BCBS 239 Principles (2013):}
\begin{enumerate}
\item Governance: Clear ownership and accountability
\item Data Architecture: Robust and flexible
\item Accuracy and Integrity: Automated controls
\item Completeness and Timeliness: Comprehensive and fast
\item Adaptability: Support ad-hoc requests
\item Distribution: Appropriate access and security
\end{enumerate}

\vspace{2mm}
\textbf{Modern Solutions:}
\begin{itemize}
\item Data lakes and real-time streaming (Kafka)
\item Cloud-based risk engines (AWS, Azure, GCP)
\item In-memory computing (GridGain, Hazelcast)
\item Standardized data models (FINOS, CDM)
\end{itemize}
\end{columns}
\end{frame}

\begin{frame}{Real-Time Risk Management}
\begin{columns}[T]
\column{0.48\textwidth}
\textbf{Intraday Risk Monitoring:}
\begin{itemize}
\item \textbf{Pre-Trade Checks:} Order price/size limits (microseconds)
\item \textbf{Incremental VaR:} Add/remove trade impact
\item \textbf{Greeks Monitoring:} Real-time delta, gamma, vega
\item \textbf{Stress Ladder:} Continuous recalculation
\item \textbf{P\&L Attribution:} Explain intraday P\&L moves
\end{itemize}

\vspace{2mm}
\textbf{Technology Stack:}
\begin{itemize}
\item \textbf{In-Memory Grids:} Apache Ignite, Hazelcast
\item \textbf{Stream Processing:} Kafka, Flink, Spark Streaming
\item \textbf{GPUs:} Massively parallel Monte Carlo
\item \textbf{FPGA:} Ultra-low latency risk calculations
\end{itemize}

\column{0.48\textwidth}
\textbf{Use Cases:}
\begin{itemize}
\item \textbf{Algorithmic Trading:} Real-time position limits
\item \textbf{Market Making:} Inventory risk management
\item \textbf{Prime Brokerage:} Client margin calculations
\item \textbf{Treasury:} Intraday liquidity risk
\end{itemize}

\vspace{2mm}
\textbf{Performance Requirements:}
\begin{itemize}
\item Pre-trade checks: under 1 millisecond
\item Incremental VaR: under 100 milliseconds
\item Full portfolio VaR: under 5 minutes (10k+ positions)
\item Stress tests: under 15 minutes
\end{itemize}

\vspace{2mm}
\textit{Leading trading firms: Full risk recalculation every 100-500 milliseconds for active portfolios}
\end{columns}
\end{frame}

\section{Regulatory Capital and Risk Metrics}

\begin{frame}{Basel III Market Risk Framework}
\begin{columns}[T]
\column{0.48\textwidth}
\textbf{Standardized Approach (SA):}
\begin{itemize}
\item Risk-weighted buckets by asset class
\item Sensitivity-based method (Delta, Vega, Curvature)
\item Residual risk add-on (exotic options)
\item Simple, transparent, less risk-sensitive
\end{itemize}

\vspace{2mm}
\textbf{Internal Models Approach (IMA):}
\begin{itemize}
\item Expected Shortfall (ES) replaces VaR
\item 97.5\% ES over 10-day horizon
\item Stressed ES (calibrated to stress period)
\item Default risk charge (jump-to-default)
\end{itemize}

\vspace{2mm}
\textbf{IMA Formula:}
$$\text{Capital} = \max(\text{ES}_t, mc \cdot \text{ES}_{avg}) + \text{SES} + \text{DRC}$$
where $mc$ = multiplier (1.5+), $\text{SES}$ = Stressed ES, $\text{DRC}$ = Default Risk Charge

\column{0.48\textwidth}
\textbf{P\&L Attribution Test:}
\begin{itemize}
\item Compare theoretical P\&L (risk system) to actual (front office)
\item \textbf{Unexplained P\&L:} Absolute difference
\item \textbf{Threshold:} Exceed on max 12 days/year
\item Failure $\rightarrow$ desk removed from IMA
\end{itemize}

\vspace{2mm}
\textbf{Backtesting (ES):}
\begin{itemize}
\item More challenging than VaR (tail events)
\item Traffic light approach adapted for ES
\item Greater reliance on P\&L attribution
\end{itemize}

\vspace{2mm}
\textbf{Implementation (FRTB):}
\begin{itemize}
\item Fundamental Review of Trading Book
\item Effective: January 2023 (extended to 2025 for some)
\item Capital increase: 20-70\% vs Basel II.5
\item Driven by: tighter liquidity horizons, default risk, stressed calibration
\end{itemize}
\end{columns}
\end{frame}

\begin{frame}{Counterparty Credit Risk Metrics}
\begin{columns}[T]
\column{0.48\textwidth}
\textbf{Exposure Metrics:}
\begin{itemize}
\item \textbf{Current Exposure (CE):} Replacement cost today
\item \textbf{Potential Future Exposure (PFE):} High percentile (95\%, 97.5\%) of future exposure distribution
\item \textbf{Expected Positive Exposure (EPE):} Average exposure over time
\item \textbf{Effective EPE:} Non-decreasing EPE for capital
\end{itemize}

\vspace{2mm}
\textbf{EPE Calculation:}
$$\text{EPE}(t) = \mathbb{E}[\max(V(t), 0)]$$
where $V(t)$ = mark-to-market value at time $t$

\vspace{2mm}
\textbf{CVA (Credit Valuation Adjustment):}
$$\text{CVA} = (1-R) \sum_{i=1}^{n} \text{EE}(t_i) \cdot \text{PD}(t_{i-1}, t_i)$$
where $R$ = recovery rate, $\text{EE}$ = expected exposure, $\text{PD}$ = default probability

\column{0.48\textwidth}
\textbf{XVA Framework:}
\begin{itemize}
\item \textbf{CVA:} Credit risk of counterparty
\item \textbf{DVA:} Credit risk of own entity
\item \textbf{FVA:} Funding cost of uncollateralized exposure
\item \textbf{MVA:} Margin valuation adjustment (initial margin cost)
\item \textbf{KVA:} Capital valuation adjustment
\end{itemize}

\vspace{2mm}
\textbf{Calculation Challenges:}
\begin{itemize}
\item Computationally intensive (nested Monte Carlo)
\item Wrong-way risk (exposure-default correlation)
\item Collateral modeling (CSA agreements)
\item Netting set aggregation
\end{itemize}

\vspace{2mm}
\textit{Major banks: CVA desks actively hedge CVA exposure via CDS and equity positions}
\end{columns}
\end{frame}

\begin{frame}{Summary and Key Takeaways}
\begin{columns}[T]
\column{0.48\textwidth}
\textbf{Value at Risk:}
\begin{itemize}
\item 99\% VaR: Maximum loss exceeded 1\% of days
\item Methodologies: Parametric, Historical, Monte Carlo
\item Backtesting via traffic light approach
\item Limitations: Non-coherent, ignores tail
\item Expected Shortfall replacing VaR (Basel III)
\end{itemize}

\vspace{2mm}
\textbf{Stress Testing:}
\begin{itemize}
\item Sensitivity, scenario, and reverse stress tests
\item Regulatory: CCAR (US), EBA (EU) stress tests
\item Historical scenarios (2008, COVID-19) inform design
\item Reverse stress tests identify failure points
\end{itemize}

\column{0.48\textwidth}
\textbf{Model Risk Management:}
\begin{itemize}
\item SR 11-7: Independent validation mandatory
\item Conceptual soundness, ongoing monitoring, outcomes analysis
\item ML models: Explainability and drift challenges
\item Model governance: 3 lines of defense
\end{itemize}

\vspace{2mm}
\textbf{Enterprise Risk Systems:}
\begin{itemize}
\item Real-time risk aggregation and limit monitoring
\item BCBS 239: Data governance principles
\item Cloud and in-memory computing for speed
\item Basel III FRTB: ES replaces VaR, capital increase 20-70\%
\end{itemize}
\end{columns}
\end{frame}

\end{document}
