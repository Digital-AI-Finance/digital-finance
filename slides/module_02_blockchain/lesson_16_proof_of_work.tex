\documentclass[8pt,aspectratio=169]{beamer}
\usetheme{Madrid}
\usepackage{graphicx,booktabs,adjustbox,multicol,amsmath,amssymb}
\definecolor{mlblue}{RGB}{0,102,204}
\definecolor{mlpurple}{RGB}{51,51,178}
\definecolor{mllavender}{RGB}{173,173,224}
\definecolor{mllavender2}{RGB}{193,193,232}
\definecolor{mllavender3}{RGB}{204,204,235}
\definecolor{mllavender4}{RGB}{214,214,239}
\definecolor{mlorange}{RGB}{255,127,14}
\definecolor{mlgreen}{RGB}{44,160,44}
\definecolor{mlred}{RGB}{214,39,40}
\setbeamercolor{palette primary}{bg=mllavender3,fg=mlpurple}
\setbeamercolor{palette secondary}{bg=mllavender2,fg=mlpurple}
\setbeamercolor{palette tertiary}{bg=mllavender,fg=white}
\setbeamercolor{structure}{fg=mlpurple}
\setbeamercolor{frametitle}{fg=mlpurple,bg=mllavender3}
\setbeamertemplate{navigation symbols}{}
\setbeamertemplate{itemize items}[circle]
\setbeamersize{text margin left=5mm,text margin right=5mm}

\title{Lesson 16: Proof of Work}
\subtitle{Module 2: Blockchain Fundamentals}
\author{Digital Finance}
\date{}

\begin{document}

\begin{frame}
\titlepage
\end{frame}

\begin{frame}{The Double-Spending Problem}
\begin{columns}[T]
\scriptsize
\column{0.48\textwidth}
\textbf{Digital Money Challenge:}
\begin{itemize}
\item Digital files are copyable
\item How to prevent spending same coin twice?
\item Traditional solution: Central authority (bank)
\end{itemize}

\vspace{3mm}
\textbf{Decentralized Challenge:}
\begin{itemize}
\item No central ledger
\item Network latency
\item Conflicting transactions
\item Malicious actors
\end{itemize}

\column{0.48\textwidth}
\includegraphics[width=\textwidth]{charts/lesson_16/double_spending_scenario.pdf}
\end{columns}
\end{frame}

\begin{frame}{Consensus Problem: Agreeing on Transaction Order}
\begin{center}
\includegraphics[width=0.85\textwidth]{charts/lesson_16/consensus_problem.pdf}
\end{center}
\vspace{-3mm}
\textbf{Key Questions:}
\begin{itemize}
\item Which transaction came first?
\item Who decides the canonical order?
\item How to prevent censorship or manipulation?
\item How to incentivize honest behavior?
\end{itemize}
\end{frame}

\begin{frame}{Proof of Work: The Solution}
\begin{columns}[T]
\scriptsize
\column{0.48\textwidth}
\textbf{Core Idea:}
\begin{itemize}
\item Make block creation expensive
\item Require computational work
\item Probability-based selection
\item Longest chain wins
\end{itemize}

\vspace{3mm}
\textbf{Properties:}
\begin{itemize}
\item Sybil resistance (one CPU = one vote)
\item Objective chain selection rule
\item Economic security
\item No coordination needed
\end{itemize}

\column{0.48\textwidth}
\includegraphics[width=\textwidth]{charts/lesson_16/pow_concept.pdf}
\end{columns}
\end{frame}

\begin{frame}{Hash Puzzle: Finding the Nonce}
\textbf{Mining Goal:} Find nonce such that block hash is below target

\[
\text{SHA256}(\text{Block Header}) < \text{Target}
\]

\vspace{3mm}
\textbf{Block Header Contains:}
\begin{itemize}
\item Previous block hash
\item Merkle root (transaction summary)
\item Timestamp
\item Difficulty target
\item \textcolor{mlred}{Nonce} (number to vary)
\end{itemize}

\vspace{3mm}
\textbf{Example:}
\small
\begin{align*}
\text{Target:} & \quad 0000000000000000000f1a2b3c4d... \\
\text{Hash attempt 1:} & \quad 8a3f2e1d9c... \quad \text{(too high)} \\
\text{Hash attempt 2:} & \quad 0000000000000000000a1b2c3d... \quad \text{(success!)}
\end{align*}
\end{frame}

\begin{frame}{Mining Process Visualization}
\begin{center}
\includegraphics[width=0.9\textwidth]{charts/lesson_16/mining_process.pdf}
\end{center}
\vspace{-3mm}
\textbf{Steps:}
\begin{enumerate}
\item Collect transactions from mempool
\item Build Merkle tree, create block header
\item Try different nonces (brute force search)
\item Hash until target reached
\item Broadcast valid block to network
\end{enumerate}
\end{frame}

\begin{frame}{Difficulty Target: Controlling Block Time}
\begin{columns}[T]
\scriptsize
\column{0.48\textwidth}
\textbf{Target Representation:}
\[
\text{Target} = \text{coefficient} \times 2^{8(\text{exponent}-3)}
\]

\vspace{3mm}
\textbf{Difficulty:}
\[
\text{Difficulty} = \frac{\text{Max Target}}{\text{Current Target}}
\]

\vspace{3mm}
\textbf{Bitcoin:}
\begin{itemize}
\item Target block time: 10 minutes
\item Adjusts every 2016 blocks (2 weeks)
\item Difficulty $\propto$ total hashrate
\end{itemize}

\column{0.48\textwidth}
\includegraphics[width=\textwidth]{charts/lesson_16/difficulty_adjustment.pdf}
\end{columns}
\end{frame}

\begin{frame}{Difficulty Over Time: Bitcoin Example}
\begin{center}
\includegraphics[width=0.95\textwidth]{charts/lesson_16/bitcoin_difficulty_history.pdf}
\end{center}
\vspace{-3mm}
\textbf{Observations:}
\begin{itemize}
\item Exponential growth from 2009 to 2024
\item Difficulty in 2024: $\sim 10^{13}$ times harder than 2009
\item Hashrate: From CPU mining to specialized ASICs
\end{itemize}
\end{frame}

\begin{frame}{Mining Probability and Expected Time}
\textbf{Probability of Success per Hash:}
\[
P(\text{success}) = \frac{\text{Target}}{2^{256}}
\]

\vspace{3mm}
\textbf{Expected Number of Hashes:}
\[
E[\text{hashes}] = \frac{2^{256}}{\text{Target}} = \text{Difficulty} \times 2^{32}
\]

\vspace{3mm}
\textbf{Expected Time to Find Block:}
\[
T = \frac{\text{Difficulty} \times 2^{32}}{\text{Hashrate}}
\]

\vspace{3mm}
\textbf{Example:} Difficulty = 50 trillion, Hashrate = 100 TH/s
\[
T = \frac{50 \times 10^{12} \times 2^{32}}{100 \times 10^{12}} \approx 2147 \text{ seconds} \approx 36 \text{ minutes}
\]
\end{frame}

\begin{frame}{Mining Difficulty vs Hashrate}
\begin{center}
\includegraphics[width=0.9\textwidth]{charts/lesson_16/hashrate_vs_difficulty.pdf}
\end{center}
\vspace{-3mm}
\textbf{Relationship:}
\begin{itemize}
\item Hashrate increases $\rightarrow$ blocks found faster
\item Difficulty adjusts upward $\rightarrow$ restores 10-min average
\item Self-regulating system maintains predictable issuance
\end{itemize}
\end{frame}

\begin{frame}{Blockchain Security: The 51\% Attack}
\begin{columns}[T]
\scriptsize
\column{0.48\textwidth}
\textbf{Attack Scenario:}
\begin{itemize}
\item Attacker controls $>$50\% hashrate
\item Can create longest chain
\item Rewrite transaction history
\item Double-spend attack
\end{itemize}

\vspace{3mm}
\textbf{Limitations:}
\begin{itemize}
\item Cannot steal others' coins
\item Cannot create coins from nothing
\item Cannot change protocol rules
\end{itemize}

\column{0.48\textwidth}
\includegraphics[width=\textwidth]{charts/lesson_16/51_percent_attack.pdf}
\end{columns}
\end{frame}

\begin{frame}{Confirmation Depth: Security Over Time}
\begin{center}
\includegraphics[width=0.85\textwidth]{charts/lesson_16/confirmation_depth.pdf}
\end{center}
\vspace{-3mm}
\textbf{Probability of Reversal:}
\[
P(\text{reorg after } z \text{ blocks}) \approx \left(\frac{q}{p}\right)^z
\]
where $p$ = honest hashrate fraction, $q$ = attacker hashrate fraction

\vspace{2mm}
\textbf{Bitcoin Standard:} 6 confirmations ($\sim$1 hour) for high-value transactions
\end{frame}

\begin{frame}{Mining Economics: Costs and Rewards}
\textbf{Revenue:}
\begin{itemize}
\item Block reward: 3.125 BTC (as of 2024, halves every 4 years)
\item Transaction fees: Variable (0.1--2 BTC per block)
\end{itemize}

\vspace{3mm}
\textbf{Costs:}
\begin{itemize}
\item Hardware (ASICs): \$3,000--\$15,000 per unit
\item Electricity: 3--6 cents per kWh (industrial rates)
\item Cooling, maintenance, facility
\end{itemize}

\vspace{3mm}
\textbf{Profitability Equation:}
\[
\text{Profit} = (\text{Block Reward} + \text{Fees}) \times \text{BTC Price} - \text{Electricity Cost}
\]

\vspace{2mm}
\textbf{Break-even:} Electricity cost $\approx$ 40--60\% of revenue at scale
\end{frame}

\begin{frame}{Mining Hardware Evolution}
\begin{center}
\includegraphics[width=0.9\textwidth]{charts/lesson_16/mining_hardware_evolution.pdf}
\end{center}
\vspace{-3mm}
\begin{itemize}
\item \textbf{2009--2010:} CPU mining (1--10 MH/s)
\item \textbf{2010--2012:} GPU mining (100--500 MH/s)
\item \textbf{2013+:} ASIC mining (1--200+ TH/s)
\item \textbf{Modern ASICs:} Antminer S19 Pro (110 TH/s, 3250W)
\end{itemize}
\end{frame}

\begin{frame}{Energy Consumption: The Elephant in the Room}
\begin{columns}[T]
\scriptsize
\column{0.48\textwidth}
\textbf{Bitcoin Network (2024):}
\begin{itemize}
\item Total hashrate: $\sim$600 EH/s
\item Power consumption: $\sim$150 TWh/year
\item Comparable to Argentina or Netherlands
\end{itemize}

\vspace{3mm}
\textbf{Per Transaction:}
\begin{itemize}
\item $\sim$700 kWh per transaction
\item vs Visa: $\sim$0.001 kWh
\item But: Bitcoin = settlement layer
\end{itemize}

\column{0.48\textwidth}
\includegraphics[width=\textwidth]{charts/lesson_16/energy_consumption.pdf}
\end{columns}
\end{frame}

\begin{frame}{Energy Sources: Renewable Debate}
\begin{center}
\includegraphics[width=0.85\textwidth]{charts/lesson_16/mining_energy_sources.pdf}
\end{center}
\vspace{-3mm}
\textbf{Estimates (2024):}
\begin{itemize}
\item Renewable energy: 40--60\% (hydroelectric, wind, solar)
\item Natural gas: 20--30\%
\item Coal: 10--20\%
\item Nuclear: 5--10\%
\end{itemize}
\textbf{Trend:} Miners seek cheap, stranded renewable energy (e.g., flare gas, curtailed hydro)
\end{frame}

\begin{frame}{Environmental Criticisms and Counterarguments}
\begin{columns}[T]
\scriptsize
\column{0.48\textwidth}
\textbf{Criticisms:}
\begin{itemize}
\item Massive carbon footprint
\item E-waste from obsolete ASICs
\item Inefficient compared to databases
\item Competes with useful computing
\end{itemize}

\column{0.48\textwidth}
\textbf{Counterarguments:}
\begin{itemize}
\item Energy = security (makes attacks expensive)
\item Incentivizes renewable buildout
\item Banking system also energy-intensive
\item Enables censorship-resistant money
\end{itemize}
\end{columns}

\vspace{4mm}
\textbf{Trade-off:} Security vs energy efficiency (Proof of Stake addresses this)
\end{frame}

\begin{frame}{Mining Centralization Risks}
\begin{center}
\includegraphics[width=0.85\textwidth]{charts/lesson_16/mining_pool_distribution.pdf}
\end{center}
\vspace{-3mm}
\textbf{Concerns:}
\begin{itemize}
\item Top 5 pools control $>$70\% hashrate
\item Geographic concentration (China historically dominant, now US/Kazakhstan)
\item Pool operators could censor transactions
\end{itemize}
\textbf{Mitigation:} Miners can switch pools, Stratum V2 protocol improves decentralization
\end{frame}

\begin{frame}{Selfish Mining Attack}
\begin{columns}[T]
\scriptsize
\column{0.48\textwidth}
\textbf{Strategy:}
\begin{enumerate}
\item Miner finds block, keeps secret
\item Continues mining on private chain
\item Reveals when ahead by 2+ blocks
\item Honest chain orphaned
\end{enumerate}

\vspace{3mm}
\textbf{Result:}
\begin{itemize}
\item Unfair revenue (more than hashrate share)
\item Effective with $>$25\% hashrate
\item Wastes other miners' work
\end{itemize}

\column{0.48\textwidth}
\includegraphics[width=\textwidth]{charts/lesson_16/selfish_mining.pdf}
\end{columns}
\end{frame}

\begin{frame}{Alternatives to Proof of Work}
\begin{center}
\begin{tabular}{@{}p{3cm}p{4cm}p{4cm}p{3cm}@{}}
\toprule
\textbf{Mechanism} & \textbf{Selection} & \textbf{Pros} & \textbf{Cons} \\
\midrule
Proof of Work & Computational power & Proven security, decentralized & Energy intensive \\
\midrule
Proof of Stake & Staked capital & Energy efficient & Rich get richer, slashing risk \\
\midrule
Proof of Authority & Approved validators & Fast, low energy & Centralized, permissioned \\
\midrule
Proof of Space & Disk storage & Lower energy than PoW & New, unproven security \\
\bottomrule
\end{tabular}
\end{center}

\vspace{3mm}
\textbf{Note:} Ethereum switched from PoW to PoS in 2022 (The Merge)
\end{frame}

\begin{frame}{Summary}
\begin{itemize}
\item \textbf{Double-Spending Problem:} Solved by probabilistic consensus via PoW
\item \textbf{Mining:} Find nonce making block hash $<$ target (SHA256 puzzle)
\item \textbf{Difficulty:} Auto-adjusts to maintain constant block time (10 min for Bitcoin)
\item \textbf{Security:} 51\% attack possible but expensive; confirmation depth increases safety
\item \textbf{Economics:} Revenue (block reward + fees) vs costs (hardware + electricity)
\item \textbf{Energy Debate:} $\sim$150 TWh/year, trade-off between security and efficiency
\end{itemize}

\vspace{4mm}
\textbf{Next Lesson:} Proof of Stake -- energy-efficient alternative consensus
\end{frame}

\end{document}
