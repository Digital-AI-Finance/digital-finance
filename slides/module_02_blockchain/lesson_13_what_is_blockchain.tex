\documentclass[8pt,aspectratio=169]{beamer}
\usetheme{Madrid}
\usepackage{graphicx}
\usepackage{booktabs}
\usepackage{adjustbox}
\usepackage{multicol}
\usepackage{amsmath}
\usepackage{amssymb}

\definecolor{mlblue}{RGB}{0,102,204}
\definecolor{mlpurple}{RGB}{51,51,178}
\definecolor{mllavender}{RGB}{173,173,224}
\definecolor{mllavender2}{RGB}{193,193,232}
\definecolor{mllavender3}{RGB}{204,204,235}
\definecolor{mllavender4}{RGB}{214,214,239}
\definecolor{mlorange}{RGB}{255, 127, 14}
\definecolor{mlgreen}{RGB}{44, 160, 44}
\definecolor{mlred}{RGB}{214, 39, 40}

\setbeamercolor{palette primary}{bg=mllavender3,fg=mlpurple}
\setbeamercolor{palette secondary}{bg=mllavender2,fg=mlpurple}
\setbeamercolor{palette tertiary}{bg=mllavender,fg=white}
\setbeamercolor{structure}{fg=mlpurple}
\setbeamercolor{frametitle}{fg=mlpurple,bg=mllavender3}
\setbeamertemplate{navigation symbols}{}
\setbeamertemplate{itemize items}[circle]
\setbeamersize{text margin left=5mm,text margin right=5mm}

% Bottom note command for key takeaways
\newcommand{\bottomnote}[1]{%
\vfill
\vspace{-2mm}
\textcolor{mllavender2}{\rule{\textwidth}{0.4pt}}
\vspace{1mm}
\footnotesize
\textbf{#1}
}
\title{Lesson 13: What is Blockchain?}
\subtitle{Module 2: Blockchain and Cryptocurrencies}
\author{Digital Finance}
\date{\today}

\begin{document}

\begin{frame}
\titlepage
\end{frame}

\begin{frame}[t]{The Trust Problem in Digital Transactions}
\vspace{-2mm}
\begin{columns}[T]
\column{0.48\textwidth}
\textbf{Traditional Digital Payments:}
\begin{itemize}
\item Require trusted intermediary (bank)
\item Centralized ledger control
\item Single point of failure
\item Gatekeeping and censorship risk
\item High transaction fees
\end{itemize}

\column{0.48\textwidth}
\textbf{The Double-Spending Problem:}
\begin{itemize}
\item Digital files can be copied
\item Same money spent twice
\item Who determines truth?
\item Intermediaries solve this... at a cost
\end{itemize}
\end{columns}

\vspace{5mm}
\centering
\textit{``How can strangers transact without trusting each other or a central authority?''}
\bottomnote{Digital transactions require trust mechanisms---blockchain removes the need for intermediaries.}
\end{frame}

\begin{frame}[t]{The Evolution of Digital Cash}
\vspace{-2mm}
\begin{table}[h]
\centering
\small
\begin{tabular}{lll}
\toprule
\textbf{Year} & \textbf{Innovation} & \textbf{Limitation} \\
\midrule
1983 & DigiCash (Chaum) & Required central bank \\
1997 & Hashcash (Back) & No transfer mechanism \\
1998 & b-money (Dai) & Theoretical only \\
2005 & Bit Gold (Szabo) & No implementation \\
2008 & \textcolor{mlpurple}{\textbf{Bitcoin (Nakamoto)}} & \textcolor{mlgreen}{\textbf{First working solution}} \\
\bottomrule
\end{tabular}
\end{table}

\vspace{3mm}
\textbf{Key Insight:} All prior attempts failed to solve Byzantine Generals Problem in decentralized networks
\bottomnote{Understanding history helps predict future developments in the technology.}
\end{frame}

\begin{frame}[t]{Satoshi Nakamoto's Breakthrough (October 2008)}
\vspace{-2mm}
\begin{columns}[T]
\column{0.58\textwidth}
\textbf{Bitcoin Whitepaper:}
\begin{itemize}
\item ``Bitcoin: A Peer-to-Peer Electronic Cash System''
\item 9 pages, published on cryptography mailing list
\item Combined existing cryptographic primitives in novel way
\item Genesis block mined January 3, 2009
\end{itemize}

\vspace{3mm}
\textbf{Core Innovations:}
\begin{itemize}
\item Proof-of-Work consensus
\item Decentralized timestamp server
\item Longest chain rule
\item Economic incentives (mining rewards)
\end{itemize}

\column{0.38\textwidth}
\textbf{Mystery Identity:}
\begin{itemize}
\item Unknown person/group
\item Disappeared April 2011
\item Owns ~1M BTC (never moved)
\item Multiple theories, no proof
\end{itemize}

\vspace{3mm}
\small
\textit{Genesis block message:}\\
``The Times 03/Jan/2009\\
Chancellor on brink of\\
second bailout for banks''
\end{columns}
\bottomnote{Bitcoin combined existing cryptographic primitives in a novel way to solve double-spending.}
\end{frame}

\begin{frame}[t]{What is a Blockchain? Core Definition}
\vspace{-2mm}
\textbf{Blockchain:} A distributed, immutable ledger of transactions organized in cryptographically linked blocks

\vspace{3mm}
\begin{columns}[T]
\column{0.48\textwidth}
\textbf{Key Components:}
\begin{enumerate}
\item \textbf{Blocks:} Batches of transactions
\item \textbf{Chain:} Cryptographic links between blocks
\item \textbf{Network:} Distributed nodes maintaining copies
\item \textbf{Consensus:} Agreement mechanism (PoW/PoS)
\item \textbf{Cryptography:} Hash functions + digital signatures
\end{enumerate}

\column{0.48\textwidth}
\textbf{Essential Properties:}
\begin{itemize}
\item \textcolor{mlpurple}{\textbf{Decentralization:}} No single controller
\item \textcolor{mlpurple}{\textbf{Transparency:}} All transactions visible
\item \textcolor{mlpurple}{\textbf{Immutability:}} Cannot alter history
\item \textcolor{mlpurple}{\textbf{Security:}} Cryptographic protection
\item \textcolor{mlpurple}{\textbf{Pseudonymity:}} Addresses, not names
\end{itemize}
\end{columns}
\bottomnote{Clear definitions are essential for understanding complex technical concepts.}
\end{frame}

\begin{frame}[t]{Centralized vs Decentralized Systems}
\vspace{-2mm}
\begin{columns}[T]
\column{0.48\textwidth}
\textbf{Centralized (Traditional):}
\begin{itemize}
\item Single authority controls ledger
\item Fast transaction processing
\item Easy to upgrade/modify
\item Single point of failure
\item Requires trust in intermediary
\item Examples: Banks, PayPal, Visa
\end{itemize}

\vspace{3mm}
\textbf{Advantages:}
\begin{itemize}
\item Efficiency and speed
\item Clear governance
\item Customer support
\end{itemize}

\column{0.48\textwidth}
\textbf{Decentralized (Blockchain):}
\begin{itemize}
\item Multiple nodes maintain ledger
\item Slower (consensus overhead)
\item Difficult to change rules
\item No single point of failure
\item Trustless operation
\item Examples: Bitcoin, Ethereum
\end{itemize}

\vspace{3mm}
\textbf{Advantages:}
\begin{itemize}
\item Censorship resistance
\item Transparency
\item No intermediary needed
\end{itemize}
\end{columns}
\bottomnote{Centralized systems trade trust for efficiency; decentralized systems trade efficiency for trustlessness.}
\end{frame}

\begin{frame}[t]{The Blockchain Trilemma}
\vspace{-2mm}
\centering
\textbf{Impossible to maximize all three simultaneously:}

\vspace{5mm}
\begin{columns}[c]
\column{0.32\textwidth}
\centering
\textcolor{mlpurple}{\textbf{DECENTRALIZATION}}\\
\small
Number of independent validators\\
Resistance to control

\column{0.32\textwidth}
\centering
\textcolor{mlpurple}{\textbf{SECURITY}}\\
\small
Cost to attack network\\
Immutability guarantees

\column{0.32\textwidth}
\centering
\textcolor{mlpurple}{\textbf{SCALABILITY}}\\
\small
Transactions per second\\
Low fees
\end{columns}

\vspace{8mm}
\begin{table}[h]
\centering
\footnotesize
\begin{tabular}{lccc}
\toprule
\textbf{Network} & \textbf{Decentralization} & \textbf{Security} & \textbf{Scalability} \\
\midrule
Bitcoin & High & High & Low (7 TPS) \\
Ethereum & High & High & Medium (15-30 TPS) \\
BSC & Low & Medium & High (100+ TPS) \\
Solana & Medium & Medium & Very High (3000+ TPS) \\
\bottomrule
\end{tabular}
\end{table}
\bottomnote{The blockchain trilemma forces trade-offs between decentralization, security, and scalability.}
\end{frame}

\begin{frame}[t]{How Blockchain Works: Simplified Flow}
\vspace{-2mm}
\textbf{Transaction Lifecycle (6 Steps):}

\vspace{3mm}
\begin{enumerate}
\item \textbf{Initiation:} User broadcasts transaction to network
\item \textbf{Validation:} Nodes verify signature and sufficient balance
\item \textbf{Mempool:} Valid transactions wait in memory pool
\item \textbf{Block Creation:} Miner/validator selects transactions for new block
\item \textbf{Consensus:} Network agrees on new block (PoW/PoS)
\item \textbf{Finalization:} Block added to chain, transaction confirmed
\end{enumerate}

\vspace{5mm}
\textbf{Typical Confirmation Times:}
\begin{itemize}
\item Bitcoin: ~10 minutes per block (6 blocks for finality = 1 hour)
\item Ethereum: ~12 seconds per block (32 blocks for finality = 6-7 minutes)
\item Solana: ~400ms per block (instant practical finality)
\end{itemize}
\bottomnote{Understanding the process flow is key to identifying optimization opportunities.}
\end{frame}

\begin{frame}[t]{Public vs Private Blockchains}
\vspace{-2mm}
\begin{table}[h]
\centering
\small
\begin{tabular}{p{3cm}p{5cm}p{5cm}}
\toprule
\textbf{Feature} & \textbf{Public (Permissionless)} & \textbf{Private (Permissioned)} \\
\midrule
Access & Anyone can join & Invited participants only \\
Validators & Anyone can become validator & Pre-approved validators \\
Transparency & Fully transparent & Controlled visibility \\
Speed & Slower (global consensus) & Faster (known validators) \\
Energy & High (PoW) or Medium (PoS) & Low (simple consensus) \\
Use Cases & Cryptocurrencies, DeFi & Enterprise, supply chain \\
Examples & Bitcoin, Ethereum & Hyperledger, R3 Corda \\
Trust Model & Trustless & Trust in consortium \\
\bottomrule
\end{tabular}
\end{table}

\vspace{3mm}
\textbf{Hybrid Models:} Some networks (e.g., VeChain) combine public chain with private enterprise features
\bottomnote{Public and private blockchains serve different use cases with different trust models.}
\end{frame}

\begin{frame}[t]{Blockchain Use Cases Beyond Cryptocurrency}
\vspace{-2mm}
\begin{columns}[T]
\column{0.48\textwidth}
\textbf{Financial Services:}
\begin{itemize}
\item Cross-border payments (Ripple)
\item Securities settlement (ASX)
\item Trade finance (we.trade)
\item Insurance claims (Etherisc)
\end{itemize}

\vspace{3mm}
\textbf{Supply Chain:}
\begin{itemize}
\item Food traceability (Walmart + IBM)
\item Pharmaceutical tracking
\item Luxury goods authentication
\item Carbon credit tracking
\end{itemize}

\column{0.48\textwidth}
\textbf{Digital Identity:}
\begin{itemize}
\item Self-sovereign identity (DID)
\item Academic credentials
\item Government IDs (Estonia)
\end{itemize}

\vspace{3mm}
\textbf{Other Applications:}
\begin{itemize}
\item Voting systems
\item Real estate registries
\item Intellectual property
\item Healthcare records (HIPAA-compliant)
\item Energy grid management
\end{itemize}
\end{columns}
\bottomnote{Real-world applications demonstrate the practical value of blockchain technology.}
\end{frame}

\begin{frame}[t]{Real-World Example: Walmart Food Traceability}
\vspace{-2mm}
\textbf{Problem:} 2018 E. coli outbreak in romaine lettuce took weeks to trace source

\vspace{3mm}
\textbf{Solution:} Walmart + IBM Food Trust (Hyperledger Fabric)

\vspace{3mm}
\begin{columns}[T]
\column{0.48\textwidth}
\textbf{Before Blockchain:}
\begin{itemize}
\item Manual record keeping
\item 7 days to trace mango origin
\item Paper-based documentation
\item Information silos
\item Difficult recalls
\end{itemize}

\column{0.48\textwidth}
\textbf{After Blockchain:}
\begin{itemize}
\item Digital immutable records
\item 2.2 seconds to trace origin
\item Real-time visibility
\item Shared data access
\item Precise, fast recalls
\end{itemize}
\end{columns}

\vspace{5mm}
\textbf{Impact:} Reduced food waste, improved consumer safety, lower liability costs
\bottomnote{Case studies provide concrete evidence of technology impact and adoption patterns.}
\end{frame}

\begin{frame}[t]{Limitations and Challenges}
\vspace{-2mm}
\begin{columns}[T]
\column{0.48\textwidth}
\textbf{Technical Limitations:}
\begin{itemize}
\item \textcolor{mlred}{Scalability:} Low TPS vs Visa (24,000 TPS)
\item \textcolor{mlred}{Energy:} Bitcoin uses ~150 TWh/year
\item \textcolor{mlred}{Storage:} Bitcoin blockchain > 500 GB
\item \textcolor{mlred}{Finality:} Long confirmation times
\item \textcolor{mlred}{Irreversibility:} No undo for mistakes
\end{itemize}

\column{0.48\textwidth}
\textbf{Adoption Barriers:}
\begin{itemize}
\item Regulatory uncertainty
\item User experience complexity
\item Integration with legacy systems
\item Lack of interoperability
\item Environmental concerns (PoW)
\item Volatility (for crypto)
\end{itemize}
\end{columns}

\vspace{5mm}
\centering
\textbf{Key Insight:} Blockchain is not a universal solution - use only when decentralization and immutability are critical requirements
\bottomnote{Understanding limitations helps identify appropriate use cases and avoid over-engineering.}
\end{frame}

\begin{frame}[t]{Blockchain vs Traditional Database}
\vspace{-2mm}
\begin{table}[h]
\centering
\footnotesize
\begin{tabular}{p{3.5cm}p{5cm}p{5cm}}
\toprule
\textbf{Criterion} & \textbf{Traditional Database} & \textbf{Blockchain} \\
\midrule
Control & Centralized administrator & Distributed consensus \\
CRUD Operations & Create, Read, Update, Delete & Create, Read only (append) \\
Performance & Very fast (ms latency) & Slow (seconds to minutes) \\
Data Integrity & Trust in administrator & Cryptographic guarantees \\
Transparency & Opaque to external parties & Transparent to all participants \\
Cost & Low operational cost & High (consensus overhead) \\
Failure Tolerance & Backup/replication needed & Inherently redundant \\
Auditability & Depends on logging & Complete audit trail \\
\textbf{Best For} & Most business applications & Multi-party distrust scenarios \\
\bottomrule
\end{tabular}
\end{table}

\vspace{3mm}
\textbf{Decision Rule:} Use blockchain ONLY if multiple parties need shared write access without mutual trust
\bottomnote{Comparative analysis helps identify the right tool for specific requirements.}
\end{frame}

\begin{frame}[t]{The Hype Cycle: Where Are We?}
\vspace{-2mm}
\textbf{Gartner Hype Cycle for Blockchain (2015-2024):}

\vspace{3mm}
\begin{itemize}
\item \textbf{2015-2017:} Peak of Inflated Expectations - ``Blockchain will change everything''
\item \textbf{2018-2020:} Trough of Disillusionment - ICO crash, failed enterprise pilots
\item \textbf{2021-2022:} Slope of Enlightenment - Real use cases emerge (DeFi, NFTs, CBDCs)
\item \textbf{2023-2024:} Plateau of Productivity - Mature applications in specific domains
\end{itemize}

\vspace{5mm}
\textbf{Current Reality (2024):}
\begin{itemize}
\item Cryptocurrencies: Established asset class (total market cap ~\$2T)
\item DeFi: \$50B+ total value locked, real financial infrastructure
\item Enterprise: Selective adoption where justified (supply chain, trade finance)
\item CBDCs: 130+ countries exploring, 11 launched (e.g., Nigeria eNaira, Bahamas Sand Dollar)
\end{itemize}
\bottomnote{Technology adoption follows predictable patterns---timing matters for investment decisions.}
\end{frame}

\begin{frame}[t]{Bitcoin Network Statistics (2024)}
\vspace{-2mm}
\begin{columns}[T]
\column{0.48\textwidth}
\textbf{Network Metrics:}
\begin{itemize}
\item \textbf{Hash Rate:} ~600 EH/s
\item \textbf{Active Addresses:} ~1M/day
\item \textbf{Transactions:} ~400k/day
\item \textbf{Block Size:} 1-2 MB average
\item \textbf{Nodes:} ~17,000 reachable
\item \textbf{Mining Difficulty:} Adjusts every 2016 blocks
\end{itemize}

\column{0.48\textwidth}
\textbf{Economic Metrics:}
\begin{itemize}
\item \textbf{Market Cap:} ~\$850B
\item \textbf{Circulating Supply:} 19.5M BTC
\item \textbf{Max Supply:} 21M (hard cap)
\item \textbf{Block Reward:} 6.25 BTC (halves every 4 years)
\item \textbf{Fees:} \$2-50 per transaction
\item \textbf{Energy:} ~150 TWh/year (0.5\% global)
\end{itemize}
\end{columns}

\vspace{3mm}
\footnotesize
\textit{Next halving: April 2024 (reward drops to 3.125 BTC)}
\bottomnote{Network metrics provide objective measures of adoption and ecosystem health.}
\end{frame}

\begin{frame}[t]{Ethereum Network Statistics (2024)}
\vspace{-2mm}
\begin{columns}[T]
\column{0.48\textwidth}
\textbf{Post-Merge Metrics:}
\begin{itemize}
\item \textbf{Consensus:} Proof-of-Stake (Sept 2022)
\item \textbf{Validators:} ~950,000
\item \textbf{Staked ETH:} ~32M (~27\% of supply)
\item \textbf{Transactions:} ~1.2M/day
\item \textbf{Smart Contracts:} ~50M deployed
\item \textbf{Energy:} 99.95\% reduction vs PoW
\end{itemize}

\column{0.48\textwidth}
\textbf{DeFi Ecosystem:}
\begin{itemize}
\item \textbf{TVL:} ~\$25B
\item \textbf{DEX Volume:} ~\$50B/month
\item \textbf{NFT Sales:} ~\$500M/month
\item \textbf{Gas Fees:} \$1-20 (varies)
\item \textbf{ERC-20 Tokens:} ~500k
\item \textbf{Layer 2 Adoption:} Growing (Arbitrum, Optimism)
\end{itemize}
\end{columns}

\vspace{3mm}
\footnotesize
\textit{EIP-4844 (Proto-Danksharding) expected 2024 - major scalability upgrade}
\bottomnote{Network metrics provide objective measures of adoption and ecosystem health.}
\end{frame}

\begin{frame}[t]{Key Terminology Summary}
\vspace{-2mm}
\begin{multicols}{2}
\footnotesize
\textbf{Block:} Batch of transactions\\[1mm]
\textbf{Blockchain:} Chain of cryptographically linked blocks\\[1mm]
\textbf{Node:} Computer maintaining blockchain copy\\[1mm]
\textbf{Miner:} Node creating new blocks (PoW)\\[1mm]
\textbf{Validator:} Node validating blocks (PoS)\\[1mm]
\textbf{Consensus:} Agreement mechanism\\[1mm]
\textbf{Hash:} Cryptographic fingerprint\\[1mm]
\textbf{Nonce:} Number used once (PoW)\\[1mm]
\textbf{Difficulty:} Mining puzzle hardness\\[1mm]
\textbf{Mempool:} Pending transactions pool\\[1mm]
\textbf{UTXO:} Unspent transaction output\\[1mm]
\textbf{Gas:} Transaction fee unit (Ethereum)\\[1mm]
\textbf{Smart Contract:} Self-executing code\\[1mm]
\textbf{DeFi:} Decentralized finance\\[1mm]
\textbf{Layer 1:} Base blockchain\\[1mm]
\textbf{Layer 2:} Scaling solution on top\\[1mm]
\textbf{Fork:} Protocol rule change\\[1mm]
\textbf{51\% Attack:} Majority control threat\\[1mm]
\end{multicols}
\end{frame}

\begin{frame}[t]{Next Lesson Preview}
\vspace{-2mm}
\begin{center}
\Large\textbf{Lesson 14: Blocks and Cryptographic Hashing}
\end{center}

\vspace{5mm}
\textbf{What We'll Cover:}
\begin{itemize}
\item Block structure and anatomy
\item SHA-256 hash function in depth
\item Avalanche effect demonstration
\item Hash pointers and Merkle trees
\item Why blockchain is immutable
\item Practical examples and calculations
\end{itemize}

\vspace{5mm}
\textbf{Prepare:}
\begin{itemize}
\item Review basic binary and hexadecimal notation
\item Understand exponential growth (important for hash space)
\item Install Bitcoin Core or blockchain explorer for hands-on exploration
\end{itemize}
\end{frame}

\begin{frame}[t]{Summary: Key Takeaways}
\vspace{-2mm}
\begin{enumerate}
\item \textbf{Trust Problem:} Blockchain solves double-spending without intermediaries
\item \textbf{Satoshi's Innovation:} Combined existing cryptography with economic incentives
\item \textbf{Core Properties:} Decentralization, transparency, immutability, security
\item \textbf{Trilemma:} Cannot maximize decentralization, security, and scalability simultaneously
\item \textbf{Not a Panacea:} Use only when multiple parties need shared, tamper-proof records
\item \textbf{Real Adoption:} Cryptocurrencies, DeFi, supply chain, identity - but still early stage
\item \textbf{Public vs Private:} Different trust models and use cases
\item \textbf{Evolution:} From hype (2017) to practical applications (2024)
\end{enumerate}

\vspace{3mm}
\centering
\textit{``Blockchain is a solution looking for the right problems - choose wisely.''}
\end{frame}

\end{document}
