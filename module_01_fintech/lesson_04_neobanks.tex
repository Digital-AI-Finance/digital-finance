\documentclass[8pt,aspectratio=169]{beamer}
\usetheme{Madrid}
\usepackage{graphicx,booktabs,adjustbox,multicol,amsmath,amssymb}
\definecolor{mlblue}{RGB}{0,102,204}
\definecolor{mlpurple}{RGB}{51,51,178}
\definecolor{mllavender}{RGB}{173,173,224}
\definecolor{mllavender2}{RGB}{193,193,232}
\definecolor{mllavender3}{RGB}{204,204,235}
\definecolor{mllavender4}{RGB}{214,214,239}
\definecolor{mlorange}{RGB}{255,127,14}
\definecolor{mlgreen}{RGB}{44,160,44}
\definecolor{mlred}{RGB}{214,39,40}
\setbeamercolor{palette primary}{bg=mllavender3,fg=mlpurple}
\setbeamercolor{palette secondary}{bg=mllavender2,fg=mlpurple}
\setbeamercolor{palette tertiary}{bg=mllavender,fg=white}
\setbeamercolor{structure}{fg=mlpurple}
\setbeamercolor{frametitle}{fg=mlpurple,bg=mllavender3}
\setbeamertemplate{navigation symbols}{}
\setbeamertemplate{itemize items}[circle]
\setbeamersize{text margin left=5mm,text margin right=5mm}

% Bottom note command for key takeaways
\newcommand{\bottomnote}[1]{%
\vfill
\vspace{-2mm}
\textcolor{mllavender2}{\rule{\textwidth}{0.4pt}}
\vspace{1mm}
\footnotesize
\textbf{#1}
}
\title{Lesson 4: Neobanks and Challenger Banks}
\subtitle{Module 1: FinTech Fundamentals}
\author{Digital Finance}
\date{}

\begin{document}

\begin{frame}
\titlepage
\end{frame}

% =============================================================================
\section{Neobank Fundamentals}
% =============================================================================

\begin{frame}[t]{Neobank Definition}
\begin{columns}[T]
\scriptsize
\column{0.48\textwidth}
\textbf{Digital-First Banking}
\begin{itemize}
\item No physical branches
\item Mobile app primary interface
\item Cloud-native architecture
\item Lower operating costs
\end{itemize}

\column{0.48\textwidth}
\includegraphics[width=\textwidth]{figures/neobank_vs_traditional.pdf}
\end{columns}
\bottomnote{Neobanks operate without branches---cloud-native architecture enables 10x lower costs.}
\end{frame}

\begin{frame}[t]{Market Leaders}
\begin{center}
\includegraphics[width=0.60\textwidth]{figures/neobank_market_leaders.pdf}
\end{center}
\bottomnote{Revolut leads Europe (35M users), Nubank dominates Latin America (80M users).}
\end{frame}

% =============================================================================
\section{Case Studies}
% =============================================================================

\begin{frame}[t]{Revolut: Case Study}
\begin{columns}[T]
\scriptsize
\column{0.48\textwidth}
\textbf{Growth Trajectory}
\begin{itemize}
\item Founded: 2015 (UK)
\item 35M customers (2023)
\item \$33B valuation
\item 38 countries operational
\end{itemize}

\column{0.48\textwidth}
\includegraphics[width=\textwidth]{figures/revolut_growth.pdf}
\end{columns}
\bottomnote{Revolut: Founded 2015, 35M customers, \$33B valuation, operating in 38 countries.}
\end{frame}

\begin{frame}[t]{Revolut Product Suite}
\begin{center}
\includegraphics[width=0.62\textwidth]{figures/revolut_products/revolut_products.pdf}
\end{center}
\bottomnote{Super-app strategy: payments, crypto, stocks, insurance, travel---all in one app.}
\end{frame}

\begin{frame}[t]{N26: European Challenger}
\begin{columns}[T]
\scriptsize
\column{0.48\textwidth}
\textbf{German Banking License}
\begin{itemize}
\item Founded: 2013 (Berlin)
\item 8M customers
\item EU passporting rights
\item Premium tier focus
\end{itemize}

\column{0.48\textwidth}
\includegraphics[width=\textwidth]{figures/n26_expansion/n26_expansion.pdf}
\end{columns}
\bottomnote{N26: German banking license enables EU passporting---8M customers across Europe.}
\end{frame}

% =============================================================================
\section{Business Models and Economics}
% =============================================================================

\begin{frame}[t]{Neobank Business Models}
\begin{columns}[T]
\scriptsize
\column{0.48\textwidth}
\textbf{Three Archetypes}
\begin{itemize}
\item Freemium (Revolut)
\item Subscription (N26 Metal)
\item Transaction-based (Chime)
\item Hybrid approaches
\end{itemize}

\column{0.48\textwidth}
\includegraphics[width=\textwidth]{figures/neobank_business_models/neobank_business_models.pdf}
\end{columns}
\bottomnote{Three models: freemium (Revolut), subscription (N26 Metal), transaction-based (Chime).}
\end{frame}

\begin{frame}[t]{Unit Economics Analysis}
\begin{center}
\includegraphics[width=0.60\textwidth]{figures/neobank_unit_economics.pdf}
\end{center}
\bottomnote{Unit economics: CAC \$20-80 (vs \$200-400 traditional), LTV depends on cross-sell success.}
\end{frame}

\begin{frame}[t]{Customer Acquisition Cost (CAC)}
\begin{columns}[T]
\scriptsize
\column{0.48\textwidth}
\textbf{Acquisition Metrics}
\begin{itemize}
\item Traditional bank: \$200-400
\item Neobank: \$20-80
\item Viral coefficient: 0.3-0.7
\item Payback period: 6-18 months
\end{itemize}

\column{0.48\textwidth}
\includegraphics[width=\textwidth]{figures/cac_comparison.pdf}
\end{columns}
\bottomnote{Neobank CAC \$20-80 vs traditional \$200-400---viral coefficient 0.3-0.7 drives growth.}
\end{frame}

\begin{frame}[t]{Revenue Streams}
\begin{center}
\includegraphics[width=0.62\textwidth]{figures/neobank_revenue_breakdown/neobank_revenue_breakdown.pdf}
\end{center}
\bottomnote{Revenue mix: interchange (40-60\%), subscriptions (20-30\%), lending (10-20\%), FX (10\%).}
\end{frame}

\begin{frame}[t]{Interchange Revenue Model}
\begin{columns}[T]
\scriptsize
\column{0.48\textwidth}
\textbf{Debit Card Monetization}
\begin{itemize}
\item Interchange fee: 0.2\%-2\%
\item Merchant pays to issuer
\item Volume-dependent profit
\item EU cap: 0.2\% (Durbin)
\end{itemize}

\column{0.48\textwidth}
\includegraphics[width=\textwidth]{figures/interchange_flow/interchange_flow.pdf}
\end{columns}
\bottomnote{Interchange: 0.2-2\% per transaction---EU cap 0.2\% limits European neobank revenue.}
\end{frame}

\begin{frame}[t]{Profitability Challenges}
\begin{columns}[T]
\scriptsize
\column{0.48\textwidth}
\textbf{Path to Breakeven}
\begin{itemize}
\item High growth vs profit trade-off
\item Regulatory capital requirements
\item Scale threshold: 5M+ users
\item Cross-sell dependency
\end{itemize}

\column{0.48\textwidth}
\includegraphics[width=\textwidth]{figures/neobank_profitability/neobank_profitability.pdf}
\end{columns}
\bottomnote{Profitability requires 5M+ users---high growth vs profit trade-off is central challenge.}
\end{frame}

% =============================================================================
\section{Licensing and Partnerships}
% =============================================================================

\begin{frame}[t]{Banking License Strategies}
\begin{center}
\includegraphics[width=0.58\textwidth]{figures/license_strategies/license_strategies.pdf}
\end{center}
\bottomnote{License options: full bank license (N26), e-money license (Revolut EU), or BaaS partner.}
\end{frame}

\begin{frame}[t]{Partner Banking Model}
\begin{columns}[T]
\scriptsize
\column{0.48\textwidth}
\textbf{BaaS Relationships}
\begin{itemize}
\item Chime + Bancorp Bank
\item Faster time to market
\item Lower regulatory burden
\item Revenue sharing (30-50\%)
\end{itemize}

\column{0.48\textwidth}
\includegraphics[width=\textwidth]{figures/baas_partnership/baas_partnership.pdf}
\end{columns}
\bottomnote{BaaS model: Chime + Bancorp Bank---faster time to market but 30-50\% revenue share.}
\end{frame}

% =============================================================================
\section{Competition and Future}
% =============================================================================

\begin{frame}[t]{Competitive Landscape}
\begin{center}
\includegraphics[width=0.62\textwidth]{figures/neobank_positioning/neobank_positioning.pdf}
\end{center}
\bottomnote{Market positioning: premium vs mass market, full-stack vs specialized services.}
\end{frame}

\begin{frame}[t]{Traditional Bank Response}
\begin{columns}[T]
\scriptsize
\column{0.48\textwidth}
\textbf{Digital Transformation}
\begin{itemize}
\item Marcus by Goldman Sachs
\item JPMorgan Chase Mobile
\item Core banking modernization
\item Acquisitions (BBVA $\to$ Simple)
\end{itemize}

\column{0.48\textwidth}
\includegraphics[width=\textwidth]{figures/incumbent_digital_investment/incumbent_digital_investment.pdf}
\end{columns}
\bottomnote{Incumbents respond: Marcus by Goldman, JPMorgan Chase Mobile, core banking modernization.}
\end{frame}

\begin{frame}[t]{Future Outlook}
\begin{columns}[T]
\scriptsize
\column{0.48\textwidth}
\textbf{Market Consolidation}
\begin{itemize}
\item M\&A acceleration expected
\item Super-app convergence
\item Embedded finance integration
\item Regulatory clarity needed
\end{itemize}

\column{0.48\textwidth}
\includegraphics[width=\textwidth]{figures/neobank_future_scenarios/neobank_future_scenarios.pdf}
\end{columns}
\bottomnote{Future: M\&A consolidation, super-app convergence, embedded finance integration.}
\end{frame}

% =============================================================================
\section{Summary}
% =============================================================================

\begin{frame}[t]{Key Takeaways}
\begin{itemize}
\item \textbf{Cost Advantage}: 10x lower CAC than traditional banks
\item \textbf{Business Models}: Interchange + subscriptions + lending
\item \textbf{Unit Economics}: Profitability requires 5M+ scale
\item \textbf{Licensing}: Partner vs own license trade-offs
\item \textbf{Competition}: Incumbents investing heavily in digital
\end{itemize}
\bottomnote{Neobanks disrupt with 10x lower CAC, but profitability at scale remains the key challenge.}
\end{frame}

\end{document}
