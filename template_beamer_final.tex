\documentclass[8pt,aspectratio=169]{beamer}
\usetheme{Madrid}
\usepackage{graphicx}
\usepackage{booktabs}
\usepackage{adjustbox}
\usepackage{multicol}
\usepackage{amsmath}
\usepackage{amssymb}

% Color definitions
\definecolor{mlblue}{RGB}{0,102,204}
\definecolor{mlpurple}{RGB}{51,51,178}
\definecolor{mllavender}{RGB}{173,173,224}
\definecolor{mllavender2}{RGB}{193,193,232}
\definecolor{mllavender3}{RGB}{204,204,235}
\definecolor{mllavender4}{RGB}{214,214,239}
\definecolor{mlorange}{RGB}{255, 127, 14}
\definecolor{mlgreen}{RGB}{44, 160, 44}
\definecolor{mlred}{RGB}{214, 39, 40}
\definecolor{mlgray}{RGB}{127, 127, 127}

% Additional colors for template compatibility
\definecolor{lightgray}{RGB}{240, 240, 240}
\definecolor{midgray}{RGB}{180, 180, 180}

% Apply custom colors to Madrid theme
\setbeamercolor{palette primary}{bg=mllavender3,fg=mlpurple}
\setbeamercolor{palette secondary}{bg=mllavender2,fg=mlpurple}
\setbeamercolor{palette tertiary}{bg=mllavender,fg=white}
\setbeamercolor{palette quaternary}{bg=mlpurple,fg=white}

\setbeamercolor{structure}{fg=mlpurple}
\setbeamercolor{section in toc}{fg=mlpurple}
\setbeamercolor{subsection in toc}{fg=mlblue}
\setbeamercolor{title}{fg=mlpurple}
\setbeamercolor{frametitle}{fg=mlpurple,bg=mllavender3}
\setbeamercolor{block title}{bg=mllavender2,fg=mlpurple}
\setbeamercolor{block body}{bg=mllavender4,fg=black}

% Remove navigation symbols
\setbeamertemplate{navigation symbols}{}

% Clean itemize/enumerate
\setbeamertemplate{itemize items}[circle]
\setbeamertemplate{enumerate items}[default]

% Reduce margins for more content space
\setbeamersize{text margin left=5mm,text margin right=5mm}

% Command for bottom annotation (Madrid-style)
\newcommand{\bottomnote}[1]{%
\vfill
\vspace{-2mm}
\textcolor{mllavender2}{\rule{\textwidth}{0.4pt}}
\vspace{1mm}
\footnotesize
\textbf{#1}
}

% Command for compact list spacing
\newcommand{\compactlist}{%
\setlength{\itemsep}{0pt}%
\setlength{\parskip}{0pt}%
\setlength{\parsep}{0pt}%
}

% Command for chart placeholder with safe sizing
\newcommand{\chartplaceholder}[2][5cm]{%
\begin{center}
\begin{adjustbox}{max width=0.95\textwidth, max height=#1}
\framebox[\textwidth][c]{%
\rule{0pt}{#1}%
\textcolor{midgray}{[#2]}%
}
\end{adjustbox}
\end{center}
}

\title{Beamer Template Collection}
\subtitle{28 Professional Slide Layouts with Madrid Theme}
\author{Template System}
\institute{Academic \& Professional Presentations}
\date{\today}

\begin{document}

% ==================== LAYOUT 1: PLAIN TITLE ====================
\begin{frame}[plain]
\vspace{2cm}
\begin{center}
{\Huge Main Title}\\[0.5cm]
{\Large Subtitle or Description}\\[2cm]
{\normalsize Additional Information}
\end{center}
\end{frame}

% ==================== LAYOUT 2: STANDARD TITLE ====================
\begin{frame}[plain]
\titlepage
\end{frame}

% ==================== TABLE OF CONTENTS ====================
\begin{frame}[t]{Template Overview}
\tableofcontents
\end{frame}

% ==================== SECTION: CONTENT LAYOUTS ====================
\section{Content Layouts}

\begin{frame}[t]
\vfill
\centering
\begin{beamercolorbox}[sep=8pt,center]{title}
\usebeamerfont{title}\Large Content Layouts\par
\end{beamercolorbox}
\vfill
\end{frame}

% ==================== LAYOUT 3: TWO COLUMNS TEXT (FIXED) ====================
\begin{frame}[t]{Two Column Layout - Text}
\begin{columns}[T]
\column{0.48\textwidth}
\textbf{Left Column Header}

Main content for the left side.

Key points:
\begin{itemize}
\item First key point
\item Second key point
\item Third key point
\end{itemize}

\column{0.48\textwidth}
\textbf{Right Column Header}

Supporting content for the right side.

\begin{center}
\framebox[0.85\columnwidth][c]{
\rule{0pt}{4cm}
\textcolor{midgray}{[Supporting Visual]}
}
\end{center}
\end{columns}

\bottomnote{Key takeaway in one sentence}
\end{frame}

% ==================== LAYOUT 4: TWO COLUMNS WITH MATH ====================
\begin{frame}[t]{Two Column Layout - Mathematics}
\begin{columns}[T]
\column{0.48\textwidth}
\textbf{Definition}

A mathematical concept defined:
$$f(x) = ax^2 + bx + c$$

Properties:
\begin{itemize}
\item Property: $a \neq 0$
\item Vertex: $x = -\frac{b}{2a}$
\item Discriminant: $\Delta = b^2 - 4ac$
\end{itemize}

\column{0.48\textwidth}
\textbf{Example}

Specific instance:
$$f(x) = 2x^2 + 3x + 1$$

Result: Minimum at $x = -\frac{3}{4}$

\begin{center}
\framebox[0.85\columnwidth][c]{
\rule{0pt}{3.5cm}
\textcolor{midgray}{[Parabola Graph]}
}
\end{center}
\end{columns}

\bottomnote{Theory paired with visual example}
\end{frame}

% ==================== LAYOUT 5: VISUAL COMPARISON (REDESIGNED) ====================
\begin{frame}[t]{Visual Comparison}
\begin{columns}[T]
\column{0.48\textwidth}
\textbf{Approach A}

\begin{center}
\framebox[0.9\columnwidth][c]{
\rule{0pt}{4.5cm}
\textcolor{midgray}{[Visual: Method A]}
}
\end{center}

\begin{itemize}
\item Key characteristic 1
\item Key characteristic 2
\end{itemize}

\column{0.48\textwidth}
\textbf{Approach B}

\begin{center}
\framebox[0.9\columnwidth][c]{
\rule{0pt}{4.5cm}
\textcolor{midgray}{[Visual: Method B]}
}
\end{center}

\begin{itemize}
\item Key characteristic 1
\item Key characteristic 2
\end{itemize}
\end{columns}

\bottomnote{Use visuals to show differences}
\end{frame}

% ==================== LAYOUT 6: THREE COLUMN VISUAL (REDESIGNED) ====================
\begin{frame}[t]{Three-Way Visual Comparison}
\begin{columns}[T]
\column{0.31\textwidth}
\begin{center}
\textbf{Category A}
\vspace{0.3em}

\framebox[0.95\columnwidth][c]{
\rule{0pt}{4cm}
\textcolor{midgray}{[Chart A]}
}
\end{center}

\begin{itemize}
\item Point 1
\item Point 2
\end{itemize}

\column{0.31\textwidth}
\begin{center}
\textbf{Category B}
\vspace{0.3em}

\framebox[0.95\columnwidth][c]{
\rule{0pt}{4cm}
\textcolor{midgray}{[Chart B]}
}
\end{center}

\begin{itemize}
\item Point 1
\item Point 2
\end{itemize}

\column{0.31\textwidth}
\begin{center}
\textbf{Category C}
\vspace{0.3em}

\framebox[0.95\columnwidth][c]{
\rule{0pt}{4cm}
\textcolor{midgray}{[Chart C]}
}
\end{center}

\begin{itemize}
\item Point 1
\item Point 2
\end{itemize}
\end{columns}

\bottomnote{Three-way comparisons need strong visuals}
\end{frame}

% ==================== SECTION: VISUAL LAYOUTS ====================
\section{Visual Layouts}

\begin{frame}[t]
\vfill
\centering
\begin{beamercolorbox}[sep=8pt,center]{title}
\usebeamerfont{title}\Large Visual Layouts\par
\end{beamercolorbox}
\vfill
\end{frame}

% ==================== LAYOUT 7: FULL WIDTH WITH IMAGE ====================
\begin{frame}[t]{Full Width Content with Image}
\textbf{Main Topic Introduction}

Key concepts:
\begin{itemize}
\item Concept one with brief explanation
\item Concept two with additional details
\end{itemize}

\begin{center}
\framebox[0.9\textwidth][c]{
\rule{0pt}{4cm}
\textcolor{midgray}{[Image/Chart Placeholder]}
}
\end{center}

\bottomnote{Visuals complement textual content}
\end{frame}

% ==================== LAYOUT 8: COLUMNS WITH IMAGE (FIXED) ====================
\begin{frame}[t,shrink=10]{Mixed Media Layout}
\begin{columns}[T]
\column{0.48\textwidth}
\textbf{Text Content}

Explanation of concept.

Important points:
\begin{itemize}
\item First observation
\item Second observation
\item Third observation
\end{itemize}

Formula: $E = mc^2$

\column{0.48\textwidth}
\begin{center}
\framebox[0.9\columnwidth][c]{
\rule{0pt}{5.5cm}
\textcolor{midgray}{[Visual Element]}
}
\end{center}
\end{columns}

\bottomnote{Combine text and visuals}
\end{frame}

% ==================== SECTION: COMPARISONS AND ANALYSIS ====================
\section{Comparisons and Analysis}

\begin{frame}[t]
\vfill
\centering
\begin{beamercolorbox}[sep=8pt,center]{title}
\usebeamerfont{title}\Large Comparisons and Analysis\par
\end{beamercolorbox}
\vfill
\end{frame}

% ==================== LAYOUT 9: DEFINITION-EXAMPLE (SIMPLIFIED) ====================
\begin{frame}[t]{Definition and Example}
\begin{columns}[T]
\column{0.48\textwidth}
\textbf{Definition}

Formal statement of concept.

\textbf{Key Properties}
\begin{itemize}
\item Essential property 1
\item Essential property 2
\end{itemize}

\column{0.48\textwidth}
\textbf{Visual Example}

\begin{center}
\framebox[0.9\columnwidth][c]{
\rule{0pt}{5cm}
\textcolor{midgray}{[Diagram/Example]}
}
\end{center}

Result: \textcolor{mlgreen}{Verified}
\end{columns}

\bottomnote{One clear example beats multiple confusing ones}
\end{frame}

% ==================== LAYOUT 10: COMPARISON (SIMPLIFIED) ====================
\begin{frame}[t]{Method Comparison}
\begin{columns}[T]
\column{0.48\textwidth}
\textbf{Method A}

\begin{center}
\framebox[0.85\columnwidth][c]{
\rule{0pt}{4cm}
\textcolor{midgray}{[Method A Visual]}
}
\end{center}

\textbf{Strengths}
\begin{itemize}
\item Top advantage 1
\item Top advantage 2
\end{itemize}

\textbf{Limitation:} Key constraint

\column{0.48\textwidth}
\textbf{Method B}

\begin{center}
\framebox[0.85\columnwidth][c]{
\rule{0pt}{4cm}
\textcolor{midgray}{[Method B Visual]}
}
\end{center}

\textbf{Strengths}
\begin{itemize}
\item Top advantage 1
\item Top advantage 2
\end{itemize}

\textbf{Limitation:} Key constraint
\end{columns}

\bottomnote{Focus on the most important trade-off}
\end{frame}

% ==================== LAYOUT 11: PROCESS FLOW (VISUAL-FIRST) ====================
\begin{frame}[t]{Process Flow}
\begin{center}
\framebox[0.95\textwidth][c]{
\rule{0pt}{4.5cm}
\textcolor{midgray}{[Process Flow Diagram: Input -> Step 1 -> Step 2 -> Output]}
}
\end{center}

\begin{columns}[T]
\column{0.31\textwidth}
\textbf{Input:} Data source

\column{0.31\textwidth}
\textbf{Process:} Main action

\column{0.31\textwidth}
\textbf{Output:} Result
\end{columns}

\bottomnote{Process flows must be visual}
\end{frame}

% ==================== LAYOUT 12: FORMULA REFERENCE ====================
\begin{frame}[t]{Formula Reference}
\begin{columns}[T]
\column{0.31\textwidth}
\textbf{Category 1}

$$a + b = c$$
$$x^2 + y^2 = r^2$$

\column{0.31\textwidth}
\textbf{Category 2}

$$\int_a^b f(x)\,dx$$
$$\sum_{i=1}^n i = \frac{n(n+1)}{2}$$

\column{0.31\textwidth}
\textbf{Category 3}

$$\nabla \times \vec{F} = 0$$
$$E = \hbar\omega$$
\end{columns}

\vspace{0.5em}
\begin{center}
\framebox[0.8\textwidth][c]{
\rule{0pt}{3.5cm}
\textcolor{midgray}{[Formula Visualization/Diagram]}
}
\end{center}

\bottomnote{Quick reference formulas by category}
\end{frame}

% ==================== LAYOUT 13: SUMMARY (SIMPLIFIED) ====================
\begin{frame}[t,shrink=10]{Summary Layout}
\begin{columns}[T]
\column{0.48\textwidth}
\textbf{Key Concepts}
\begin{itemize}
\item Main idea 1
\item Main idea 2
\item Main idea 3
\end{itemize}

\column{0.48\textwidth}
\textbf{Applications}
\begin{itemize}
\item Real-world use 1
\item Real-world use 2
\item Next steps / Further reading
\end{itemize}
\end{columns}

\vspace{0.5em}
\begin{center}
\framebox[0.95\textwidth][c]{
\rule{0pt}{4cm}
\textcolor{midgray}{[Summary Dashboard]}
}
\end{center}

\bottomnote{Summaries consolidate learning}
\end{frame}

% ==================== SECTION: SPECIALIZED FORMATS ====================
\section{Specialized Formats}

\begin{frame}[t]
\vfill
\centering
\begin{beamercolorbox}[sep=8pt,center]{title}
\usebeamerfont{title}\Large Specialized Formats\par
\end{beamercolorbox}
\vfill
\end{frame}

% ==================== LAYOUT 14: Q&A STYLE ====================
\begin{frame}[t]{Question and Answer Format}
\begin{columns}[T]
\column{0.48\textwidth}
\textit{Q1: What is the main purpose?}

Answer explaining the primary goal.

\vspace{0.5em}
\textit{Q2: How does it work?}

Brief explanation of the mechanism.

\column{0.48\textwidth}
\textit{Q3: When should it be used?}

Scenarios for application.

\vspace{0.5em}
\begin{center}
\framebox[0.9\columnwidth][c]{
\rule{0pt}{4cm}
\textcolor{midgray}{[FAQ Diagram]}
}
\end{center}
\end{columns}

\bottomnote{Anticipating questions improves comprehension}
\end{frame}

% ==================== LAYOUT 15: CLOSING SLIDE ====================
\begin{frame}[plain]
\vspace{3cm}
\begin{center}
{\Large Thank you}\\[2cm]
{\normalsize Questions?}\\[1cm]
{\small contact@example.com}
\end{center}
\end{frame}

% ==================== LAYOUT 16: COURSE ROADMAP (VISUAL-FIRST) ====================
\begin{frame}[t]{Course Roadmap}
\begin{center}
\framebox[0.95\textwidth][c]{
\rule{0pt}{4.5cm}
\textcolor{midgray}{[Visual Roadmap: Foundations -> Intermediate -> Advanced -> Applications]}
}
\end{center}

\begin{columns}[T]
\column{0.23\textwidth}
\textbf{Part 1:} Foundations

\column{0.23\textwidth}
\textbf{Part 2:} Intermediate

\column{0.23\textwidth}
\textbf{Part 3:} Advanced

\column{0.23\textwidth}
\textbf{Part 4:} Applications
\end{columns}

\bottomnote{Roadmaps should be visual journeys}
\end{frame}

% ==================== LAYOUT 17: CODE EXAMPLE (SIMPLIFIED) ====================
\begin{frame}[t]{Code Example Layout}
\begin{columns}[T]
\column{0.48\textwidth}
\textbf{Input Code}

\texttt{def function(x):}\\
\texttt{~~~~if x > 0:}\\
\texttt{~~~~~~~~return x * 2}\\
\texttt{~~~~else:}\\
\texttt{~~~~~~~~return -x}

\vspace{0.5em}
\textbf{Explanation}

Doubles positive, negates negative.

\column{0.48\textwidth}
\textbf{Output}: \texttt{10}

\textbf{Examples}
\begin{itemize}
\item $f(5) = 10$
\item $f(-4) = 4$
\item $f(0) = 0$
\end{itemize}

\begin{center}
\framebox[0.85\columnwidth][c]{
\rule{0pt}{3cm}
\textcolor{midgray}{[Function Plot]}
}
\end{center}
\end{columns}

\bottomnote{Code with visual output}
\end{frame}

% ==================== LAYOUT 18: TRADE-OFFS (SIMPLIFIED) ====================
\begin{frame}[t]{Key Trade-offs}
\begin{columns}[T]
\column{0.48\textwidth}
\textbf{Strengths}
\begin{itemize}
\item[\textcolor{mlgreen}{+}] Most important benefit
\item[\textcolor{mlgreen}{+}] Second key benefit
\item[\textcolor{mlgreen}{+}] Third key benefit
\end{itemize}

\column{0.48\textwidth}
\textbf{Limitations}
\begin{itemize}
\item[\textcolor{mlorange}{-}] Most important drawback
\item[\textcolor{mlorange}{-}] Second key drawback
\item[\textcolor{mlorange}{-}] Third key drawback
\end{itemize}
\end{columns}

\vspace{0.5em}
\begin{center}
\framebox[0.7\textwidth][c]{
\rule{0pt}{4cm}
\textcolor{midgray}{[Trade-off Visualization]}
}
\end{center}

\bottomnote{Recommendation: One sentence guidance}
\end{frame}

% ==================== LAYOUT 19: TIMELINE (VISUAL-FIRST) ====================
\begin{frame}[t]{Project Timeline}
\begin{center}
\framebox[0.95\textwidth][c]{
\rule{0pt}{6cm}
\textcolor{midgray}{[Gantt Chart: Phases 1-4 with milestones]}
}
\end{center}

\vspace{0.5em}
\textbf{Key Milestones:} Prototype (Week 6), Beta (Week 15), Launch (Week 18)

\bottomnote{Generate timeline charts using Python}
\end{frame}

% ==================== LAYOUT 20: REFERENCES (SIMPLIFIED) ====================
\begin{frame}[t,shrink=10]{References and Resources}
\begin{columns}[T]
\column{0.48\textwidth}
\textbf{Primary Sources}
\begin{itemize}
\item Author (2024): \textit{Main Title}
\item Researcher (2023): \textit{Key Paper}
\item Expert (2023): \textit{Foundation}
\end{itemize}

\column{0.48\textwidth}
\textbf{Online Resources}
\begin{itemize}
\item Official documentation
\item Video tutorials
\item Community forums
\end{itemize}
\end{columns}

\vspace{0.5em}
\begin{center}
\framebox[0.6\textwidth][c]{
\rule{0pt}{3cm}
\textcolor{midgray}{[QR Code to Resources]}
}
\end{center}

\bottomnote{Curated resources accelerate learning}
\end{frame}

% ==================== SECTION: DATA VISUALIZATION ====================
\section{Data Visualization}

\begin{frame}[t]
\vfill
\centering
\begin{beamercolorbox}[sep=8pt,center]{title}
\usebeamerfont{title}\Large Data Visualization\par
\end{beamercolorbox}
\vfill
\end{frame}

% ==================== LAYOUT 21: FULL-SIZE CHART (FIXED) ====================
\begin{frame}[t]{Full-Size Chart Layout}
\begin{center}
\framebox[0.95\textwidth][c]{
\rule{0pt}{6cm}
\textcolor{midgray}{[Full-Size Chart/Visualization]}
}
\end{center}

\bottomnote{Key insight from the visualization}
\end{frame}

% ==================== LAYOUT 22: CHART WITH EXPLANATIONS ====================
\begin{frame}[t]{Chart with Bottom Explanations}
\begin{center}
\framebox[0.95\textwidth][c]{
\rule{0pt}{5cm}
\textcolor{midgray}{[Main Chart/Visualization]}
}
\end{center}

\vspace{0.5em}
\textbf{Key Observations:}
\begin{itemize}
\item Trend 1: First pattern or insight
\item Trend 2: Second pattern or insight
\item Trend 3: Third pattern or insight
\end{itemize}

\bottomnote{Methodology notes about the data}
\end{frame}

% ==================== NEW SECTION: MULTI-CHART LAYOUTS ====================
\section{Multi-Chart Layouts}

\begin{frame}[t]
\vfill
\centering
\begin{beamercolorbox}[sep=8pt,center]{title}
\usebeamerfont{title}\Large Multi-Chart Layouts\par
\end{beamercolorbox}
\vfill
\end{frame}

% ==================== LAYOUT 23: DUAL CHARTS SIDE-BY-SIDE ====================
\begin{frame}[t]{Dual Chart Comparison}
\begin{columns}[T]
\column{0.48\textwidth}
\begin{center}
\textbf{Chart A Title}\\[0.3em]
\framebox[\columnwidth][c]{
\rule{0pt}{5.5cm}
\textcolor{midgray}{[Chart A]}
}
\end{center}

\column{0.48\textwidth}
\begin{center}
\textbf{Chart B Title}\\[0.3em]
\framebox[\columnwidth][c]{
\rule{0pt}{5.5cm}
\textcolor{midgray}{[Chart B]}
}
\end{center}
\end{columns}

\bottomnote{Direct visual comparison of two related visualizations}
\end{frame}

% ==================== LAYOUT 24: 2x2 CHART GRID ====================
\begin{frame}[t]{Multi-Panel Chart Grid}
\vspace{-0.5em}
\begin{columns}[T]
\column{0.48\textwidth}
\begin{center}
\textbf{Panel A}\\[0.1em]
\framebox[\columnwidth][c]{
\rule{0pt}{2.6cm}
\textcolor{midgray}{[Chart A]}
}
\end{center}

\vspace{0.1em}
\begin{center}
\textbf{Panel C}\\[0.1em]
\framebox[\columnwidth][c]{
\rule{0pt}{2.6cm}
\textcolor{midgray}{[Chart C]}
}
\end{center}

\column{0.48\textwidth}
\begin{center}
\textbf{Panel B}\\[0.1em]
\framebox[\columnwidth][c]{
\rule{0pt}{2.6cm}
\textcolor{midgray}{[Chart B]}
}
\end{center}

\vspace{0.1em}
\begin{center}
\textbf{Panel D}\\[0.1em]
\framebox[\columnwidth][c]{
\rule{0pt}{2.6cm}
\textcolor{midgray}{[Chart D]}
}
\end{center}
\end{columns}

\bottomnote{Small multiples pattern for comprehensive overview}
\end{frame}

% ==================== LAYOUT 25: VERTICAL STACKED CHARTS ====================
\begin{frame}[t]{Vertical Chart Sequence}
\vspace{-0.5em}
\begin{center}
\textbf{Top: Overview Metric}\\[0.1em]
\framebox[0.9\textwidth][c]{
\rule{0pt}{2.8cm}
\textcolor{midgray}{[Chart 1: Primary Visualization]}
}
\end{center}

\vspace{0.1em}

\begin{center}
\textbf{Bottom: Detail Breakdown}\\[0.1em]
\framebox[0.9\textwidth][c]{
\rule{0pt}{2.8cm}
\textcolor{midgray}{[Chart 2: Detailed Analysis]}
}
\end{center}

\bottomnote{Hierarchical: overview followed by detail}
\end{frame}

% ==================== LAYOUT 26: CHART WITH SIDE TABLE ====================
\begin{frame}[t]{Chart with Supporting Data}
\begin{columns}[T]
\column{0.65\textwidth}
\begin{center}
\framebox[\columnwidth][c]{
\rule{0pt}{6cm}
\textcolor{midgray}{[Main Visualization]}
}
\end{center}

\column{0.31\textwidth}
\textbf{Key Statistics}

\begin{table}[h]
\footnotesize
\begin{tabular}{lr}
\toprule
\textbf{Metric} & \textbf{Value} \\
\midrule
Mean & 42.5 \\
Median & 41.2 \\
StdDev & 8.3 \\
N & 1,250 \\
\bottomrule
\end{tabular}
\end{table}

\vspace{0.3em}
\textbf{Legend}
\begin{itemize}
\item[\textcolor{mlblue}{$\blacksquare$}] Series A
\item[\textcolor{mlorange}{$\blacksquare$}] Series B
\end{itemize}
\end{columns}

\bottomnote{Combine visualization with quantitative summary}
\end{frame}

% ==================== LAYOUT 27: TRIPLE CHART HORIZONTAL ====================
\begin{frame}[t]{Three-Way Visual Comparison}
\begin{columns}[T]
\column{0.31\textwidth}
\begin{center}
\textbf{Scenario 1}\\[0.2em]
\framebox[\columnwidth][c]{
\rule{0pt}{5cm}
\textcolor{midgray}{[Chart 1]}
}
\end{center}

\column{0.31\textwidth}
\begin{center}
\textbf{Scenario 2}\\[0.2em]
\framebox[\columnwidth][c]{
\rule{0pt}{5cm}
\textcolor{midgray}{[Chart 2]}
}
\end{center}

\column{0.31\textwidth}
\begin{center}
\textbf{Scenario 3}\\[0.2em]
\framebox[\columnwidth][c]{
\rule{0pt}{5cm}
\textcolor{midgray}{[Chart 3]}
}
\end{center}
\end{columns}

\vspace{0.3em}
\textbf{Insight:} Key differences across scenarios

\bottomnote{Side-by-side enables pattern recognition}
\end{frame}

% ==================== LAYOUT 28: BEFORE/AFTER COMPARISON ====================
\begin{frame}[t]{Before/After Analysis}
\begin{columns}[T]
\column{0.48\textwidth}
\begin{center}
\textbf{Before: Baseline}\\[0.3em]
\framebox[\columnwidth][c]{
\rule{0pt}{5cm}
\textcolor{midgray}{[Before State]}
}
\end{center}

Metric A: 45.2\\
Metric B: 23.8

\column{0.48\textwidth}
\begin{center}
\textbf{After: Post-Intervention}\\[0.3em]
\framebox[\columnwidth][c]{
\rule{0pt}{5cm}
\textcolor{midgray}{[After State]}
}
\end{center}

Metric A: 52.7 \textcolor{mlgreen}{(+16.6\%)}\\
Metric B: 19.3 \textcolor{mlgreen}{(-18.9\%)}
\end{columns}

\bottomnote{Visualizing change reveals intervention effectiveness}
\end{frame}

\end{document}
