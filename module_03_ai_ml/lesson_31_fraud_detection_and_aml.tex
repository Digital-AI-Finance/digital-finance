\documentclass[8pt,aspectratio=169]{beamer}
\usetheme{Madrid}
\usepackage{graphicx,booktabs,adjustbox,multicol,amsmath,amssymb}
\definecolor{mlblue}{RGB}{0,102,204}
\definecolor{mlpurple}{RGB}{51,51,178}
\definecolor{mllavender}{RGB}{173,173,224}
\definecolor{mllavender2}{RGB}{193,193,232}
\definecolor{mllavender3}{RGB}{204,204,235}
\definecolor{mllavender4}{RGB}{214,214,239}
\definecolor{mlorange}{RGB}{255,127,14}
\definecolor{mlgreen}{RGB}{44,160,44}
\definecolor{mlred}{RGB}{214,39,40}
\setbeamercolor{palette primary}{bg=mllavender3,fg=mlpurple}
\setbeamercolor{palette secondary}{bg=mllavender2,fg=mlpurple}
\setbeamercolor{palette tertiary}{bg=mllavender,fg=white}
\setbeamercolor{structure}{fg=mlpurple}
\setbeamercolor{frametitle}{fg=mlpurple,bg=mllavender3}
\setbeamertemplate{navigation symbols}{}
\setbeamertemplate{itemize items}[circle]
\setbeamersize{text margin left=5mm,text margin right=5mm}

% Bottom note command for key takeaways
\newcommand{\bottomnote}[1]{%
\vfill
\vspace{-2mm}
\textcolor{mllavender2}{\rule{\textwidth}{0.4pt}}
\vspace{1mm}
\footnotesize
\textbf{#1}
}
\title{Digital Finance 3: Technology in Finance}
\subtitle{Lesson 31: Fraud Detection and AML}
\author{FHGR}
\date{\today}

\begin{document}

\begin{frame}
\titlepage
\bottomnote{Fraud detection is a continuous arms race between attackers and defenders.}
\end{frame}

\begin{frame}[t]{Learning Objectives}
By the end of this lesson, you will be able to:
\begin{itemize}
\item Classify different types of financial fraud
\item Design ML-based fraud detection systems
\item Understand real-time scoring architectures
\item Apply anomaly detection techniques
\item Balance precision and recall in fraud detection
\item Explain AML compliance requirements
\end{itemize}
\bottomnote{Anomaly detection identifies novel fraud patterns without labeled examples.}
\end{frame}

\begin{frame}[t]{Fraud Types Distribution}
\begin{center}
\includegraphics[width=0.60\textwidth]{figures/fraud_types_distribution/fraud_types_distribution.pdf}
\end{center}
\bottomnote{Credit card fraud, identity theft, and account takeover are the most common fraud types in finance.}
\end{frame}

\begin{frame}[t]{Fraud Detection Pipeline}
\begin{center}
\includegraphics[width=0.60\textwidth]{figures/fraud_detection_pipeline/fraud_detection_pipeline.pdf}
\end{center}
\bottomnote{Modern fraud detection systems combine rules-based filtering with ML models for real-time scoring.}
\end{frame}

\begin{frame}[t]{Fraud Cost-Benefit Analysis}
\begin{center}
\includegraphics[width=0.60\textwidth]{figures/fraud_cost_benefit/fraud_cost_benefit.pdf}
\end{center}
\bottomnote{Optimal fraud thresholds balance fraud losses against customer friction from false positives.}
\end{frame}

\begin{frame}[t]{Anomaly Detection Techniques}
\begin{center}
\includegraphics[width=0.60\textwidth]{figures/anomaly_detection/anomaly_detection.pdf}
\end{center}
\bottomnote{Unsupervised methods detect anomalies by identifying patterns that deviate from normal behavior.}
\end{frame}

\begin{frame}[t]{Real-Time Scoring Architecture}
\begin{center}
\includegraphics[width=0.60\textwidth]{figures/real_time_scoring/real_time_scoring.pdf}
\end{center}
\bottomnote{Real-time fraud systems must score transactions in milliseconds to avoid payment delays.}
\end{frame}

\begin{frame}[t]{Summary}
\textbf{Key Takeaways:}
\begin{itemize}
\item ML significantly improves fraud detection accuracy
\item Real-time scoring essential (sub-100ms latency)
\item Class imbalance major challenge (0.1-1\% fraud rate)
\item Anomaly detection for new fraud patterns
\item Cost-benefit analysis determines thresholds
\item AML compliance requires explainable models
\end{itemize}

\vspace{1em}
\textbf{Next Lesson:} NLP in Finance
\bottomnote{Fraud evolves constantly - models require continuous monitoring and retraining.}
\end{frame}

\end{document}
