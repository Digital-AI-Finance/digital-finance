\documentclass[8pt,aspectratio=169]{beamer}
\usetheme{Madrid}
\usepackage{graphicx,booktabs,adjustbox,multicol,amsmath,amssymb}
\definecolor{mlblue}{RGB}{0,102,204}
\definecolor{mlpurple}{RGB}{51,51,178}
\definecolor{mllavender}{RGB}{173,173,224}
\definecolor{mllavender2}{RGB}{193,193,232}
\definecolor{mllavender3}{RGB}{204,204,235}
\definecolor{mllavender4}{RGB}{214,214,239}
\definecolor{mlorange}{RGB}{255,127,14}
\definecolor{mlgreen}{RGB}{44,160,44}
\definecolor{mlred}{RGB}{214,39,40}
\setbeamercolor{palette primary}{bg=mllavender3,fg=mlpurple}
\setbeamercolor{palette secondary}{bg=mllavender2,fg=mlpurple}
\setbeamercolor{palette tertiary}{bg=mllavender,fg=white}
\setbeamercolor{structure}{fg=mlpurple}
\setbeamercolor{frametitle}{fg=mlpurple,bg=mllavender3}
\setbeamertemplate{navigation symbols}{}
\setbeamertemplate{itemize items}[circle]
\setbeamersize{text margin left=5mm,text margin right=5mm}

% Bottom note command for key takeaways
\newcommand{\bottomnote}[1]{%
\vfill
\vspace{-2mm}
\textcolor{mllavender2}{\rule{\textwidth}{0.4pt}}
\vspace{1mm}
\footnotesize
\textbf{#1}
}
\title{Digital Finance 3: Technology in Finance}
\subtitle{Lesson 30: Credit Scoring and Risk Models}
\author{FHGR}
\date{\today}

\begin{document}

\begin{frame}
\titlepage
\bottomnote{Summary of key concepts presented above.}
\end{frame}

\begin{frame}[t]{Learning Objectives}
By the end of this lesson, you will be able to:
\begin{itemize}
\item Compare ML credit scoring with traditional methods (FICO)
\item Understand gradient boosting and tree-based models
\item Recognize fairness and bias issues in credit decisioning
\item Explain explainability requirements (GDPR, SHAP)
\item Describe regulatory frameworks (EBA guidelines)
\item Calculate PD and LGD metrics
\end{itemize}
\bottomnote{Summary of key concepts presented above.}
\end{frame}

\begin{frame}[t]{Traditional vs. ML Credit Scoring}
\begin{columns}[T]
\column{0.48\textwidth}
\textbf{FICO Score (Traditional):}
\begin{itemize}
\item Range: 300-850
\item 5 factors with fixed weights
\item Linear scorecard model
\item Only credit bureau data
\item AUC: 0.65-0.75
\item Transparent, regulated
\end{itemize}

\vspace{0.5em}
\textbf{Limitations:}
\begin{itemize}
\item Linear assumptions
\item One-size-fits-all
\item Limited for thin-file borrowers
\end{itemize}

\column{0.48\textwidth}
\textbf{ML Approach:}
\begin{itemize}
\item 100s-1000s of features
\item Bureau + alternative data
\item Non-linear (XGBoost, Random Forest)
\item AUC: 0.75-0.85
\item Higher accuracy
\end{itemize}

\vspace{0.5em}
\textbf{Business Impact:}
\begin{itemize}
\item 10-30\% default reduction
\item Or 15-25\% approval increase (same risk)
\end{itemize}

\vspace{0.5em}
\textbf{Trade-offs:}
\begin{itemize}
\item Accuracy vs. interpretability
\item Regulatory uncertainty
\item Fairness concerns
\end{itemize}
\end{columns}
\bottomnote{Comparative analysis helps identify the right tool for specific requirements.}
\end{frame}

\begin{frame}[t]{Fairness and Bias}
\begin{columns}[T]
\column{0.48\textwidth}
\textbf{Protected Attributes:}
\begin{itemize}
\item Race, color, national origin
\item Religion
\item Sex, gender
\item Age
\item Marital status
\end{itemize}

\vspace{0.5em}
\textbf{The Problem:}
\begin{itemize}
\item ML can learn biased patterns
\item Proxy variables (zip code = race)
\item Historical discrimination baked in
\end{itemize}

\column{0.48\textwidth}
\textbf{Fairness Metrics:}
\begin{itemize}
\item Demographic parity
\item Equal opportunity
\item Equalized odds
\end{itemize}

\vspace{0.5em}
\textbf{Mitigation:}
\begin{itemize}
\item Audit for disparate impact
\item Re-weighting samples
\item Constrained optimization
\item Regular monitoring
\end{itemize}

\vspace{0.5em}
\textbf{Laws:} ECOA, Fair Housing Act, GDPR Article 22
\end{columns}
\bottomnote{AI and ML are transforming financial services through automation and prediction.}
\end{frame}

\begin{frame}[t]{FICO Score Ranges and Risk Categories}
\begin{center}
\includegraphics[width=0.60\textwidth]{figures/fico_score_ranges/fico_score_ranges.pdf}
\end{center}
\bottomnote{FICO scores range from 300 to 850, with higher scores indicating lower default risk.}
\end{frame}

\begin{frame}[t]{Credit Score Distribution}
\begin{center}
\includegraphics[width=0.60\textwidth]{figures/credit_score_distribution/credit_score_distribution.pdf}
\end{center}
\bottomnote{Credit scores typically follow a bell-shaped distribution, skewed toward higher scores.}
\end{frame}

\begin{frame}[t]{Credit Model Comparison: Traditional vs. ML}
\begin{center}
\includegraphics[width=0.60\textwidth]{figures/credit_model_comparison/credit_model_comparison.pdf}
\end{center}
\bottomnote{Machine learning models consistently outperform traditional scorecards in predictive accuracy.}
\end{frame}

\begin{frame}[t]{Default Probability Curves}
\begin{center}
\includegraphics[width=0.60\textwidth]{figures/default_probability_curve/default_probability_curve.pdf}
\end{center}
\bottomnote{Default probability decreases non-linearly with credit score, demonstrating risk stratification.}
\end{frame}

\begin{frame}[t]{Feature Importance in Credit Models}
\begin{center}
\includegraphics[width=0.60\textwidth]{figures/feature_importance_credit/feature_importance_credit.pdf}
\end{center}
\bottomnote{Payment history and credit utilization are typically the most predictive features in credit models.}
\end{frame}

\begin{frame}[t]{Algorithm Performance Comparison}
\begin{center}
\includegraphics[width=0.60\textwidth]{figures/algorithm_comparison/algorithm_comparison.pdf}
\end{center}
\bottomnote{Gradient boosting methods like XGBoost consistently achieve the highest AUC scores in credit scoring.}
\end{frame}

\begin{frame}[t]{Summary}
\textbf{Key Takeaways:}
\begin{itemize}
\item ML improves credit scoring accuracy by 10-20\%
\item Gradient boosting (XGBoost) dominates
\item Fairness and bias are critical concerns
\item Explainability required (SHAP values)
\item Regulatory frameworks evolving (EBA guidelines)
\item PD and LGD are core risk metrics
\end{itemize}

\vspace{1em}
\textbf{Next Lesson:} Fraud Detection and AML
\bottomnote{Summary of key concepts presented above.}
\end{frame}

\end{document}
