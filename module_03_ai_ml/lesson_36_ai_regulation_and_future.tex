\documentclass[8pt,aspectratio=169]{beamer}
\usetheme{Madrid}
\usepackage{graphicx,booktabs,adjustbox,multicol,amsmath,amssymb}
\definecolor{mlblue}{RGB}{0,102,204}
\definecolor{mlpurple}{RGB}{51,51,178}
\definecolor{mllavender}{RGB}{173,173,224}
\definecolor{mllavender2}{RGB}{193,193,232}
\definecolor{mllavender3}{RGB}{204,204,235}
\definecolor{mllavender4}{RGB}{214,214,239}
\definecolor{mlorange}{RGB}{255,127,14}
\definecolor{mlgreen}{RGB}{44,160,44}
\definecolor{mlred}{RGB}{214,39,40}
\setbeamercolor{palette primary}{bg=mllavender3,fg=mlpurple}
\setbeamercolor{palette secondary}{bg=mllavender2,fg=mlpurple}
\setbeamercolor{palette tertiary}{bg=mllavender,fg=white}
\setbeamercolor{structure}{fg=mlpurple}
\setbeamercolor{frametitle}{fg=mlpurple,bg=mllavender3}
\setbeamertemplate{navigation symbols}{}
\setbeamertemplate{itemize items}[circle]
\setbeamersize{text margin left=5mm,text margin right=5mm}

% Bottom note command for key takeaways
\newcommand{\bottomnote}[1]{%
\vfill
\vspace{-2mm}
\textcolor{mllavender2}{\rule{\textwidth}{0.4pt}}
\vspace{1mm}
\footnotesize
\textbf{#1}
}
\title{Digital Finance 3: Technology in Finance}
\subtitle{Lesson 36: AI Regulation and Future}
\author{FHGR}
\date{\today}

\begin{document}

\begin{frame}
\titlepage
\bottomnote{AI regulation shapes the future of innovation in financial services.}
\end{frame}

\begin{frame}[t]{Learning Objectives}
By the end of this lesson, you will be able to:
\begin{itemize}
\item Explain the EU AI Act and risk-based framework
\item Understand financial sector-specific AI regulations
\item Navigate GDPR Article 22 (automated decisions)
\item Evaluate emerging regulatory trends
\item Anticipate future AI developments in finance
\item Assess career opportunities in AI finance
\end{itemize}
\bottomnote{EU AI Act classifies AI systems by risk level with strictest rules for high-risk uses.}
\end{frame}

\begin{frame}[t]{AI Regulation Landscape}
\begin{center}
\includegraphics[width=0.60\textwidth]{figures/ai_regulation_landscape/ai_regulation_landscape.pdf}
\end{center}
\bottomnote{Global AI regulation is converging toward risk-based frameworks prioritizing high-risk financial applications.}
\end{frame}

\begin{frame}[t]{EU AI Act Implementation Timeline}
\begin{center}
\includegraphics[width=0.60\textwidth]{figures/eu_ai_act_timeline/eu_ai_act_timeline.pdf}
\end{center}
\bottomnote{The EU AI Act enters force in phases, with full compliance required by 2026 for most financial systems.}
\end{frame}

\begin{frame}[t]{Summary}
\textbf{Key Takeaways:}
\begin{itemize}
\item EU AI Act establishes risk-based framework (2024-2026)
\item Financial ML systems mostly high-risk (credit, insurance)
\item GDPR Article 22: right to explanation for automated decisions
\item Explainability and fairness now regulatory requirements
\item Future trends: LLMs, federated learning, quantum ML
\item Career opportunities: ML engineers, model validators, AI ethicists
\end{itemize}

\vspace{1em}
\textbf{Module 3 Complete!} Next: Module 4 - Traditional Finance Technology
\bottomnote{AI regulation varies globally with EU taking the most prescriptive approach.}
\end{frame}

\end{document}
